\documentclass{article}
\usepackage{graphicx} % Required for inserting images
\usepackage[utf8]{inputenc}
\usepackage{mathrsfs,amsmath}
\usepackage[shortlabels]{enumitem}
\usepackage[margin=2cm]{geometry}
\usepackage{physics}
\usepackage{xcolor}
\usepackage{titlesec}
\usepackage{stmaryrd}
\usepackage{hyperref}
\usepackage{blindtext}
\titleformat*{\subparagraph}{}

\newcommand{\im}[1]{\text{im }{#1}}
\newcommand{\coker}[1]{\text{coker }{#1}}
\newcommand{\dash}{\text{-}}
\newcommand{\fps}[1]{#1 \llbracket x \rrbracket}
\newcommand{\floor}[1]{\lfloor #1 \rfloor}
\newcommand{\ceil}[1]{\lceil #1 \rceil}


\NeedsTeXFormat{LaTeX2e}
\ProvidesPackage{quiver}[2021/01/11 quiver]

% `tikz-cd` is necessary to draw commutative diagrams.
\RequirePackage{tikz-cd}
% `amssymb` is necessary for `\lrcorner` and `\ulcorner`.
\RequirePackage{amssymb}
% `calc` is necessary to draw curved arrows.
\usetikzlibrary{calc}
% `pathmorphing` is necessary to draw squiggly arrows.
\usetikzlibrary{decorations.pathmorphing}

% A TikZ style for curved arrows of a fixed height, due to AndréC.
\tikzset{curve/.style={settings={#1},to path={(\tikztostart)
    .. controls ($(\tikztostart)!\pv{pos}!(\tikztotarget)!\pv{height}!270:(\tikztotarget)$)
    and ($(\tikztostart)!1-\pv{pos}!(\tikztotarget)!\pv{height}!270:(\tikztotarget)$)
    .. (\tikztotarget)\tikztonodes}},
    settings/.code={\tikzset{quiver/.cd,#1}
        \def\pv##1{\pgfkeysvalueof{/tikz/quiver/##1}}},
    quiver/.cd,pos/.initial=0.35,height/.initial=0}

% TikZ arrowhead/tail styles.
\tikzset{tail reversed/.code={\pgfsetarrowsstart{tikzcd to}}}
\tikzset{2tail/.code={\pgfsetarrowsstart{Implies[reversed]}}}
\tikzset{2tail reversed/.code={\pgfsetarrowsstart{Implies}}}
% TikZ arrow styles.
\tikzset{no body/.style={/tikz/dash pattern=on 0 off 1mm}}



\title{Algebra: Chapter 0 - Aluffi}
\author{Conan Pickford \texttt{<pickfordconan@gmail.com>}}
\date{October 2023}

\begin{document}

\maketitle

\section*{I - Categories}
\subsection*{1.3 - Categories} 
\paragraph{1.} Let $\mathsf{C}$ be a category. Consider the structure $\mathsf{C}^{\text{op}}$ with $\mathsf{obj}(\mathsf{C}^{\text{op}}) := \mathsf{obj}(\mathsf{C})$ and for objects $A,B$ of $\mathsf{C}^{\text{op}}$, we have that $\text{Hom}_{\mathsf{C}^{\text{op}}}(A,B) := \text{Hom}_{\mathsf{C}}(B,A)$. For objects $A,B,C$ of $\mathsf{C}$ and $f \in \text{Hom}_{\mathsf{C}^{\text{op}}}(A,B)$ and $g \in \text{Hom}_{\mathsf{C}^{\text{op}}}(B,C)$, define the composition law $\circ_{\mathsf{C}^{\text{op}}}$ by $g \circ_{\mathsf{C}^{\text{op}}} f := f \circ_{\mathsf{C}} g \in \text{Hom}_{\mathsf{C}}(C,A) = \text{Hom}_{\mathsf{C}^{\text{op}}}(A,C)$. Let $A,B,C,D$ be objects in $\mathsf{C}^{\text{op}}$. Let $f \in \text{Hom}_{\textsf{C}^{\text{op}}}(A,B), g \in \text{Hom}_{\textsf{C}^{\text{op}}}(B,C)$ and $h \in \text{Hom}_{\textsf{C}^{\text{op}}}(C,D)$. Then, $(h \circ_{\mathsf{C}^{\text{op}}} g) \circ_{\mathsf{C}^{\text{op}}} f = f \circ_{\mathsf{C}} (g \circ_{\mathsf{C}} h) = (f \circ_{\mathsf{C}} g) \circ_{\mathsf{C}} h = h \circ_{\mathsf{C}^{\text{op}}} (g \circ_{\mathsf{C}^{\text{op}}} f)$ as $\circ_{\mathsf{C}}$ is an associative operation. Hence, $\circ_{\mathsf{C}^{\text{op}}}$ is associative. For each object $A$ of $\mathsf{C}^{\text{op}}$, we have that there is an identity morphism $1_{A} \in \text{Hom}_{\mathsf{C}}(A,A) = \text{Hom}_{\mathsf{C}^{\text{op}}}(A,A)$ as $\mathsf{C}$ is a category. Let $f \in \text{Hom}_{\mathsf{C}^{\text{op}}}(A,B)$, we have that $f \circ_{\mathsf{C}^{\text{op}}} 1_{A} = 1_{A} \circ_{\mathsf{C}} 1_{A} = f$ and $1_{B} \circ_{\mathsf{C}^{\text{op}}} f = f \circ_{\mathsf{C}} 1_{B} = f$ as $1_{A}, 1_{B}$ are the identities under $\circ_{\mathsf{C}}$. Lastly, we have that for all objects $A,B,C,D$ of $\mathsf{C}^{\text{op}}$, we have that $\text{Hom}_{\textsf{C}^{\text{op}}}(A,B)$ and $\text{Hom}_{\textsf{C}^{\text{op}}}(C,D)$ are disjoint as $\text{Hom}_{\textsf{C}}(B,A)$ and $\text{Hom}_{\textsf{C}}(D,C)$ are disjoint as $\mathsf{C}$ is a category. It follows that $\mathsf{C}^{\text{op}}$ is a category. 

\paragraph{2.} Let $A$ be a finite set such that $|A| = n \in \mathbb{N}$. We have that $\text{Hom}_{\mathsf{Set}}(A,B)$ is the set of all functions from $A$ to $B$, in which there are $|B|^{|A|}$ of them. Then, $|\text{End}_{\mathsf{Set}}(A)| = |\text{Hom}_{\mathsf{Set}}(A,A)| = |A|^{|A|} = n^{n}$. 

\paragraph{3.} Let $S$ be a set and $\sim$ a relation on $S$ such that $\sim$ is reflexive and transitive. Let $\mathsf{S}$ be the category with $\mathsf{obj}(\mathsf{S}) = S$ and morphisms as if $a,b \in S$, then $\text{Hom}_{\mathsf{S}}(a,b)$ be the set consisting of $(a,b) \in S \times S$ if $a \sim b$ and let $\text{Hom}(a,b) = \emptyset$ otherwise. Define the compisition law $\circ_{\mathsf{S}}$ by if $f \in \text{Hom}_{\mathsf{S}}(a,b)$ and $g \in \text{Hom}_{\mathsf{S}}(b,c)$, then $g \circ_{\mathsf{S}} f = (a,c) \in \text{Hom}_{\mathsf{S}}(a,c)$. For each object $a$ of $\mathsf{S}$, let $1_{a} = (a,a)$. Let $f \in \text{Hom}_{\mathsf{S}}(a,b)$, we have that $f \circ_{\mathsf{S}} 1_{a} = (a,b) \circ_{\mathsf{S}} (a,a) = (a,b) = f$ and $f \circ_{\mathsf{S}} 1_{b} = (a,b) \circ_{\mathsf{S}} (b,b) = (a,b) = f$. Hence, $(a,a) \in \text{Hom}_{\mathsf{S}}(a,a)$ is the identity morphism on $a$. 

\paragraph{4.} As $<$ is not reflexive, we cannot define a category on the set $\mathbb{Z}$ in the style of the previous exercise. 

\paragraph{5.} Let $S$ be a set and define a relation on $\mathcal{P}(S)$ by $A \sim B$ if and only if $A \subseteq B$. We have that $\sim$ is reflexive and transitive. We define a category on $\mathcal{P}(S)$ in the style of (3). 

\paragraph{6.} 

\paragraph{7.} Let $\mathsf{C}$ be a category. Let $A$ be an onject in $\mathsf{C}$. Consider the structure $\mathsf{C}^{A}$ where $\mathsf{obj}(\mathsf{C}^{A})$ are all morphisms from $A$ to any object of $\mathsf{C}$; thus, an object of $\mathsf{C}^{A}$ is a morphism $f \in \text{Hom}_{\mathsf{C}}(A, Z)$ for some object $Z$ of $\mathsf{C}$. Let $f_{1}, f_{2}$ be objects of $\mathsf{C}^{A}$, that is, two arrows, 
\[\begin{tikzcd}
	A & A \\
	{Z_{1}} & {Z_{2}}
	\arrow["{f_{1}}"', from=1-1, to=2-1]
	\arrow["{f_{2}}", from=1-2, to=2-2]
\end{tikzcd}\]
Define morphisms $f_{1} \to f_{2}$ to be commutative diagrams 
\[\begin{tikzcd}
	& A \\
	{Z_{1}} && {Z_{2}}
	\arrow["{f_{2}}", from=1-2, to=2-3]
	\arrow["{f_{1}}"', from=1-2, to=2-1]
	\arrow["\sigma"', from=2-1, to=2-3]
\end{tikzcd}\]
That is, morphisms $f \to g$ correspond precisely to those morphisms $\sigma: Z_{1} \to Z_{2}$ in $\mathsf{C}$ such that $\sigma f = g$. For $f\in \text{Hom}_{\mathsf{C}^{A}}(f_{1},f_{2})$ and $g \in \text{Hom}_{\mathsf{C}^{A}}(f_{2},f_{3})$, thats is, the commutative diagrams
\[\begin{tikzcd}
	& A &&&& A \\
	{Z_{1}} && {Z_{2}} && {Z_{2}} && {Z_{3}}
	\arrow["{f_{1}}"', from=1-2, to=2-1]
	\arrow["{f_{2}}", from=1-2, to=2-3]
	\arrow["\sigma"', from=2-1, to=2-3]
	\arrow["\tau"', from=2-5, to=2-7]
	\arrow["{f_{2}}"', from=1-6, to=2-5]
	\arrow["{f_{3}}", from=1-6, to=2-7]
\end{tikzcd}\]
respecitvely, define the composition $\circ_{\mathsf{C}^{A}}$ by $g \circ_{\mathsf{C}^{A}} f$ is the commutative diagram
\[\begin{tikzcd}
	& A \\
	{Z_{1}} && {Z_{3}}
	\arrow["{f_{1}}"', from=1-2, to=2-1]
	\arrow["{f_{3}}", from=1-2, to=2-3]
	\arrow["\sigma\tau"', from=2-1, to=2-3]
\end{tikzcd}\]
For $f \in \text{Hom}_{\mathsf{C}^{A}}(f_{1},f_{2})$, let $1_{f_{1}} \in \text{Hom}_{\mathsf{C}^{A}}(f_{1},f_{1})$ and $1_{f_{2}}\in \text{Hom}_{\mathsf{C}^{A}}(f_{2},f_{2})$ be the commutative diagrams
\[\begin{tikzcd}
	& A &&&& A \\
	{Z_{1}} && {Z_{1}} && {Z_{2}} && {Z_{2}}
	\arrow["{f_{1}}"', from=1-2, to=2-1]
	\arrow["{f_{1}}", from=1-2, to=2-3]
	\arrow["{1_{Z_{1}}}"', from=2-1, to=2-3]
	\arrow["{f_{2}}"', from=1-6, to=2-5]
	\arrow["{f_{2}}", from=1-6, to=2-7]
	\arrow["{1_{Z_{2}}}"', from=2-5, to=2-7]
\end{tikzcd}\]
respectively. We have that $f \circ_{\mathsf{C}^{A}} 1_{f_{1}} = f$ and $1_{f_{2}} \circ_{\mathsf{C}^{A}} f = f$. Finally, we also note that associaitvity holds as $\circ_{\mathsf{C}}$ is associative in $\mathsf{C}$. Therefore, $\mathsf{C}^{A}$ is a category. 

\paragraph{8.} Let $\mathsf{Set}_{\infty}$ be a structure such that $\mathsf{obj}(\mathsf{Set}_{\infty})$ are infinite sets and for $A,B \in \mathsf{obj}(\mathsf{Set}_{\infty})$, we have that $\text{Hom}_{\mathsf{Set}_{\infty}}(A,B)$ are the set functions from $A$ to $B$. Define the composition law $\circ_{\mathsf{Set}_{\infty}}$ as function composition. Thus, $\circ_{\mathsf{Set}_{\infty}}$ is an associative operation. For $f \in \text{Hom}_{\mathsf{Set}_{\infty}}(A,B)$ define $1_{A}$ to be the identity function on $A$. Then, $f \circ_{\mathsf{Set}_{\infty}} 1_{A} = f$ and $1_{B} \circ_{\mathsf{Set}_{\infty}} f = f$. Therefore, $\mathsf{Set}_{\infty}$ is a category. We have that $\mathsf{Set}_{\infty}$ is a subcategory of $\mathsf{Set}$ as $\mathsf{obj}(\mathsf{Set}_{\infty}) \subseteq \mathsf{obj}(\mathsf{Set})$ and $\text{Hom}_{\mathsf{Set}_{\infty}}(A,B) \subseteq \text{Hom}_{\mathsf{Set}}(A,B)$ for all objects $A,B \in \mathsf{obj}(\mathsf{Set}_{\infty})$. In fact, $\text{Hom}_{\mathsf{Set}_{\infty}}(A,B) = \text{Hom}_{\mathsf{Set}}(A,B)$ as both are classes of all set functions from $A$ to $B$. It follows that $\mathsf{Set}_{\infty}$ is a full subcategory of $\mathsf{Set}$.

\paragraph{9.}
\paragraph{10.}
\paragraph{11.}

\subsection*{1.4 - Morphisms} 
\paragraph{1.}

\paragraph{2.} Let $S$ be a set and $\sim$ be an equivalence relation on $S$. Define a category $\mathsf{C}$ such that $\mathsf{obj}(\mathsf{C}) = S$ and for objects $A,B$ of $\mathsf{C}$, define $\text{Hom}_{\mathsf{C}}(A,B) = (A,B)$ if $A \sim B$ and $\text{Hom}_{\mathsf{C}}(A,B) = \emptyset$ otherwise. Define the composition law as before. Let $f \in \text{Hom}_{\mathsf{C}}(A,B)$. As $\text{Hom}_{\mathsf{C}}(A,B)$ is non-empty, we have that $A \sim B$ and so $B \sim A$ as $\sim$ is symmetric. There then exists a $g \in \text{Hom}_{\mathsf{C}}(B,A)$ where $g = (B,A)$. We have that $gf = (a,a) = 1_{A} \in \text{End}_{\mathsf{C}}(A)$ and $fg = (b,b) = 1_{B} \in \text{End}_{\mathsf{C}}(B)$. Therefore, $f$ is an isomorphism. As $f$ was arbitrary, it follows that $\mathsf{C}$ is a groupoid. 

\paragraph{3.} Let $A,B$ be objects of a category $\mathsf{C}$ and let $f \in \text{Hom}_{\mathsf{C}}(A,B)$ be a morphism such that $f$ has a right inverse. Thus, there exists a $g \in \text{Hom}_{\mathsf{C}}(B,A)$ such that $fg = 1_{B}$ where $1_{B} \in \text{End}_{\mathsf{C}}(B)$ is the identity morphism on $B$. Let $Z$ be an object of $\mathsf{C}$ and $\beta', \beta'' \in \text{Hom}_{\mathsf{C}}(B,Z)$ be morphisms from $B$ to $Z$ such that $\beta'f = \beta''f$. Then, $\beta' = \beta'1_{B} = \beta'(fg) = (\beta'f)g = (\beta''f)g = \beta''(fg) = \beta''1_{B} = \beta''$. Therefore, $\beta' = \beta''$ and so $f$ is an epimorphism. The converse is not true, however. Let $\mathsf{C}$ be a category such that $\mathsf{obj}(\mathsf{C}) = \mathbb{Z}$ and for each pair of objects $A,B$ of $\mathsf{C}$, we have $\text{Hom}_{\mathsf{C}}(A,B) = (A,B) \in \mathbb{Z} \times \mathbb{Z}$ if $A \leq B$ and $\text{Hom}_{\mathsf{C}}(A,B) = \empty$ otherwise. Let $f = (0,1) \in \text{Hom}_{\mathsf{C}}(0,1)$. Let $Z$ be an object of $\mathsf{C}$ and let $\beta',\beta'' \in \text{Hom}_{\mathsf{C}}(1,Z)$ such that $\beta' \circ f = \beta'' \circ f$. We have that $\text{Hom}_{\mathsf{C}}(1,Z)$ is non-empty and will contain only one element, namely, $(1,Z) \in \mathbb{Z} \times \mathbb{Z}$. Then, $\beta' = \beta'' = (1, Z)$. Therefore, $f$ is an epimorphism. Suppose there exists a $g \in \text{Hom}_{\mathsf{C}}(1,0)$, then we have that $1 \leq 0$. This is ofcourse ridiculous, hence, $g$ cannot exist. There cannot possibly exist a right inverse of $f$. 

\paragraph{4.} Let $\mathsf{C}$ be a category. Let $f \in \text{Hom}_{\mathsf{C}}(A,B), g \in \text{Hom}_{\mathsf{C}}(B,C)$ be monomorphisms. Let $gf \in \text{Hom}_{\mathsf{C}}(A,C)$ be their composition. Let $Z$ be an object of $\mathsf{C}$ and $\alpha, \alpha' \in \text{Hom}_{\mathsf{C}}(Z,A)$ such that $gf\alpha = gf\alpha'$. We have that $f\alpha, f\alpha' \in \text{Hom}_{\mathsf{C}}(Z,B)$ and since $g$ is a monomorphism, we have that $f\alpha = f\alpha'$. As $f$ is a monomorphism, we have that $\alpha = \alpha'$. Therefore, $gf$ is a monomorphism. Consider the structure $\mathsf{C}_{\text{mono}}$ where $\mathsf{obj}(\mathsf{C}_{\text{mono}}) = \mathsf{obj}(\mathsf{C})$ and for objects $A,B$ of $\mathsf{C}_{\text{mono}}$, let $\text{Hom}_{\mathsf{C}_{\text{mono}}}(A,B)$ be the monomorphisms of $\text{Hom}_{\mathsf{C}}(A,B)$. For $f \in \text{Hom}_{\mathsf{C}_{\text{mono}}}(A,B), g \in \text{Hom}_{\mathsf{C}_{\text{mono}}}(B,C)$, define their composition $g \circ_{\mathsf{C}_{\text{mono}}} f = g \circ_{\mathsf{C}} f \in \text{Hom}_{\mathsf{C}_{\text{mono}}}(A,C)$. We have that $\circ_{\mathsf{C}_{\text{mono}}}$ is associative as $\circ_{\mathsf{C}}$ is associative. For object $A$ of $\mathsf{C}_{\text{mono}}$, let $1_{A} \in \text{End}_{\mathsf{C}_{\text{mono}}}(A)$ be the identity morphism of $A$ in $\mathsf{C}$. We verify that $1_{A}$ is a monomorphism. Let $Z$ be an object in $\mathsf{C}_{\text{mono}}$ and $\alpha, \alpha' \in \text{Hom}_{\mathsf{C}_{\text{mono}}}(Z,A)$ such that $1_{A} \circ_{\mathsf{C}_{\text{mono}}} \alpha = 1_{A} \circ_{\mathsf{C}_{\text{mono}}} \alpha'$. Then, $1_{A} \circ_{\mathsf{C}_{\text{mono}}} \alpha = 1_{A} \circ_{\mathsf{C}_{\text{mono}}} \alpha' \implies 1_{A} \circ_{\mathsf{C}} \alpha = 1_{A} \circ_{\mathsf{C}} \alpha' \implies \alpha = \alpha'$ as $1_{A} \in \text{End}_{\mathsf{C}}(A)$ is the identity morphism. Hence, $1_{A}$ is a monomorphism. Therefore, $1_{A} \in \text{End}_{\mathsf{C}_{\text{mono}}}(A)$. Let $f \in \text{Hom}_{\mathsf{C}_{\text{mono}}}(A,B)$. Then, $f \circ_{\mathsf{C}_{\text{mono}}} 1_{A} = f \circ_{\mathsf{C}} 1_{A} = f$ and $1_{B} \circ_{\mathsf{C}_{\text{mono}}} f = 1_{B} \circ_{\mathsf{C}} f = f$. Therefore, the identity morphisms are identities with respect to composition in $\mathsf{C}_{\text{mono}}$. It follows that $\mathsf{C}_{\text{mono}}$ is a category and is a subcategory of $\mathsf{C}$.

\paragraph{5.}

\subsection*{1.5 - Universal Properties} 
\paragraph{1.} Let $\mathsf{C}$ be a category and $\mathsf{C}^{\text{op}}$ be its opposite category. Let $F$ be a final object of $\mathsf{C}$. We have that $F$ is an object of $\mathsf{C}^{\text{op}}$. Let $A$ be an object of $\mathsf{C}^{\text{op}}$. Then, $\text{Hom}_{\mathsf{C}^{\text{op}}}(F,A) = \text{Hom}_{\mathsf{C}}(A,F)$ is singleton as $F$ is final in $\mathsf{C}$. Therefore, $F$ is initial in $\mathsf{C}^{\text{op}}$. 

\paragraph{2.} Let $\emptyset$ be the empty set in the category $\mathsf{Set}$. For all objects $A$ of $\mathsf{Set}$, we have that there is exactly one set function from $\emptyset$ to $A$, namely, the empty graph. Therefore, $\text{Hom}_{\mathsf{Set}}(\emptyset,A)$ is a singleton for all objects $A$. Thus, $\emptyset$ is inital in $\mathsf{Set}$. Suppose there exists an object $I$ of $\mathsf{Set}$ such that $I$ is non-empty and $I$ is an initial object. As $I$ is an initial object, we have that $\text{Hom}_{\mathsf{Set}}(I,A)$ is singleton for all objects $A$. However, $\text{Hom}_{\mathsf{Set}}(I, \emptyset)$ is empty as $I$ is non-empty. Hence, $I$ cannot possibly be initial in $\mathsf{Set}$. It follows that $\emptyset$ is the unique initial object of $\mathsf{Set}$. 

\paragraph{3.} Let $\mathsf{C}$ be a category and let $F, F'$ be final objects of $\mathsf{C}$. Let $f \in \text{Hom}_{\mathsf{C}}(F,F')$ and $g \in \text{Hom}_{\mathsf{C}}(F',F)$. We note $f \in \text{Hom}_{\mathsf{C}}(F,F')$ is unique and $g \in \text{Hom}_{\mathsf{C}}(F',F)$ is unique as $F,F'$ are final objects. We have that $gf \in \text{Hom}_{\mathsf{C}}(F,F)$ and $1_{F} \in \text{Hom}_{\mathsf{C}}(F,F)$, thus, $gf = 1_{F}$ as $F$ is final and so $\text{Hom}_{\mathsf{C}}(F,F)$ is a singleton. Similarly, $\text{Hom}_{\mathsf{C}}(F',F') \ni fg = 1_{F'}$. It follows that $f$ and $g$ are isomorphisms and $F$ is isomorphic to $F'$. 

\paragraph{4.} Let $\mathsf{Set}^{*}$ be the category of pointed sets. Let $f:\{*\} \to S$ be an object of $\mathsf{Set}^{*}$ such that $S$ is singleton. Let $g:\{*\} \to A$ be an object of $\mathsf{Set}^{*}$. We have that a morphism $f \to g$ would correspond to a set function $\sigma:S \to A$ such that $\sigma f = g$. We have that there exists only one choice of $\sigma$, namely, the map $\alpha \mapsto g(*)$ where $\alpha \in S$. Thus, $\text{Hom}_{\mathsf{Set}^{*}}(f,g)$ is singleton. Similarly, a morphism $g \to f$ corresponds to a set function $\tau:A \to S$ such that $\tau g = f$ and there exists only one possible choice of $\tau$, namely, the constant function. Hence, $\text{Hom}_{\mathsf{Set}^{*}}(g,f)$ is singleton. We have that $f$ is an initial and final object of $\mathsf{Set}^{*}$. 

\paragraph{5.} Let $\sim$ be an equivalence relation defined on a set $A$. Let $\mathsf{C}$ be a category where $\mathsf{obj}(\mathsf{C})$ is the class of set functions $\varphi:A \to Z$ where $Z$ is a set and for $a,a' \in A$, if $a \sim a'$, then $\varphi(a) = \varphi(a')$. For $f,g$ objects of $\mathsf{C}$, let a morphism $f \to g$ correspond to a set function $\sigma$ such that $\sigma f = g$. Let $\varphi:A \to Z$ be an object of $\mathsf{C}$ and let $f: A \to \{*\}$ be an object such that $\{*\}$ is singleton. In the commutative diagram
\[\begin{tikzcd}
	& A \\
	Z && {\{*\}}
	\arrow["\varphi"', from=1-2, to=2-1]
	\arrow["f", from=1-2, to=2-3]
	\arrow["\sigma"', from=2-1, to=2-3]
\end{tikzcd}\]
there is only one possibility for $\sigma:Z \to \{*\}$. Therefore, $\text{Hom}_{\mathsf{C}}(\varphi, f)$ is singleton. It follows that $f$ is a final object in $\mathsf{C}$. 

\paragraph{6.} Let $\mathsf{C}$ be the category corresponding to endowing $\mathbb{Z}^{+}$ with the relation $\sim$ where $a \sim a'$ if $a \mid a'$. Let $a,b$ be objects in $\mathsf{C}$. Let $(g_{1},g_{2})$ be the object in $\mathsf{C}_{a,b}$ corresponding to the diagram 
\[\begin{tikzcd}
	& a \\
	{\gcd(a,b)} \\
	& b
	\arrow["{g_{1}}", from=2-1, to=1-2]
	\arrow["{g_{2}}"', from=2-1, to=3-2]
\end{tikzcd}\]
Let $(f_{1},f_{2})$ be an object in $\mathsf{C}_{a,b}$. We have that a morphism $(f_{1},f_{2}) \to (g_{1},g_{2})$ corresponds to a commutative diagram
\[\begin{tikzcd}[ampersand replacement=\&]
	\&\& a \\
	c \& {\gcd(a,b)} \\
	\&\& b
	\arrow["{f_{1}}", curve={height=-12pt}, from=2-1, to=1-3]
	\arrow["\sigma"', from=2-1, to=2-2]
	\arrow["{f_{2}}"', curve={height=12pt}, from=2-1, to=3-3]
	\arrow["{g_{1}}"', from=2-2, to=1-3]
	\arrow["{g_{2}}", from=2-2, to=3-3]
\end{tikzcd}\]
As there is a morphism $c \to a$ and a morphism $c \to b$ in $\mathsf{C}$, we have that $c \mid a$ and $c \mid b$. Hence, $c \mid \gcd(a,b)$ by definition. It follows that $\text{Hom}_{\mathsf{C}}(c,\gcd(a,b))$ is non-empty and is then singleton. Therefore, there is a unique $\sigma$ which makes the above diagram commute. Hence, $\text{Hom}_{\mathsf{C}_{a,b}}((f_{1},f_{2}),(g_{1},g_{2}))$ is singleton and so $(g_{1},g_{2})$ is a final object of $\mathsf{C}_{a,b}$. Let $(h_{1}, h_{2})$ be the object in $\mathsf{C}^{a,b}$ corresponding to the diagram
\[\begin{tikzcd}
	& a \\
	{\text{lcm}(a,b)} \\
	& b
	\arrow["{h_{1}}"', from=1-2, to=2-1]
	\arrow["{h_{2}}", from=3-2, to=2-1]
\end{tikzcd}\]
Let $(f_{1},f_{2})$ be an object in $\mathsf{C}^{a,b}$. We have that a morphism $(f_{1},f_{2}) \to (h_{1},h_{2})$ corresponds to a commutative diagram
\[\begin{tikzcd}[ampersand replacement=\&]
	\&\& a \\
	c \& {\text{lcm}(a,b)} \\
	\&\& b
	\arrow["\tau", from=2-2, to=2-1]
	\arrow["{h_{1}}", from=1-3, to=2-2]
	\arrow["{h_{2}}"', from=3-3, to=2-2]
	\arrow["{f_{1}}"', curve={height=12pt}, from=1-3, to=2-1]
	\arrow["{f_{2}}", curve={height=-12pt}, from=3-3, to=2-1]
\end{tikzcd}\]
As there is a morphism $a \to c$ and a morphism $b \to c$ in $\mathsf{C}$, we have that $a \mid c$ and $b \mid c$. Hence, $\text{lcm}(a,b) \mid c$. It follows that $\text{Hom}_{\mathsf{C}}(\text{lcm}(a,b),c)$ is non-empty and is then singleton. Therefore, there is a unique $\tau$ which makes the above diagram commute. Hence, $\text{Hom}_{\mathsf{C}^{a,b}}((f_{1},f_{2}),(h_{1},h_{2}))$ is singleton and so $(h_{1},h_{2})$ is a final object of $\mathsf{C}^{a,b}$. We can conclude that $\mathsf{C}$ has products and coproducts. 

\paragraph{7.} 

\paragraph{8.} 

\paragraph{9.}
\paragraph{10.}
\paragraph{11.}
\paragraph{12.} \textbf{Not Done} Let $\alpha:A \to C$ and $\beta:B \to C$ be morphisms in $\mathsf{Set}$. Let $A \times_{\alpha,\beta} B$ be the subset of $A \times B$ defined by $A \times_{\alpha,\beta} B = \{(x,y) \mid \alpha(x) = \beta(y)\}$. Let 
\[\begin{tikzcd}[ampersand replacement=\&]
	\& A \\
	Z \&\& C \\
	\& B
	\arrow["g"', from=2-1, to=3-2]
	\arrow["f", from=2-1, to=1-2]
	\arrow["\alpha", from=1-2, to=2-3]
	\arrow["\beta"', from=3-2, to=2-3]
\end{tikzcd}\]
be a morphism in $\mathsf{Set}_{\alpha,\beta}$. We have that the following diagram is commutative as a morphism $(Z,f,g) \to (A \times_{\alpha,\beta} B, \pi_{A}, \pi_{B})$ in $\mathsf{Set}_{\alpha,\beta}$
\[\begin{tikzcd}[ampersand replacement=\&]
	\&\& A \\
	Z \& {A\times_{\alpha,\beta}B} \&\& C \\
	\&\& B
	\arrow["\sigma", from=2-1, to=2-2]
	\arrow["{\pi_{A}}"', from=2-2, to=1-3]
	\arrow["\alpha", from=1-3, to=2-4]
	\arrow["{\pi_{B}}", from=2-2, to=3-3]
	\arrow["\beta"', from=3-3, to=2-4]
	\arrow["f", curve={height=-12pt}, from=2-1, to=1-3]
	\arrow["g"', curve={height=12pt}, from=2-1, to=3-3]
\end{tikzcd}\]
As the diagram is commutative, $f = \pi_{A}\sigma$ and $g = \pi_{B}\sigma$, hence, $\sigma(z) = (f(z),g(z))$ for all $z \in Z$. Thus, $\sigma$ is unique. We also have that $\sigma$ is well-defined as for each $z \in Z$, we have that $(\alpha f)(z) = (\beta g)(z)$, so $(f(z), g(z)) \in A \times_{\alpha,\beta} B$. It follows that $(A \times_{\alpha,\beta} B, \pi_{A}, \pi_{B})$ is final in $\mathsf{Set}_{\alpha,\beta}$, thus, $\mathsf{Set}$ has fibered products. Similarly, let $\alpha:C \to A, \beta:C \to B$ be morphisms in $\mathsf{Set}$. Let $\sim$ be an equivalence relation defined on $A \amalg B$ generated by the set of $(0,\alpha(x)) \sim (1,\beta(x))$ for all $x \in C$. Let
\[\begin{tikzcd}
	& A \\
	C && Z \\
	& B
	\arrow["\alpha", from=2-1, to=1-2]
	\arrow["\beta"', from=2-1, to=3-2]
	\arrow["f", from=1-2, to=2-3]
	\arrow["g"', from=3-2, to=2-3]
\end{tikzcd}\]
be a morphism in $\mathsf{Set}$. We have that the following diagram is commutative as a morphism $(A \amalg B/\sim, i_{A}, i_{B}) \to (Z,f,g)$
\[\begin{tikzcd}
	& A \\
	C && {A \amalg B/\sim} & Z \\
	& B
	\arrow["\alpha", from=2-1, to=1-2]
	\arrow["\beta"', from=2-1, to=3-2]
	\arrow["{i_{A}}", from=1-2, to=2-3]
	\arrow["{i_{B}}"', from=3-2, to=2-3]
	\arrow["\sigma", from=2-3, to=2-4]
	\arrow["f", curve={height=-12pt}, from=1-2, to=2-4]
	\arrow["g"', curve={height=12pt}, from=3-2, to=2-4]
\end{tikzcd}\]
where $i_{A}:A \to A \amalg B/ \sim$ is defined by $i_{A}(a) = [(0,a)]_{\sim}$ and $i_{A}:B \to A \amalg B/ \sim$ is defined by $i_{B}(b) = [(1,b)]_{\sim}$. 


\section*{II - Groups, first encounter}
\subsection*{2.1 - Definition of Group}

\paragraph{1.}
\paragraph{2.}
\paragraph{3.} Let $G$ be a group and $h,g \in G$. We have that $(hg)(g^{-1}h^{-1}) = hgg^{-1}h^{-1} = he_{G}h^{-1} = hh^{-1} = e_{G}$. Therefore, $(hg)^{-1} = g^{-1}h^{-1}$. 

\paragraph{4.} Let $G$ be a group such that for each $g \in G$, we have that $g^{2} = e_{G}$. Let $x,y \in G$, we have that $x^{2} = e_{G}$ and $y^{2} = e_{G}$. Then, $x^{2}y^{2} = e_{G}$. We also have that $xy \in G$ and so $xyxy = (xy)^{2} = e_{G}$. Hence, $x^{2}y^{2} = xyxy$. By left and right cancellation, it follows that $xy = yx$. Therefore, $G$ is abelian. 

\paragraph{5.} Let $G$ be a group such that there exists $x,y,z$ such that $xz = yz$. By right cancellation, we must have that $x = y$. Hence, in a groups multiplication table, every column and every row must contain all the elements of the group exactly once. 

\paragraph{6.}

\paragraph{7.} Let $g \in G$ be an element of finite order and let $N \in \mathbb{Z}$. Suppose that $g^{N} = e$. By Lemma 1.10, we have that $|g| \mid N$. For the converse, suppose that $|g| \mid N$. Then, $N = k|g|$ for some $k \in \mathbb{N}$. We have that $g^{N} = g^{k|g|} = (g^{|g|})^{k} = e^{k} = e$. 


\paragraph{8.} Let $G$ be a finite abelian group with exactly one element $f \in G$ of order $2$. For each non-trivial $g$ with $g \neq f$, we have that $g$ has a unique inverse in $G$ that is not itself. If $g$ was self inverse, then $g$ has order $2$, which is not possible. Hence, $\prod_{g\in G}g = f$. 

\paragraph{9.} Let $G$ ve a finite group of order $n$ with $m$ elements of order $2$. We must have that $n-m-1$ is even as this number represents the number of elements in $G$ that are not self inverse. As every inverse of $g \in G$ is unique for those $n-m-1$ elements, we must have that $n-m-1$ is even as for every $g$, there is a unique element $g^{-1} \in G$ that is also not of order $2$. Hence, $n-m$ is odd. We deduce that if $n$ is even, then, as $n-m$ is odd, we must have that $G$ necessarily contains an element of order $2$ as $m \geq 1$. 

\paragraph{10.}

\paragraph{11.} Let $G$ be a group and $g,h \in G$. Suppose $g$ has order $n$ and $hgh^{-1}$ has order $m$. We have that $(hgh^{-1})^{n} = hg^{n}h^{-1} = heh^{-1} = hh^{-1} = e$, hence, $n$ is a multiple of $m$. We have that $e = (hgh^{-1})^{m} = hg^{m}h^{-1} \implies g^{m} = e$. Thus, $m$ is a multiple of $n$. It follows that $m = n$. Therefore, $|gh| = |h(gh)h^{-1}| = |hghh^{-1}| = |hge| = |hg|$. 

\paragraph{12.}

\paragraph{13.} Let $G = \mathbb{Z}_{8}$ and $g = h = [4] \in G$. Note that $G$ is abelian. We have that $|gh| = |[4] + [4]| = |[8]| = |[0]| = 1$, however, $|g| = |h| = |[4]| = 2$. We have $|gh| = 1$ and $\text{lcm}(|g|,|h|) = 2$.

\paragraph{14.} Let $G$ be a group and let $g,h \in G$ such that $gh = hg$ and $\gcd(|g|,|h|) = 1$. By Proposition 1.14, we have that $|gh|$ divides $|g||h|$. We have that $e = (gh)^{|gh||h|} = g^{|gh||h|}h^{|gh||h|} = g^{|gh||h|}$. Hence, $|g|$ divides $|gh||h|$. As $\gcd(|g|,|h|) = 1$, we have that $|g|$ divides $|gh|$. Similarly, $e = (gh)^{|gh||g|} = h^{|gh||g|}$ and so $|h|$ divides $|gh|$. It follows that $|g||h|$ divides $|gh|$. Therefore, $|gh| = |g||h|$. 

\paragraph{15.} Let $G$ be an abelian group and let $g \in G$ be an element of $G$ with maximal finite order. Let $h \in H$ have finite order and suppose, for contradiction, $|h|$ does not divide $|g|$. There is then a prime $p$ such that $|g| = p^{m}r$ and $|h| = p^{n}s$ with $r,s$ coprime to $p$ and $m < n$. We have that the element $g^{p^{m}}$ has order $r$ and the element $h^{s}$ has order $p^{n}$. As $\gcd(r,p) = 1$, we have that $\gcd(|g^{p^{m}}|, |h^{s}|) = 1$. By the previous exercise, $|g^{p^{m}}h^{s}| = |g^{p^{m}}||h^{s}| = rp^{n} > rp^{m} = |g|$, which contradicts the assumption that $g$ has maximal finite order. Therefore, $|h|$ must divide $|g|$. 

\subsection*{2.2 - Examples of Groups}
\paragraph{1.}

\paragraph{2.} Let $d \leq n \in \mathbb{N}$. Let $\sigma_{d} \in S_{n}$ be a permutation such that $\sigma(i) = i+1$ for all $i < d$, $\sigma(i) = i$ for all $i > d$ and $\sigma(d) = 1$. We have that $\sigma_{d} \in S_{n}$ is of order $d$.  

\paragraph{3.} Let $d\in \mathbb{N}$. Let $\sigma_{d} \in S_{\mathbb{N}}$ be a permutation such that $\sigma(i) = i+1$ for all $i < d$, $\sigma(i) = i$ for all $i > d$ and $\sigma(d) = 1$. We have that $\sigma_{d} \in S_{n}$ is of order $d$.

\paragraph{4.}
\paragraph{5.}
\paragraph{6.}
\paragraph{7.}
\paragraph{8.}

\paragraph{9.} Let $n \in \mathbb{Z}$ and consider the relation on $\mathbb{Z}$ defined by $a \equiv b \mod n \iff n \mid (b-a)$. We have that $n \mid 0 = a - a$, so $a \equiv a \mod n$. Furthermore, suppose that $n \mid (b-a)$. Then, $b-a = kn$ for some $k$. Thus, $a-b = -kn$ and so $n \mid (a-b)$. Therefore, $a \equiv b \mod n \iff b \equiv a \mod n$. Finally, suppose that $a \equiv b \mod n$ and $b \equiv c \mod n$. We have that $n \mid (b-a)$ and $n \mid (c-b)$. We have that $b-a = kn$ and $c-b = k'n$ for some $k,k' \in \mathbb{Z}$. We have that $c-a = (b-a) + (c-b) = kn+k'n = (k+k')n$. Therefore, $n \mid c-a$ and $a \equiv c \mod n$. It follows that the relation is an equivalence relation.

\paragraph{10.} 

\paragraph{11.} Let $n = 2k+1 \in \mathbb{Z}$ be an odd integer. We have that $(2k+1)^{2} = 4k^{2} + 4k + 1 = 4(k^{2}+k) + 1$. Let $k = 2m$ be even. Then, $k^{2} + k = 4m^{2} + 2m = 2(2m^{2} + m)$ is divisible by $2$. Let $k = 2m+1$ be odd. Then, $k^{2} + k = (2m+1)^{2} + (2m+1) = 4m^{2} + 6m + 2 = 2(2m^{2} + 3m + 1)$ is divisible by $2$. Hence, $(2k+1)^{2} = 4(k^{2} + k) + 1 = 8z + 1$ for some $z \in \mathbb{Z}$. It follows that $(2k+1)^{2} \equiv 1 \mod 8$ for all $k \in \mathbb{Z}$. 

\paragraph{12.} Let $a,b,c$ be non-zero integers such that $a^{2} + b^{2} = 3c^{2}$. We have that $[a^{2}]_{4} + [b^{2}]_{4} \in \{[0]_{4},[1]_{4},[2]_{4}\} \subseteq \mathbb{Z}/4\mathbb{Z}$ as $[n^{2}]_{4} \in \{[0]_{4},[1]_{4}\}$ for any $n \in \mathbb{Z}$. This forces $[3c^{2}]_{4} = [a^{2}] = [b^{2}] = [0]_{4}$ as $[c^{2}]_{4}$ can only take $[0]_{4}$ or $[1]_{4}$. Hence, $3c^{2}$ must be divisible by $4$. As $\gcd(4,3) = 1$, $4$ divides $c^{2}$ and so $2$ divides $c$. Similarly, $2$ divides $a$ and $b$. We then have that $a/2,b/2,c/2$ are non-zero integers such that $(a/2)^{2} + (b/2)^{2} = 3(c/2)^{2}$. By induction, we have that $(a/2^{n},b/2^{n},c/2^{n})$ are integers solutions for all integers $n > 0$. This contradicts that $a,b,c$ are non-zero integers. 

\paragraph{13.} Let $m,n \in \mathbb{Z}$ such that $\gcd(m,n) = 1$. By Corollary $2.5$, we have that there exists an $a \in \mathbb{Z}$ such that $a[m]_{n} = [1]_{n}$. Thus, $am = 1 \mod n$. By definition, $n \mid (am - 1)$. Therefore, $am - 1 = bn$ for some $b \in \mathbb{Z}$ and so $am - bn = 1$. Conversely, suppose that $am + bn = 1$ for some $a,b \in \mathbb{Z}$. Suppose, for contradiction, that $k = \gcd(m,n) > 1$. Then, $m = km'$ and $n = kn'$ for integer $n' \neq 1,m' \neq 1$. We have that $1 = am + bn = k(am' + bn')$, which is a contradiction as $k > 1$ and $am' + bn' \in \mathbb{Z}$. It follows that $\gcd(m,n) = 1$. 

\paragraph{14.} Suppose $x \equiv x' \mod n$ and $y \equiv y' \mod n$. We have that $n \mid (x - x')$ and $n \mid (y - y')$. Then, $x-x' = nk$ and $y-y' = nl$ for some $k,l \in \mathbb{Z}$. We have that $xy - x'y' = xy - xy' + xy' - x'y' = x(y - y') + y'(x - x') = xnl + y'nk = n(xl + y'k)$. Hence, $n \mid (xy - x'y')$. Therefore, $xy = x'y' \mod n$. 

\paragraph{15.} Let $n > 0$ be an odd integer.
\subparagraph{(i)} Suppose that $\gcd(m,n) = 1$. Let $k = \gcd(2m+n,2n)$. Assume $2 \mid k$. We have that $k \mid (2m+n)$ and so $2 \mid n$, which is a contradiction as $n$ is odd. Thus, $k$ is odd. We have that $k \mid 2n$, which is follows that $k \mid n$. Then, $k\mid (2m+n)$ implies that $k \mid m$ as $k$ is odd. Hence, $k \mid \gcd(m,n) = 1$. It follows that $k = 1$. 

\subparagraph{(ii)} Suppose $\gcd(r,2n) = 1$. Let $k = \gcd(\frac{r+n}{2},n)$. We have that $k \mid n$ and so $k \mid 2n$ and $2k \mid 2n$. Furthermore, $k \mid \frac{r+n}{2}$ and so $2k \mid r+n$. Thus, $2k \mid r+n-2n = r-n$. Then, $2k \mid (r-n)+(r+n) = 2r$. Hence, $k \mid r$. It follows that $k \mid \gcd(r,2n) = 1$. Therefore, $k = 1$. 

\subparagraph{(iii)} Define the map $\varphi:(\mathbb{Z}/n\mathbb{Z})^{*} \to (\mathbb{Z}/2n\mathbb{Z})^{*}$ by $\varphi([m]_{n}) = [2m + n]_{2n}$. Suppose $[x]_{n} = [y]_{n}$. Then, $x = y + kn$ for some $k \in \mathbb{Z}$. Hence, $2x + n = 2y + n + 2kn$ and so $[2x+n]_{2n} = [2y+n]_{2n}$. Thus, $\varphi$ is well-defined. Now, suppose $[2x+n]_{2n} = [2y+n]_{2n}$. Then, $2x+n = 2y+n + 2kn$ for some $k \in \mathbb{Z}$. We then have that $x = y + kn$ and so $[x]_{n} = [y]_{n}$. Therefore, $\varphi$ is injective. Finally, let $[x]_{2n} \in (\mathbb{Z}/2n\mathbb{Z})^{*}$. We have that $\gcd(2n,x) = 1$ and so, by the previous exercise, $\gcd(\frac{n+x}{2},n) = 1$. Hence, $[\frac{n+x}{2}]_{n} \in (\mathbb{Z}/n\mathbb{Z})^{*}$. We have that $\varphi([\frac{n+x}{2}]_{n}) = [(n+x)+n]_{2n} = [x + 2n]_{2n} = [x]_{2n}$. And so $\varphi$ is surjective. It follows that $\varphi$ is a bijection. 

\paragraph{16.} The last digit of $x \in \mathbb{Z}$ corresponds to the least residue of $x$ in $\mathbb{Z}/10\mathbb{Z}$. We have that $1238237 \equiv 7 \mod 10$ and so $1238237^{18238456} \equiv 7^{18238456} \mod 10$. We then have that $7^{2} \equiv - 1 \mod 10$. Hence, $7^{18238456} = 49^{9119228} \equiv (-1)^{9119228} \mod 10 = 1 \mod 10$. Therefore, the last digit of $1238237^{18238456}$ is $1$. 

\paragraph{17.} Suppose $m \equiv m' \mod n$. We set to prove $\gcd(m,n) = 1 \iff \gcd(m',n) = 1$. As $m \equiv m' \mod n \iff m' \equiv m \mod n$, it suffices to only prove one direction. Suppose that $\gcd(m,n) = 1$. Then, $am + bn = 1$ for some $a,b \in \mathbb{Z}$. As $m \equiv m' \mod n$, we have that $m = m' + kn$ for some $k \in \mathbb{Z}$. Then, $a(m'+kn) + bn = 1$. Hence, $am' + (ak + b)n = 1$. Therefore, $\gcd(m',n) = 1$. 

\paragraph{18.}
\paragraph{19.}

\subsection*{2.3 - The Category $\mathsf{Grp}$}
\paragraph{1.} 
\paragraph{2.}

\paragraph{3.} Let $A,B$ be abelian groups in $\mathsf{Ab}$. Let $Z$ be an object in $\mathsf{Ab}$ and $f:A \to Z, g:B \to Z$ be morphisms in $\mathsf{Ab}$. We have that the following diagram commutes for some $\sigma:A \times B \to Z$ where $i_{A}:A \to A \times B, i_{B}:B \to A \times B$ are defined by $i_{A}(x) = (x,0_{B})$ and $i_{B}(x) = (0_{A},x)$. 
\[\begin{tikzcd}[ampersand replacement=\&]
	A \\
	\& {A \times B} \&\& Z \\
	B
	\arrow["{i_{A}}", from=1-1, to=2-2]
	\arrow["{i_{B}}"', from=3-1, to=2-2]
	\arrow["\sigma"{description}, from=2-2, to=2-4]
	\arrow["f", curve={height=-12pt}, from=1-1, to=2-4]
	\arrow["g"', curve={height=12pt}, from=3-1, to=2-4]
\end{tikzcd}\]
We have that $i_{A}$ and $i_{B}$ are homomorphisms as $i_{A}(x+_{A}y) = (x+_{A}y,0_{B}) = (x,0_{B}) +_{A \times B} (y,0_{B}) = i_{A}(x) +_{A\times B} i_{A}(y)$ and $i_{B}(x'+_{B}y') = (0_{A},x'+_{B}y') = (0_{A},x') +_{A \times B} (0_{A},y') = i_{B}(x') + i_{B}(y')$ for all $x,y \in A$ and $x',y' \in B$. As the diagram commutes, we have that $f = \sigma i_{A}$ and $g = \sigma i_{B}$. We have that $\sigma((a,0)) = f(a)$ and $\sigma((0,b)) = g(b)$ for $a\in A$ and $b \in B$. As $\sigma$ is a homomorphism, for $(x,y) \in A \times B$, we have that 
\begin{align*}
\sigma((x,y)) &= \sigma((x,0_{B}) +_{A\times B} (0_{A},y)) \\
&= \sigma((x,0_{B})) +_{Z} \sigma((0_{A},y)) \\
&= f(x) +_{Z} g(y) 
\end{align*}
Let $\sigma:A \times B \to Z$ be defined by $\sigma((a,b)) = f(a) +_{Z} g(b)$. We have that for $(x,y), (x',y') \in A \times B$, 
\begin{align*}
\sigma((x,y) +_{A\times B} (x',y')) &= \sigma((x+_{A} x', y+_{B}y')) \\
&= f(x +_{A} x') +_{Z} g(y +_{B} y') \\
&= (f(x) +_{Z} f(x')) +_{Z} (g(y) +_{Z} g(y')) \\
&= f(x) +_{Z} f(x') +_{Z} g(y) +_{Z} g(y') \\
&= f(x) +_{Z} g(y) +_{Z} f(x') +_{Z} g(y') \\
&= (f(x) +_{Z} g(y)) +_{Z} (f(x') +_{Z} g(y')) \\
&= \sigma((x,y)) +_{Z} \sigma((x',y')) 
\end{align*}
Hence, $\sigma$ is a homomorphism and is unique. Therefore, $A \times B$ with the maps $i_{A}$ and $i_{B}$ is the coproduct of $A$ and $B$ in $\mathsf{Ab}$.

\paragraph{4.}

\paragraph{5.} Suppose that there exists nontrivial groups $G,H$ such that $\mathbb{Q} \cong G \times H$. Let $f:\mathbb{Q} \to G \times H$ be an isomorphism. Let $h \in H$ be a nontrivial element in $H$. As $f$ is an isomorphism, there exists a nontrivial $p/q \in \mathbb{Q}$ such that $f(p/q) = (0,h) \in G \times H$. Let $\pi_{G}:G \times H \to G$ be the projection onto $G$ and let $g = \pi_{G} \circ f$. Then, $g(p/q) = 0$. We have that $qg(p/q) = g(p) = pg(1)$ as $g$ is a homomorphism. The only element in $\mathbb{Q}$ with finite order is $0$, hence, $g(1) = 0$. Let $m/n \in \mathbb{Q}$, it follows that $ng(m/n) = g(m) = mg(1) = 0$, and so $g(m/n) = 0$. $g$ is then the zero map, which means $G \times \{0\} \subseteq \ker f$. This is a contradiction as $f$ is an isomorphism. Therefore, $\mathbb{Q}$ cannot be written as the direct product of two nontrivial groups. 

\paragraph{6.} Suppose $S_{3}$ is the coproduct of $\mathbb{Z}/2\mathbb{Z}$ and $\mathbb{Z}/3\mathbb{Z}$. Define $f:\mathbb{Z}/2\mathbb{Z} \to S_{3}$ by $f([0]_{2}) = \text{id}$ and $f([1]_{2}) = (1 \ 2)$. Define $g:\mathbb{Z}/3\mathbb{Z} \to S_{3}$ by $g([0]_{3}) = \text{id}$,$g([1]_{3}) = (1 \ 2 \ 3)$ and $g([2]_{3}) = (1 \ 3 \ 2)$. We have that the following diagram commutes for some homomorphisms $\sigma, i_{2}, i_{3}$. 
\[\begin{tikzcd}[ampersand replacement=\&]
	{\mathbb{Z}/2\mathbb{Z}} \\
	\& {\mathbb{Z}/2\mathbb{Z} \times \mathbb{Z}/3\mathbb{Z}} \&\& {S_{3}} \\
	{\mathbb{Z}/3\mathbb{Z}}
	\arrow["{i_{A}}", from=1-1, to=2-2]
	\arrow["{i_{B}}"', from=3-1, to=2-2]
	\arrow["\sigma"{description}, from=2-2, to=2-4]
	\arrow["f", curve={height=-12pt}, from=1-1, to=2-4]
	\arrow["g"', curve={height=12pt}, from=3-1, to=2-4]
\end{tikzcd}\]
We have that 
\begin{align*}
(1 \ 3) &= (1 \ 2 \ 3)(1 \ 2) \\ 
&= g([1]_{3})f([1]_{2}) \\ 
&= \sigma(i_{B}([1]_{3}))\sigma(i_{A}([1]_{2})) \\
&= \sigma(i_{B}([1]_{3}) + i_{A}([1]_{2})) \\
&= \sigma(i_{A}([1]_{2}) + i_{B}([1]_{3})) \\
&= \sigma(i_{A}([1]_{2}))\sigma(i_{B}([1]_{3})) \\
&= f([1]_{2})g([1]_{3}) \\ 
&= (1 \ 2)(1 \ 2 \ 3) \\
&= (2 \ 3)
\end{align*}
as $\mathbb{Z}/2\mathbb{Z} \times \mathbb{Z}/3\mathbb{Z}$ is abelian. Hence, such a commutative diagram cannot exist. Therefore, $S_{3}$ is not the coproduct of $\mathbb{Z}/2\mathbb{Z}$ and $\mathbb{Z}/3\mathbb{Z}$. 

\paragraph{7.}

\paragraph{8.} Let $G$ be a group defined by two generators $x,y$ subject only to the relations $x^{2} = 1_{G}$ and $y^{3} = 1_{G}$. Let $Z$ be an object in $\mathsf{Ab}$ and let $f:\mathbb{Z}/2\mathbb{Z} \to Z, g:\mathbb{Z}/3\mathbb{Z} \to Z$ be morphisms. Let $\sigma:G \to Z$ be a homomorphism such that $\sigma(x) = f([1]_{2})$ and $\sigma(y) = g([1]_{3})$. We have that for any $w \in G$, $\sigma(w) = \sigma(\prod_{i=1}^{k}x_{i}^{n_{i}}) = \prod_{i=1}^{k}\sigma(x_{i})^{n_{i}} = \prod_{i=1}^{k}\sigma(x_{i})^{n_{i}}$
where $x_{i} \in \{x,y\}, n_{i} \in \mathbb{Z}, k \in \mathbb{N}$ and $w$ is generated by $x,y$ and $\sigma$ is a homomorphism. We have that each $w$ has a unique output as $\sigma(x_{i})$ is uniquely determined by $x_{i}$. Hence, $\sigma$ is unique. We have that the following diagram commutes
\[\begin{tikzcd}[ampersand replacement=\&]
	{\mathbb{Z}/2\mathbb{Z}} \\
	\& G \&\& Z \\
	{\mathbb{Z}/3\mathbb{Z}}
	\arrow["{i_{3}}", from=3-1, to=2-2]
	\arrow["{i_{2}}"', from=1-1, to=2-2]
	\arrow["f", curve={height=-12pt}, from=1-1, to=2-4]
	\arrow["g"', curve={height=12pt}, from=3-1, to=2-4]
	\arrow["\sigma"{description}, from=2-2, to=2-4]
\end{tikzcd}\]
where $i_{2}([n]_{2}) = x^{n}$ and $i_{3}([n]_{3}) = y^{n}$. We have that $f = \sigma i_{2}$ and $g = \sigma i_{3}$. We have that $\sigma(x) = f([1]_{2})$ and $\sigma(y) = g([1]_{3})$. It follows $\sigma$ in the diagram and $G$ is the coproduct of $\mathbb{Z}/2\mathbb{Z}$ and $\mathbb{Z}/3\mathbb{Z}$. 

\paragraph{9.} \textbf{Not Done} Let $A,B$ be abelian groups and $\alpha:A \to C, \beta:B \to C$ be morphisms in $\mathsf{Ab}$. Let $E = \{(x,y) \in A \times B\mid \alpha(x) = \beta(y)\}$ with a binary operation, $+_{E}$, inherited from $A\times B$. We note that $+_{E}$ is an associative and commutative operation as $+_{A \times B}$ is associative and commutative. We have that $(0_{A},0_{B}) \in E$ as $\alpha(0_{A}) = 0_{C} = \beta(0_{B})$ as $\alpha,\beta$ are homomorphisms. Let $(x,y) \in E$. Then, $\alpha(-x) = -\alpha(x) = -\beta(y) = -\beta(-y)$. Hence, $(-x,-y) \in E$. It follows that $E$ is an abelian group. We note that for each $(x,y)\in E$, $(\alpha \circ \pi_{A})((x,y)) = \alpha(x) = \beta(y) = (\beta \circ \pi_{B})((x,y))$. Let $Z$ be an abelian group and let $f:Z \to A$ and $g:Z \to B$ be morphisms in $\mathsf{Ab}$ such that $\alpha f = \beta g$. We have that the following diagram commutes
\[\begin{tikzcd}[ampersand replacement=\&]
	\&\&\& A \\
	Z \&\& E \&\& C \\
	\&\&\& B
	\arrow["\sigma", from=2-1, to=2-3]
	\arrow["{\pi_{A}}", from=2-3, to=1-4]
	\arrow["\alpha", from=1-4, to=2-5]
	\arrow["{\pi_{B}}", from=2-3, to=3-4]
	\arrow["\beta"', from=3-4, to=2-5]
	\arrow["f", curve={height=-12pt}, from=2-1, to=1-4]
	\arrow["g"', curve={height=12pt}, from=2-1, to=3-4]
\end{tikzcd}\]
As the diagram commutes, $f = \pi_{A}\sigma$ and $g = \pi_{B}\sigma$, hence, $\sigma(x) = (f(x),g(x))$. By assumption, $\alpha f = \beta g$, thus, $\sigma(x) \in E$ for all $x \in Z$. Hence, $\sigma$ is well-defined. It follows that fiber products exist in $\mathsf{Ab}$. 

\subsection*{2.4 - Group Homomorphisms}
\paragraph{1.}
\paragraph{2.}

\paragraph{3.} Suppose that $G$ is a group of order $n$ such that $G$ contains an element $g \in G$ such that $g$ has order $n$. We verify $\varphi:\mathbb{Z}/n\mathbb{Z} \to G$ defined by $\varphi([x]_{n}) = g^{x}$ is an isomorphism. We have that for $[x]_{n},[y]_{n} \in \mathbb{Z}/n\mathbb{Z}$, $\varphi([x]_{n} + [y]_{n}) = \varphi([x+y]_{n}) = g^{x+y} = g^{x}g^{y} = \varphi([x]_{n})\varphi([y]_{n})$. Furthermore, suppose $\varphi([x]_{n}) = \varphi([y]_{n})$. Then, $g^{x} = g^{y}$ and so $g^{x-y} = 1_{G}$. We must have that the order of $g \in G$ divides $x-y$ so $n \mid x-y$. Therefore, $[x]_{n} = [y]_{n}$. Finally, let $h \in G$. Suppose there did not exist an $[x]_{n} \in \mathbb{Z}/n\mathbb{Z}$ such that $\varphi([x]_{n}) = h$. Hence, $g^{x} = h$ does not hold for any $x \in \mathbb{Z}$. This leads to contradiction as $G$ would not contain $n$ elements as $\langle g \rangle = \{1_{G},g,g^{2},...,g^{n-1}\} \subseteq G$. It follows that $\varphi$ must be an isomorphism. For the converse, suppose that $G$ is a group of order $n$ that is isomorphic to $\mathbb{Z}$. We have that there exists an isomorphism $\psi:\mathbb{Z}/n\mathbb{Z} \to G$. We have that $[1]_{n} \in \mathbb{Z}/n\mathbb{Z}$ has order $n$. Then, $\psi([1]_{n}) \in G$ will also be of order $n$ by Proposition 4.8. Hence, $G$ contains an element of order $n$. 

\paragraph{4.}

\paragraph{5.} Suppose, for contradiction, there exists an isomorphism $\varphi:(\mathbb{C} - \{0\}, \cdot) \to (\mathbb{R} - \{0\}, \cdot)$. We have that $i$ has order $4$ in $(\mathbb{C} - \{0\}, \cdot)$ and so $\varphi(i)$ must have order $4$ in $(\mathbb{R} - \{0\}, \cdot)$. There must exist an $x \in (\mathbb{R} - \{0\}, \cdot)$ such that $|x| = 4$. Such an $x$ must be a solution to the equation $x^{4} = 1$, however, the only solutions in $\mathbb{R}$ to the equation are $1$ and $-1$, which have order $1$ and $2$, respectively. As such an $x$ cannot exist, we have a contradiction. Therefore, an isomorphism between $(\mathbb{C} - \{0\}, \cdot)$ and $(\mathbb{R} - \{0\}, \cdot)$ cannot exist. 

\paragraph{6.}

\paragraph{7.} Let $G$ be a group and define $\varphi:G \to G$ by $\varphi(g) = g^{-1}$. Suppose that $\varphi$ is a homomorphism. Then, for each $x,y \in G$, we have that 
$$y^{-1}x^{-1} = (xy)^{-1} = \varphi(xy) = \varphi(x)\varphi(y) = x^{-1}y^{-1}$$
It follows that $xy = yx$. Hence, $G$ is abelian. For the converse, suppose that $G$ is abelian. We have that for each $x,y \in G$,
$$\varphi(xy) = (xy)^{-1} = y^{-1}x^{-1} = x^{-1}y^{-1} = \varphi(x)\varphi(y)$$
Hence, $\varphi$ is a homomorphism. Now, define $\psi:G \to G$ by $\psi(g) = g^{2}$. Suppose $\psi$ is a homomorphism. Then, for $x,y \in G$
$$xyxy = (xy)^{2} = \psi(xy) = \psi(x)\psi(y) = x^{2}y^{2}$$
It follows that $xy = yx$. For the converse, suppose that $G$ is abelian. Then, for $x,y \in G$, 
$$\psi(xy) = (xy)^{2} = xyxy = xxyy = x^{2}y^{2} = \psi(x)\psi(y)$$
Therefore, $\psi$ is a homomorphism. 

\paragraph{8.} Let $G$ be a group and $g \in G$. Define $\gamma_{g}:G \to G$ by $\gamma_{g}(x) = gxg^{-1}$. Suppose $\gamma_{g}(x) = \gamma_{g}(y)$ for some $x,y \in G$. Then, $gxg^{-1} = gyg^{-1}$ and so $x = y$. We also have that for each $x \in G$, $\gamma_{g}(g^{-1}xg) = g(g^{-1}xg)g^{-1} = gg^{-1}xgg^{-1} = 1x1 = x$. It follows $\gamma_{g}$ is bijective. Furthermore, for $x,y \in G$, we have that $\gamma_{g}(xy) = gxyg^{-1} = gxg^{-1}gyg^{-1} = \gamma_{g}(x)\gamma_{g}(y)$. Hence, $\gamma_{g}$ is a homomorphism. Therefore, $\gamma_{g}(y)$ is an automorphism. Define the function $\psi:G \to \text{Aut}_{\mathsf{Grp}}(G)$ by $\psi(g) = \gamma_{g}$. For $g,g' \in G$, we have that $\psi(gg')[x] = \gamma_{gg'}[x] = gg'x(gg')^{-1} = gg'xg'^{-1}g = (\gamma_{g} \circ \gamma_{g'})[x] = (\psi(g) \circ \psi(g'))[x]$. Hence, $\psi$ is a homomorphism. Suppose $\psi$ is trivial. We have that for each $x,g \in G$, $\psi(g)[x] = \text{id}[x]$ and so $gxg^{-1} = x$. It follows that $gx = xg$ and so $G$ is abelian. For the converse, suppose that $G$ is abelian. Then, for each $x,g \in G$, $\psi(g)[x] = \gamma_{g}(x) = gxg^{-1} = xgg^{-1} = x1 = x$. Hence, $\psi$ is trivial. 

\paragraph{9.} Let $m,n \in \mathbb{N}$ such that $\gcd(m,n) = 1$. Let $k$ be the order of $([1]_{m},[1]_{n}) \in \mathbb{Z}/m\mathbb{Z} \times \mathbb{Z}/n\mathbb{Z}$. Then, $([0]_{m},[0]_{n}) = k([1]_{m},[1]_{n}) = ([k]_{m},[k]_{n})$. Hence, $m \mid k$ and $n \mid k$. As $\gcd(m,n) = 1$, we have that $mn \mid k$. As $mn([1]_{m},[1]_{n}) = ([mn]_{m},[mn]_{n}) = ([0]_{m},[0]_{n})$, it follows that $k = mn$. As $\mathbb{Z}/m\mathbb{Z} \times \mathbb{Z}/n\mathbb{Z}$ contains an element of order $mn$ and is also of order $mn$, $\mathbb{Z}/m\mathbb{Z} \times \mathbb{Z}/n\mathbb{Z}$ is isomorphic to $\mathbb{Z}/mn\mathbb{Z}$. 

\paragraph{10.} 

\paragraph{11.}

\paragraph{12.} Suppose $x^{3} - 9 = 0$ has a solution, $c$, in $\mathbb{Z}/31\mathbb{Z}$. We have that $c^{3} = 9$ and so $[c]_{31}^3 = [9]_{31}$. Note the order of $[9]_{31}$ in $(\mathbb{Z}/31\mathbb{Z})^{*}$ is $15$. We have that $[1]_{31} = [c]_{31}^{30} = [9]_{31}^{10}$, which contradicts the fact that the order of $[9]_{31}$ is $15$. Hence, such a $c$ cannot exist. 

\paragraph{13.} Let $V = \{1,a,b,c\}$ be the Klein four-group. There are exactly 6 distinct bijection set-functions from $V$ to $V$ that fix $1$. Let $f:V \to V$ be such a function. We have that $f(a),f(b),f(c)$ are unqiue by assumption. Then, if $x,y \in V$ such that $x \neq y$ and $x \neq 1, y \neq 1$, we have that $f(xy) = f(x)f(y)$. We also have that $f(x1) = f(x) = f(x)1 = f(x)f(1)$. This suffices to show $f$ is a homomorphism as $V$ is abelian. It follows that each $f$ is an isomorphism and so $\text{Aut}_{\mathsf{Grp}}(V)$ has $6$ elements. We have that each $f \in \text{Aut}_{\mathsf{Grp}}(V)$ corresponds to a different permutation on a 3-set and an isomorphism between $\text{Aut}_{\mathsf{Grp}}(V)$ and $S_{3}$ is trivial. 

\paragraph{14.} Let $\varphi:\mathbb{Z}/n\mathbb{Z} \to \mathbb{Z}/n\mathbb{Z}$ be a homomorphism. For each $[x]_{n} \in \mathbb{Z}/n\mathbb{Z}$, $\varphi([x]_{n}) = \varphi(x[1]_{n}) = x\varphi([1]_{n})$. Hence, $\varphi$ is uniquely determined by where $[1]_{n}$ is mapped. Suppose $\varphi([1]_{n}) = [m]_{n}$ where $\gcd(m,n) = 1$. Let $[x]_{n}, [y]_{n} \in \mathbb{Z}/n\mathbb{Z}$ such that $\varphi([x]_{n}) = \varphi([y]_{n})$. Then, $\varphi([x-y]_{n}) = [0]_{n}$ and so $(x-y)[m]_{n} = [0]_{n}$. We have that $n \mid m(x-y)$. As $\gcd(m,n) = 1$, $n \mid x-y$ and so $[x]_{n} = [y]_{n}$. It follows that $\varphi$ is bijective, hence, an automorphism. Now suppose that $\varphi([1]_{n}) = [m]_{n}$ where $\gcd(m,n) > 1$. Assume, for contradiction, that $\varphi$ is an isomorphism, then, $n = |[1]_{n}| = |\varphi([1]_{n})| = |[m]_{n}| = \frac{n}{\gcd(m,n)} < n$. Therefore, $\varphi$ cannot be an isomorphism. It follows that $|\text{Aut}_{\mathsf{Grp}}(\mathbb{Z}/n\mathbb{Z})| = \phi(n)$. 

\paragraph{15.} 

\paragraph{16.}
\paragraph{17.}

\paragraph{18.} Let $\varphi:G \to H$ be an isomorphism in of groups $G$ and $H$. Suppose $G$ is abelian. Let $h,h' \in H$. As $\varphi$ is an isomorphism, there are elements $g,g' \in G$ such that $\varphi(g) = h$ and $\varphi(g) = h'$. We have that 
$$hh' = \varphi(g)\varphi(g') = \varphi(gg') = \varphi(g'g) = \varphi(g')\varphi(g) = h'h$$
Therefore, $H$ is abelian. For the converse, suppose $H$ is abelian. We have that $\varphi^{-1}:H \to G$ exists and is an isomorphism. With the same argument, we can deduce $G$ is abelian. 

\subsection*{2.5 - Free Groups}

\paragraph{1.} Let $A$ be a set and let $\mathscr{F}^{A}$ be a category where $\mathsf{obj}(\mathscr{F}^{A})$ are pairs $(j,G)$, where $G$ is a group and $j:A \to G$ is a set function from $A$ to $G$ and morphisms $(j_{1}, G_{1}) \to (j_{2},G_{2})$ are commutative diagrams 
\[\begin{tikzcd}[ampersand replacement=\&]
	{G_{1}} \& {G_{2}} \\
	A
	\arrow["\varphi", from=1-1, to=1-2]
	\arrow["{j_{1}}", from=2-1, to=1-1]
	\arrow["{j_{2}}"', from=2-1, to=1-2]
\end{tikzcd}\]
where $\varphi$ is a group homomophism. Let $E$ be the trivial group and $i$ the identity map. We have that a morphism $(j,G) \to (i,E)$ is the commutative diagram
\[\begin{tikzcd}[ampersand replacement=\&]
	G \& E \\
	A
	\arrow["\varphi", from=1-1, to=1-2]
	\arrow["j", from=2-1, to=1-1]
	\arrow["i"', from=2-1, to=1-2]
\end{tikzcd}\]
We have that $\varphi$ must be the identity homomorphism as it is a homomorphism to the trivial group. Therefore, $\mathsf{Hom}_{\mathscr{F}^{A}}((j,G), (i,E))$ is singleton. We have that $\mathscr{F}^{A}$ has final objects. 

\paragraph{2.}

\paragraph{3.} Let $A$ be a set and $F(A)$ be the free group associated with $A$. We have that there is a unique homomorphism such that the following diagram commutes
\[\begin{tikzcd}
	{F(A)} & {F(A)} \\
	A
	\arrow["i"', from=2-1, to=1-2]
	\arrow["j", from=2-1, to=1-1]
	\arrow["\varphi", from=1-1, to=1-2]
\end{tikzcd}\]
where $i$ is the inclusion map. Let $x,y \in A$ such that $j(x) = j(y)$. We note that $\varphi \circ j = i$, hence, 
$$x = i(x) = (\varphi \circ j)(x) = \varphi(j(x)) = \varphi(j(y)) = (\varphi \circ j)(y) = i(y) = y$$
Therefore, $j$ is injective

\paragraph{4.}

\paragraph{5.} Let $H$ be an abelian group and $A$ be a set. Define 
$$H^{\oplus A} = \{\alpha:A \to H \mid \alpha(x) \neq 0_{H} \text{ for finitely many elements } x \in A\}$$
with the binary operation, $+$, inherited from $H^{A}$. We first note $+$ is an associative operation on $H^{\oplus A}$ as $H^{A}$ is a group. Let $\alpha \in H^{\oplus A}$. Then, $\alpha(x) \neq 0_{H}$ for finitely many elements $x \in A$. It then follows that $-\alpha(x) \neq 0_{H}$ for finitely many $x \in A$. Let $i:A \to H$ be a map defined by $i(x) = 0_{H}$ for all $x \in A$. We have that $i(x) \neq 0_{H}$ for no $x \in A$. Therefore, $i \in H^{\oplus A}$. Suppose that $A$ is a finite set. Then, if $\alpha:A \to H$ is a set function in $H^{A}$, we must have that $\alpha \in H^{\oplus A}$. And so $H^{\oplus A} = H^{A}$. Now, suppose that $A$ is an infinite set. Let $\alpha, \beta:A \to H$ be elements of $H^{\oplus A}$. Suppose, for contradiction, that $\alpha(x)  +\beta(x) \neq 0_{H}$ for infinitely many $x \in A$. We then have that $\alpha(x) \neq \beta(x)$ for infinitely many $x \in A$. However, $\alpha(x) \neq 0_{H}$ for finitely many $x \in A$ and $\beta(x) \neq 0_{H}$ for finitely many $x \in A$. Necessarily, $\alpha(x) = \beta(x)$ for infinitely many $x \in A$. Hence, $\alpha(x) + \beta(x) \neq 0_{H}$ for finitely many $x \in A$. Therefore, $\alpha + \beta \in H^{\oplus A}$. It follows that $H^{\oplus A}$ is a group.

\paragraph{6.}

\paragraph{7.}

\paragraph{8.} \textbf{NOT  DONE} Let $A, B$ be sets and let $A \amalg B$ be their coproduct. Denote the free group associated with a set $S$ by $F(S)$. Using the universal property for free groups, there is a $j_{A}, j_{B}, j_{A \amalg B}$ such that for any pair of groups $G_{1}, G_{2}, G_{3}$ and set functions $f:A \to G_{1}, g:B \to G_{2}, h:A \amalg B \to G_{3}$, there exists unique $\varphi_{1}, \varphi_{2}, \varphi_{3}$ such that the following diagrams commute
\[\begin{tikzcd}[ampersand replacement=\&]
	{F(A)} \&\& {G_{1}} \&\& {F(B)} \&\& {G_{2}} \&\& {F(A \amalg B)} \&\& {G_{3}} \\
	\\
	A \&\&\&\& B \&\&\&\& {A \amalg B}
	\arrow["f"{description}, from=3-1, to=1-3]
	\arrow["{j_{A}}"{description}, from=3-1, to=1-1]
	\arrow["{\varphi_{1}}"{description}, from=1-1, to=1-3]
	\arrow["{j_{B}}"{description}, from=3-5, to=1-5]
	\arrow["g"{description}, from=3-5, to=1-7]
	\arrow["{\varphi_{2}}"{description}, from=1-5, to=1-7]
	\arrow["h"{description}, from=3-9, to=1-11]
	\arrow["{j_{A \amalg B}}"{description}, from=3-9, to=1-9]
	\arrow["{\varphi_{3}}"{description}, from=1-9, to=1-11]
\end{tikzcd}\]
Using the universal property for coproducts, there are $i_{A}:A \to A \amalg B, i_{B}:B \to A \amalg B$ such that for any set $Z$ and pair of set functions $f:A \to Z, g:B \to Z$, there is a unique $\varphi$ such that the following diagram commutes
\[\begin{tikzcd}[ampersand replacement=\&]
	A \\
	\& {A \amalg B} \&\& Z \\
	B
	\arrow["{i_{A}}", from=1-1, to=2-2]
	\arrow["{i_{B}}"', from=3-1, to=2-2]
	\arrow["\varphi"{description}, from=2-2, to=2-4]
	\arrow["f"{description}, curve={height=-12pt}, from=1-1, to=2-4]
	\arrow["g"{description}, curve={height=12pt}, from=3-1, to=2-4]
\end{tikzcd}\] 
There then exists unique homomorphisms $i_{F(A)}, i_{F(B)}$ such that the following diagrams commute
\[\begin{tikzcd}[ampersand replacement=\&]
	{F(A)} \&\& {F(A \amalg B)} \&\& {F(B)} \&\& {F(A \amalg B)} \\
	\& {A \amalg B} \&\&\&\& {A \amalg B} \\
	A \&\&\&\& B
	\arrow["{j_{A}}", from=3-1, to=1-1]
	\arrow["{i_{F(A)}}", from=1-1, to=1-3]
	\arrow["{j_{B}}", from=3-5, to=1-5]
	\arrow["{i_{F(B)}}", from=1-5, to=1-7]
	\arrow["{i_{B}}"', from=3-5, to=2-6]
	\arrow["{j_{A \amalg B}}"', from=2-6, to=1-7]
	\arrow["{i_{A}}"', from=3-1, to=2-2]
	\arrow["{j_{A \amalg B}}"', from=2-2, to=1-3]
\end{tikzcd}\]
We then have that the following diagram commutes 
\[\begin{tikzcd}[ampersand replacement=\&]
	A \&\& {F(A)} \\
	\& {A \amalg B} \&\& {F(A \amalg B)} \\
	B \&\& {F(B)}
	\arrow["{i_{A}}"', from=1-1, to=2-2]
	\arrow["{i_{B}}", from=3-1, to=2-2]
	\arrow["{j_{A}}"', from=1-1, to=1-3]
	\arrow["{i_{F(A)}}", from=1-3, to=2-4]
	\arrow["{j_{B}}", from=3-1, to=3-3]
	\arrow["{i_{F(B)}}"', from=3-3, to=2-4]
	\arrow["j_{A \amalg B}"{description}, from=2-2, to=2-4]
\end{tikzcd}\]







Let $G$ be a group and $f:F(A) \to G, g:F(B) \to G$ be homomorphisms. 
\[\begin{tikzcd}[ampersand replacement=\&]
	A \&\& {F(A)} \\
	\& {A \amalg B} \&\& {F(A \amalg B)} \&\& G \\
	B \&\& {F(B)}
	\arrow["{i_{A}}"', from=1-1, to=2-2]
	\arrow["{i_{B}}", from=3-1, to=2-2]
	\arrow["{j_{A}}"', from=1-1, to=1-3]
	\arrow["{i_{F(A)}}", from=1-3, to=2-4]
	\arrow["{j_{B}}", from=3-1, to=3-3]
	\arrow["{i_{F(B)}}"', from=3-3, to=2-4]
	\arrow["{j_{A \amalg B}}"{description}, from=2-2, to=2-4]
	\arrow["g", curve={height=12pt}, from=3-3, to=2-6]
	\arrow["f"', curve={height=-12pt}, from=1-3, to=2-6]
	\arrow["\psi"{description}, from=2-4, to=2-6]
\end{tikzcd}\]












\paragraph{9.}

\paragraph{10.} Let $A$ be a set and let $F = F^{\text{ab}}(A)$. Define an equivalence relation on $F$ by setting $f \sim f'$ if and only if $f - f' = 2g$ for some $g \in F$. By Proposition 5.6, we have that $F \cong \mathbb{Z}^{\oplus A}$. Let $\textbf{x}, \textbf{y} \in F$. We have that $\textbf{x} = \sum_{a \in A}m_{a}j_{a}, \textbf{y} = \sum_{a\in A}n_{a}j_{a}$ where $m_{a},n_{a} \in \mathbb{Z}$ and is non-zero for finite $a \in A$ and $j_{a}(x) = 1$ where $x = a$ and $j_{a}(x) = 0$ otherwise. Then, 
\begin{align*}
\textbf{x} \sim \textbf{y} &\iff \exists g \in F, \ \textbf{x} - \textbf{y} = 2g \\
&\iff \forall a \in A, \exists k_{a} \in \mathbb{Z}, \ \sum_{a \in A}m_{a}j_{a} - \sum_{a \in A}n_{a}j_{a} = 2\sum_{a \in A}k_{a}j_{a} \\
&\iff \forall a \in A, \exists k_{a} \in \mathbb{Z}, \ \sum_{a\in A}(m_{a} - n_{a})j_{a} = 2\sum_{a \in A}k_{a}j_{a} \\
&\iff \forall a \in A, \ 2 \mid (m_{a} - n_{a}) \\
\end{align*}
Let $\mathfrak{C} = \{[\sum_{a\in A}\delta_{a}j_{a}]_{\sim} \mid \delta_{a} \in \{0,1\} \text{ and } \delta_{a} \neq 0 \text{ for finitely many } a \in A\}$. Let $\textbf{x} \in F$. Then, $\textbf{x} = \sum_{a\in A}m_{a}j_{a}$. Let $\textbf{x}' = \sum_{a \in A}m_{a}'j_{a}$ where $m_{a}'$ is the least residue of $m_{a}$ modulo $2$. We have $\textbf{x} \sim \textbf{x}'$ and $\textbf{x} \in [\textbf{x}']_{\sim} \in \mathfrak{C}$. Let $[\textbf{x}]_{\sim}, [\textbf{y}]_{\sim} \in \mathfrak{C}$ such that there exists $\textbf{z} \in [\textbf{x}]_{\sim} \cap [\textbf{y}]_{\sim}$. Let $\textbf{x} = \sum_{a\in A}x_{a}j_{a}, \textbf{y} = \sum_{a\in A}y_{a}j_{a}$ and $\textbf{z} = \sum_{a\in A}z_{a}j_{a}$. Then, $2 \mid (x_{a} - z_{a})$ and $2 \mid (z_{a} - y_{a})$ and so $2 \mid (x_{a} - y_{a})$. Therefore, $[\textbf{x}]_{\sim} = [\textbf{y}]_{\sim}$. It follows that $\mathfrak{C}$ is a disjoint partition of $F$ and so $\mathfrak{C} = F/\sim$. It is also clear that $|F/\sim| = 2^{|A|}$ and so $F/\sim$ is finite if and only if $A$ is finite. Now suppose that $F^{\text{ab}}(A) \cong F^{\text{ab}}(B)$ and that $A$ is finite. We then have that $F^{\text{ab}}(A)/\sim$ is finite with $2^{|A|}$ elements. We then have that $F^{\text{ab}}(B)/\sim$ is finite with $2^{|A|}$ elements and so $F^{\text{ab}}(B)$ is finite. We then have that $2^{|A|} = 2^{|B|}$, hence, $|A| = |B|$. 

\subsection*{2.6 - Subgroups}
\paragraph{1.} 

\paragraph{2.}

\paragraph{3.} 

\paragraph{4.} Let $G$ be a group. Let $\epsilon_{g}:\mathbb{Z} \to G$ be the exponential map. Suppose $g$ has order $n$. Define $\varphi:\epsilon_{g}(\mathbb{Z}) \to \mathbb{Z}/n\mathbb{Z}$ by $\varphi(g^{k}) = [k]_{n}$. Suppose $g^{k} = g^{k'}$. Then, $n \mid k - k'$. We then have that $\varphi(g^{k}) = [k]_{n} = [k']_{n} = \varphi(g^{k'})$. We also have that $\varphi(g^{x}g^{y}) = \varphi(g^{x+y}) = [x+y]_{n} = [x]_{n} + [y]_{n} = \varphi(g^{x}) + \varphi(g^{y})$ and for any $[k]_{n} \in \mathbb{Z}/n\mathbb{Z}$, $\varphi(g^{k}) = [k]_{n}$. It follows that $\varphi$ is an isomorphism. Hence, $\epsilon_{g}(\mathbb{Z}) \cong \mathbb{Z}/n\mathbb{Z}$. Now suppose $g$ has infinite order. Define $\varphi:\epsilon_{g}(\mathbb{Z}) \to \mathbb{Z}$ by $\varphi(g^{k}) = k$. We have that $\varphi$ is an isomorphism and $\epsilon_{g}(\mathbb{Z}) \cong \mathbb{Z}$. 

\paragraph{5.} Let $\varphi:G \to G'$ be a homomorphism. Let $H$ be a subgroup of $G$. Let $x,y \in \varphi(H)$. We have that there are $x',y'$ such that $\varphi(x') = x$ and $\varphi(y') = y$ where $x', y' \in H$. We have that $x'y'^{-1} \in H$ and then $xy^{-1} = \varphi(x')\varphi(y')^{-1} = \varphi(x')\varphi(y'^{-1}) = \varphi(x'y'^{-1}) \in \varphi(H)$. Therefore, $\varphi(H)$ is a subgroup of $G'$. Let $n > 0$ be an integer. Define $\psi:G \to G$ by $\psi(x) = x^{n}$. We have that $\text{im} \ G = \{g^{n} \mid g \in G\}$ and so $\{g^{n} \mid g \in G\}$ is a subgroup of $G$. 

\paragraph{6.} Let $H, H'$ be subgroups of $G$. Suppose that $H \subseteq H'$ or $H' \subseteq H$, then $H \cup H' = H$ or $H \cup H' = H'$, which are both subgroups of $G$ in either case. For the converse, suppose that $H \cup H'$ is a subgroup of $G$. Assume, for contradiction, that $H$ is not contained fully in $H'$ and $H'$ is not contained fully in $H$. Then, there exists $x \in H - H'$ and $y \in H' - H$. We have that $xy \in H \cup H'$ and so $xy \in H'$ or $xy \in H$. We have that $x^{-1} \in H$ as $x \in H$ and we also have that $y^{-1} \in H'$ as $y \in H'$. If $xy \in H$, then $y = x^{-1}xy \in H$ and if $xy \in H'$, then $x = xyy^{-1} \in H'$. In both cases, we lead to contradiction. It follows that $H \subseteq H'$ or $H' \subseteq H$. Now, let $H_{1} \subseteq H_{2} \subseteq ...$ be subgroups of a group $G$. Let $x,y \in H = \bigcup_{i \in \mathbb{N}}H_{i}$. Then, $x \in H_{n}$ and $y \in H_{m}$ for some $n,m \in \mathbb{N}$. Without loss of generality, assume $n \geq m$. Then, $x,y \in H_{n}$ as $H_{m} \subseteq H_{n}$. As $H_{n}$ is a subgroup of $G$, we have that $xy^{-1} \in H_{n}$. Therefore, $xy^{-1} \in H$. We must have that $H$ is a subgroup of $G$. 

\paragraph{7.} Let $\gamma_{g}, \gamma_{h} \in \text{Inn}(G)$. We have that $(\gamma_{g} \circ \gamma_{h})(x) = \gamma_{g}(\gamma_{h}(x)) = \gamma_{g}(hxh^{-1}) = ghxh^{-1}g^{-1} = ghx(gh)^{-1} = \gamma_{gh}(x) \in \text{Inn}(G)$. We have that the inverse of $\gamma_{g}$ is $\gamma_{g^{-1}}$ as $(\gamma_{g} \circ \gamma_{g^{-1}})(x) = gg^{-1}xgg^{-1} = x = \text{id}$ and $\gamma_{g^{-1}} \in \text{Inn}(G)$. Then, $\text{Inn}(G)$ is a subgroup of $\text{Aut}(G)$. Suppose that $\text{Inn}(G)$ is cyclic. There then exists an $a \in G$ such that for any $g \in G$ such that $\gamma_{g} = \gamma_{a}^{n}$ for some $n \in \mathbb{N}$. Then, for any $x \in G$, $gxg^{-1} = a^{n}xa^{-n}$. We then have that $gag^{-1} = a$. We then have that $\gamma_{g}(x) = \gamma_{a}^{n}(x) = a^{n}xa^{-n} = x = \text{id}$. Therefore, any inner automorphism of $G$ is trivial and so $\text{Inn}(G)$ is trivial. As $\text{Inn}(G)$ is trivial, we have that $\gamma_{g} = \text{id}$ for all $g \in G$. Let $x,y \in G$. Then, $\gamma_{x}(y) = y$ and so $xyx^{-1} = y$. Hence, $xy = yx$ and so $G$ is abelian. Suppose $G$ is abelian. Then, for any $g \in G$, we have that $\gamma_{g}(x) = gxg^{-1} = xgg^{-1} = x = \text{id}$. Then, $\text{Inn}(G)$ is trivial, and thus cyclic. It follows that $\text{Inn}(G)$ is cyclic if and only if $\text{Inn}(G)$ is trivial if and only if $G$ is abelian. Finally, assume $\text{Aut}(G)$ is cyclic. Then, as $\text{Inn}(G)$ is a subgroup of $\text{Aut}(G)$, $\text{Inn}(G)$ is cyclic. Hence, $G$ is abelian.

\paragraph{8.} Let $G$ be an abelian group. Suppose that $G$ is finitely generated. Then, $G = \langle A \rangle$ where $A$ is a finite set, $A = \{a_{1}, ..., a_{n}\}$ say. Define $\varphi:\mathbb{Z}^{\oplus n} \to G$ by $\varphi(\textbf{x}) = a_{1}^{x_{1}}...a_{n}^{x_{n}}$. We have that $\varphi$ is surjective as $G$ is generated by $A$. Let $\textbf{x}, \textbf{y} \in \mathbb{Z}^{\oplus n}$. We have that $\varphi(\textbf{x} + \textbf{y}) = a_{1}^{x_{1}+y_{1}}...a_{n}^{x_{n}+y_{n}} = a_{1}^{x_{1}}a_{1}^{y_{1}}...a_{n}^{x_{n}}a_{n}^{y_{n}} = a_{1}^{x_{1}}...a_{n}^{x_{n}}a_{1}^{y_{1}}...a_{n}^{y_{n}} = \varphi(\textbf{x})\varphi(\textbf{y})$. Hence, $\varphi$ is a homomorphism. Therefore, there exists a surjective homomorphism $\varphi:\mathbb{Z}^{\oplus n} \to G$. For the converse, suppose there exists a surjective homomorphism $\psi: \mathbb{Z}^{\oplus n} \to G$. Let $g \in G$. Then, $g = \psi(\textbf{x})$ for some $\textbf{x} \in \mathbb{Z}^{\oplus n}$. As $\psi$ is a homomorphism, we have that $g = \sum_{i=1}^{n}x_{i}(\psi \circ j_{i})$ where $j_{i} = (0,...,1,...,0)$ where $1$ is placed in the $i$th place. We claim $G$ is generated by the set $A = \{\varphi \circ j_{i} \mid i \in [n]\}$. Trivially, we have $\langle A \rangle$ is a subgroup of $G$. Let $g \in G$. By before, $g = \sum_{i=1}^{n}x_{i}(\psi \circ j_{i})$, and so $G$ is contained in $\langle A \rangle$. It follows that $G$ is finitely generated. 

\paragraph{9.} Let $\langle A \rangle$ be a finitely generated subgroup of the additive group $\mathbb{Q}$. There exists a surjective homomorphism $\varphi:\mathbb{Z}^{\oplus n} \to \langle A \rangle$ for some $n \in \mathbb{N}$ as $\mathbb{Q}$ is abelian, so $\langle A \rangle$ is abelian. For each $a \in \langle A \rangle$, there is an $\textbf{x} \in \mathbb{Z}^{\oplus n}$ such that $\varphi(\textbf{x}) = a$. Then, $a = \varphi(\textbf{x}) = \sum_{i=1}^{n}x_{i}\varphi(j_{i}) = \sum_{i=1}^{n}x_{i}\frac{a_{i}}{b_{i}}$ where $a_{i} \in \mathbb{Z}$ and $b_{i} \in \mathbb{N}$. Then, $a = \frac{k}{b_{1}...b_{n}}$ for some $k \in \mathbb{Z}$. We have that $a \in \langle g \rangle$ where $g = \frac{1}{b_{1}...b_{n}}$. Thus, $\langle A \rangle$ is cyclic as a subgroup of a cyclic group. It follows $\mathbb{Q}$ is not finitely generated as $\mathbb{Q}$ is not cyclic. 

\paragraph{10.}
\paragraph{11.}

\paragraph{12.} Let $m,n \in \mathbb{Z}$ and $d = \gcd(m,n)$. Let $kd \in d\mathbb{Z}$. We have that $m = m'd$ and $n = n'd$ and $\gcd(m,n) = \gcd(m'd,n'd) = d\gcd(m',n')$, hence, $\gcd(m',n') = 1$. There then exists $x,y \in \mathbb{Z}$ such that $xm' + yn' = 1$ and so $xm + yn = d$. Then, $kd = kxm + kyn \in \langle m, n\rangle $. Let $x \in \langle m,n \rangle$. We have that $x = pm + qn = d(pm' + qn') \in d\mathbb{Z}$ for some $p,q \in \mathbb{Z}$. Therefore, $d\mathbb{Z} = \langle m,n\rangle$.  

\paragraph{13.}
\paragraph{14.}

\paragraph{15.} Let $\varphi:G \to G'$ be a group homomorphism such that there exists a group homomorphism $\psi:G' \to G$ with $\psi \circ \varphi = \text{id}_{G}$. Let $x \in \ker\varphi$. We have that $\varphi(x) = 1_{G'}$. Then, $1_{G} = \psi(1_{G'}) = \psi(\varphi(x)) = (\psi \circ \varphi)(x) = \text{id}_{G}(x) = x$. It follows that $\ker\varphi$ is trivial. By Proposition 6.12, $\varphi$ is a monomorphism. 

\paragraph{16.}

\subsection*{2.7 - Quotient Groups}
\paragraph{1.} 

\paragraph{2.} Define a homomorphism $\varphi:\mathbb{Z}/2\mathbb{Z} \to S_{3}$ by $\varphi([0]_{2}) = \text{id}$ and $\varphi([1]_{2}) = (1 \ 2)$. We have that $\varphi(\mathbb{Z}/2\mathbb{Z})$ is not normal in $S_{3}$ as $(1 \ 3)(1 \ 2)(1 \ 3)^{-1} = (2 \ 3) \notin \varphi(\mathbb{Z}/2\mathbb{Z})$. Hence, the image of a homomorphism is not necessarily normal. 

\paragraph{3.} Let $G$ be a group and $N$ a subgroup such that $gng^{-1} \in N$ for all $g \in G$ and $n \in N$. Let $g \in G$ and $x \in gNg^{-1}$. Then, $x = gng^{-1}$ for some $n \in N$. By assumption, $x = gng^{-1} \in N$. It follows that $gNg^{-1} \subseteq N$. Let $x \in N$. For all $g \in G$, we have that $x = gg^{-1}xgg^{-1} \in gNg^{-1}$. Then, $N \subseteq gNg^{-1}$ for any $g \in G$. It follows that $gNg^{-1} = N$. Now let $x \in gN$. We have that $x = gn$ for some $n \in N$. As $N = g^{-1}Ng$, we have that $n = g^{-1}n'g$ for some $n' \in N$. Then, $x = gn = gg^{-1}n'g = n'g \in Ng$. Let $y \in Ng$. Then, $y = ng$ for some $n \in N$. As $N = gNg^{-1}$, we have that $n = gn'g^{-1}$ for some $n' \in N$. Hence, $y = ng = gn'g^{-1}g = gn' \in gN$. It follows that $gN = Ng$. Let $n \in N$ and $g \in G$. As $gN = Ng$, we have that $gn = n'g$ for some $n' \in N$. Then, $gng^{-1} = n'gg^{-1} = n' \in N$. It follows that every definition of normal subgroups are equivalent. 

\paragraph{4.} Let $F = F^{\text{ab}}(A)$ where $A$ is a set. Define a relation $\sim$ on $F$ by $f \sim f'$ if and only if $f - f' = 2g$ for some $g \in F$. Let $x \in F$ and suppose that $f \sim f'$. There then exists a $g \in F$ such that $f - f' = 2g$. We then have that $2g = f - f' = f + 0_{F} - f' = (f - x) + (x - f')$. Hence, $f+x \sim f'+x$. We have that $\sim$ is compatible with the group structure. By previous exercises, its clear that $F/\sim \ \cong (\mathbb{Z}/2\mathbb{Z})^{\oplus A}$. 

\paragraph{5.}

\paragraph{6.} Let $G$ be an abelian group. Let $n \in \mathbb{N}$ and define a relation $\sim$ on $G$ by $a \sim b$ if and only if $ab^{-1} = g^{n}$ for some $g \in G$. For any $a \in G$, we have that $a \sim a$ as $aa^{-1} = 1_{G} = 1_{G}^{n}$. Suppose now $a \sim b$. Then, $ab^{-1} = g^{n}$ for some $g \in G$. We have that $ba^{-1} = (ab^{-1})^{-1} = (g^{n})^{-1} = (g^{-1})^{n}$. As $g^{-1} \in G$, we have that $b \sim a$. Suppose that $a \sim b$ and $b \sim c$. Then, $ab^{-1} = g^{n}$ and $bc^{-1} = h^{n}$ for some $g,h \in G$. Then, $ac^{-1} = ab^{-1}bc^{-1} = g^{n}h^{n} = (gh)^{n}$ as $G$ is abelian. Thus, $a \sim c$. Assume $a \sim b$ and let $x \in G$. Then, $ab^{-1} = g^{n}$ for some $g \in G$. We have that $g^{n} = ab^{-1} = axx^{-1}b^{-1} = (ax)(bx)^{-1}$. Therefore, $ax \sim bx$. By Proposition $7.4$, the equivalence class of $1_{G}$, $[1_{G}]_{\sim}$ is a subgroup of $G$. We claim $[1_{G}]_{\sim} = A = \{g^{n} \mid g \in G\}$. Let $x \in [1_{G}]_{\sim}$. Then, $x = g^{n}$ for some $g \in G$. Then, $x \in A$. Let $y \in A$. Then, $y = g^{n}$ for some $g \in G$ and then $y \sim 1_{G}$. We have that $y \in [1_{G}]_{\sim}$. Therefore, $[1_{G}]_{\sim} = A$. 

\paragraph{7.} Let $G$ be a group. Let $n \in \mathbb{N}$ and let $A = \{g \in G \mid |g| = n\}$. Let $H = \langle A \rangle$. Let $\gamma \in \text{Inn}(G)$. As $\gamma$ is an automorphism, we have that $\gamma$ preserves order. Let $x \in H$. Then, $x = x_{1}...x_{k}$ where $x_{i} \in A$ for each $i \in [k]$. Then, $\gamma(x) = \gamma(x_{1}...x_{k}) = \gamma(x_{1})...\gamma(x_{k}) \in H$ as $|\gamma(x_{i})| = |x_{i}| = n$ for each $i \in [k]$. It follows that $H$ is normal.

\paragraph{8.} Let $H$ be a subgroup of $G$ and define a relation $\sim_{L}$ on $G$ by $a \sim_{L} b$ if and only if $a^{-1}b \in H$. For each $a \in G$, we have that $a \sim_{L} a$ as $1_{G} = a^{-1}a \in H$. Suppose $a \sim_{L} b$. Then, $a^{-1}b \in H$. We then have that $b^{-1}a = (a^{-1}b)^{-1} \in H$. Then, $b \sim_{L} a$. Finally, suppose that $a \sim_{L} b$ and $b \sim_{L} c$. Then, $a^{-1}b, b^{-1}c \in H$. We then have that $a^{-1}c = a^{-1}bb^{-1}c \in H$. Therefore, $\sim_{L}$ is an equivalence relation. Suppose that $a \sim_{L} b$ and let $g \in G$. Then, $a^{-1}b \in H$. We then have that $(ga)^{-1}(gb) = a^{-1}g^{-1}gb = a^{-1}b \in H$. Therefore, $ga \sim_{L} gb$.

\paragraph{9.}

\paragraph{10.} Let $G$ be a group and $H \subseteq G$ be a subgroup. Suppose $H$ is normal in $G$. Let $\gamma \in \text{Inn}(G)$. Let $x \in \gamma(H)$. Then, $x = ghg^{-1}$ for some $g \in G$ and $h \in H$. As $H$ is normal, $ghg^{-1} \in H$. Therefore, $\gamma(H) \subseteq H$. For the converse, suppose that $\gamma(H) \subseteq H$ for all $\gamma \in \text{Inn}(G)$. Let $h \in H$ and $g \in G$. Then, $ghg^{-1} = \gamma_{g}(h) \in H$. Therefore, $H$ is normal. 

\paragraph{11.} Let $G$ be a group and $[G,G]$ be its commutator. Let $\gamma \in \text{Inn}(G)$. Let $x \in [G,G]$. Then, $x = [x_{1},y_{1}][x_{2},y_{2}]...[x_{n},y_{n}]$ where $x_{i}, y_{i} \in G$ for all $i \in [n]$ and $[g,h] = ghg^{-1}h^{-1}$. We have that $\gamma([g,h]) = \gamma(ghg^{-1}h^{-1}) = \gamma(g)\gamma(h)\gamma(g)^{-1}\gamma(h)^{-1} = [\gamma(g),\gamma(h)]$. As $\gamma$ is a homomorphism, we have that $\gamma(x) = \gamma([x_{1},y_{1}]...[x_{n},y_{n}]) = \gamma([x_{1},y_{1}])...\gamma([x_{n},y_{n}]) = [\gamma(x_{1}),\gamma(y_{1})]...[\gamma(x_{n}),\gamma(y_{n})] \in [G,G]$. We have that $\gamma([G,G]) \subseteq [G,G]$. Therefore, $[G,G]$ is normal. Now, let $x[G,G], y[G,G] \in G/[G,G]$. Then, 
$$x[G,G]y[G,G] = xy[G,G] = xy(y^{-1}x^{-1}yx)[G,G] = yx[G,G] = y[G,G]x[G,G]$$
Therefore, $G/[G,G]$ is abelian

\paragraph{12.} Let $F = F(A)$ be the free group associated with a set $A$ and let $f:A \to G$ be a set function from the set $A$ to an abelian group $G$. By the universal property of free groups, there is a unique homomorphism $\psi:F \to G$ such that $\psi \circ j_{A} = f$ where $j_{A}$ is the canonical map from $A$ set to its free group. Let $[F,F]$ be the commutator of $F$ and let $x = [a_{1},b_{1}]...[a_{n},b_{n}] \in [F,F]$ where $a_{i},b_{i} \in F$ for all $i \in [n]$. Then, 
$$\psi(x) = \psi([a_{1},b_{1}]...[a_{n},b_{n}]) = \psi([a_{1},b_{1}])...\psi([a_{n},b_{n}]) = [\psi(a_{1}),\psi(b_{1})]...[\psi(a_{n}),\psi(b_{n})] = 1_{G}...1_{G} = 1_{G}$$
as $G$ is abelian. Hence, $[F,F] \subseteq \ker \psi$. By Theorem 7.12, there is a unique homomorphism $\varphi:F/[F,F] \to G$ such that the following diagram commutes
\[\begin{tikzcd}[ampersand replacement=\&]
	A \\
	F \& G \\
	{F/[F,F]}
	\arrow["{j_{A}}"', from=1-1, to=2-1]
	\arrow["\pi"', from=2-1, to=3-1]
	\arrow["\psi", from=2-1, to=2-2]
	\arrow["f", from=1-1, to=2-2]
	\arrow["\varphi"', from=3-1, to=2-2]
\end{tikzcd}\]
Suppose that $\varphi':F/[F,F] \to G$ is a homomorphism such that $\varphi' \circ \pi \circ j_{A} = f$. As $\psi:F \to G$ is the unique homomorphism satisfying $\psi \circ j_{A} = f$, we must have that $\varphi' \circ \pi = \psi$. We have that $\varphi' \circ \pi = \psi$ and $\varphi' \circ \pi \circ j_{A} = f$. Hence, $\varphi'$ also makes the diagram above commute. By uniqueness of $\varphi$, we must have that $\varphi = \varphi'$. It follows that $\varphi$ is the unique homomorphism such that $\varphi \circ \pi \circ j_{A} = f$ and so the pair $(F/[F,F], \pi \circ j_{A})$ satisfies the universal property of the free abelian group of $A$. Therefore, $F^{\text{ab}} \cong F/[F,F]$. 

\paragraph{13.} 

\subsection*{2.8 - Canonical Decomposition and Lagrange's Theorem}

\paragraph{1.}

\paragraph{2.} Let $G$ be a group and $H$ be a subgroup of index $2$. Let the left cosets of $H$ be $\{H, gH\}$ and the right cosets be $\{H,Hg\}$. We must have that $gH = Hg$ as left cosets, aswell as right cosets, form a disjoint partition of $G$. By a previous exercise, $H$ is normal in $G$. 

\paragraph{3.} Let $G = \{g_{1}, ..., g_{n}\}$ be a finite group. Let $R = \{g_{i}g_{j}g_{k}^{-1} \in A \mid g_{i}g_{j}g_{k}^{-1} = 1, g_{i}, g_{j}, g_{k} \in G\}$. Clearly, $\langle G \mid R \rangle \cong G$ and so $G$ is finitely presented. 

\paragraph{4.}
\paragraph{5.}
\paragraph{6.}
\paragraph{7.}

\paragraph{8.} Define the homomorphism by $\varphi:\text{GL}_{n}(\mathbb{R}) \to \text{SL}_{n}(\mathbb{R})$ by $\varphi(M) = \frac{1}{\det M}M$. We have that $\varphi$ is clearly surjective with $\ker \varphi = \{a I_{n} \mid a \in \mathbb{R} - \{0\}\}$ where $I_{n}$ is the $n \times n$ identity matrix. There is a clear isomorphism between $\ker \varphi$ and $\mathbb{R}^{\times}$. By the First Isomorphism Theorem, $\text{GL}_{n}(\mathbb{R})/\text{SL}_{n}(\mathbb{R}) \cong \mathbb{R}^{\times}$. 

\paragraph{9.}
\paragraph{10.}

\paragraph{11.} Let $G$ be a group and $N$ be a normal subgroup of $G$ containing the subgroup $H$ of $G$. Suppose that $N$ is normal in $G$. Let $xH \in N/H$ and $gH \in G/H$. Then, $(gH)(xH)(g^{-1}H) = gxg^{-1}H \in N/H$ as $gxg^{-1} \in N$ as $N$ is normal in $G$. For the converse, suppose that $N/H$ is normal in $G/H$. Let $g \in G$ and $x \in N$. We have that $(gH)(xH)(g^{-1}H) = gxg^{-1}H \in N/H$. Then, $gxg^{-1} \in N$. We have that $N$ is normal in $G$. 

\paragraph{12.} 

\paragraph{13.}

\paragraph{14.} Let $G$ be a group of order $n$. Let $k \in \mathbb{Z}$ such that $\gcd(n,k) = 1$. As $\gcd(n,k) = 1$, there exists $x,y \in \mathbb{Z}$ such that $nx + ky = 1$. For any $g \in G$, 
$$g = g^{nx + ky} = g^{nx}g^{ky} = (g^{x})^{n}(g^{y})^{k} = (g^{y})^{k}$$
As $g^{y} \in G$, we have that there exists some $g' \in G$ such that $g = g'^{k}$. Therefore, the function $\varphi:G \to G$ given by $\varphi(g) = g^{K}$ is surjective. 

\paragraph{15.} Let $n,a$ be positive integers. Let $G = \mathbb{Z}/(a^{n} - 1)\mathbb{Z}$. We note that the order of $G$ is $\phi(a^{n} - 1)$. Let $k = \gcd(a^{n} - 1, a)$. Then, $k \mid a^{n} - 1$ and $k \mid a$. We have that $k \mid a^{n}$ as $k \mid a$ and so $k \mid a^{n} - (a^{n} - 1) = 1$. It follows $\gcd(a^{n}-1,a) = 1$. Hence, $a \in G$. We have that $a^{n} - 1 \equiv 0 \mod a^{n} - 1$ and so $a^{n} \equiv 1 \mod a^{n} - 1$. We must have that the order of $a$ must divide $n$. As $a^{k} < a^{n} - 1$ for all $k < n$, we must have that the order of $a$ is $n$. By Lagrange's Theorem, $n = |\langle a \rangle |$ must divide $|G| = \phi(a^{n} - 1)$. Therefore, $n \mid \phi(a^{n} - 1)$. 

\paragraph{16.}

\paragraph{17.} Let $G$ be a non-trivial group and let $x \in G$ be a non-trivial element in $G$. As $|x| > 1$, we must have that $|x| = qm$ for some prime $q$ and some $m \in \mathbb{Z}$. Then, $(x^{m})^{q} = x^{qm} = 1_{G}$ and so $|x^{m}| \mid q$. It must follow that $|x^{m}| = q$ as $q$ is prime and $x^{m}$ is non-trivial as $|x| = qm$ and $m < qm$. Hence, every non-trivial group contains an element of prime order. We set to prove that if $G$ is an abelian group of order $n$ and $p$ is a prime dividing $n$, then $G$ contains an element of order $p$. Let $G$ be an abelian group of order $2$. We must have that $G$ contains and element of prime order, hence, $G$ must contain an element of order $2$ as $2$ is the only prime dividing $2$. Let $n \in \mathbb{N}$. Assume for all abelian groups $G$ of order $k < n$, if $p$ is a prime dividing $k$, then $G$ contains an element of order $p$. Let $G$ be an abelian group of order $n$ and let $p$ be a prime dividing $n$. Let $x \in G$ be an element of prime order, $q$. If $q = p$, then we are done. If $p \neq q$, then consider the group $G/H$ where $H = \langle x \rangle$. We have that $|G/H| = n/q < n$ and $p \mid n/q$, then, there is some $gH \in G/H$ such that $|gH| = p$. We have that $g^{p}H = (gH)^{p} = H$ and so $g^{p} \in H$. As $p \neq q$, we have that $g^{p} \neq 1_{G}$ and so $g^{p} = x^{d}$ for some $d \in \mathbb{Z}$. We have that $(g^{q})^{p} = g^{pq} = x^{dpq} = 1_{G}$ and so $|g^{q}| \mid p$. As $g^{q}$ is non-trivial, we have that $|g^{q}| = p$. It follows that $G$ contains an element of order $p$. By the principle of strong induction, the claim holds. 

\paragraph{18.} Let $G$ be an abelian group of order $2n$ where $n$ is odd. By the previous exercise, there exists an element of order $2$, $x$ say. Suppose there exists a non-trivial element $y \in G$ such that $y$ has order $2$ and $x \neq y$. Let $H = \{1_{G}, x, y, xy\}$. We have that $H$ is a subgroup of $G$ and is of order $4$. By Lagrange's Theorem, the order of $H$ must divide the order of $G$. Hence, $4 \mid 2n$. This is a contradiction as $n$ is odd, hence, such an element $y$ cannot exist. It follows that $x$ is unique. 

\paragraph{19.}

\paragraph{20.} Let $G$ be a finite abelian group of order $1$. Then, $G$ is the trivial group and for every $d$ dividing $1$ (which is just $1$), there is a subgroup of order $d$, $H$, of $G$. Assume for all finite abelian groups of order $k < n$, if $d \mid n$, then $G$ contains a subgroup of order $d$. Let $G$ be a finite abelian group of order $n$ and let $d$ be a divisor of $n$. If $d = 1$, then $G$ clearly has a subgroup of order $d$. Suppose $d > 1$. Then, $d = pm$ for some prime $p$ and $m \in \mathbb{N}$. We have that $G$ contains an element of order $p$, $x$ say, and so $\langle x \rangle$ is a subgroup of order $p$. We have that $G/\langle x\rangle$ is a group of order $n/p$ and $d/p \mid n/p$. As $n/p < n$, we have that $G/\langle x \rangle$ contains a subgroup of order $d/p$, $H/\langle x\rangle$. By the Third Isomorphism Theorem, $H$ is a subgroup of $G$, and $|H| = |H/\langle x\rangle||\langle x \rangle| = d/p \cdot p = d$ by Lagrange's Theorem. Therefore, $G$ has a subgroup of order $d$. By the principle of strong induction, if $d$ is a divisor of $|G|$, where $G$ is a finite abelian group, then $G$ contains a subgroup of order $d$. 

\paragraph{21.}

\paragraph{22.} Let $G, G'$ be groups and let $\varphi:G \to G'$ be a homomorphism. Let $L$ be a group and let $\alpha:G' \to L$ be a homomorphism such that $\alpha \circ \varphi = 0$. Let $N$ be the normal closure of $\im \varphi$ and $\pi:G' \to G'/N$ be the canonical projection. As $\alpha \circ \varphi = 0$, we have that $\im \varphi \subseteq \ker \alpha$. We have that $\ker \alpha$ is a normal subgroup of $G'$ and, by definition, $\im \varphi \subseteq N \subseteq \ker \alpha$. By Theorem 7.12, there is a unique $\psi$ such that the following diagram commutes, 
\[\begin{tikzcd}[ampersand replacement=\&]
	G \& {G'} \& L \\
	\& {G'/N}
	\arrow["\varphi", from=1-1, to=1-2]
	\arrow["\pi"', two heads, from=1-2, to=2-2]
	\arrow["\psi"', from=2-2, to=1-3]
	\arrow["\alpha", from=1-2, to=1-3]
	\arrow["0", curve={height=-18pt}, from=1-1, to=1-3]
\end{tikzcd}\]
Therefore, in $\mathsf{Grp}$, $\text{coker }\varphi \cong G'/N$


\paragraph{23.}
\paragraph{24.}
\paragraph{25.}

\subsection*{2.9 - Group Actions}
\paragraph{1.}
\paragraph{2.}

\paragraph{3.} Let $G = (G, *)$ be a group and define the opposite group of $G$, $G^{\circ} = (G, \cdot)$, supported on the same set $G$, by prescribing $\forall g,h \in G$, $g \cdot h = h * g$. We verify $G^{\circ}$ is a group. We have that for all $x,y,z \in G^{\circ}$, $x \cdot (y \cdot z) = (y \cdot z) * x = (z * y) * x = z * (y * x) = (y * x) \cdot z = (x \cdot y) \cdot z$. Let $1_{G}$ be the identity element in $G$. We have that $1_{G} \in G^{\circ}$, and for each $x \in G^{\circ}$, $1_{G}\cdot x = x * 1_{G} = x$ and $x \cdot 1_{G} = 1_{G} * x = x$. Hence, $1_{G^{\circ}} = 1_{G}$. Furthermore, for each $x \in G^{\circ}$, let $x^{-1} \in G$ be the inverse of $x \in G$. We have that $x \cdot x^{-1} = x^{-1} * x = 1_{G} = 1_{G^{\circ}}$. It follows that $G^{\circ}$ is a group. Let $i:G^{\circ} \to G$ be the identity map, sending $g \in G^{\circ}$ to $g \in G$. $i$ is trivially bijective. For all $x,y \in G$, we have that $i$ is a homomorphism if and only if $i(x \cdot y) = i(x) * i(y)$ if and only if $x\cdot y = x * y$ if and only if $y * x = x * y$. Hence, $i$ is an isomorphism if and only if $G$ is abelian. Define $\varphi:G^{\circ} \to G$ by $\varphi(g) = g^{-1}$. We have that $\varphi$ is clearly bijective. As well as that, 
$$\varphi(x\cdot y) = (x \cdot y)^{-1} = (y * x)^{-1} = x^{-1} * y^{-1} = \varphi(x) * \varphi(y)$$
Hence, $\varphi$ is an isomorphism. It follows that $G \cong G^{\circ}$. 

\paragraph{4.}

\paragraph{5.} Let $G$ be a group and let $A$ be the underlying set. Let $\rho:G \times A \to A$ be the action by left multiplication. Suppose there is a $g \in G$ such that for all $a \in A$, $\rho(g,a) = a$. Then, we have that, in $G$, $ga = a$ for all $a \in A$. It follows from right multiplication, $g = 1_{G}$. Hence, $\rho$ is free. 

\paragraph{6.} Let $G$ be a group and let $a \in G$. Let $\rho:G \times A \to A$ be an action of $G$ on a set $A$. Let $\text{Orb}_{G}(a)$ be the orbit of $a \in G$ under $\rho$. Let $\rho'$ be the restriction of $\rho$ to $G \times \text{Orb}_{G}(a)$. Let $g, h \in \text{Orb}_{G}(a)$. Then, $g = \rho(g',a)$ and $h = \rho(h', a)$ for some $g',h' \in G$. We have that 
$$\rho'(h'g'^{-1}, g) = \rho'(h', \rho'(g'^{-1},g)) = \rho'(h', \rho'(g'^{-1},\rho(g',a))) = \rho'(h', \rho'(1_{G},\rho(1_{G},a))) = \rho'(h', \rho'(1_{G},a)) = \rho'(h', a) = h$$
It follows that there exists an $a \in G$ such that $\rho'(a,g) = h$. Therefore, the action $\rho'$ is transitive. 

\paragraph{7.} Let $G$ be a group and let $a \in G$. Let $\rho:G \times A \to A$ be a left action of $G$ on a set $A$. Define $\text{Stab}_{G}(a) = \{g \in G \mid \rho(g,a) = a\}$. Let $g,h \in \text{Stab}_{G}(a)$. We have that 
$$\rho(gh^{-1}, a) = \rho(gh^{-1}, \rho(h,a)) = \rho(gh^{-1}h, a) = \rho(g,a) = a$$
Therefore, $\text{Stab}_{G}(a)$ is a subgroup of $G$. 

\paragraph{8.} Let $G$ be a group. Consider the structure $G\dash\mathsf{Set}$ where $\mathsf{obj}(G\dash\mathsf{Set})$ is the class of pairs $(\rho, A)$ where $\rho:G \times A \to A$ is a left action of $G$ on a set $A$ and a morphism $(\rho, A) \to (\rho', A')$ corresponds to a set function $\varphi:A \to A'$ such that the following diagram commutes, 
\[\begin{tikzcd}[ampersand replacement=\&]
	{G \times A} \&\& {G \times A'} \\
	A \&\& {A'}
	\arrow["{\text{id}_{G}\times \varphi}", from=1-1, to=1-3]
	\arrow["\varphi"', from=2-1, to=2-3]
	\arrow["{\rho'}", from=1-3, to=2-3]
	\arrow["\rho"', from=1-1, to=2-1]
\end{tikzcd}\]
i.e for all $g \in G, a \in A$, $\varphi$ is a set function such that $\varphi(\rho(g,a)) = \rho'(g,\varphi(a))$. Let $(\rho, A)$ be an object of $G\dash\mathsf{Set}$. We have that there is a $1_{(\rho, A)} \in \text{Hom}((\rho, A),(\rho, A))$, namely, the identity set function $\text{id}_{A}:A \to A$, as $\text{id}_{A}(\rho(g,a)) = \rho(g,a) = \rho(g,\text{id}_{A}(a))$ for all $g\in G, a \in A$. Let $(\rho, A), (\rho', A'), (\rho'', A'')$ be objects in $G\dash\mathsf{Set}$ and $f$ be a morphism $(\rho, A) \to (\rho',A)$ and $g$ be a morphism $(\rho', A) \to (\rho'',A'')$. Let $\varphi:A \to A'$ correspond to $f$ and $\psi:A' \to A''$ correspond to $g$. Define the composition $gf$ as the set function $\psi \circ \varphi$. We have that 
$$(\varphi \circ \psi)(\rho(g,a)) = \varphi(\psi(\rho(g,a))) = \varphi(\rho'(g,\psi(a))) = \rho''(g, \varphi(\psi(a))) = \rho''(g,(\varphi \circ \psi)(a))$$
for all $g \in G, a \in A$. Hence, their composition is $G\dash$equivariant. We note this composition is associative as function composition is associative. Let $f \in \text{Hom}(A,B)$, we have that $f1_{A} = f$ and $1_{B}f = f$ as $1_{A}, 1_{B}$ are simply identity functions. It follows $G\dash\mathsf{Set}$ is a category. 

\paragraph{9.} Let $(\rho, A), (\rho', A')$ be objects in the category $G\dash\mathsf{Set}$. Let $\rho \times \rho':G \times (A \times A') \to A \times A'$ be a map defined by $(\rho \times \rho')(g,(a,a')) = \rho(g,a) \times \rho'(g,a')$. We have that $(\rho \times \rho')(1_{G},(a,a')) = \rho(1_{G},a) \times \rho'(1_{G},a') = (a,a')$ and
$$(\rho \times \rho')(gh,(a,a')) = \rho(gh,a) \times \rho'(gh,a') = \rho(g,\rho(h,a)) \times \rho'(g,\rho(h,a')) = (\rho \times \rho')(g, (\rho(h,a), \rho(h,a')))$$
$$=(\rho \times \rho')(g, (\rho \times \rho')(h,(a,a')))$$
for all $g,h \in G$ and $(a,a') \in A \times A'$. Hence, $\rho \times \rho'$ is an action, so $(\rho \times \rho', A \times A')$ is an object of $G\dash\mathsf{Set}$. Let $\pi_{A}:A \times A' \to A$ be the projection map. We have that
$$\pi_{A}((\rho \times \rho')(g,(a,a'))) = \pi_{A}(\rho(g,a) \times \rho'(g,a')) = \rho(g,a) = \rho(g,\pi_{A}(a,a'))$$
for all $g \in G$ and $(a,a') \in A \times A'$. Therefore, $\pi_{A}$ is $G$-equivariant. Let $(\sigma, Z)$ be an object in $G\dash\mathsf{Set}$ and $f,g$ be morphisms from $(\sigma, Z)$ to $(\rho,A)$ and $(\sigma, Z)$ to $(\rho', A')$, respectively. We claim there exists a unique morphism $\varphi$ such that the following diagram commutes,
\[\begin{tikzcd}[ampersand replacement=\&]
	\&\&\& {(\rho,A)} \\
	{(\sigma,Z)} \&\& {(\rho\times\rho',A \times A')} \\
	\&\&\& {(\rho',A')}
	\arrow["\varphi", from=2-1, to=2-3]
	\arrow["{\pi_{A'}}"', from=2-3, to=3-4]
	\arrow["{\pi_A}", from=2-3, to=1-4]
	\arrow["f", curve={height=-12pt}, from=2-1, to=1-4]
	\arrow["g"', curve={height=12pt}, from=2-1, to=3-4]
\end{tikzcd}\]
We have that $\pi_{A} \circ \varphi = f$ and $\pi_{A'} \circ \varphi = g$. Then, $\varphi:Z \to A \times A'$ must take form of $f \times g$. Furthermore, 
$$\varphi(\sigma(h, a)) = f(\sigma(h,a)) \times g(\sigma(h,a)) = \rho(h,f(a)) \times \rho'(h,g(a)) = (\rho \times \rho')(h, (f(a),g(a))) = (\rho \times \rho')(h, \varphi(a))$$
for all $h\in G$ and $a \in Z$. Hence, $\varphi$ is $G$-equivariant. Therefore, $\varphi$ is a unique morphism that makes the above diagram commute. We conclude that the product of $(\rho, A)$ and $(\rho',A')$ in $G\dash\mathsf{Set}$ exists, and is $(\rho \times \rho', A \times A')$. Next, we show $G\dash\mathsf{Set}$ has coproducts. Let $(\rho, A), (\rho', A')$ be objects in $G\dash\mathsf{Set}$. Define $\rho \amalg \rho':G \times A \amalg A' \to A \amalg A'$ by 
$$(\rho \amalg \rho')(g,(a,i)) = \begin{cases} (\rho(g,a),1) & \text{if } i = 1 \\ (\rho'(g,a),2) &\text{if } i = 2\end{cases}$$
We have that 
$$(\rho \amalg \rho')(1_{G},(a,1)) = (\rho(1_{G},a),1) = (a,1)$$
$$(\rho \amalg \rho')(1_{G},(a',2)) = (\rho'(1_{G},a'),2) = (a',2)$$
for all $a \in A$ and $a' \in A'$. Furthermore, for each $g,h \in G$, 
$$(\rho \amalg \rho')(gh,(a,1)) = (\rho(gh,a),1) = (\rho(g,\rho(h,a)),1) = (\rho \amalg \rho')(g,(\rho(h,a),1)) = (\rho \amalg \rho')(g,(\rho \amalg \rho')(h,(a,1)))$$
$$(\rho \amalg \rho')(gh,(a',2)) = (\rho'(gh,a'),2) = (\rho'(g,\rho'(h,a')),2) = (\rho \amalg \rho')(g,(\rho'(h,a'),2)) = (\rho \amalg \rho')(g,(\rho \amalg \rho')(h,(a',2)))$$
Hence, $\rho \amalg \rho'$ is an action, and so $(\rho \amalg \rho', A \amalg A')$ is an object in $G\dash\mathsf{Set}$. Let $i_{A}:A \to A \amalg A', i_{A'}:A' \to A\amalg A'$ be the canonical inclusions. We have that 
$$i_{A}(\rho(g,a)) = (\rho(g,a), 1) = (\rho \amalg \rho')(g, (a,1)) = (\rho \amalg \rho')(g, i_{A}(a))$$
$$i_{A'}(\rho'(g,a')) = (\rho'(g,a'), 2) = (\rho \amalg \rho')(g, (a',2)) = (\rho \amalg \rho')(g, i_{A'}(a))$$
Hence, $i_{A}, i_{A'}$ are $G$-equivariant. Let $(\sigma, Z)$ be an object in $G\dash\mathsf{Set}$ and $f,g$ be morphisms to $(\sigma, Z)$ from $(\rho,A)$ and to $(\sigma, Z)$ from $(\rho', A')$, respectively. We claim there exists a unique morphism $\psi$ such that the following diagram commutes,
\[\begin{tikzcd}[ampersand replacement=\&]
	\&\&\& {(\rho, A)} \\
	{(\sigma, Z)} \&\& {(\rho \amalg \rho', A \amalg A')} \\
	\&\&\& {(\rho', A')}
	\arrow["\psi"', from=2-3, to=2-1]
	\arrow["{i_{A}}"', from=1-4, to=2-3]
	\arrow["{i_{A'}}", from=3-4, to=2-3]
	\arrow["g", curve={height=-12pt}, from=3-4, to=2-1]
	\arrow["f"', curve={height=12pt}, from=1-4, to=2-1]
\end{tikzcd}\]
We have that $\psi \circ i_{A} = f$ and $\psi \circ i_{A'} = g$. Hence, 
$$\psi(a,i) = \begin{cases} f(a) & \text{if } i = 1 \\ g(a) & \text{if } i = 2\end{cases}$$
We have that
$$\psi((\rho \amalg \rho')(h,(a,1))) = \psi((\rho(h,a),1)) = f(\rho(h,a)) = \sigma(h,f(a)) = \sigma(h,\psi(a,1))$$
$$\psi((\rho \amalg \rho')(h,(a',2))) = \psi((\rho(h,a'),2)) = g(\rho(h,a)) = \sigma(h,g(a)) = \sigma(h,\psi(a',2))$$
for all $h \in G$ and $a \in A, a' \in A'$. Therefore, $\psi$ is $G$-equivariant. It follows that $G\dash\mathsf{Set}$ has coproducts, and the coproduct of $(\rho,A)$ and $(\rho',A')$ is $(\rho \amalg \rho', A \amalg A')$. 

\paragraph{10.}

\paragraph{11.} Let $G$ be a finite group and let $H$ be a subgroup of index $p$ where $p$ is the smallest prime dividing $|G|$. Let $\rho:G \times G/H \to G/H$ be left multiplication of $G$ on $G/H$. We have that $\rho$ is an action and induces a homomorphism $\sigma:G \to S_{G/H}$ given $\sigma(g)(aH) = (ga)H$. We have that $G/\ker \sigma$ is isomorphic to a subgroup of $S_{G/H}$ by the First Isomorphism Theorem. Let $x \in \ker \sigma$. Then, $\sigma(x)(aH) = (xa)H = H$ for all $a \in G$. For $a = 1_{G}$, we deduce that $xH = H$. Hence, $x \in H$. We have that $|G/\ker \sigma|$ must divide $p!$ as $G/\ker \sigma$ is isomorphic to a subgroup of $S_{G/H}$. We must also have that $|G/\ker \sigma|$ must also divide $|G|$ as $|G| = |G/\ker\sigma||\ker \sigma|$. As $p$ is the smallest prime dividing $|G|$, we must have that $|G/\ker \sigma| = p$. Then, $p = |G/\ker \sigma| = [G:\ker\sigma] = [G:H][H:\ker\sigma] = p[H:\ker\sigma]$. Hence, $[H:\ker\sigma] = 1$. As $\ker\sigma \subseteq H$, we must have that $\ker\sigma = H$. As $\ker\sigma$ is normal, $H$ is therefore normal in $G$. 

\paragraph{12.} Let $G$ be a finite group and $H \subseteq G$ a subgroup of index $n$. Let $\rho: G \times G/H \to G/H$ be left multiplication where $G/H$ is the set of left cosets of $H$ in $G$. We have that $\rho$ is an action and induces a homomorphism $\sigma: G \to S_{G/H}$ given by $\sigma(g)(aH) = (ga)H$. We have that $G/\ker \sigma$ is isomorphic to a subgroup of $S_{G/H}$ by the First Isomorphism Theorem. Let $x \in \ker \sigma$. Then, $\sigma(x)(aH) = (xa)H = H$ for all $a \in G$. For $a = 1_{G}$, we deduce that $xH = H$. Hence, $x \in H$. We have that $|G/\ker \sigma|$ must divide $n!$ as $G/\ker \sigma$ is isomorphic to a subgroup of $S_{G/H}$. We must also have that $|G/\ker \sigma|$ must also divide $|G|$ as $|G| = |G/\ker\sigma||\ker \sigma|$. Then, $[G:\ker \sigma]$ is a divisor of $|G|$ and $n!$, hence, divides the greatest common divisor of $|G|$ and $n!$. Therefore, there exists a normal subgroup $K \subseteq H$ such that $[G:K] \leq n!$. 

\paragraph{13.} Let $G$ be a group and let $H$ be a subgroup of $G$. Let $G/H$ be the set of left cosets of $H$ in $G$ and $G/gHg^{-1}$ be the set of left cosets of $gHg^{-1}$ in $G$ for some $g \in G$. Let $\rho:G \times G/H \to G/H$ and $\rho':G \times G/gHg^{-1} \to G/gHg^{-1}$ be left multiplication. Define $\varphi:G/H \to G/gHg^{-1}$ by $\varphi(xH) = (xg^{-1})gHg^{-1}$. Suppose that $xH = yH$. We have that
$$\varphi(xH) = (xg^{-1})gHg^{-1} = xHg^{-1} = yHg^{-1} = (yg^{-1})gHg^{-1} = \varphi(yH)$$
Now, suppose that $\varphi(xH) = \varphi(yH)$. Then, $xHg^{-1} = yHg^{-1}$. Hence, $xH = yH$. Let $agHg^{-1} \in G/gHg^{-1}$. Then, $\varphi(agH) = (agg^{-1})gHg^{-1} = agHg^{-1}$. It follows that $\varphi$ is well-defined and bijective. Finally, 
$$\rho'(x, \varphi(aH)) = \rho(x, (ag^{-1})gHg^{-1}) = xaHg^{-1} = xag^{-1}(gHg^{-1}) =\varphi(xaH) = \varphi(\rho(x,aH))$$
for all $x \in G$ and $aH \in G/H$. Therefore, $\varphi$ is $G$-equivariant. It follows that $G/H \cong G/gHg^{-1}$ in $G\dash\mathsf{Set}$.

\paragraph{14.}
\paragraph{15.}
\paragraph{16.}

\paragraph{17.} Let $G$ be a group. Consider $G$ as an object in $G\dash\mathsf{Set}$ along with left multiplication. I claim $\text{Aut}_{G\dash\mathsf{Set}}(G) = \{\varphi_{g}(x) = xg \mid g \in G\}$ where $\varphi_{g}\varphi_{h} = \varphi_{gh}$. Let $\psi \in \text{Aut}_{G\dash\mathsf{Set}}(G)$. We have that $\psi$ is $G$-equivariant, that is, for each $a,g \in G$, $\psi(ga) = g\psi(a)$. For all $x \in G$, it follows that $\psi(x) = \psi(x1_{G}) = x\psi(1_{G})$. Hence, $\psi = \varphi_{g}$ for some $g \in G$. Now, let $\varphi_{g} \in \{\varphi_{g} = xg \mid g \in G\}$. Its clear that $\varphi_{g}$ is bijective. We show $\varphi_{g}$ is $G$-equivariant. For each $x,a \in G$, we have that $\varphi(xa) = (xa)g = x(ag) = x\varphi(a)$. Therefore, $\varphi_{g}$ is $G$-equivariant, and the claim follows. Define $f:G \to \text{Aut}_{G\dash\mathsf{Set}}(G)$ by $f(g) = \varphi_{g}$. $f$ is clearly surjective. Let $h,g \in G$ such that $f(g) = f(h)$. Then, $\varphi_{g}(x) = \varphi_{h}(x)$ for each $x \in G$. Hence, $xg = xh$ and so $g = h$. Thus, $f$ is bijective. Finally, we have that for all $g,h \in G$, $f(gh) = \varphi_{gh} = \varphi_{g}\varphi_{h} = f(g)f(h)$. Therefore, $f$ is an isomorphism. We have that $\text{Aut}_{G\dash\mathsf{Set}} \cong G$. 

\paragraph{18.}

\subsection*{2.10 - Group Objects In Categories}

\paragraph{1.}

\paragraph{2.} 

\paragraph{3.}
\paragraph{4.}
\paragraph{5.}

\section*{III - Rings, and Modules}
\subsection*{3.1 - Definition of Ring}
\paragraph{1.} Let $R$ be a ring such that $0_{R} = 1_{R}$. Let $x \in R$. Then, $x = x1_{R} = x0_{R} = 0_{R}$. It follows that $R$ is the zero-ring.

\paragraph{2.} 
\paragraph{3.}
\paragraph{4.}
\paragraph{5.} 

\paragraph{6.} Let $R$ be a ring where $x,y \in R$ such that $xy = yx$. We first prove the Binomial Theorem, that is, 
$$(x+y)^{n} = \sum_{i=0}^{n}\binom{n}{i}x^{n-i}y^{i}$$
for all $n \in \mathbb{N}$. We see this holds for $n = 1$ easily. Assume 
$$(x + y)^{k} = \sum_{i=0}^{k}\binom{n}{i}x^{n-i}y^{i}$$
for some $k \in \mathbb{N}$. We have that 
$$(x+y)^{k+1} = (x+y)(x+y)^{k} = (x+y)\sum_{i=0}^{k}\binom{k}{i}x^{k-i}y^{i} = \sum_{i=0}^{k}\binom{k}{i}x^{n-i}y^{i}(x+y)$$
$$=\sum_{i=0}^{k}\binom{k}{i}[x^{k+1-i}y^{i}+x^{k-i}y^{i+1}] = \sum_{i=0}^{k+1}\qty[\binom{k}{i-1} + \binom{k}{i}]x^{k+1-i}y^{i} = \sum_{i=0}^{k+1}\binom{k+1}{i}x^{k+1-i}y^{i}$$
By the principle of mathematical induction, the claim holds. Let $R$ be a ring where $a,b \in R$ are nilpotent and commute. We have that $a^{n} = 0_{R}$ and $b^{m} = 0_{R}$ for some $n,m \in \mathbb{N}$. We have that 
$$(a + b)^{n+m} = \sum_{i=0}^{n+m}\binom{n+m}{i}a^{n+m-i}b^{i}$$
as $a$ and $b$ commute. For $i < m$, we have that $a^{n+m-i}b^{i} = a^{n}a^{m-i}b^{i} = 0_{R}$ and for $i \geq m$, $a^{n+m-i}b^{i} = a^{n+m-i}b^{i-m}b^{m} = 0_{R}$. Hence, $(a+b)^{n+m} = 0_{R}$. It follows that $a+b$ is nilpotent. 

\paragraph{7.} Let $n \in \mathbb{N}$ and let $[m] \in \mathbb{Z}/n\mathbb{Z}$. Suppose that $[m]$ is nilpotent. There then exists some $q \in \mathbb{N}$ such that $[m]^{q} = [0]$. Hence, $n \mid m^{q}$. Let $p$ be a prime dividing $n$. Then, $p \mid m^{q}$. As $p$ is prime, $p \mid m$. It follows that $m$ is divisible by each prime divisor of $n$. For the converse, suppose that $m \in \mathbb{N}$ such that for all $p$ prime dividing $n$, $p \mid m$. Let $A = \{k \in \mathbb{N} \mid p^{k} \text{ divides } n \text{ and } p^{k+1} \nmid n, p \text{ prime}\}$. Let $M = \max A$. We have that $n \mid m^{M}$, hence, $[m]^{M} = [0]$. Therefore, $[m]$ is nilpotent. 

\paragraph{8.} Let $R$ be an integral domain and $x \in R$ such that $x^{2} = 1_{R}$. We then have that $x^{2} - 1_{R} = 0_{R}$ and so $(x - 1_{R})(x + 1_{R}) = 0_{R}$. As $R$ is an intgeral domain, $x - 1_{R} = 0_{R}$ or $x + 1_{R} = 0$. Hence, $x = 1_{R}$ or $x_{R} = -1_{R}$. Consider the ring $\mathcal{M}_{2}(\mathbb{R})$. We have that the following matrix is a solution to the equation $x^{2} = I$ where the matrix is not $I$ or $-I$:
$$\begin{pmatrix} 2 & 1 \\ -3 & -2 \end{pmatrix}$$

\paragraph{9.} Let $u \in R$ be a unit such that $u$ has inverses $v$ and $v'$. We have that
$$v = v1_{R} = v(uv') = (vu)v' = 1_{R}v' = v'$$
Hence, the inverse of $u$ is unique. Let $R^{*} = \{u \in R \mid u \text{ is a unit in } R\}$ equipped with multiplication induced by $R$. We note that multiplication is associative as $R$ is a ring. Let $x,y \in R^{*}$, we have that there exists $x',y' \in R^{*}$ such that $xx' = x'x = 1_{R}$ and $yy' = y'y = 1_{R}$. Then, $(xy)(y'x') = x(yy')x' = x1_{R}x' = xx' = 1_{R}$ and $(y'x')(xy) = y'(x'x)y = y'1_{R}y = y'y = 1_{R}$. Hence, $xy$ is a unit and $R^{*}$ is closed under multiplication. We note that $1_{R} \in R^{*}$ as $1_{R}1_{R} = 1_{R}$. Furthermore, for each $x \in R^{*}$, there is a $x'$ such that $xx' = x'x = 1_{R}$. We have that $x'$ is an inverse of $x$ and $x'$ is a unit in $R$. It follows that $R^{*}$ is a group under multiplication. 

\paragraph{10.} Let $R$ be a ring and $a \in R$ such that $a$ is a right unit and has atleast two left inverses. Suppose, for contradiction, that there is a non-zero $b \in R$ such that $ab = 0$. As $a$ is a right unit, there is a $b'$ such that $b'a = 1$. We then have that $b = 1b = (b'a)b = b'(ab) = b'0 = 0$, which contradicts the original assumption. Hence, $a$ must be a left-zero divisor. Now let $x,x' \in R$ be left inverses of $a$ such that $x \neq x'$. We have that $x - x' \neq 0$ and $(x - x')a = xa - x'a = 1 - 1 = 0$. Therefore, $a$ is a right zero-divisor. 

\paragraph{11.}
\paragraph{12.}
\paragraph{13.}
\paragraph{14.} Let $R$ be a ring and $f(x), g(x) \in R[x]$ be non-zero polynomials. We have that $f(x) = \sum_{i\geq 0}a_{i}x^{i}$ and $g(x) = \sum_{i\geq 0}b_{i}x^{i}$ for some $n,m \in \mathbb{N}$. Then, 
$$f(x) + g(x) = \sum_{i\geq 0}(a_{i}+b_{i})x^{i}$$
WLOG assume $m = \deg(f) \geq \deg(g) = n$. Then, $a_{k} + b_{k} = 0$ for all $k > m$. We have that $a_{m} + b_{m}$ is $0$ or not. Hence, $\deg(f + g) \leq \max\{\deg(f),\deg(g)\}$. Furthermore, assume $R$ is an integral domain. We have that
$$f(x)g(x) = \sum_{k \geq 0}\sum_{i+j=k}a_{i}b_{j}x^{i+j}$$
We have that for all $k > m+n$, $a_{i}b_{j} = 0$ for all $i,j$ such that $i+j = k$. As $R$ is an integral domain, if $a_{m}b_{n} = 0$, then $a_{m} = 0$ or $b_{n} = 0$. It follows that the coefficient of $x^{n+m}$ is non-zero as $f$ and $g$ are polynomials of degree $m$ and $n$ respectively. Hence, $\deg(fg) = \deg(f) + \deg(m)$.

\paragraph{15.} Let $R$ be a ring. Suppose that $R$ is an integral domain. Let $f,g \in R[x]$ such that $fg = 0$. We have that $0 = \deg(fg) = \deg(f) + \deg(g)$. It follows that $\deg(f) = 0$ and $\deg(g) = 0$ as $\deg(p) \geq 0$ for all $p \in R[x]$. We must have that $f(x) = r$ and $g(x) = r'$ for some $r,r' \in R$. Then, $0 = fg = rr'$. As $R$ is an integral domain, $f(x) = r = 0$ or $g(x) = r' = 0$. Therefore, $R[x]$ is an integral domain. For the converse, suppose that $R[x]$ is an integral domain. Let $r,r' \in R$ such that $rr' = 0$. We then have that $f(x) = r \in R[x]$ and $g(x) = r' \in R[x]$ and so $fg = rr' = 0$. As $R[x]$ is an integral domain, $f = 0$ or $g = 0$. It follows that $R$ is an integral domain. 

\paragraph{16.}
\subparagraph{(i)} Let $R$ be a ring. Let $f = \sum_{i\geq 0}a_{i}x^{i} \in \fps{R}$ be a unit. We have that there exists a $g = \sum_{i\geq 0}b_{i}x^{i} \in \fps{R}$ such that $gf = fg = 1 \in \fps{R}$. We that that 
$$1 = \sum_{k\geq 0}\sum_{i+j=k}a_{i}b_{j}x^{k}$$
and so $a_{0}b_{0} = 1$. Therefore, $a_{0}\in R$ is a unit. For the converse, let $f = \sum_{i\geq 0}a_{i}x^{i} \in \fps{R}$ where $a_{0}$ is a unit. Define the sequence $b_{n}$ where 
$$b_{n} = -a_{0}^{-1}\qty(\sum_{i=0}^{n-1}a_{n-i}b_{i})$$
and $b_{0} = a_{0}^{-1}$. We have that $\qty(\sum_{i\geq 0}a_{i}x^{i})\qty(\sum_{i\geq 0}b_{i}x^{i}) = 1$. The inverse of $1 - x$ is then $1 + x + x^{2} + x^{3} + ...$. 

\subparagraph{(ii)} Let $R$ be a ring. Suppose that $\fps{R}$ is an integral domain. Let $r,r' \in R$ be non-zero elements in $R$ such that $rr' = 0$. We have that $f(x) = r$ and $g(x) = r'$ are elements in $\fps{R}$ and we have that $fg = 0$. By assumption, we have that $f = 0$ or $g = 0$. Hence, $r = 0$ or $r' = 0$. Therefore, $R$ is an integral domain. For the converse, suppose that $R$ is an integral domain. Let $f, g \in \fps{R}$ such that $fg = 0$. We have that 
$$\sum_{i=0}^{n}a_{i}b_{n-i} = 0$$
for all $n \geq 0$. Then, $a_{0}b_{0} = 0$ and so $a_{0} = 0$ or $b_{0} = 0$. Without loss of generality, assume that $a_{0} = 0$ and $b_{0} \neq 0$. Then, we have that $a_{0}b_{1} + a_{1}b_{0} = a_{1}b_{0} = 0$. Hence, $a_{1} = 0$. Via strong induction, we have that $a_{n} = 0$ for all $n \in \mathbb{N}$. It follows that $f = 0$. Similiarly, if $b_{0} = 0$, then $g = 0$. Therefore, $\fps{R}$ is an integral domain. 

\paragraph{17.}

\subsection*{3.2 - The Category $\mathsf{Ring}$}
\paragraph{1.} Let $R$ be a ring and $\textbf{0}$ the zero-ring. Suppose there exists a homomorphism $\varphi:\textbf{0} \to R$. Let $0 \in \textbf{0}$. We have that $\varphi(0) = 1_{R}$. Then, $1_{R} = \varphi(0) = \varphi(0 + 0) = \varphi(0) + \varphi(0) = 1_{R} + 1_{R}$. Hence, $1_{R} = 0_{R}$. Therefore, $R$ is the zero-ring. 

\paragraph{2.} Let $R$ and $S$ be rings. Let $\varphi:R \to S$ be a function preserving both additive and multiplicative operations.

\subparagraph{(i)} Suppose that $\varphi$ is surjective. We have that there exists an $x \in R$ such that $\varphi(x) = 1_{S}$. Then, 
$$1_{S} = \varphi(x) = \varphi(x1_{R}) = \varphi(x)\varphi(1_{R}) = 1_{S}\varphi(1_{R}) = \varphi(1_{R})$$
Hence, $\varphi$ is a ring homomorphism. 

\subparagraph{(ii)} Suppose that $\varphi \neq 0$ and $S$ is an integral domain. We have that $\varphi(1_{R}) = \varphi(1_{R}1_{R}) = \varphi(1_{R})\varphi(1_{R})$. Hence, $\varphi(1_{R})^{2} - \varphi(1_{R}) = 0_{S}$. By the distributive law, $\varphi(1_{R})(\varphi(1_{R}) - 1_{S}) = 0_{S}$. As $S$ is an integral domain, $\varphi(1_{R}) = 0_{S}$ or $\varphi(1_{R}) = 1_{S}$. Suppose that $\varphi(1_{R}) = 0_{S}$. Let $x \in R$. Then, $\varphi(x) = \varphi(1_{R}x) = \varphi(1_{R})\varphi(x) = 0_{S}\varphi(x) = 0_{S}$. This cannot happen as, by assumption, $\varphi \neq 0$. Therefore, $\varphi(1_{R}) = 1_{S}$. It follows that $\varphi$ is a ring homomorphism. 

\paragraph{3.}
\paragraph{4.}
\paragraph{5.}

\paragraph{6.} Let $\alpha:R \to S$ be a fixed ring homomorphism and let $s \in S$ such that $s\alpha(r) = \alpha(r)s$ for all $r \in R$. Let $\overline{\alpha}:R[x] \to S$ be a ring homomorphism that extends $\alpha$ and sends $x$ to $s$. Let $f(x) = \sum_{i \geq 0}a_{i}x^{i} \in R[x]$. We have that
$$\overline{\alpha}(f(x)) = \overline{\alpha}\qty(\sum_{i\geq 0}a_{i}x^{i}) = \sum_{i\geq 0}\overline{\alpha}(a_{i}x^{i}) = \sum_{i\geq 0}\overline{\alpha}(a_{i})\overline{\alpha}(x)^{i} = \sum_{i\geq 0}\alpha(a_{i})s^{i}$$
We note that $\overline{\alpha}(f) = \sum_{i\geq 0}\alpha(a_{i})s^{i}$ fits our criteria and is unique. 

\paragraph{7.}

\paragraph{8.} Let $F$ be a field and $S$ a subring of $F$. Let $x,y \in S$ such that $xy = 0$. As $x, y \in S$, and $S$ is a subring of $F$, we have that $x,y \in F$ and $xy = 0$. As $F$ is a field, it is an integral domain, hence, $x = 0$ or $y = 0$. It follows that $S$ is an integral domain. 

\paragraph{9.} 
\subparagraph{(i)} Let $R$ be a ring and $Z(R) = \{a \in R \mid \forall r \in R, ar = ra\}$ be the centre of $R$. Let $x,y \in Z(R)$. For each $r \in R$, we have that $(x-y)r = xr - yr = rx - ry = r(x-y)$. Thus, $x-y \in Z(R)$. We also have that $0_{R} \in Z(R)$ as $0_{R}$ commutes with all $r \in R$. We also have that $1_{R} \in Z(R)$ for the same reason. Furthermore, $r(xy) = (rx)y = (xr)y = x(ry) = x(yr) = (xy)r$. Therefore, $xy \in Z(R)$. It follows that $Z(R)$ is a subring of $R$. 

\subparagraph{(ii)} Let $R$ be a division ring. Let $x,y \in Z(R)$. Then, $xr = rx$ and $yr = ry$ for all $r \in R$. Thus, $xy = yx$. We have that $Z(R)$ is a commutative ring. Let $x \in Z(R)$. Then, $x \in R$ and so there is a $x^{-1}$ such that $xx^{-1} = x^{-1}x = 1_{R}$. We have that $xr = rx \implies rx^{-1} = x^{-1}r$ and so $x^{-1} \in Z(R)$. Hence, $x$ is a unit in $Z(R)$. As $x$ was arbitrary, it follows that $Z(R)$ is a division ring. Therefore, $Z(R)$ is a field.

\paragraph{10.} 
\subparagraph{(i)} Let $R$ be a ring and let $a \in R$. Let $C(a) = \{r \in R \mid ar = ra\}$. Let $x,y \in C(a)$, then $a(x-y) = ax - ay = xa - ya = (x-y)a$. Hence, $x-y \in C(a)$. We also have that $a1_{R} = a = 1_{R}a$ and $a0_{R} = 0_{R} = 0_{R}a$, and so $0_{R}, 1_{R} \in C(a)$. Furthermore, $a(xy) = (ax)y = (xa)y = x(ay) = x(ya) = (xy)a$. Thus, $xy \in C(a)$. Therefore, $C(a)$ is a subring of $R$ for all $a \in R$. 

\subparagraph{(ii)} Let $x \in Z(a)$. Then, $xr = rx$ for all $r \in R$. We must have that $x \in C(r)$ for all $r \in R$. Therefore, $x \in \bigcap_{a \in R}C(a)$. Now, let $x \in \bigcap_{a\in R}C(a)$. Hence, $x \in C(a)$ for all $a \in R$. Thus, $xa = ax$ for all $a \in R$ and so $x \in Z(R)$. Therefore, 
$$Z(R) = \bigcap_{a\in R}C(a)$$

\subparagraph{(iii)} Let $R$ be a division ring and let $a \in R$. Let $C(a)$ be the centraliser of $a$. Let $x \in C(a)$. As $R$ is a division ring, there is an $x^{-1} \in R$ such that $xx^{-1} = x^{-1}x = 1_{R}$. As $x \in C(a)$, $xa = ax$. Then, $x^{-1}a = ax^{-1}$, which means $x^{-1} \in C(a)$. It follows that $C(a)$ is a division ring. 

\paragraph{11.} Let $R$ be a division ring consisting of $p^{2}$ elements where $p$ is prime. Suppose, for contradiction, $R$ is non-commutative. As $R$ is non-commutative, $Z(R) \neq R$. As $Z(R)$ is a subring of $R$, $0_{R}$ and $1_{R}$ are contained in $Z(R)$. We have that $1 < Z(R) < p^{2}$. The only divisors of $p^{2}$ are $1,p$ and $p^{2}$, hence, $|Z(R)| = p$ by Lagrange's Theorem. Let $r \in R$ such that $r \notin Z(R)$. We have that $Z(R) \subseteq C(r)$ as $Z(R) = \bigcap_{a\in R}C(a)$. As $C(r)$ is a subring of $R$, which contains $Z(R)$ and $r \in R$, we must have that $C(r) = R$ by Lagrange's Theorem. Let $x,y \in R$. If $x \in Z(R)$, then $xy = yx$. If $x \notin Z(R)$, then $C(x) = R$ and so $xy = yx$. In both cases, $xy = yx$. Therefore, $R$ is commutative. This is a contradiction. It follows that $R$ is a field. 

\paragraph{12.}

\paragraph{13.} Let $R_{1}, R_{2}$ be rings. Let $R_{1} \times R_{2}$ be their 'componentwise' product. Let $S$ be a ring and $f:R_{1} \to S, g:R_{2} \to S$ be ring homomorphisms. Let $\varphi:S \to R_{1} \times R_{2}$ be a ring homomorphism such that $f = \pi_{R_{1}} \circ \varphi$ and $g = \pi_{R_{2}} \circ \varphi$. We must have that $\varphi = (f,g)$. We verify $\varphi = (f,g)$ is infact a ring homomorphism. Let $x,y \in S$, we have that 
$$\varphi(xy) = (f(xy),g(xy)) = (f(x)f(y),g(x)g(y)) = (f(x),g(x))(f(y),g(y)) = \varphi(x)\varphi(y)$$
$$\varphi(x+y) = (f(x+y),g(x+y)) = (f(x) + f(y),g(x) + g(y)) = (f(x),g(x)) + (f(y),g(y)) = \varphi(x) + \varphi(y)$$
$$\varphi(1_{S}) = (f(1_{S}), g(1_{S})) = (1_{R_{1}}, 1_{R_{2}}) = 1_{R_{1} \times R_{2}}$$
Therefore, $\varphi$ is a ring homomorphism. It follows that $R_{1} \times R_{2}$ satisfies the universal property for the product of $R_{1}$ and $R_{2}$ in $\mathsf{Ring}$. 

\paragraph{14.}
\paragraph{15.}
\paragraph{16.}
\paragraph{17.}
\paragraph{18.}
\paragraph{19.}

\subsection*{3.3 - Ideals and Quotient Rings}
\paragraph{1.} Let $\varphi:R \to S$ be a ring homomorphism. Let $x,y \in \varphi(R)$. There then exists $x', y'$ such that $\varphi(x') = x$ and $\varphi(y') = y$. We have that $\varphi(x' - y') = \varphi(x') - \varphi(y') = x - y$. Hence, $x-y \in \varphi(R)$. Furthermore, $\varphi(x'y') = \varphi(x')\varphi(y') = xy$. Hence, $xy \in \varphi(R)$. Finally, as $\varphi$ is a ring homomorphism, $\varphi(1_{R}) = 1_{S}$ and so $1_{S} \in \varphi(R)$. It follows that $\varphi(R)$ is a subring of $S$. Suppose that $\varphi(R)$ is an ideal of $S$. By definition, for all $s \in S$ and $x \in \varphi(R)$, we have that $xs \in \varphi(R)$ and $sx \in \varphi(R)$. Setting $x = 1_{S}$, we have that $s \in \varphi(R)$ for all $s \in S$. It follows that $\varphi$ is surjective. Suppose that $\ker \varphi$ is a subring of $R$. We have that $1_{R} \in \ker\varphi$ and so $\varphi(1_{R}) = 0_{S}$. For any $x \in R$, we then have that $\varphi(x) = \varphi(x1_{R}) = \varphi(x)\varphi(1_{R}) = \varphi(x)0_{S} = 0_{S}$. Hence, $\varphi$ is the zero map.

\paragraph{2.}  Let $\varphi:R \to S$ be a ring homomorphism and $J$ an ideal of $S$. Set $I = \varphi^{-1}(J)$. Let $x,y \in I$. Then, $\varphi(x),\varphi(y) \in J$. As $J$ is an ideal, $\varphi(x) - \varphi(y) \in J$. Hence, $\varphi(x-y) \in J$. We must have that $x-y \in I$. Now, let $r \in R$ and $x \in I$. We have that $\varphi(r) \in S$ and $\varphi(x) \in J$. As $J$ is an ideal, $\varphi(rx) = \varphi(r)\varphi(x) \in J$ and $\varphi(xr) = \varphi(x)\varphi(r) \in J$. It follows that $rx, xr \in I$ and so $I$ is an ideal. 

\paragraph{3.} Let $\varphi:R \to S$ be a ring homomorphism and let $J$ be an ideal of $R$

\subparagraph{(i)} Consider the inclusion map $i:\mathbb{Z} \to \mathbb{Q}$. The inclusion map is a ring homomorphism as $\mathbb{Z}$ is a subring of $\mathbb{Q}$. We have that $\mathbb{Z}$ is an ideal of $\mathbb{Z}$, however, $i(\mathbb{Z}) = \mathbb{Z}$ is not an ideal of $\mathbb{Q}$. 

\subparagraph{(ii)} Assume that $\varphi$ is surjective. Let $x,y \in \varphi(J)$. We have that there exists $x', y' \in J$ such that $\varphi(x') = x$ and $\varphi(y') = y$. As $J$ is an ideal, $x'-y' \in J$ and so $x-y = \varphi(x')-\varphi(y') = \varphi(x'-y') \in \varphi(J)$. Let $r \in S$ and $x \in \varphi(J)$. As $\varphi$ is surjective, there is an $r' \in R$ such that $\varphi(r') = r$ and there is an $x' \in J$ such that $\varphi(x') = x$. As $J$ is an ideal, $r'x', x'r' \in J$. Hence, $rx = \varphi(r')\varphi(x') = \varphi(r'x') \in \varphi(J)$ and $xr = \varphi(x')\varphi(r') = \varphi(x'r') \in \varphi(J)$. Therefore, $\varphi(J)$ is an ideal of $S$. 

\subparagraph{(iii)} 

\paragraph{4.} Let $R$ be a ring of characteristic $n$ such that every subgroup of $(R, +)$ is an ideal. By definition, $\ker f = n\mathbb{Z}$ where $f:\mathbb{Z} \to R$ is the map defined by $f(x) = x\cdot 1_{R}$. We have that $f(\mathbb{Z})$ is a subring of $R$, hence, $1_{R} \in f(\mathbb{Z})$. By assumption, $f(\mathbb{Z})$ is an ideal of $R$. $f(\mathbb{Z})$ of $R$ is an ideal that contains $1_{R}$, thus, $f(\mathbb{Z}) = R$. By the first isomorphism theorem, $\mathbb{Z}/n\mathbb{Z} \cong R$. 

\paragraph{5.}

\paragraph{6.}

\paragraph{7.} Let $R$ be a ring and let $a \in R$. Let $xa,ya \in Ra$. We have that $xa - ya = (x-y)a \in Ra$ as $x-y \in R$. Let $r \in R$ and $xa \in Ra$. We have that $rxa \in Ra$ as $rx \in R$. Hence, $Ra$ is a left ideal of $R$. Suppose that $a$ is a right unit. Then, there exists an $a^{-1} \in R$ such that $a^{-1}a = 1_{R}$. We have that $a = 1_{R}a \in Ra$. As $Ra$ is a left ideal, we have that $1_{R} = a^{-1}a \in Ra$. Hence, $R = Ra$. For the converse, suppose that $R = Ra$. Then, $1_{R} \in R$ can be represented as $ra$ for some $r \in R$. It follows that $a$ is a right unit in $R$. With a similar argument, we can also show $aR$ is a right ideal of $R$ and $R = aR$ if and only if $a$ is a left unit in $R$. 

\paragraph{8.} Let $R$ be a ring. Suppose that $R$ is a division ring. We have that $\{0\}$ is an ideal of $R$. Let $I$ be a left ideal of $R$ with atleast two elements. Let $x \in I$ such that $x \neq 0_{R}$. If $x = 1_{R}$, then $I = R$ and we are done. Suppose $x \neq 1_{R}$. As $R$ is a division ring, there exists an $x^{-1} \in R$ such that $xx^{-1} = x^{-1}x = 1_{R}$. As $I$ is a left ideal, $1_{R} = x^{-1}x \in I$. Hence, $I = R$. With a similar argument, we can also show that every right ideal of $R$ with atleast two elements is equal to $R$. Therefore, $R$ only contains $R$ and $\{0\}$ as left and right ideals. For the converse, suppose that $R$ only has $R$ and $\{0\}$ as its right and left ideals. Let $x \in R$. By the previous exercise, we have that $xR$ is a right ideal of $R$. If $x \neq 0_{R}$, then $xR$ is forced to be $R$ by assumption. Then, by the previous exercise, $x$ is a left unit. Similarly, $Rx$ is a left ideal of $R$. If $x \neq 0_{R}$, then $Rx$ is forced to be $R$. Then, $x$ is a right unit. $x$ is then a unit in $R$. As $x$ was arbitrary, it follows that $R$ is a division ring. 

\paragraph{9.}

\paragraph{10.} Let $\varphi:k \to R$ be a ring homomorphism where $k$ is a field and $R$ is a non-zero ring. As $k$ is a field, it is a division ring, and so its only ideals are $k$ and $\{0\}$. We have that $\ker \varphi$ is an ideal of $k$, thus, $\ker \varphi = k$ or $\ker \varphi = \{0\}$. If $\ker \varphi = k$, then $\varphi$ is the zero map. As $\varphi$ is a ring homomorphism, $1_{R} = \varphi(1_{k}) = 0_{R}$. Hence, $R$ is the zero-ring, which is not permitted by assumption. This forces $\ker \varphi = \{0\}$. By Proposition 2.4, $\varphi$ is injective. 

\paragraph{11.}

\paragraph{12.} Let $R$ be a commutative ring and $N$ be set of nilpotent elements of $R$. Let $x,y \in N$. There exists a $k \in \mathbb{N}$ such that $y^{k} = 0_{R}$. We have that $(-y)^{k} = (-1)^{k}y^{k} = (-1)^{k}0_{R} = 0_{R}$. Hence, $-y \in N$. By a previous exercise, $x-y \in N$. Let $x \in N$ and $r \in R$. There exists a $m \in \mathbb{N}$ such that $x^{m} = 0_{R}$. We have that $(rx)^{m} = r^{m}x^{m} = r^{m}0_{R} = 0_{R}$. Therefore, $rx \in N$. It follows that $N$ is an ideal of $R$.

\paragraph{13.} Let $R$ be a commutative ring and $N$ the nilradical of $R$. Suppose there exists an $x + N \in R/N$ such that there is a $k \in \mathbb{N}$ such that $(x+N)^{k} = N$. Then, $x^{k} + N = N$ and so $x^{k} \in N$. Hence, $x^{k}$ is nilpotent in $R$. There then exists an $m \in \mathbb{N}$ such that $(x^{k})^{m} = 0_{R}$. Therefore, $x \in N$ as $x^{km} = 0_{R}$. Hence, $x + N = N$. It follows that the nilradical of $R/N$ is trivial. 

\paragraph{14.} Let $R$ be an integral domain with $\text{char} R > 0$. Suppose that $\text{char} R = mn$ for $1 < m,n < mn$. Let $f:\mathbb{Z} \to R$ be the ring homomorphism given by $f(x) = x\cdot 1_{R}$. Then, $0_{R} = f(mn) = f(m)f(n)$. As $R$ is an integral domain, $f(m) = 0_{R}$ or $f(n) = 0_{R}$. In either case, this contradicts $\text{char} R = mn$ as $m, n < mn$. Therefore, $\text{char} R$ cannot be composite. $\text{char} R$ must be prime or $0$. 

\paragraph{15.} 
\paragraph{16.}
\paragraph{17.}

\subsection*{3.4 - Ideals and Quotient Rings: Remarks and Examples}
\paragraph{1.} Let $R$ be a ring and let $\{I_{\alpha}\}_{\alpha \in A}$ be a family of ideals in $R$. Let 
$$\sum_{\alpha \in A}I_{\alpha} = \biggl\{\sum_{\alpha\in A}r_{\alpha} \mid r_{\alpha} \in I_{\alpha} \text{ and } r_{\alpha} = 0_{R} \text{ for all but finitely many } \alpha \in A\biggl\}$$
Let $x = \sum_{\alpha \in A}r_{\alpha}, y = \sum_{\alpha}r_{\alpha}'$ be elements in $\sum_{\alpha \in A}I_{\alpha}$. We have that 
$$x - y = \sum_{\alpha\in A}r_{\alpha} - \sum_{\alpha \in A}r_{\alpha}' = \sum_{\alpha \in A}(r_{\alpha} - r_{\alpha}') \in \sum_{\alpha\in A}I_{\alpha}$$
as $r_{\alpha} - r_{\alpha}' \in I_{\alpha}$ for all $\alpha \in A$ and $r_{\alpha} - r_{\alpha}' = 0_{R}$ for all but finitely many $\alpha \in A$. Let $s \in R$ and $x = \sum_{\alpha \in A}r_{\alpha} \in \sum_{\alpha \in A}I_{\alpha}$. Then, 
$$sx = s\qty(\sum_{\alpha \in A}r_{\alpha}) = \sum_{\alpha \in A}sr_{\alpha} \in \sum_{\alpha \in A}I_{\alpha}$$
$$xs = \qty(\sum_{\alpha \in A}r_{\alpha})s = \sum_{\alpha \in A}sr_{\alpha} \in \sum_{\alpha \in A}I_{\alpha}$$
as $I_{\alpha}$ is an ideal for all $\alpha \in A$ and $sr_{\alpha} = 0_{R}, r_{\alpha}s = 0_{R}$ for all but finitely many $\alpha \in A$. Therefore, $\sum_{\alpha\in A}I_{\alpha}$ is an ideal in $R$. Let $J$ be an ideal of $R$ containing $I_{\alpha}$ for all $\alpha \in A$. Then, as $J$ is closed under addition, $J$ must contain $\sum_{\alpha \in A}r_{\alpha}$ where $r_{\alpha} = 0_{R}$ for all but finitely many $\alpha \in A$ and $r_{\alpha} \in I_{\alpha}$. Hence, $\sum_{\alpha \in A}I_{\alpha} \subseteq J$. It follows that $\sum_{\alpha \in A}I_{\alpha}$ is the smallest ideal containing each $I_{\alpha}$. 

\paragraph{2.} Let $\varphi:R \to S$ is a surjective ring homomorphism where $R$ is a Noetherian ring. Let $J$ be an ideal of $S$. We have that $I = \varphi^{-1}(J)$ is an ideal of $R$, and since $R$ is Noetherian, $I$ is finitely generated. Hence, $I = (a_{1}, ..., a_{n})$ for some $a_{1}, ..., a_{n} \in R$. Let $x \in J$. Then, there exists a $x' \in I$ such that $\varphi(x') = x$. We have that $x' = r_{1}a_{1} + ... + r_{n}a_{n}$ for some $r_{1},..., r_{n} \in R$. Then, $x = \varphi(x') = \varphi(r_{1}x_{1} + ... + r_{n}x_{n}) = \varphi(r_{1})\varphi(x_{1}) + ... + \varphi(r_{n})\varphi(x_{n})$. Hence, $J \subseteq (\varphi(x_{1}), ..., \varphi(x_{n}))$. Now, let $x \in (\varphi(x_{1}), ..., \varphi(x_{n}))$. We have that $x = s_{1}\varphi(x_{1}) + ... + s_{n}\varphi(x_{n})$. As $\varphi$ is surjective, for each $s_{i}$, there is an $r_{i} \in R$ such that $\varphi(r_{i}) = s_{i}$. We have that $x = \varphi(r_{1}x_{1} + ... + r_{n}x_{n})$. Note that $r_{1}x_{1} + ... + r_{n}x_{n} \in I$, and so $x \in J$. It follows that $J = (\varphi(x_{1}), ..., \varphi(x_{n}))$. Therefore, $J$ is finitely generated. It follows that $S$ is Noetherian. 

\paragraph{3.} Let $(2,x) \in \mathbb{Z}[x]$ be the ideal generated by $2$ and $x$ in $\mathbb{Z}[x]$. Suppose that $(2,x) = (f(x))$ for some $f \in \mathbb{Z}[x]$. As $2 \in (2,x)$, we must have that $2 = fg$ for some $g \in \mathbb{Z}[x]$. We have that $\text{deg}(fg) = \text{deg}(2) = 0$ and, as $\mathbb{Z}$ is an integral domain, $\text{deg}(f) + \text{deg}(g) = 0$. It follows that $\text{deg}(f) = \text{deg}(g) = 0$. Hence, $f(x) = a$ and $g(x) = b$ for some $a,b \in \mathbb{Z}$. We have that $2 = fg = ab$. This forces $f = \pm 1$ or $f = \pm 2$. If $f = 1$ or $f = -1$, then $(f) = \mathbb{Z}[x] \neq (2,x)$. If $f = 2$ or $f = -2$, then $(f)$ is the set of all polynomials with even coefficients. Thus, $x \not\in (f)$. In both cases, $(f) \neq (2,x)$. It follows that such an $f$ cannot exist. Therefore, $(2,x)$ is not principal in $\mathbb{Z}[x]$.  

\paragraph{4.} Let $k$ be a field and let $I$ be an ideal of $k[x]$. If $I = (0)$, then $I$ is principal. If $I \neq (0)$, then let $f \in I$ be a non-zero polynomial of minimal degree. Let $a$ be the leading coefficient of $f$. We have that $g = a^{-1}f$ is a monic polynomial. Let $h \in I$. Then, $h = gp + r$ for some $p,r \in k[x]$ where $\deg r < \deg g$. As $g \in I$, we have that $gp \in I$ and so $r = h - gp \in I$. By minimality of $g$, $r$ must be the zero polynomial. Hence, $h = gp$. It follows that $I = (g(x))$. Therefore, $k[x]$ is a principal ideal domain.  

\paragraph{5.} Let $I, J$ be ideal of $R$, where $R$ is a commutative ring, such that $I + J = (1_{R})$. Let $x \in I \cap J$. Then, $x \in I$ and $y \in J$. As $I + J = (1)$, we have that $1_{R} = i + j$ for some $i \in I$ and $j \in J$. Then, $x = x1_{R} = x(i + j) = xi + xj = ix + xj$. We have that $ix \in IJ$ and $xj \in IJ$ and so $xi + ix \in IJ$. Hence, $x \in IJ$. Therefore, $I \cap J \subseteq IJ$. It follows that $IJ = I \cap J$. 

\paragraph{6.} Let $R$ be a commutative ring and $I,J$ be ideals of $R$ such that $I \cap J \neq IJ$. Then, there exists an $x \in I \cap J$ such that $x \not\in IJ$. As $x \not\in IJ$, $x + IJ$ is not the zero element in $R/IJ$. As $x \in I \cap J$, we have that $x \in I$ and $x \in J$, hence, $x^{2} \in IJ$. Therefore, $(x + IJ)^{2} = x^{2} + IJ = IJ$. Thus, $R/IJ$ contains nilpotent elements. By taking the contrapositive statement, it follows that if $R/IJ$ is reduced, then $I \cap J = IJ$.

\paragraph{7.} Let $k$ be a field and $I$ an ideal of $k[x]$. Suppose that $I = (f(x)) = (g(x))$. Then, $f(x) = P(x)g(x)$ and $g(x) = Q(x)f(x)$. Hence, $f(x) = P(x)Q(x)g(x)$. It follows that $P(x)Q(x) = 1$. We must have that $P(x) = a$ and $Q(x) = a^{-1}$ for $a \in k$. Thus, $f(x) = ag(x)$. It follows that there is a unique monic polynomial that generates $I$. 

\paragraph{8.} Let $R$ be a ring and let $f \in R[x]$ be a monic polynomial. Let $g \in R[x]$ be a polynomial of degree $n \geq 0$. Let $a_{n}$ be the leading coefficient of $g$. Then, the leading coefficient of $fg$ is $a_{n} \neq 0$. We have that $\deg(fg) = \deg(f) + \deg(g)$. Let $h \in R[x]$ be a polynomial such that $fh = 0$. We have that $0 = \deg(0) = \deg(fh) = \deg(f) + \deg(h)$. It follows that $\deg(f) = 0$, so $f = 1$ as $f$ is a monic polynomial of degree $0$. As $fh = 0$, we must have that $0 = fh = 1h = h$. Hence, $h$ is the zero polynomial. Therefore, $f$ cannot be a left zero divisor. A similar argument can also show that $f$ cannot be a right zero divisor. 

\paragraph{9.}

\paragraph{10.} Let $d$ be a non-square integer. Let $\mathbb{Q}(\sqrt{d}) = \{x + y\sqrt{d} \mid x,y \in \mathbb{Z}\}$. We have that for each $x + y\sqrt{d} \in \mathbb{Q}(\sqrt{d})$, $x + y\sqrt{d} + (-x - y\sqrt{d}) = 0$. Let $x_{1} + y_{1}\sqrt{d}, x_{2} + y_{2}\sqrt{d} \in \mathbb{Q}(\sqrt{d})$. We have that 
$$(x_{1} + y_{1}\sqrt{d}) + (-x_{2} - y_{2}\sqrt{d}) = (x_{1} - x_{2}) + (y_{1} - y_{2})\sqrt{d} \in \mathbb{Q}(\sqrt{d})$$
$$(x_{1} + y_{1}\sqrt{d})(x_{2} + y_{2}\sqrt{d}) = x_{1}x_{2} + dy_{1}y_{2} + (x_{1}y_{2} + x_{2}y_{1})\sqrt{d} \in \mathbb{Q}(\sqrt{d})$$
Hence, $\mathbb{Q}(\sqrt{d})$ is a subring of $\mathbb{C}$. Define $N:\mathbb{Q}(\sqrt{d}) \to \mathbb{Q}$ by $N(x + y\sqrt{d}) = a^{2} - b^{2}d$. For $z = z_{1} + z_{2}\sqrt{d}, w = w_{1} + w_{2}\sqrt{d} \in \mathbb{Q}(\sqrt{d})$, we have that 
$$N(zw) = N((z_{1} + z_{2}\sqrt{d})(w_{1} + w_{2}\sqrt{d}) = N(z_{1}w_{1} + dz_{2}w_{2} + (z_{1}w_{2} + z_{2}w_{1})\sqrt{d}) = (z_{1}w_{1} + dz_{2}w_{2})^{2} - (z_{1}w_{2} + z_{2}w_{1})^{2}d$$
$$= (z_{1}^{2}w_{1}^{2} + 2z_{1}w_{1}z_{2}w_{2} + z_{2}^{2}w_{2}^{2}d^{2}) - (z_{1}^{2}w_{2}^{2} + 2z_{1}w_{2}w_{1}z_{2} + z_{2}^{2}w_{1}^{2})d = z_{1}^{2}w_{1}^{2} - z_{1}^{2}w_{2}^{2}d + z_{2}^{2}w_{2}^{2}d^{2} - z_{2}^{2}w_{1}^{2}d$$
$$=(z_{1}^{2} - z_{2}^{2}d)(w_{1}^{2} - w_{2}^{2}d) = N(z)N(w)$$
\paragraph{11.}
\paragraph{12.}
\paragraph{13.}
\paragraph{14.}

\paragraph{15.} Let $\varphi:R \to S$ be a homomorphism of commutative rings and let $I$ be a prime ideal of $S$. Let $x,y \in R$ such that $xy \in \varphi^{-1}(I)$. We have that $\varphi(x)\varphi(y) = \varphi(xy) \in I$. By primality of $I$, $\varphi(x) \in I$ or $\varphi(y) \in I$. Hence, $x \in \varphi^{-1}(I)$ or $y \in \varphi^{-1}(I)$. Therefore, $\varphi^{-1}(I)$ is a prime ideal of $R$. Let $(0) \subseteq \mathbb{Q}$. As $\mathbb{Q}$ is a field, $(0)$ is a maximal ideal. Let $i: \mathbb{Z} \to \mathbb{Q}$ be the inclusion map. We have that $i^{-1}((0)) = (0) \subseteq \mathbb{Z}$ is not maximal in $\mathbb{Z}$. Therefore, the inverse image of a maximal ideal is not necessarily maximal. 

\paragraph{16.} Let $R$ be a commutative ring and let $P$ be a prime ideal of $R$. Suppose that $0$ is the only zero-divisor of $R$ contained in $P$. Let $x,y \in R$ such that $xy = 0$. As $xy = 0 \in P$, we have that $x = 0$ or $y = 0$. By assumption, $x = 0$ or $y = 0$. Therefore, $R$ is an integral domain. 

\paragraph{17.} Let $K$ be a compact topological space and $R$ the ring of all real-valued functions on $K$, with addition and multiplication defined pointwise. 

\subparagraph{(i)} For $p \in K$, defined $M_{p} = \{f \in R \mid f(p) = 0\}$. Define $\varphi:R \to \mathbb{R}$ by $\varphi(f) = f(p)$. We have that $M_{p}$ is precisely the kernel of this map. For any $r \in \mathbb{R}$, we have that $f:K \to \mathbb{R}$ defined by $f(x) = r$ is continuous, hence, $f \in R$. Thus, $\varphi$ is surjective. By the first isomorphism theorem, $R/M_{p} \cong \mathbb{R}$, and so $R/M_{p}$ is a field. Therefore, $M_{p}$ is a maximal ideal. 

\subparagraph{(ii)} Let $n > 1$ and let $f_{1}, ..., f_{n} \in R$ such that they have no common zeros. Define $g = f_{1}^{2} + ... + f_{n}^{2}$. We have that $g$ has no zeros in $K$ and $g \in (f_{1}, ..., f_{n})$, the ideal generated by $f_{1}, ..., f_{n}$. As $g$ has no zeros, $1/g \in R$. Thus, $1 = g(1/g) \in (f_{1}, ..., f_{n})$. Therefore, $(1) = (f_{1}, ..., f_{n})$

\subparagraph{(iii)} 

\paragraph{18.} Let $R$ be a commutative ring and $N$ be the nilradical of $R$. Let $P$ be a prime ideal of $R$. Let $x \in N$. Then, $x^{n} = 0_{R}$ for some $n \in \mathbb{N}$. As $P$ is an ideal, $0_{R} \in P$. Then, $x^{n-1}x \in P$. As $P$ is prime, $x^{n-1} \in P$ or $x \in P$. If $x \in P$, then we are done. If $x^{n-1} \in P$, then $x = x^{2-n}x^{n-1} \in P$ as $P$ is an ideal. Therefore, $N \subseteq P$. 

\paragraph{19.} 
\subparagraph{(i)} Let $R$ be a commutative ring with prime ideal $P$. Let $I,J$ be ideals of $R$ such that $IJ \subseteq P$. Without loss of generality, suppose that $J \nsubseteq P$. Then, there is an $x \in J$ such that $x \notin P$. Let $y \in I$. We have that $yx \in IJ \subseteq P$, and so $y \in P$ or $x \in P$. This forces $y \in P$. Therefore, $I \subseteq P$. Now, suppose that $I_{1}, ..., I_{n}$ are ideals of $R$ such that $I_{1}...I_{n} \subseteq P$. Then, $I_{k} \subseteq P$ or $I_{1}...I_{k-1}I_{k+1}...I_{n} \subseteq P$. By repeating this argument, we have that $I_{k} \subseteq P$ for some $k$. 

\subparagraph{(ii)} Consider the ideals $I_{n} = (n)$ of $\mathbb{Z}$. We have that $\bigcap_{n=1}^{\infty}I_{n} = (0) \subseteq (0)$, however, $(0)$ does not contain any of $I_{n}$. 

\paragraph{20.} Let $M$ be a two-sided ideal in a ring $R$. Suppose $M$ is maximal. We have that the only ideals containing $M$ are $R$ and $M$ itself. Hence, the only ideals of $R/M$ are $R/M$ and $M/M$ as there are a bijection between ideals containing $M$ and ideals of $R/M$. It follows $R/M$ is simple. For the converse, suppose that $R/M$ is simple. Again, as there are a bijection between ideals containing $M$ and ideals of $R/M$, the only ideals containing $M$ are $R$ and $M$ itself, so the maximality of $M$ follows. 

\paragraph{21.}Let $k$ be an algebraically closed field and let $I$ be an ideal of $k[x]$. Suppose that $I$ is maximal. As $k$ is a field, it is a PID, and so $k[x]$ is a PID. Then, $I = (f(x))$ for some $f \in k[x]$. We note that $f$ is not a constant polynomial, otherwise $I = k[x]$ as $k$ is a field. There then exists an $a \in k$ such that $f(a) = 0_{k}$ as $a$ is algebraically closed. Hence, $f(x) = g(x)(x-a)$ for some $g \in k[x]$. We have that $f \in (x - a)$ and so $I \subseteq (x-a)$. By maximality, $I = (x-a)$. For the converse, suppose that $I = (x-a)$. By Proposition 4.6, $k[x]/I \cong k$. As $k$ is a field, $I$ is maximal. 

\paragraph{22.} We have that $\mathbb{R}[x]/(x^{2}+1) \cong \mathbb{R} \oplus \mathbb{R} \cong \mathbb{C}$ by Proposition 4.6. As $\mathbb{C}$ is a field, $(x^{2}+1)$ is a maximal ideal. 

\paragraph{23.}
\paragraph{24.} Consider the following chain of ideals of $\mathbb{Z}[x]$
$$(0) \subset (x) \subset (2,x) \subset \mathbb{Z}[x]$$
We have that $(x)$ is a prime ideal as $\mathbb{Z}[x]/(x) \cong \mathbb{Z}$ is an integral domain. Aswell as that, $(2,x)$ is a prime ideal as 
$$\mathbb{Z}[x]/(2,x) \cong \frac{\mathbb{Z}[x]/(x)}{(2,x)/(x)} \cong \frac{\mathbb{Z}}{(2)} \cong \mathbb{Z}/2\mathbb{Z}$$
is also an integral domain. Therefore, the Krull dimension of $\mathbb{Z}[x]$ is atleast $2$. 

\subsection*{3.5 - Modules Over a Ring}

\paragraph{1.}
\paragraph{2.}

\paragraph{3.} Let $M$ be a module over a ring $R$. For all $m \in M$, we have that $0_{R}\cdot m = (0_{R} + 0_{R})\cdot m = 0_{R}\cdot m + 0_{R} \cdot m$. Hence, $0_{R}\cdot m = 0_{M}$. Furthermore, $(-1) \cdot m + m = (-1)\cdot m + 1\cdot m = (-1 + 1)\cdot m = 0_{R}\cdot m = 0$. Hence, $(-1)\cdot m = -m$. 

\paragraph{4.} Let $M, N$ be simple $R$-modules and let $\varphi:M \to N$ be a homomorphism of $R$-modules. As $M$ is simple, $\ker \varphi = 0$ or $\ker \varphi = M$. If $\ker \varphi = M$, then immediately $\varphi = 0$. Suppose $\ker\varphi = 0$. As $N$ is simple, $\im{\varphi} = 0$ or $\im{\varphi} = N$. If $\im{\varphi} = 0$, then $\varphi = 0$ immediately. Suppose that $\im{\varphi} = N$. Then, $\varphi$ is surjective. Let $x,y \in M$ such that $\varphi(x) = \varphi(y)$. Then, $\varphi(x) - \varphi(y) = 0_{N}$ and so $\varphi(x-y) = 0_{N}$. We have that $x-y \in \ker\varphi$ and so $x-y = 0_{M}$. Hence, $x=y$. Therefore, $\varphi$ is injective. It follows that $\varphi$ is bijective and, thus, an isomorphism. 

\paragraph{5.} Let $R$ be a commutative ring, viewed as a module over itself. Define $\varphi:M \to \text{Hom}_{R\text{-}\mathsf{Mod}}(R,M)$ by $\varphi(m) = \lambda_{m}$ where $\lambda_{m}:R \to M$ is defined by $\lambda_{m}(x) = x\cdot m$. We have that for all $m,n \in M$ and $r \in R$,
$$\varphi(m+n) = \lambda_{m+n} = x\cdot (m+n) = x\cdot m + x \cdot n = \lambda_{m} + \lambda_{n} = \varphi(m) + \varphi(n)$$
$$\varphi(r\cdot m) = \lambda_{r\cdot m} = x\cdot(r \cdot m) = (xr) \cdot m = (rx) \cdot m = r(x\cdot m) = r\lambda_{m} = r\varphi(m)$$
Suppose that $\sigma \in \text{Hom}_{R\text{-}\mathsf{Mod}}(R,M)$. Then, for all $x \in R$, $\sigma(x) = \sigma(x \cdot 1) = x\sigma(1) = \lambda_{\sigma(1)}$. Hence, $\varphi$ is surjective. Suppose that $\lambda_{m}, \lambda_{n} \in \text{Hom}_{R\text{-}\mathsf{Mod}}(R,M)$ such that $\lambda_{m} = \lambda_{n}$ for all $x \in R$. By setting $x = 1$, we obtain $m = n$. Hence, $\varphi$ is injective. It follows that $\varphi$ is an isomorphism.   

\paragraph{6.} Let $G$ be an abelian group with the structure of a $\mathbb{Q}$-vector space. We have that the inclusion homomorphism $i:\mathbb{Z} \to \mathbb{Q}$ is an epimorphism. Suppose that there exists homomorphisms $\sigma_{1},\sigma_{2}:\mathbb{Q} \to \text{End}_{\mathsf{Ab}}(G)$. We have that $(\sigma_{1} \circ i)(x) = \sigma_{1}(x) = \sum_{x}\sigma_{1}(1) = \sum_{x}1_{\text{End}_{\mathsf{Ab}}(G)}$ and $(\sigma_{2} \circ i)(x) = \sigma_{2}(x) = \sum_{x}\sigma_{2}(1) = \sum_{x}1_{\text{End}_{\mathsf{Ab}}(G)}$. Hence, $\sigma_{1} \circ i = \sigma_{2} \circ i$. As $i$ is an epimorphism, $\sigma_{1} = \sigma_{2}$. As a module is determined by such a homomorphism, the $\mathbb{Q}$-vector space structure on $G$ is unique. 

\paragraph{7.}

\paragraph{8.}
\paragraph{9.}
\paragraph{10.}
\paragraph{11.}

\paragraph{12.} Let $R$ be a ring and let $M,N$ be $R$-modules. Let $\varphi:M \to N$ be a homomorphism of modules such that it is bijective as a set function. For $n_{1}, n_{2} \in N$, we have that there exists $m_{1}, m_{2} \in M$ such that $\varphi(m_{1}) = n_{1}$ and $\varphi(m_{2}) = n_{2}$. We have that $\varphi(m_{1} + m_{2}) = \varphi(m_{1}) + \varphi(m_{2}) = n_{1} + n_{2}$. Then, $\varphi^{-1}(n_{1}) + \varphi^{-1}(n_{2}) = m_{1} + m_{2} = \varphi^{-1}(n_{1} + n_{2})$. Furthermore, let $r \in R$ and $n \in N$. Then, there is a unique $m \in M$ such that $\varphi(m) = n$. As $\varphi$ is a homomorphism, we have that $\varphi(rm) = r\cdot \varphi(m) = r\cdot n$, and so $r\varphi^{-1}(n) = rm = \varphi^{-1}(rn)$. It follows that $\varphi^{-1}$ is a module homomorphism. 

\paragraph{13.} Let $R$ be an integral domain, and let $I = (a)$ be a non-zero principal ideal of $R$. Define $\varphi: R \to I$ by $\varphi(x) = ax$. We have that
$$\varphi(m+n) = a(m+n) = am + an = \varphi(m) + \varphi(n)$$
$$\varphi(rm) = a(rm) = (ar)m = (ra)m = r(am) = r\varphi(m)$$
Hence, $\varphi$ is an $R$-module homomorphism. If $\varphi(b) = 0$ for some $b \in R$, then $ab = 0$. As $a$ is assumed non-zero, $b = 0$ as $R$ is an integral domain. Hence, the kernel of the homomorphism is trivial. Let $x \in I$. Then, $x = ax'$ for some $x' \in R$. Hence, $\varphi$ is surjective. By the first isomorphism theorem, $R \cong I$. 

\paragraph{14.} Let $N, P$ be submodules of an  $R$-module $M$. We have that $N + P$ is a subgroup of $M$. Let $r \in R$ and $x = n + p \in N + P$. Then, $rx = r(n + p) = rn + rp \in N + P$ as $rn \in N$ and $rp \in P$. Hence, $N + P$ is a submodule of $M$. Furthermore, we have that $N \cap P$ is a submodule of $P$. Indeed, $N \cap P$ is a subgroup of $P$ and if $r \in R$ and $x \in N \cap P$, then $rx \in N \cap P$ as $rx \in N$ and $rx \in P$ given that $N$ and $P$ are submodules of $M$. Define the map $\varphi:P \to N+P/N$ by $\varphi(p) = pN$. We have that 
$$\varphi(x + y) = (x+y)N = xN + yN = \varphi(x) + \varphi(y)$$
$$\varphi(r\cdot x) = (r\cdot x)N = r\cdot(xN) = r\cdot \varphi(x)$$
Therefore, $\varphi$ is an $R$-module homomorphism. Next, we have that 
$$\ker\varphi = \{p \in P \mid \varphi(p) = 1_{N+P/N}\} = \{p \in P \mid pN = N\} = \{p \in P \mid p \in N\} = P \cap N$$
Hence, by the first isomorphism theorem, $P/P \cap N \cong (N+P)/N$. 

\paragraph{15.}

\paragraph{16.} Let $R$ be a commutative ring, $M$ an $R$-module, and let $a \in R$ be a nilpotent element in $R$. Suppose that $M = 0$. Then, $aM = 0 = M$. For the converse, suppose that $aM = M$. Let $m \in M$. We have that there exists an $m'$ such that $m = am'$. Then, there exists an $m''$ such that $m = a(am'') = a^{2}m''$. For each $k \in \mathbb{N}$, there is some $n \in M$ such that $m = a^{k}n$. As $a$ is nilpotent, it follows that $m = 0$. Therefore, $M$ is the trivial module. 

\paragraph{17.}
\paragraph{18.}

\subsection*{3.6 - Products, Coproducts, etc., In $R\text{-}\mathsf{Mod}$}

\paragraph{1.} Let $A$ be a set and $R$ a ring. Let $\alpha \in R^{\oplus A}$. Then, $\alpha(x) = \sum_{a\in A}\alpha(a)j_{a}(x)$. Hence, $\alpha(x) \in \langle j_{a} \mid a \in A\rangle$. We have that $R^{\oplus A} = \langle j_{a} \mid a \in A\rangle$. Define $j:A \to R^{\oplus A}$ by $j(a) = j_{a}$. Let $f:A \to M$ be a set function, where $M$ is an $R$-module, and let $\varphi:R^{\oplus A} \to M$ be an $R$-module homomorphism such that $\varphi \circ j = f$. For each $a \in A$, we have tht $f(a) = (\varphi \circ j)(a) = \varphi(j_{a})$. Hence, for any $\sum_{a \in A}r_{a}j_{a} \in R^{\oplus}$, we have that $\varphi(\sum_{a \in A}r_{a}j_{a}) = \sum_{a\in A}r_{a}\varphi(j_{a}) = \sum_{a\in A}r_{a}f(a)$. We have that $\varphi$ is unique. Therefore, $R^{\oplus A}$ satisfies the universal property for the free $R$-module generated by $A$. Thus, $R^{\oplus A} \cong F^{R}(A)$

\paragraph{2.}
\paragraph{3.} Let $R$ be a ring, $M$ an $R$-modul, and $p:M \to M$ an $R$-module homomorphism such that $p^{2} = p$. Define $\varphi:M \to \ker p \oplus \im{p}$ by $\varphi(m) = (m - p(m), p(m))$. We have that $\varphi(m) \in \ker p \oplus \im{p}$ for all $m \in M$ as $p(m - p(m)) = p(m) - p(p(m)) = p(m) - p(m) = 0$, hence, $m - p(m) \in \ker p$ and $p(m) \in \im{p}$. Furthermore, 
$$\varphi(m+n) = (m+n - p(m+n), p(m+n)) = (m+n-p(m) - p(n),p(m) + p(n)) = (m - p(m), p(m)) + (n - p(n), p(n))$$
$$= \varphi(m) + \varphi(n)$$
$$\varphi(r\cdot m) = (r\cdot m - p(r \cdot m), p(r \cdot m)) = (r\cdot m - r\cdot p(m), r\cdot p(m)) = (r\cdot ( m - p(m)), r\cdot p(m)) = r\cdot(m-p(m), p(m)) = r\cdot \varphi(m)$$
for all $r \in R$ and $m,n \in M$. Thus, $\varphi$ is an $R$-module homomorphism. Next, we have that 
$$\ker \varphi = \{m \in M \mid \varphi(m) = 0\} = \{m \in M \mid m - p(m) = 0 \text{ and } p(m) = 0\} = \{m \in M \mid m = 0\} = \{0\}$$
Hence, $\varphi$ is injective by Proposition 6.2. Let $(x,y) \in \ker p \oplus \im{p}$. Then, $p(x) = 0$ and there is a $z \in M$ such that $p(z) = y$. We have that 
$$\varphi(x+p(z)) = (x+p(z)-p(x+p(z)), p(x+p(z))) = (x + p(z) - p(x) - p(p(z)), p(x) + p(p(z)))$$
$$=(x + p(z) - 0 - p(z), 0 + p(z)) = (x, y)$$
Hence, $\varphi$ is surjective. We have that $\varphi$ is a bijective $R$-module homomorphism. Therefore, $M \cong \ker p \oplus \im{p}$. 

\paragraph{4.} Let $R$ be a ring and let $n > 1$. View $R^{\oplus (n-1)}$ as a submodule of $R^{\oplus n}$ via the homomorphism $R^{\oplus (n-1)} \to R^{\oplus n}$ defined by $(r_{1}, ..., r_{n-1}) \mapsto (r_{1}, ..., r_{n-1},0)$. Define $\varphi:R^{\oplus n} \to R$ by $\varphi(r_{1}, ..., r_{n}) = r_{1}$. We have that $\varphi$ is a surjective homomorphism and $\ker \varphi \cong R^{\oplus (n-1)}$. Hence, $R^{\oplus n}/R^{\oplus (n-1)} \cong R$ by the first isomorphism theorem.

\paragraph{5.}

\paragraph{6.} Let $R$ be a commutative ring and let $F = R^{\oplus n}$ be a finitely generated free $R$-module. Let $\lambda \in \text{Hom}_{R\text{-}\mathsf{Mod}}(F,R)$. Note, for all $\textbf{x} \in F$, we have that 
\begin{equation}
    \lambda(\textbf{x}) = \lambda\qty(\sum_{i=1}^{n}x_{i}j_{i}) = \sum_{i=1}^{n}x_{i}\lambda(j_{i}) \tag{\ast}
\end{equation}
where $j_{i} = (0,...,1,...,0)$ where $1 \in R$ is placed in the $i$th position. Define $\varphi:F \to \text{Hom}_{R\text{-}\mathsf{Mod}}(F,R)$ by sending $\textbf{r} \in F$ to the homomorphism sending $j_{i}$ to $r_{i}$. $\varphi$ is clearly surjective by $(\ast)$. For all $\textbf{x} \in F$, we have that for all $\textbf{r}, \textbf{s} \in F$ and $a \in R$, 
$$\varphi(\textbf{r} + \textbf{s}) = \sum_{i=1}^{n}x_{i}(r_{i} + s_{i}) = \sum_{i=1}^{n}(x_{i}r_{i} + x_{i}s_{i}) = \sum_{i=1}^{n}x_{i}r_{i} + \sum_{i=1}^{n}x_{i}s_{i} = \varphi(\textbf{r}) + \varphi(\textbf{s})$$
$$\varphi(a\textbf{r}) = \sum_{i=1}^{n}x_{i}ar_{i} = \sum_{i=1}^{n}ax_{i}r_{i} = a\sum_{i=1}^{n}x_{i}r_{i} = a\varphi(\textbf{r})$$
Thus, $\varphi$ is a homomorphism. Let $\textbf{r} \in F$ such that $\varphi(\textbf{r}) = 0$. Then, 
$$\varphi(\textbf{r}) = 0 \implies \forall \textbf{x} \in F, \  \sum_{i=1}^{n}x_{i}r_{i} = 0 \implies \textbf{r} = 0$$
Thus, $\varphi$ is injective. It follows that $\varphi$ is an isomorphism. Therefore, $\text{Hom}_{R\text{-}\mathsf{Mod}}(F,R) \cong F$. View $\mathbb{Q}$ as  a $\mathbb{Z}$-module. Let $\psi \in \text{Hom}_{\mathbb{Z}\text{-}\mathsf{Mod}}(\mathbb{Q},\mathbb{Z})$. We have that $\psi(1) = 2^{n}\psi(1/2^{n})$ for all $n \in \mathbb{N}$. Hence, $2^{n} \mid \psi(1)$ for all $n \in \mathbb{N}$. Therefore, $\psi(1) = 0$. Let $x \in \mathbb{Q}$. We have that $\psi(x) = x\psi(1) = x0 = 0$. Thus, $\psi$ is the trivial homomorphism. We must have that $\text{Hom}_{\mathbb{Z}\text{-}\mathsf{Mod}}(\mathbb{Q},\mathbb{Z}) = 0$. 

\paragraph{7.}

\paragraph{8.}

\paragraph{9.} Let $R$ be a ring, $F$ a non-zero free $R$-module, and let $\varphi:M \to N$ be an $R$-module homomorphism. Suppose that $\varphi$ is onto. Let $\alpha:F \to N$ be an $R$-module homomorphism. As $F$ is free, $F \cong R^{\oplus A} = \langle j_{a} \mid a \in A \rangle$ for some set $A$. For each $\alpha(a) \in N$, as $\varphi$ is onto, there is some $m_{a} \in M$ such that $\varphi(m_{a}) = \alpha(j_{a})$. Define $\beta:F \to M$ by $\beta(\sum_{a \in A}r_{a}j_{a}) = \sum_{a\in A}r_{a}m_{a}$. We have that $\beta$ is a well-defined $R$-module homomorphism such that for all $x \in F$
$$(\varphi \circ \beta)(x) = \varphi\qty(\beta\qty(x)) = \varphi\qty(\beta\qty(\sum_{a\in A}x_{a}j_{a})) = \varphi\qty(\sum_{a\in A}x_{a}m_{a}) = \sum_{a\in A}x_{a}\alpha(j_{a}) = \alpha\qty(\sum_{a\in A}x_{a}j_{a}) = \alpha(x)$$
For the converse, suppose that for all homomorphisms $\alpha:F \to N$, there exists a $\beta:F \to M$ such that $\alpha = \varphi \circ \beta$. Let $x \in N$. Define $\alpha:F \to N$ sending some $j_{a}$ to $x$ and sending all other $j_{a}$ to $0$. We have that for some $a \in A$, $x = \alpha(j_{a}) = (\varphi \circ \beta)(j_a)$, hence, $x \in \im{\varphi}$. Thus, $\varphi$ is onto.

\paragraph{10.} Let $M, N$ and $Z$ be $R$-modules and let $\mu:M \to Z, \nu:N \to Z$ be $R$-module homomorphisms. Let $\varphi:M \oplus N \to Z$ be a map defined by $\varphi((m,n)) = \mu(m) - \nu(n)$. We have that for all $(m,n), (m',n') \in M \oplus N$ and $r \in R$, 
\begin{align*} \varphi((m,n) + (m',n')) &= \varphi((m+m',n+n')) \\
&= \mu(m+m') - \nu(n+n')\\
&= \mu(m) + \mu(m') - \nu(n) - \nu(n')\\ 
&= \mu(m) - \nu(n) + \mu(m') - \nu(n')\\
&= \varphi((m,n)) + \varphi((m',n'))
\end{align*}
\begin{align*}
\varphi(r(m,n)) &= \varphi((rm,rn)) \\
&= \mu(rm) - \nu(rn)\\
&= r\mu(m) - r\nu(n)\\
&= r(\mu(m) - \nu(n))\\
&= r\varphi((m,n))
\end{align*}
Hence, $\varphi$ is an $R$-module homomorphism. Define
$$M \times_{Z} N = \ker\varphi = \{(m,n) \in M \oplus N \mid \mu(m) - \nu(n) = 0\} = \{(m,n) \in M \oplus N \mid \mu(m) = \nu(n)\}$$
We note that $M \times_{Z} N$ is a submodule of $M \oplus N$ as it is a kernel of a homomorphism, hence, an $R$-module. Let $\pi_{M}:M \times_{Z} N \to M$ be the projection map restricted to $M \times_{Z} N$ and $\pi_{N}:M \times_{Z} N \to N$ be the projection map restricted to $M\times_{Z} N$. We note both are $R$-module homomorphisms as they are restrictions of $R$-module homomorphisms to a submodule. Let $P$ be an $R$-module and $\varphi_{M}:P \to M, \varphi_{N}:P \to N$ be $R$-module homomorphisms such that $\nu\varphi_{N} = \mu\varphi_{M}$. Let $(m,n) \in M \times_{Z} N$. Then, $\nu\pi_{N}(m,n) = \nu(n) = \mu(m) = \mu\pi_{M}(m,n)$. Hence, $\nu\pi_{N} = \mu\pi_{M}$. We now set to prove there exists a unique $R$-module homomorphism $\psi$ such that the following diagram commutes,
\[\begin{tikzcd}[ampersand replacement=\&]
	P \\
	\& {M \times_{Z} N} \& N \\
	\& M \& Z
	\arrow["{\pi_{M}}"', from=2-2, to=3-2]
	\arrow["{\pi_{N}}", from=2-2, to=2-3]
	\arrow["\mu"', from=3-2, to=3-3]
	\arrow["\nu", from=2-3, to=3-3]
	\arrow["{\varphi_{M}}"', curve={height=12pt}, from=1-1, to=3-2]
	\arrow["{\varphi_{N}}", curve={height=-12pt}, from=1-1, to=2-3]
	\arrow["\psi", from=1-1, to=2-2]
\end{tikzcd}\]
We have that $\pi_{N}\psi = \varphi_{N}$ and $\pi_{M}\psi = \varphi_{M}$. Hence, $\psi = (\varphi_{N}, \varphi_{M})$. We have that $\psi$ is an $R$-module homomorphism as $\varphi_{N}, \varphi_{M}$ are $R$-module homomorphisms. We also note $\psi$ is unique. The claim follows. Therefore, $R\text{-}\mathsf{Mod}$ has fibered products. 

\paragraph{11.} Let $M,N$ and $Z$ be $R$-modules and let $\mu:Z \to M, \nu:Z \to N$ be $R$-module homomorphisms. Let $I = \langle (\mu(x),-\nu(x)) \mid x \in Z\rangle$ be an ideal of $M \oplus N$ generated by elements of the form $(\mu(x), -\nu(x))$. Define $M \oplus_{Z} N$ to be the quotient module $(M \oplus N)/ I$. Define $i_{N}:N \to M \oplus_{Z} N$ by $i_{N}(n) = (0,n) + I$ and $i_{M}:M \to M \oplus_{Z} N$ by $i_{M}(m) = (m,0) + I$. Note that for any $x \in Z$, 
\begin{align*}(i_{M}\circ \mu - i_{N} \circ \nu)(x) &= i_{M}(\mu(x)) - i_{N}(\nu(x)) \\
&= ((\mu(x), 0) + I) - ((0, \nu(x)) + I) \\
&= (\mu(x), 0) - (0,\nu(x)) + I \\
&= (\mu(x), -\nu(x)) + I \\
&= I
\end{align*}
Therefore, $i_{M}\circ \mu = i_{N} \circ \nu$. Let $P$ be a $R$-module and let $\varphi_{N}:N \to P$ and $\varphi_{M}:M \to P$ be homomorphisms such that $\varphi_{N}\circ \nu = \varphi_{M}\circ \mu$. Suppose the following diagram commutes for some homomorphism $\psi:M \oplus_{Z} N \to P$,
\[\begin{tikzcd}[ampersand replacement=\&]
	P \\
	\& {M \oplus_{Z}N} \& N \\
	\& M \& Z
	\arrow["\nu"', from=3-3, to=2-3]
	\arrow["\mu", from=3-3, to=3-2]
	\arrow["{i_{M}}", from=3-2, to=2-2]
	\arrow["{i_{N}}"', from=2-3, to=2-2]
	\arrow["{\varphi_{M}}", curve={height=-12pt}, from=3-2, to=1-1]
	\arrow["{\varphi_{N}}"', curve={height=12pt}, from=2-3, to=1-1]
	\arrow["\psi"', from=2-2, to=1-1]
\end{tikzcd}\]
For all $x \in N$, we have that $\varphi_{N} = \psi\circ i_{N}$ so $\varphi_{N}(x) = \psi((0,x) + I)$. For all $x \in M$, we have that $\varphi_{M} = \psi \circ i_{M}$ so $\varphi_{M}(x) = \psi((x,0) + I)$. Hence, 
\begin{align*}
\psi((m,n) + I) &= \psi((m,0) + I + (0,n) + I) \\
&= \psi((m,0) + I) + \psi((0,n) + I) \\
&= \varphi_{M}(m) + \varphi_{N}(n)
\end{align*}
$\psi$ is the a unique $R$-module homomorphism for which the diagram commutes. It follows $R\text{-}\mathsf{Mod}$ has fibered coproducts. 

\paragraph{12.}

\paragraph{13.} Let $M, N$ be $R$-modules and let $\varphi:M \to N$ be a surjective homomorphism. Suppose further that $M$ is finitely generated. Then, $M = \langle m_{1}, ..., m_{n}\rangle$ for some $m_{1}, ..., m_{n} \in M$. Let $x \in N$. As $\varphi$ is surjective, there is some $x' \in M$ such that $\varphi(x') = x$. We have that $x' = \sum_{i=1}^{n}r_{i}m_{i}$ for some $r_{i} \in R$. Thus, 
$$x = \varphi(x') = \varphi\qty(\sum_{i=1}^{n}r_{i}m_{i}) = \sum_{i=1}^{n}r_{i}\varphi(m_{i})$$
Therefore, $x \in \langle \varphi(m_{1}), ..., \varphi(m_{n})\rangle$. It follows that $N$ is finitely generated. 

\paragraph{14.} Suppose that $(x_{1}, x_{2}, ...)$ is finitely generated as a submodule of $\mathbb{Z}[x_{1}, x_{2}, ...]$. Then, $(x_{1}, x_{2}, ...) = \langle a_{1}, a_{2}, ..., a_{n}\rangle$ for some $a_{1}, a_{2}, ..., a_{n} \in \mathbb{Z}[x_{1}, x_{2}, ...]$. We have that each $a_{i}$ is a finite polynomial, and so, there exists an $x_{k}$ such that the coefficient of $x_{k}$ and all $x_{l}$ for all $l > k$ is $0$. There is then a maximum $x_{N}$ among all $a_{i}$, which means that $x_{N} \not\in \langle a_{1}, a_{2}, ..., a_{n}\rangle$. Therefore, $(x_{1}, x_{2}, ...)$ cannot be finitely generated. 

\paragraph{15.}

\paragraph{16.} Let $R$ be a ring and let $M$ be a simple $R$-module. Let $x \in M$ such that $x \neq 0$. We have that $\langle x \rangle$ is a submodule of $M$. As $M$ is simple, $\langle x \rangle = M$ as $\langle x \rangle$ contains $x \neq 0$. Thus, $M$ is cyclic. For the next part, suppose that $M$ is an $R$-module that is cyclic. We have that $M = \langle x \rangle$ for some $x \in M$. View $R$ as a module over itself and define the $R$-module homomorphism $\varphi:R \to M$ by $\varphi(r) = rx$. $\varphi$ is clearly surjective and so $M \cong R/I$, where $I = \ker \varphi$, by the first isomorphism theorem. For the converse, suppose that $M \cong R/I$ for some ideal $I$ of $R$. We have that $\langle 1 + I\rangle$ is a submodule of $R/I$. Let $x + I \in R/I$. Then, $x + I = x\cdot (1 + I) \in \langle 1 + I \rangle$. It follows that $R/I = \langle 1 + I\rangle$. Therefore, $M$ is cyclic. Finally, let $M$ be a cyclic module with submodule $N$. We have that $M \cong R/I$ for some ideal $I$ and so $N \cong S/I$ for some submodule $S$ of $R$. Then, $M/N \cong (R/I)/(S/I) \cong R/S$. Hence, $M/N$ is cyclic. 

\paragraph{17.} 
\subparagraph{(i)} Let $M$ be a cyclic $R$-module so that $M \cong R/I$ for some ideal $I$, and let $N$ be another $R$-module. Denote $\{n \in N \mid \forall a \in I, an = 0\}$ by $\text{Ann}_{I}(N)$. We note this is a submodule of $N$. Let $\lambda \in \text{Hom}_{R\text{-}\mathsf{Mod}}(R/I, N)$. Then, $\lambda(1 + I) \in \text{Ann}_{I}(N)$ as, for all $a \in I$, we have that $a\cdot \lambda(1 + I) = \lambda(a\cdot(1 + I)) = \lambda(a + I) = \lambda(I) = 0$. Define $\varphi:\text{Hom}_{R\text{-}\mathsf{Mod}}(R/I, N) \to \text{Ann}_{I}(N)$ by sending $\lambda \in \text{Hom}_{R\text{-}\mathsf{Mod}}(R/I, N)$ to $\lambda(1 + I)$. For $\lambda_{1}, \lambda_{2} \in \text{Hom}_{R\text{-}\mathsf{Mod}}(R/I, N)$, we have that $\varphi(\lambda_{1} + \lambda_{2}) = (\lambda_{1} + \lambda_{2})(1 + I) = \lambda_{1}(1 + I) + \lambda_{2}(1 + I) = \varphi(\lambda_{1}) + \varphi(\lambda_{2})$. Additionally, for $r \in R$ and $\lambda \in \text{Hom}_{R\text{-}\mathsf{Mod}}(R/I, N)$, we have that $\varphi(r\cdot \lambda) = (r\cdot \lambda)(1 + I) = r\cdot \lambda(1 + I) = r\cdot \varphi(\lambda)$. Therefore, $\varphi$ is an $R$-module homomorphism. Let $\lambda_{1}, \lambda_{2} \in \text{Hom}_{R\text{-}\mathsf{Mod}}(R/I, N)$ such that $\varphi(\lambda_{1}) = \varphi(\lambda_{2})$. Then, $\varphi(\lambda_{1} - \lambda_{2}) = 0$. Hence, $(\lambda_{1} - \lambda_{2})(1 + I) = 0$. For all $x + I \in R/I$, it follows that $(\lambda_{1} - \lambda_{2})(x + I) = (\lambda_{1} - \lambda_{2})(x \cdot (1 + I)) = x(\lambda_{1} - \lambda_{2})(1 + I) = x0 = 0$. This forces $\lambda_{1} = \lambda_{2}$ and injectivity of $\varphi$ follows. Let $n \in \text{Ann}_{I}(N)$. Define the homomorphism $\sigma:R \to N$ by $\sigma(r) = rn$. Let $i \in I$. Then, $\sigma(i) = in = 0$ as $n \in \text{Ann}_{I}(N)$. Hence, $I \subseteq \ker \sigma$. By Theorem 5.14, there exists a unique homomorphism $\overline{\sigma}:R/I \to N$ such that $\overline{\sigma}\pi = \sigma$ where $\pi: R \to R/I$ is the canonical projection. We have that $\varphi(\overline{\sigma}) = \overline{\sigma}(1 + I) = \overline{\sigma}\pi(1) = \sigma(1) = 1\cdot n = n$. $\varphi$ must then be surjective. Therefore, $\varphi$ is an isomorphism. We conclude $\text{Hom}_{R\text{-}\mathsf{Mod}}(M, N) \cong \text{Ann}_{I}(N)$. 

\subparagraph{(ii)} 

\paragraph{18.} Let $M$ be an $R$-module and $N$ a submodule such that $M/N$ and $N$ are finitely generated. As $M/N$ and $N$ are finitely generated, $M/N = \langle m_{1} + N, ..., m_{l} + N\rangle$ for $m_{1},...,m_{l} \in M$ for some $l \in \mathbb{N}$ and $N = \langle n_{1}, ..., n_{k} \rangle$ for $n_{1},...,n_{k} \in N$ for some $k \in \mathbb{N}$. Let $x \in M$. We have that
$$x + N = \sum_{i=1}^{l}x_{i}(m_{i} + N) = \sum_{i=1}^{l}(x_{i}m_{i} + N) =\qty(\sum_{i=1}^{l}x_{i}m_{i}) + N$$
for some $x_{1},...,x_{l} \in R$. Hence, $x - \sum_{i=1}^{l}x_{i}m_{i} \in N$. Thus, 
$$x - \sum_{i=1}^{l}x_{i}m_{i} = \sum_{j=1}^{k}y_{j}n_{j}$$
for some $y_{1},...y_{k} \in R$. Therefore, 
$$x = \sum_{i=1}^{l}x_{i}m_{i} + \sum_{j=1}^{k}y_{j}n_{j}$$
Hence, $x \in \langle m_{1}, ..., m_{l}, n_{1}, ..., n_{k}\rangle$. It follows that $M = \langle m_{1}, ..., m_{l}, n_{1}, ..., n_{k}\rangle$ and so $M$ is finitely generated. 

\subsection*{3.7 - Complexes and Homology}
\paragraph{1.} Let $M$ be an $R$-module such that 
$$\cdots \longrightarrow 0 \overset{d}{\longrightarrow} M \overset{d'}{\longrightarrow} 0 \longrightarrow \cdots$$
is exact. As $d:0 \to M$ is a homomorphism, we must have that $d(0) = 0_{M}$, hence, $d$ is the trivial homomorphism. As $d':M \to 0$ is a homomorphism from $M$ to $0$, we must have that every element of $M$ must be sent to the zero element, hence, $d'$ is also the trivial homomorphism. We have that $d:M \to 0$ is a map with a left and right inverse, namely $d'$, and is also a homomorphism. Therefore, $d$ is an isomorphism. Thus, $M \cong 0$. 

\paragraph{2.} Let $M, M'$ be $R$-modules and $\varphi:M\to M'$ a homomorphism such that
$$\cdots \longrightarrow 0 \longrightarrow M \overset{\varphi}{\longrightarrow} M' \longrightarrow 0 \longrightarrow \cdots$$
is exact. By exactness, $\ker\varphi = 0$ and $\im{\varphi} = M'$. Hence, $\varphi$ is a bijection. It follows that $\varphi$ is an isomorphism, thus, $M \cong M'$. 

\paragraph{3.} Let $M_{\bullet}$ be the complex 
$$\cdots 0 \longrightarrow L \overset{\psi}{\longrightarrow} M \overset{\varphi}{\longrightarrow} M' \overset{\psi'}{\longrightarrow} N \longrightarrow 0 \longrightarrow \cdots$$
Suppose $M_{\bullet}$ is exact. We have that $\ker\psi' = \im{\varphi}$ and $\psi'$ is surjective. By the first isomorphism theorem, $N \cong M'/\ker\psi' = M'/\im{\varphi} \cong \text{coker }\varphi$. Additionally, $\psi$ is injective and $\psi:L \to \ker\varphi$ is a surjective homomorphism as $\im{\psi} = \ker\varphi$. Therefore, $L \cong \ker\varphi$. 

\paragraph{4.}

\paragraph{5.} Assume that the complex
$$\cdots \longrightarrow L \overset{\psi}{\longrightarrow} M \overset{\varphi}{\longrightarrow} N \longrightarrow \cdots$$
is exact with $L,N$ Noetherian. Let $M'$ be a submodule of $M$. We have that $\varphi(M')$ is a submodule of $N$ and is finitely generated by assumption. Suppose $\varphi(M') = \langle \varphi(x_{1}), ..., \varphi(x_{n})\rangle$ for some $x_{1}, ..., x_{n} \in M'$. Let $x \in M'$. Then, $\varphi(x) = \sum_{i=1}^{n}r_{i}\varphi(x_{i})$ for some $r_{1}, ..., r_{n} \in R$. We have that
$$0 = \varphi(x) - \sum_{i=1}^{n}r_{i}\varphi(x_{i}) = \varphi(x) - \varphi\qty(\sum_{i=1}^{n}r_{i}x_{i}) = \varphi\qty(x - \sum_{i=1}^{n}r_{i}x_{i})$$
Thus, $x - \sum_{i=1}^{n}r_{i}x_{i} \in \ker\varphi = \im{\psi}$ by exactness at $M$. We have that $\psi^{-1}(M')$ is a submodule of $L$ and is finitely generated as $L$ is Noetherian. Then, $\psi^{-1}(M') = \langle y_{1}, ..., y_{m}\rangle$ for some $y_{1}, ..., y_{m} \in L$. As $x - \sum_{i=1}^{n}r_{i}x_{i} \in \im{\psi}$, we have that there is some $y \in \psi^{-1}(M')$ such that $\psi(y) = x - \sum_{i=1}^{n}r_{i}x_{i} \in \ker\varphi = \im{\psi}$. We then have that 
$$x - \sum_{i=1}^{n}r_{i}x_{i} = \psi(y) = \psi\qty(\sum_{j=1}^{m}r_{i}'y_{i}) = \sum_{j=1}^{m}r_{i}'\psi(y_{i})$$
for some $r'_{1}, ..., r'_{m} \in R$. Therefore, 
$$x = \sum_{i=1}^{n}r_{i}x_{i} + \sum_{j=1}^{m}r_{i}'\psi(y_{i})$$
It follows that $M' = \langle x_{1}, ..., x_{n}, \psi(y_{1}),..., \psi(y_{n})\rangle$, hence, $M$ is Noetherian.

\paragraph{6.}

\paragraph{7.} \subparagraph{(i)} Let 
$$0 \longrightarrow M \overset{\varphi}{\longrightarrow} N \overset{\psi}{\longrightarrow} P \longrightarrow 0$$
be a short exact sequence of $R$-modules, and let $L$ be an $R$-module. Define $f:\text{Hom}_{R\text{-}\mathsf{Mod}}(P,L) \to \text{Hom}_{R\text{-}\mathsf{Mod}}(N,L)$ by $f(\lambda) = \lambda \circ \psi$. We note that $\psi$ is an epimorphism. Let $\lambda \in \ker f$. Then, $\lambda \circ \psi = 0$. As $\psi$ is an epimorphism, we have that $\lambda = 0$. Thus, $f$ is a monomorphism. Define $g:\text{Hom}_{R\text{-}\mathsf{Mod}}(N,L) \to \text{Hom}_{R\text{-}\mathsf{Mod}}(M,L)$ by $g(\lambda) = \lambda \circ \varphi$. For any $\lambda \in \text{Hom}_{R\text{-}\mathsf{Mod}}(P,L)$, we have that $(g \circ f)(\lambda) = g(f(\lambda)) = g(\lambda \circ \psi) = \lambda \circ \psi \circ \varphi = \lambda \circ 0 = 0$. Now, let $\sigma \in \ker g$. Then, $\sigma \circ \varphi = 0$. We have that there exists a unique $\alpha:P \to L$ such that $\sigma = \alpha \circ \psi$. Hence, $\sigma \in \im f$. It follows that the chain complex 
$$0 \longrightarrow \text{Hom}_{R\text{-}\mathsf{Mod}}(P,L) \overset{f}{\longrightarrow} \text{Hom}_{R\text{-}\mathsf{Mod}}(N,L) \overset{g}{\longrightarrow}  \text{Hom}_{R\text{-}\mathsf{Mod}}(M,L)$$
is exact. 

\subparagraph{(ii)} 
\subparagraph{(iii)} 
\subparagraph{(iv)} 

\paragraph{8.} Let 
$$0 \longrightarrow M \overset{\varphi}{\longrightarrow} N \overset{\psi}{\longrightarrow} F \longrightarrow 0$$
be a short exact sequence of $R$-modules with $F$ free. We have that $\psi:N\to F$ is surjective as a set function, hence, by a previous exercise, there exists a homomorphism $\beta:F \to N$ such that $\text{id}_{F} = \psi \circ \beta$. Therefore, $\psi$ has a right inverse. Note that $M \cong M/0 \cong M/\ker\varphi \cong \im{\varphi} = \ker\psi$. Hence, by Proposition 7.5, the exact sequence must split. 

\paragraph{9.} 

\paragraph{10.} Suppose the following diagram commutes with both rows exact and $\nu, \lambda$ are isomorphisms:
\[\begin{tikzcd}[ampersand replacement=\&]
	0 \& {L_{1}} \& {M_{1}} \& {N_{1}} \& 0 \\
	0 \& {L_{0}} \& {M_{0}} \& {N_{0}} \& 0
	\arrow[from=1-1, to=1-2]
	\arrow[from=2-1, to=2-2]
	\arrow["{\alpha_{0}}", from=2-2, to=2-3]
	\arrow["{\beta_{0}}", from=2-3, to=2-4]
	\arrow[from=2-4, to=2-5]
	\arrow["\nu", from=1-4, to=2-4]
	\arrow["\mu", from=1-3, to=2-3]
	\arrow["\lambda", from=1-2, to=2-2]
	\arrow["{\alpha_{1}}", from=1-2, to=1-3]
	\arrow["{\beta_{1}}", from=1-3, to=1-4]
	\arrow[from=1-4, to=1-5]
\end{tikzcd}\]
As $\lambda, \nu$ are isomorphisms, $\ker\lambda, \ker\nu, \coker\lambda, \coker\nu$ are all trivial. By the snake lemma, there is an exact sequence
$$0 \longrightarrow 0 \longrightarrow \ker\mu \longrightarrow 0 \overset{\delta}{\longrightarrow} 0 \longrightarrow \coker\mu \longrightarrow 0 \longrightarrow 0$$
where $\delta$ is the connecting homomorphism (although in this case it is the trivial homomorphism). By exactness, $\ker\mu \cong 0$ and $\coker\mu \cong 0$. Therefore, $\mu$ is an isomorphism. 

\paragraph{11.} Let
$$0 \longrightarrow M_{1} \longrightarrow N \longrightarrow M_{2} \longrightarrow 0$$
be an exact sequence of $R$-modules. Suppose there exists an $R$-module homomorphism $\varphi: N \to M_{1} \oplus M_{2}$ such that the diagram 
\[\begin{tikzcd}[ampersand replacement=\&]
	0 \& {M_{1}} \& N \& {M_{2}} \& 0 \\
	0 \& {M_{1}} \& {M_{1} \oplus M_{2}} \& {M_{2}} \& 0
	\arrow[from=1-4, to=2-4]
	\arrow[from=1-2, to=2-2]
	\arrow["\varphi", from=1-3, to=2-3]
	\arrow[from=1-2, to=1-3]
	\arrow[from=1-3, to=1-4]
	\arrow[from=2-2, to=2-3]
	\arrow[from=2-3, to=2-4]
	\arrow[from=2-1, to=2-2]
	\arrow[from=1-1, to=1-2]
	\arrow[from=1-4, to=1-5]
	\arrow[from=2-4, to=2-5]
\end{tikzcd}\]
commutes, where the bottom row is the standard sequence of a direct sum and the morphisms $M_{1} \to M_{1}, M_{2} \to M_{2}$ are the identity maps. By the short five lemma, $\varphi$ is necessarily an isomorphism. Therefore, 
$$0 \longrightarrow M_{1} \longrightarrow N \longrightarrow M_{2} \longrightarrow 0$$
must split. 

\paragraph{12.} Suppose that the following is a commutative diagram of $R$-modules with exact rows, $\alpha$ is an epimorphism, and $\beta, \delta$ are monomorphisms:
\[\begin{tikzcd}[ampersand replacement=\&]
	{A_{1}} \& {B_{1}} \& {C_{1}} \& {D_{1}} \\
	{A_{0}} \& {B_{0}} \& {C_{0}} \& {D_{0}}
	\arrow["\alpha"', two heads, from=1-1, to=2-1]
	\arrow["\beta"', hook', from=1-2, to=2-2]
	\arrow["\gamma"', from=1-3, to=2-3]
	\arrow["\delta"', hook', from=1-4, to=2-4]
	\arrow["{f_{1}}", from=1-1, to=1-2]
	\arrow["{f_{2}}", from=1-2, to=1-3]
	\arrow["{f_{3}}", from=1-3, to=1-4]
	\arrow["{g_{1}}"', from=2-1, to=2-2]
	\arrow["{g_{2}}"', from=2-2, to=2-3]
	\arrow["{g_{3}}"', from=2-3, to=2-4]
\end{tikzcd}\]
We prove that $\gamma$ is a monomorphism. Let $x \in \ker\gamma$. By commutativity of the diagram, $\delta \circ f_{3} = g_{3} \circ \gamma$, hence, $(\delta \circ f_{3})(x) = (g_{3} \circ \gamma)(x) = g_{3}(\gamma(x)) = g_{3}(0) = 0$. As $\delta$ is a monomorphism, $f_{3}(x) = 0$. Thus, $x \in \ker f_{3} = \im f_{2}$. There then exists a $y \in B_{1}$ such that $f_{2}(y) = x$. By commutativity of the diagram, we have that $g_{2} \circ \beta = \gamma \circ f_{2}$. Thus, $(g_{2} \circ \beta)(y) = (\gamma \circ f_{2})(y) = \gamma(f_{2}(y)) = \gamma(x) = 0$. We must have that $\beta(y) \in \ker g_{2} = \im g_{1}$. Hence, there is some $z \in A_{0}$ such that $g_{1}(z) = \beta(y)$. As $\alpha$ is an epimorphism, there is a $w \in A_{1}$ such that $\alpha(w) = z$. Therefore, $\beta(y) = g_{1}(z) = (g_{1} \circ \alpha)(w) = (\beta \circ f_{1})(w)$ by commutativity of the diagram again. As $\beta$ is a monomorphism, $y = f_{1}(w)$. Finally, $x = f_{2}(y) = (f_{2} \circ f_{1})(w) = 0$ by exactness at $B_{1}$. We deduce $\ker \gamma = 0$, thus making $\gamma$ a monomorphism. 

\paragraph{13.}
Suppose that the following is a commutative diagram of $R$-modules with exact rows, $\epsilon$ is a monomorphism, and $\beta, \delta$ are epimorphisms:
\[\begin{tikzcd}[ampersand replacement=\&]
	{B_{1}} \& {C_{1}} \& {D_{1}} \& {E_{1}} \\
	{B_{0}} \& {C_{0}} \& {D_{0}} \& {E_{0}}
	\arrow["\beta"', two heads, from=1-1, to=2-1]
	\arrow["\gamma"', from=1-2, to=2-2]
	\arrow["\delta"', two heads, from=1-3, to=2-3]
	\arrow["\epsilon"', hook', from=1-4, to=2-4]
	\arrow["{f_{2}}", from=1-1, to=1-2]
	\arrow["{f_{3}}", from=1-2, to=1-3]
	\arrow["{f_{4}}", from=1-3, to=1-4]
	\arrow["{g_{2}}"', from=2-1, to=2-2]
	\arrow["{g_{3}}"', from=2-2, to=2-3]
	\arrow["{g_{4}}"', from=2-3, to=2-4]
\end{tikzcd}\]
We prove that $\gamma$ is an epimorphism. Let $x \in C_{0}$. As $\delta$ is an epimorphism, there exists a $y \in D_{1}$ such that $\delta(y) = g_{3}(x)$. By the commutativity of the diagram, $g_{4} \circ \delta = \epsilon \circ f_{4}$. Then, $(\epsilon \circ f_{4})(y) = (g_{4} \circ \delta)(y) = g_{4}(\delta(y)) = g_{4}(g_{3}(x)) = 0$ due to exactness at $D_{0}$. As $\epsilon$ is a monomorphism, it follows $f_{4}(y) = 0$. Hence, $y \in \ker f_{4} = \im f_{3}$ by exactness. There then exists a $z \in C_{1}$ such that $f_{3}(z) = y$. Define $x' = x - \gamma(z)$. We have that $g_{3}(x') = g_{3}(x - \gamma(z)) = g_{3}(x) - g_{3}(\gamma(z)) = g_{3}(x) - \delta(f_{3}(z)) = g_{3}(x) - \delta(y) = g_{3}(x) - g_{3}(x) = 0$. Thus, $x - \gamma(z) \in \ker g_{3} = \im g_{2}$ by exactness. There is a $y' \in B_{0}$ such that $g_{2}(y') = x'$. As $\beta$ is an epimorphism, there is a $z' \in B_{1}$ such that $\beta(z') = y'$. By commutativity of the diagram, $g_{2} \circ \beta = \gamma \circ f_{2}$. Hence, $(\gamma \circ f_{2})(z') = (g_{2} \circ \beta)(z') = g_{2}(\beta(z')) = g_{2}(y') = x'$. Hence, $(\gamma \circ f_{2})(z') = x' = x - \gamma(z)$. Therefore, $x = (\gamma \circ f_{2})(z') + \gamma(z) = \gamma(f_{2}(z') + z)$. We conclude that $x \in \im \gamma$. It follows that $\gamma$ is an epimorphism. 

\paragraph{14.} Suppose the following diagram commutes with $\alpha$ an epimorphism, $\epsilon$ a monomorphism, $\beta,\delta$ isomorphisms and with both rows exact:
\[\begin{tikzcd}[ampersand replacement=\&]
	{A_{1}} \& {B_{1}} \& {C_{1}} \& {D_{1}} \& {E_{1}} \\
	{A_{0}} \& {B_{0}} \& {C_{0}} \& {D_{0}} \& {E_{0}} \\
	\&\& {}
	\arrow["\alpha"', two heads, from=1-1, to=2-1]
	\arrow["\beta"', from=1-2, to=2-2]
	\arrow["\gamma"', from=1-3, to=2-3]
	\arrow["\delta"', from=1-4, to=2-4]
	\arrow[from=1-1, to=1-2]
	\arrow[from=1-2, to=1-3]
	\arrow[from=1-3, to=1-4]
	\arrow[from=2-1, to=2-2]
	\arrow[from=2-2, to=2-3]
	\arrow[from=2-3, to=2-4]
	\arrow[from=1-4, to=1-5]
	\arrow[from=2-4, to=2-5]
	\arrow["\epsilon"', hook', from=1-5, to=2-5]
\end{tikzcd}\]
By using the two versions of the four-lemma, $\gamma$ is an epimorphism and is a monomorphism. Thus, $\gamma$ is an isomorphism. 

\paragraph{15.} 
% https://q.uiver.app/#q=WzAsMjEsWzEsMCwiMCJdLFsyLDAsIjAiXSxbMywwLCIwIl0sWzQsMSwiMCJdLFs0LDIsIjAiXSxbNCwzLCIwIl0sWzAsMSwiMCJdLFswLDIsIjAiXSxbMCwzLCIwIl0sWzEsNCwiMCJdLFsyLDQsIjAiXSxbMyw0LCIwIl0sWzEsMSwiTF97Mn0iXSxbMiwxLCJNX3syfSJdLFszLDEsIk5fezJ9Il0sWzEsMiwiTF97MX0iXSxbMiwyLCJNX3sxfSJdLFszLDIsIk5fMSJdLFsxLDMsIkxfMCJdLFsyLDMsIk1fMCJdLFszLDMsIk5fMCJdLFs4LDE4XSxbMTgsOV0sWzE4LDE5LCJcXG51XzEiXSxbMTksMjAsIlxcbnVfezB9Il0sWzIwLDVdLFsxOSwxMF0sWzIwLDExXSxbNywxNV0sWzE1LDE2LCJcXG11X3sxfSJdLFsxNiwxNywiXFxtdV8wIl0sWzE1LDE4XSxbMTYsMTldLFsxNywyMF0sWzE3LDRdLFsxMiwxNV0sWzEzLDE2XSxbMTQsMTddLFs2LDEyXSxbMCwxMl0sWzEsMTNdLFsyLDE0XSxbMTIsMTMsIlxcbGFtYmRhXzEiXSxbMTMsMTQsIlxcbGFtYmRhXzAiXSxbMTQsM11d


\paragraph{16.}
\paragraph{17.}

\section*{IV - Groups, second encounter}
\subsection*{4.1 - The Conjugation Action}
\paragraph{1.} Let $p$ be a prime integer, and let $G$ be a $p$ group. Let $S$ be a finite set such that $p \nmid |S|$. Suppose that $G$ acts on the set $S$. Let $Z$ denote the set of fixed points on the action. By Corollary 1.3, $|Z| \equiv |S| \mod p$. Hence, $|Z| \not\equiv 0 \mod p$ by assumption. Therefore, $|Z| \neq 0$. We must have that the action has fixed points. 

\paragraph{2.} Let $D_{2n}$ be the dihedral group of order $n$. For $n = 1$ and $n = 2$, $D_{2n}$ has order less than $5$, hence, $D_{2n}$ is abelian. Thus, $Z(D_{2n}) = D_{2n}$. Let $n \geq 3$. Note $D_{2n} = \langle x,y \mid x^{2} = y^{n} = 1, yx = xy^{-1}\rangle$. Let $z \in Z(D_{2n})$. Then, $zx = xz$ and $zy = yz$. We have that $z = x^{i}y^{j}$ for some $i,j$ aswell as $D_{2n}$ is generated by $x,y$. Then, 
$$zy = yz \implies x^{i}y^{j}y = yx^{i}y^{j} \implies x^{i}y^{j+1} = yx^{i}y^{j} \implies x^{i}y = yx^{i}$$
If $i = 1$, then $xy = yx = xy^{-1}$, hence, $y^{2} = 1$, which is not possible. Hence, $i = 0$. Then, $z = y^{j}$ for some $j$. Then, 
$$zx = xz \implies y^{j}x = xy^{j} \implies xy^{-j} = xy^{j} \implies y^{2j} = 1$$
Hence, $n \mid 2j$. Thus, $j = n/2$ or $j = 0$. We have that $z = 1$ or $z = y^{n/2}$. If $n$ is odd, then $Z(D_{2n}) = \langle y^{n/2}\rangle$ and if $n$ is even, then $Z(D_{2n})$ is trivial. 


\paragraph{3.} Let $S_{n}$ be the group of permutations on the set $[n] = \{1,2,..., n\}$ with $n \geq 3$. Let $\tau \in S_{n}$ be a permutation sending $l$ to $k$ where $k \neq l$. Let $m \in [n]$ such that $m \neq l, k$. Let $\sigma \in S_{n}$ be the permutation solely swapping $m$ and $k$. We have that $\sigma\tau(l) = \sigma(k) = m$ and $\tau\sigma(l) = \tau(l) = k$. Then, for any non-trivial permutation of $S_{n}$, we can find a permutation such that they do not commute. Hence, $Z(S_{n})$ is trivial. 

\paragraph{4.} Let $G$ be a group, and $N$ a subgroup of the center of $G$, $Z(G)$. Let $x \in N$ and $g\in G$. As $N$ is a subgroup of $Z(G)$, $x$ is an element of $Z(G)$. Hence, $xg = gx$ for all $g \in G$. We have that $gxg^{-1} = xgg^{-1} = x \in N$. Therefore, $N$ is normal in $G$.

\paragraph{5.} Define the homomorphism $\varphi:G \to \text{Inn}(G)$ by $\varphi(g) = \lambda_{g}$ where $\lambda_{g}(x) = gxg^{-1}$. Note $\varphi$ is surjective. We have that 
$$\ker\varphi = \{g \in G \mid \lambda_{g} = \lambda_{1}\} = \{g \in G \mid \forall x \in G, gxg^{-1} = x\} = \{g \in G \mid \forall x \in G, gx = xg\} = Z(G)$$
By the first isomorphism theorem, $G/Z(G) \cong \text{Inn}(G)$. Let $G$ be a finite group, and assume $G/Z(G)$ is cyclic. We then have that $\text{Inn}(G)$ is cyclic. By a previous exercise, $\text{Inn}(G)$ is cyclic if and only if $G$ is abelian. Hence, $G$ is commutative. 

\paragraph{6.} Let $p,q$ be prime integers and let $G$ be a group of order $pq$. Suppose that $Z(G)$ is not trivial. Then, $|Z(G)|$ can either be $p,q$ or $pq$. If $|Z(G)|$ is either $p$ or $q$, then $|G/Z(G)|$ is prime, hence, $\text{Inn}(G)$ is cyclic, which makes $G$ abelian. If $Z(G) = G$, then $G$ is abelian straight away. We deduce that a group of order $pq$ is either commutative or has a nontrivial center. Suppose now $p = q$ so that $G$ has order $p^{2}$. By Corollary 1.9, $G$ has a nontrivial center being a $p$-group. Hence, $G$ is commutative.

\paragraph{7.} We have that $Q_{8}$, the Quarternion group, is a group of order $2^{3}$, however, it is not abelian. 

\paragraph{8.} Let $G$ be a group of order $p^{r}$ with $p$ prime. Suppose that $G$ is abelian. Then, with using a previous exercise, $G$ contains elements of order $p^{k}$ for every $k \leq r$, hence, contains normal subgroups of order $p^{k}$ for every $k \leq r$. Note that if $G$ is a group of order $p$, $G$ contains subgroups of order $1$ and $p$, namely, $\{1_{G}\}$ and $G$ itself, respectively. Let $n \in \mathbb{N}$, and suppose that for every group, $G$, of order $p^{r} < p^{n}$, $G$ contains a normal subgroup of order $p^{k}$ for every $k \leq r$. Let $G$ be a group of order $p^{n}$. If $G$ is abelian, then $G$ contains a subgroup of order $p^{k}$ for every $k \leq n$. Suppose that $G$ is not abelian. Then, $1 < Z(G) < p^{n}$ is a normal subgroup of $G$ of order $p^{k}$ for some $1 \leq k < n$. Under the hypothesis, $Z(G)$ contains a normal subgroup of order $p$, $H$ say. Then, $G/H$ is a group of order $p^{n-1}$. Under the hypothesis, $G/H$ contains normal subgroups of order $1, p, p^{2}, ..., p^{n-1}$. Hence, $G$ contains subgroups of order $1, p, p^{2}, p^{3}, ..., p^{n}$. By the principle of strong mathematical induction, $G$, being a group of order $p^{n}$, contains normal subgroups of order $p^{k}$ for every $k \leq n$.

\paragraph{9.} Let $p$ be a prime number and $G$ a $p$-group. Let $H$ be a non-trivial normal subgroup of $G$. Let $G$ act on $H$ via conjugation. Let $Z$ be the set of all fixed points under this action. We have that 
$$Z = \{x \in H \mid \forall g \in G, gxg^{-1} = x\} = \{x \in H \mid \forall g \in G, gx = xg\} = Z(G) \cap H$$
By Corollary 1.3, $|Z(G) \cap H| = |Z| \equiv |H| \mod p$. By Lagranges Theorem, $|H| = 0 \mod p$, hence, $|Z(G) \cap H| \equiv 0 \mod p$. As $1_{G} \in H \cap Z(G)$, $H \cap Z(G) \geq p$. 

\paragraph{10.} Let $G$ be a group of odd order and let $g \in G$ be a nontrivial element such that $g$ is conjugate to $g^{-1}$. If $g = g^{-1}$, then $g^{2} = 1$, so $2 = |g|$ divides $|G|$, which cannot occur. Suppose that $g \neq g^{-1}$. As $|G|$ is odd, the conjugacy class containing $g$ and $g^{-1}$ must contain some other element $h \in G$. We have that $g$ is conjugate to $h$, hence, there is some $x \in G$ such that $g = xhx^{-1}$. Thus, $g^{-1} = xh^{-1}x^{-1}$. $g^{-1}$ must be conjuagte to $h^{-1}$. For every $x \in [g]$, $x^{-1} \in [g]$. Therefore, $|C(g)|$ is even, which cannot happen. We must have that $g$ is the identity element. 

\paragraph{11.} 

\paragraph{12.}

\paragraph{13.} Let $G$ be a noncommutative group of order $6$. Suppose that $G$ does not have an element of order $3$. Then, as $G$ is not abelian, it cannot be cyclic, so it cannot have an element of order $6$. Hence, every element of $G$ must have order $2$ or $1$. However, by a previous exercise, $G$ is then abelian. Hence, $G$ must contain an element of order $3$. Let $y$ be such an element. As $\langle y \rangle$ has index $2$ in $G$, it must be normal. Let $[y]$ be the conjugacy class of $y$. We must have that $[y]$ has length $2$ or $3$ as $y \not\in Z(G)$ as $Z(G)$ is trivial. $\langle y \rangle$ is the union of conjugacy classes and it contains $1$. $\langle y \rangle$ also must contain $[y]$ as $y \in [y]$. As $[y]$ must have length $2$ or $3$, it is forced to have length $2$ and $y^{2}$ is also forced to be an element of $[y]$. Thus, $[y] = \{y, y^{2}\}$. As $y$ and $y^{2}$ are in the same conjugacy class, $y = xy^{2}x^{-1}$ for some $x \in G$. Then, there is an $x \in G$ such that $yx = xy^{2}$. We note that $x \neq y, y^{2}, 1$. Suppose that $x$ has order $3$. Then, $\langle x \rangle$ is normal, hence is the union of conjugacy classes. However, it is impossible for $\langle x \rangle$ to be the union of conjugacy classes as $[x]$ must be of length $3$ and cannot contain the identity. Thus, a contradiction. $x$ must be of order $2$. So far we know that $G$ contains $1,x,y,y^{2}$. We have that 
$$xy = xyy^{3} = xy^{2}y^{2} = yxy^{2} = y^{2}x$$
$$xy^{2} = yx$$
$$yx = yx$$
$$y^{2}x = y^{2}x$$
The other two elements must be $y^{2}x$ and $yx$. Note that these two elements cannot possibly equal $1,x,y,y^{2}$. Hence, $x,y$ generate $G$. We have that
$$G = \langle x,y \mid x^{2} = y^{3} = 1, xy = y^{2}x\rangle$$
Hence, $G \cong S_{3}$. 

\paragraph{14.} Let $G$ be a group and assume $[G:Z(G)] = n$ is finite. Let $A \subseteq G$ be a subset of $G$. We have that $\text{Inn}(G) \cong G/Z(G)$, hence, $|\text{Inn}(G)| = n$. Let $\lambda_{g} \in \text{Inn}(G)$ be conjugation under $g$. We have that there exists $g_{1}, ... g_{n} \in G$ such that $\lambda_{g_{1}}, ..., \lambda_{g_{n}}$ are all distinct and for all $g \in G$, $\lambda_{g} = \lambda_{g_{i}}$ for some $i$.  Let $g \in G$, we have that $gAg^{-1} = \lambda_{g}(A)$. Then, the set of all $gAg^{-1}$ is atmost $n$ as $\text{Inn}(G) = n$. 

\paragraph{15.} Let $G$ be a group with class formula $60 = 1 + 15 + 20 + 12 + 12$. Let $N$ be a normal subgroup of $G$. We have that $N$ is the union of the conjugacy classes of its elements. We have that $N$ contains the identity element, and all conjugacy classes are disjoint, hence, $|N| = 1 + ...$ where the rest of the terms are chosen from $\{15,20,12\}$ with multiplicty of $12$ accounted for. We cannot form any divisors of $60$ of this form except $1$ and $60$. Hence, the only possible normal subgroups of $G$ are $1$ and $G$, which are normal already. 

\paragraph{16.}

\paragraph{17.} Let $H$ be a proper subgroup of a finite group $G$. By Corollary 1.14, there are atmost $[G:H]$ conjugates of $H$, each with $|H|$ elements. We note that each conjugate of $H$ contains the identity. Hence, 
$$\qty|\bigcup_{g \in G}gHg^{-1}| \leq [G:H]|H| - [G:H] + 1 = |G| - [G:H] + 1 < |G|$$
as $H$ is a proper subgroup of $G$. Therefore, $G$ cannot be the union of conjugates of $H$. 

\paragraph{18.}

\paragraph{19.} Let $H$ be a proper subgroup of a finite group $G$. By a previous exercise, $G$ is not the union of conjugates of $H$. Hence, there is some $x \in G$ not contained in any conjugate of $H$. Then, for all $g \in G$, $ghg^{-1} \neq x$ for any $h \in H$. Hence, for all $g \in G$, $gxg^{-1} \neq h$ for any $h$. Therefore, $[x]$ is disjoint from $H$. 

\paragraph{20.}

\paragraph{21.} Let $H, K$ be subgroup of a group $G$ with $H \subseteq N_{G}(K)$. Let $\lambda_{h}:K \to K$ be conjugation by $h$. Note $\lambda_{h}(K) = hKh^{-1} = K$ as $h \in H \subseteq N_{G}(K)$. Define $\gamma:H \to \text{Aut}_{\mathsf{Grp}}(K)$ by $\gamma(h) = \lambda_{h}$. Let $x,y \in K$. We have that $\gamma(xy) = \lambda_{xy} = \lambda_{x} \circ \lambda_{y} = \gamma(x)\gamma(y)$. Hence, $\gamma$ is a group homomorphism. Furthermore, 
$$\ker\gamma = \{h \in H \mid \lambda_{h} = \lambda_{1}\} = \{h \in H \mid \forall x \in K, hxh^{-1} = x\} = \{h \in H \mid \forall x \in K, hx = xh\} = H \cap Z_{G}(K)$$

\paragraph{22.} Let $G$ be a finite group, and let $H$ be a cylic subgroup of $G$ of order $p$ where $p$ is the smallest prime dividing the order of $G$. Suppose further $H$ is normal in $G$. We note that $\text{Aut}_{\mathsf{Grp}}(H) \cong \text{Aut}_{\mathsf{Grp}}(\mathbb{Z}/p\mathbb{Z}) \cong \mathbb{Z}/(p-1)\mathbb{Z}$. By the previous exercise, there is a homomorphism $\gamma: G \to \mathbb{Z}/(p-1)\mathbb{Z}$ with kernel $Z_{G}(H)$. Let $x \in G$ be a non-trivial element. Then, the order of $x$ is greater than or equal to $p$ as $p$ is the smallest prime dividing $G$. As $\gamma$ is a homomorphism, the order of $\gamma(x) \in \mathbb{Z}/(p-1)\mathbb{Z}$ must divide the order of $x$. Every element in $\mathbb{Z}/(p-1)\mathbb{Z}$ has order less than or equal to $p-1$, hence, the order of $\gamma(x)$ must be $1$. Therefore, $\gamma$ is the trivial morphism. Hence, $\ker \gamma = G$. It follows that $Z_{G}(H) = G$. For all $h \in H$ and $g \in G$, $ghg^{-1} = h$. Then, $H$ is a subgroup of $Z(G)$. 

\subsection*{4.2 - The Sylow Theorems}
\paragraph{1.} With notation given in the proof of the Cauchy's Theorem, we have that $|Z| \equiv 0 \mod p$. Hence, there are $kp$ fixed elements. For every element of order $p$, the generated subgroup contains $p-1$ generators. Let $N$ be the number of subgroups of order $p$. We have that $kp = |\{1_{G}\} \cup \{x_{1}, ..., x_{1}^{p-1}\} \cup ... \cup \{x_{m}, ..., x_{m}^{p-1}\}| = 1 + N(p-1)$. Hence, $kp = 1 + Np - N$ so $N = 1 + Np - kp$. Therefore, $N \equiv 1 \mod p$. 

\paragraph{2.} 
\subparagraph{(i)}
Let $G$ be a group and suppose that $H$ is a characteristic subgroup of $G$. We have that $\varphi_{g}:G \to G$ defined by conjugation by $g \in G$ is an automorphism of $G$. Hence, for any $g \in G$, $gHg^{-1} = \varphi_{g}(H) \subseteq H$. Therefore, $H$ is normal. 

\subparagraph{(ii)} Let $H \subseteq K \subseteq G$ such that $H$ is characteristic in $K$ and $K$ is normal in $G$. Let $g \in G$. As $K$ is normal in $G$, $gKg^{-1} \subseteq K$. Define $\varphi_{g}:K \to K$ to be conjugation by $g \in G$. As $H$ is characteristic in $K$, $gHg^{-1} = \varphi_{g}(H) \subseteq H$. Therefore, $H$ is normal in $G$. 

\subparagraph{(iii)} Let $G,K$ be groups such that $G$ contains a single subgroup $H$ isomorphic to $K$. For any $g \in G$, $gHg^{-1}$ is isomorphic to $H$ via conjugation by $g$, and is also a subgroup of $G$. By assumption, we must have that $H = gHg^{-1}$ for all $g \in G$. Therefore, $H$ is normal in $G$. 

\subparagraph{(iv)} Let $K$ be a normal subgroup of a finite group $G$ such that $|K|$ and $|G/K|$ are relatively prime. Let $\varphi \in \text{Aut}_{\mathsf{Grp}}(G)$ and let $x \in \varphi(K)$. Let $\pi:G \to G/K$ be the natural projection. We have that the order of $\pi(x)$ must divide the order of $G/K$, and we also must have that the order of $\pi(x)$ divides the order of $x$. As $x \in \varphi(K)$, there is a $y \in K$ such that $\varphi(y) = x$. Then, the order of $x$ divides the order of $y$, which divides the order of $K$ by Lagranges Theorem. Thus, $\pi(x)$ divides both $|K|$ and $|G/K|$. As $|K|$ and $|G/K|$ are relatively prime, $\pi(x)$ must be $1$. Hence, $\pi(x) = 1$ for all $x \in \varphi(K)$. We must have that $xK = K$ for all $x \in \varphi(K)$, hence, $x \in K$ for all $x \in \varphi(K)$. We conclude $\varphi(K) \subseteq K$, and $K$ is characteristic. 

\paragraph{3.} Let $G$ be a nonzero finite abelian group. Suppose that $G \cong \mathbb{Z}/p\mathbb{Z}$ for some prime $p$. By Lagranges Theorem, the only possible orders of subgroups of $G$ are $1$ and $p$, both corresponding to the trivial subgroups. Hence, $G$ is simple. For the converse, suppose that $G$ is simple. Let $x \in G$ be a non-trivial element of $G$. We have that $\langle x \rangle$ is a subgroup of $G$, and as $G$ is abelian, $\langle x \rangle$ is normal, thus, $\langle x \rangle = G$. Assume, for contradiction, $|G| = mn$ where $m,n > 1$. Let $x \in G$. We have that $1 = x^{|G|} = x^{mn} = (x^{m})^{n}$. Hence, $|x^{m}|$ divides $n$. Hence, $x^{m}$ has order strictly less than $mn$. Thus, $|G|$ cannot be composite. As $|G|$ is prime and is abelian, $G \cong \mathbb{Z}/p\mathbb{Z}$ for some prime $p$. 

\paragraph{4.} Let $G$ be a simple group and let $\varphi:G \to H$ be a surjective homomorphism. As $G$ is simple, $\ker\varphi = \{1_{G}\}$ or $\ker\varphi = G$. If $\ker\varphi = \{1_{G}\}$, then, $\varphi$ is an isomorphism, hence, $H \cong G$. If $\ker\varphi = G$, then $\varphi$ is the trivial homomorphism. As $\varphi$ is surjective, $H \cong 0$. Therefore, $H \cong 0$ or $H \cong G$. 

\paragraph{5.} Let $G$ be a simple group and let $\varphi:G \to H$ be a nontrivial group homomorphism. We have that $\ker\varphi$ is a normal subgroup of $G$. As $G$ is simple, $\ker\varphi = 1$ or $\ker\varphi = G$. As $\varphi$ is nontrivial, $\ker\varphi = 1$. Therefore, $\varphi$ is injective.

\paragraph{6.} Let $p$ be a prime, and let $G$ be a group such that $|G| = p^{n}$ with $n \geq 2$. Suppose $G$ is abelian. By Cauchy's Theorem, this is an $x \in G$ with order $p$. Hence, $\langle x \rangle$ is a subgroup of $G$ of order $p$. As $G$ is abelian, $\langle x \rangle$ is normal. It follows that $G$ is not simple. Suppose now $G$ is noncommutative. We have that $Z(G) \neq G$. As $G$ is a $p$-group, $Z(G)$ is non-trivial. Thus, $1 < |Z(G)| < p^{n}$. As $Z(G)$ is normal in $G$, $G$ is not simple. We conclude that groups of order $p^{n}$ are not simple. 

\paragraph{7.} We first note that if $G$ is a finite group such that $|G| = mp$ with $1 < m < p$, then $G$ is not simple by Example 2.4. Then, groups of order $6 = 2\cdot 3, 10 = 2\cdot 5, 14 = 2\cdot 7, 15 = 3\cdot 5, 20 = 4\cdot 5, 21 = 3 \cdot 7, 22 = 2\cdot 11, 26 = 2\cdot 13, 28 = 4\cdot 7, 33 = 3\cdot 11, 34 = 2\cdot 17, 35 = 5\cdot 7, 38 = 2\cdot 19, 42 = 6\cdot 7, 44 = 4\cdot 11, 46 = 2\cdot 23, 51 = 3\cdot 17, 52 = 4\cdot 13, 55 = 5\cdot 11, 57 = 3 \cdot 19, 58 = 2 \cdot 29$ are not simple.

\paragraph{8.} Let $G$ be a finite group, and let $p$ be a prime integer. Let $N$ be the intersection of all $p$-Sylow subgroups of $G$. Let $P$ be some $p$-Sylow subgroup of $G$. Then, by the second Sylow theorem, every $p$-Sylow subgroup of $G$ is of the form $gPg^{-1}$ for some $g \in G$. We also have that for all $g \in G$, $gPg^{-1}$ is a $p$-Sylow subgroup of $G$. Hence, $N = \bigcap_{g \in G}gPg^{-1}$. Let $x \in G$. We have that
$$xNx^{-1} = x\qty(\bigcap_{g \in G}gPg^{-1})x^{-1} = \bigcap_{g \in G}xgPg^{-1}x^{-1} = \bigcap_{g \in G}(xg)P(xg)^{-1} = N$$
Therefore, $N$ is normal in $G$.

\paragraph{9.} Let $P$ be a $p$-Sylow subgroup of a finite group $G$, and let $H \subseteq G$ be a $p$-subgroup of $G$. Assume that $H \subseteq N_{G}(P)$. As $P$ is normal in $N_{G}(P)$, hence, $PH$ is a subgroup of $N_{G}(P)$. We have that $|PH||P \cap H| = |H||P|$. As $H$ is a $p$-group, $H \cap P$ must also be a $p$-group by Lagranges Theorem. As $H, P, H \cap P$ all have order of a power of $p$, $PH$ has order of a power of $p$. Hence, $PH$ is a $p$-subgroup of $N_{G}(P)$. We then have that $PH = P$ as $P$ is a maximal $p$-subgroup of $G$. Therefore, $H \subseteq PH = P$.

\paragraph{10.} Let $P$ be a $p$-Sylow subgroup of a finite group $G$, and act $P$ by conjugation on the set of all $p$-subgroups of $G$. If $P'$ is a fixed point of this action, then for all $g \in P$, $gP'g^{-1} = P'$. Thus, $P \subseteq N_{G}(P')$. By the previous exercise, $P \subseteq P'$. As $P$ is a $p$-Sylow subgroup and $P'$ is a $p$-group, we must have that $P = P'$. Therefore, $P$ is the unique fixed point of this action. 

\paragraph{11.} Let $p$ be a prime integer, and let $G$ be a finite group of order $|G| = p^{r}m$. Assume $p$ does not divide $m$. Let $P$ be a $p$-Sylow subgroup of $G$. Act $P$ by conjugation on the set of all $p$-Sylow subgroups of $G$. Let $n_{p}$ be the number of $p$-Sylow subgroups of $G$. Then, by the previous exercise, $n_{p} \cong 1 \mod p$ as the set of fixed points contains a singular element. We have that $[G:P] = m$, and by Corollary 1.14, the number of subgroups conjugate to $P$ (i.e the number of $p$-Sylow subgroups, $n_{p}$, by Sylow II) is finite and divides $m$. We conclude Sylow III holds. 

\paragraph{12.} Let $P$ be a $p$-Sylow subgroup of a finite group $G$, and let $H \subseteq G$ be a subgroup containing the normaliser $N_{G}(P)$. Act $P$ on the set $G/H$ of left cosets of $H$ in $G$ by $(g, xH) \mapsto (gxg^{-1})H$. Let $xH$ be a fixed point of this action. We have that 
$$\forall g \in P, (gxg^{-1})H = xH \iff \forall g \in P, gxg^{-1}x^{-1} = 1 \iff \forall g \in P, gx = xg \iff x \in Z_{G}(P)$$
Then, $x \in Z_{G}(P) \subseteq N_{G}(P) \subseteq H$. Therefore, $xH = H$. We must have that the set of fixed points of this action contains a singular element. Hence, $[G:H] = |G/H| \equiv 1\mod p$. 

\paragraph{13.} Let $P$ be a $p$-Sylow subgroup of a finite group $G$. 
\subparagraph{(i)} Suppose that $P$ is normal. Let $\varphi \in \text{Aut}_{\mathsf{Grp}}(G)$. Then, $\varphi(P)$ is a subgroup of $G$, and is also a $p$-group. By Sylow III, $\varphi(P) \subseteq gPg^{-1}$ for some $g \in G$. As $P$ is normal, it follows that $\varphi(P) \subseteq P$. Therefore, $P$ is characteristic. 
\subparagraph{(ii)} Let $H$ be a subgroup containing $P$, and assume $P$ is normal in $H$ and $H$ is normal in $G$. As $P$ is a normal $p$-Sylow subgroup of $H$, it is then characteristic. By a previous exercise, $P$ is normal in $G$. 
\subparagraph{(iii)} We have that $P$ is a subgroup of $N_{G}(P)$ is a subgroup of $N_{G}(N_{G}(P))$. We have that $P$ is normal in $N_{G}(P)$ and $N_{G}(P)$ is normal in $N_{G}(N_{G}(P))$. Hence, $P$ is normal in $N_{G}(N_{G}(P))$. As $N_{G}(P)$ is the largest subgroup of $G$ such that $P$ is normal in that subgroup, $N_{G}(N_{G}(P))$ is a subgroup of $N_{G}(P)$. Therefore, $N_{G}(N_{G}(P)) = N_{G}(P)$. 

\paragraph{14.} By Claim 2.12, if $G$ is a group of order $18 = 2\cdot 3^{2}, 50 = 2\cdot 5^{2}, 54 = 2\cdot 3^{3}$, then $G$ is not simple. We also have that $40 = 5 \cdot 2^{3}$ and $45 = 5\cdot 3^{2}$. As $\gcd(2,5) = \gcd(3,5) = 1$, and the only divisors $d$ of $5$ such that $d \equiv 1 \mod p$ is $1$, a group of order $40$ or $45$ is not simple. 

\paragraph{15.}
\paragraph{16.}

\paragraph{17.}

\paragraph{18.}
\paragraph{19.}

\paragraph{20.} Let $G$ be a simple group of order $168$. Note that $168 = 2^{3} \cdot 3 \cdot 7$. By Sylow I, there exists cyclic subgroups of order $7$ in $G$. By Sylow III, if $n_{7}$ is the number of cyclic subgroups of order $7$, then $n_{7} \mid 2^{3} \cdot 3$ and $n_{7} \equiv 1 \mod 7$. Then, $n_{7} \in \{1, 8\}$. As $G$ is normal, $n_{7} \neq 1$. Thus, $n_{7} = 8$. As $7$ is prime, $G$ then has $8(7 - 1) = 48$ elements of order $7$. 

\paragraph{21.} Let $G$ be a group of order $pqr$ such that $p,q,r$ are prime and $p < q < r$. By Sylow I, $G$ must contain subgroups of order $p,q$ and $r$. Let $n_{p}, n_{q}, n_{r}$ be the number of such subgroups in $G$ respectively. As $p,q,r$ are prime, any two Sylow subgroup must intersect trivially. We have that the number of distinct elements of $G$ that are an element of a Sylow subgroup is given by 
$$N = 1 + (p-1)n_{p} + (q-1)n_{q} + (r-1)n_{r}$$
By Sylow III, $n_{p} \mid qr, n_{q} \mid pr, n_{r} \mid pq$ and $n_{p} \equiv 1 \mod p, n_{q} \equiv 1 \mod q, n_{r} \equiv 1 \mod r$. As $n_{r} \equiv 1 \mod r$ and $n \mid pq$ with $p,q < r$, $n_{r}$ can only possibly be $pq$ or $1$. Suppose that $n_{r} = pq$. As $n_{q} \mid pr$ and $n_{q} \equiv 1 \mod q$, $n_{q}$ can only be $1,r,pr$. Suppose, that $n_{r} \geq pq, n_{q} \geq r, n_{p} \geq q$. Then, 
$$N \geq 1 + (p-1)q + (q-1)r + (r-1)pq = pqr + qr - q - r + 1 > pqr = |G|$$
This is not possible, hence, atleast one of $n_{p}, n_{q}, n_{r}$ is equal to one. This implies the normality of a nontrivial group in $G$. Hence, $G$ is not simple.  

\paragraph{22.} Let $G$ be a finite noncommutative group of order $n$, and let $p$ be a prime divisor. Assume that the only divisor of $n$ that is congruent to $1 \mod p$ is $1$. By Sylow I, $G$ contains a $p$-Sylow subgroup. by Sylow III, the number of $p$-Sylow subgroups is congruent to $1 \mod p$ and divides $n$. By assumption, there is only a singular $p$-Sylow subgroup of $G$. If $G$ is a not a $p$-group, then this unique Sylow subgroup is not trivial or the group itself. Hence, $G$ is abelian. If $G$ is a $p$-group, then $G$ has a nontrivial centre, and since $G$ is noncommutative, $G$ is not simple. Therefore, $G$ is not simple. 

\paragraph{23.} Let $n_{p}$ denote the number of $p$-Sylow subgroups of a group $G$. Suppose that $G$ is simple. Let $p$ be a prime divisor of $G$. As $G$ is simple, $n_{p} > 1$ by Sylow II. Act $G$ on the set of $p$-Sylow subgroups of $G$ via conjugation. This induces a homomorphism $\varphi:G \to S_{n_{p}}$. We have that $G/\ker\varphi \cong \varphi(G)$. If $\ker\varphi = 1$, then $|G|$ divides $|S_{n_{p}}| = n_{p}!$. If $\ker\varphi = G$, then a $p$-Sylow subgroup is normal, which contradicts simplicity of $G$. We conclude that $|G|$ divides $n_{p}!$ for all prime divisors $p$ of $G$. 

\paragraph{24.} \textbf{NOT FINISHED} First note that there are no noncommutative groups of order $p, p^{2}$ where $p$ prime. Then, we look at groups of order
$$6,8,10,12,14,15,16,18,20,21,22,24,26,27,28,30,32,33,34,35,36,38,39,40,42,$$
$$44,45,46,48,50,51,52,54,55,56,57,58,62,63,64,65,66,68,69,70,72,74,75,76,77,$$
$$78,80,81,82,84,85,86,87,88,90,91,92,93,94,95,96,98,99,100,102,104,105,106,$$
$$108,110,111,112,114,115,116,117,118,119,120,122,123,124,125,126,128,129,130,132,133,$$
$$134,135,136,138,140,141,142,143,144,145,146,147,148,150,152,153,154,155,156,158,$$
$$159,160,161,162,164,165,166$$
We now remove groups of order $mp$ with $1 < m < p$ with $p$ prime.
$$8,12,16,18,24,27,30,32,36,40,45,48,50,54,56,63,70,72,75,80,81,84,90,96,98,100,105,$$
$$108,112,120,125,126,128,132,135,140,144,147,150,154,160,162,165$$
Now remove $p$-groups with order greater than $p^{2}$
$$12,18,24,30,36,40,45,48,50,54,56,63,70,72,75,80,84,90,96,98,100,105,$$
$$108,112,120,126,132,135,140,144,147,150,154,160,162,165$$
Remove groups of order $mp^{r}$ with $1 < m < p$
$$12,24,30,36,40,45,48,56,63,70,72,80,84,90,96,105,$$
$$108,112,120,126,132,135,140,144,150,154,160,165$$
We now remove $40,45$ by a previous exercise
$$12,24,30,36,48,56,63,70,72,80,84,90,96,105,$$
$$108,112,120,126,132,135,140,144,150,154,160,165$$
Remove groups of order $pqr$ with $p,q,r$ prime and $p < q < r$
$$12,24,36,48,56,63,72,80,84,90,96,$$
$$108,112,120,126,132,135,140,144,150,160$$
From the text, there are no simple groups of order $12,24$
$$36,48,56,63,72,80,84,90,96,$$
$$108,112,120,126,132,135,140,144,150,160$$
Let $G$ be a group of order $p^{2}q^{2}$ with $p < q$ and $p,q$ prime. We have that $n_{q} \equiv 1 \mod q$ and $n_{q} \mid p^{2}$. As $p < q$, we must have that $n_{q} \in \{1,p^{2}\}$. Suppose that $n_{q} = p^{2}$. Then, $p^{2} \equiv 1 \mod q$ and so $q \mid p^{2} - 1$. As $q$ is prime, $q \mid p-1$ or $q \mid p+1$. As $p < q$, $q \mid p+1$, and $q = p+1$. Hence, $q = 3$ and $p = 2$. If $G$ is simple, then it must be of order $36$. Let $H$ be a group of order $36$. We have that $n_{3} \equiv 1 \mod 3$ and $n_{3} \mid 4$. Then, $n_{3} \in \{1,4\}$. Suppose $n_{3} = 4$. We have that $4! = 24$, but $36$ does not divide $24$. $H$ cannot be simple. Therefore, groups of order $p^{2}q^{2}$ are not simple. We remove these
$$48,56,63,72,80,84,90,96,$$
$$108,112,120,126,132,135,140,144,150,160$$
Let $G$ be a group of order $2^{n}\cdot 3$. We have that $n_{2} \equiv 1 \mod 2$ and $n_{2} \mid 3$. Hence, $n_{2} \in \{1,3\}$. If $n_{2} = 3$, then $n_{2}! = 6$. $|G|$ divides $6$ when $n = 1$. If $G$ is of order $6$, then $G$ is not simple. Therefore, there are no simple groups of order $2^{n}\cdot 3$. We remove these
$$56,63,72,80,84,90,$$
$$108,112,120,126,132,135,140,144,150,160$$
Let $G$ be a group of order $56 = 2^{3} \cdot 7$. Then, $n_{7} \mid 8$ and $n_{7} \equiv 1 \mod 7$. Then, $n_{7} \in \{1,8\}$. Suppose that $n_{7} = 8$. Then, there are $48$ elements of order $7$. The remaining $8$ elements must form a unique $2$-Sylow subgroup of $G$. Hence, $G$ is not simple. Remove the group of order $56$. 
$$63,72,80,84,90,108,112,120,126,132,135,140,144,150,160$$
Let $G$ be a group of order $63 = 3^{2} \cdot 7$. We have that $n_{7} \mid 9$ and $n_{7} \equiv 1 \mod 7$ and $n_{7} \mid 9$. Therefore, $n_{7} = 1$. A group of $63$ cannot be simple. 
$$72,80,84,90,108,112,120,126,132,135,140,144,150,160$$
Let $G$ be a group of order $72 = 2^{3} \cdot 3^{2}$. We have that $n_{3} \equiv 1 \mod 3$ and $n_{3} \mid 8$. Hence, $n_{3} \in \{1,4\}$. Suppose that $n_{3} = 4$. Then, $G$ does not divide $n_{3}!$, and so $G$ is not simple. We remove $72$
$$80,84,90,108,112,120,126,132,135,140,144,150,160$$
Let $G$ be a group of order $80 = 2^{4}\cdot 5$. We have $n_{2} \equiv 1 \mod 2$ and $n_{2} \mid 5$. Then, $n_{2} \in \{1,5\}$. If $n_{2} = 5$, then $n_{2}! = 120$, and $|G|$ does not divide $120$. Hence, $G$ is not simple.  
$$84,90,108,112,120,126,132,135,140,144,150,160$$
Let $G$ be a group of order $84 = 2^{2}\cdot 3\cdot 7$. We have $n_{7} \equiv 1 \mod 7$ and $n_{7} \mid 12$. We must have that $n_{7} = 1$. Therefore, $G$ is not simple. 
$$90,108,112,120,126,132,135,140,144,150,160$$
Let $G$ be a noncommutative simple group of order $90 = 2 \cdot 3^{2} \cdot 5$. We have that $n_{3} \equiv 1 \mod 3$ and $n_{3} \mid 10$. Then, $n_{3} \in \{1,10\}$. Additionally, $n_{5} \equiv 1 \mod 5$ and $n_{5} \mid 18$. Then, $n_{5} \in \{1,6\}$. As $G$ is simple, $n_{5} = 6$ and $n_{3} = 10$. Let $\{P_{1}, ..., P_{10}\}$ be the set of $3$-Sylow subgroups of $G$. Suppose $P_{i} \cap P_{j} = 1$ for all $i \neq j$. Then, $|P_{1} \cup ... \cup P_{10}| = 81$. We also have that the set of $5$-Sylow subgroups intersect trivially, hence, there are $24$ elements of order $5$. Therefore, $G$ has more than $105$ elements, which is a contradiction. There must exist $P_{i}, P_{j}$ with $i \neq j$ such that $|P_{i} \cap P_{j}| = 3$. Note that $P_{i}, P_{j}, P_{i}P_{j} \subseteq N_{G}(P_{i} \cap P_{j}) = N$. As $P_{i} \subseteq N$, $|N|$ must be a multiple of $9$. Note that $|P_{i}P_{j}| = |P_{i}||P_{j}|/|P_{i} \cap P_{j}| = 27$. Hence, $|N|$ must be greater than $27$. As $N$ is a subgroup of $G$, $|N|$ must also divide $90$. Therefore, $|N| \in \{45,90\}$. If $|N| = 45$, then $N$ has index $2$, which implies $N$ is a nontrivial normal subgroup of $G$. If $|N| = 90$, then $P_{i} \cap P_{j}$ is normal in $G$. Both cases lead to contradiction, thus, $G$ cannot be simple.  
$$108,112,120,126,132,135,140,144,150,160$$
Let $G$ be a noncommutative group of order $108 = 2^{2} \cdot 3^{3}$. We have that $n_{3} \equiv 1 \mod 3$ and $n_{3} \mid 4$. Hence, $n_{3} \in \{1,4\}$. If $n_{3} = 4$, then $108$ does not divide $n_{3}!$. Thus, $G$ cannot be simple. 
$$112,120,126,132,135,140,144,150,160$$
Let $G$ be a noncommutative simple group of order $112 = 2^{4}\cdot 7$. Then, $n_{7} \equiv 1 \mod 7$ and $n_{7} \mid 16$. Hence, $n_{7} \in \{1,8\}$. Additionally, $n_{2} \equiv 1 \mod 2$ and $n_{2} \mid 7$. Hence, $n_{2} \in \{1,7\}$. As $G$ is simple, $n_{2} = 7$ and $n_{7} = 8$. Let $\{P_{1}, ..., P_{7}\}$ be the set of $2$-Sylow subgroups of $G$. Suppose that for all $i,j$, $P_{i} \cap P_{j} = 1$. As each $P_{i}$ is unique and intersects trivially with a different $2$-Sylow subgroup, the set of $P_{i}$'s contribute to $7 \cdot 15 = 105$ nontrivial elements. We have that the $7$-Sylow subgroups are cyclic and intersect with eachother trivially. Hence, there subgroups contribute to $6 \cdot 8 = 48$ nontrivial elements of $G$. Thus, $G$ has atleast $153$ elements, which is a clear contradiction. Hence, there is $P_{i}, P_{j}$ with $i \neq j$ such that $|P_{i} \cap P_{j}| \in \{2,4,8\}$. If $|P_{i} \cap P_{j}| = 2$, then $|P_{i}P_{j}| = |P_{i}||P_{j}|/2 = 128$, which cannot occur. If $|P_{i} \cap P_{j}| = 4$, then $|P_{i}P_{j}| = 64$. Let $N = N_{G}(P_{i} \cap P_{j})$. Note $P_{i}, P_{i}P_{j} \subseteq N$. As $P_{i}P_{j} \subseteq N$, $|N| \geq 64$. Therefore, $N = G$. Hence, $P_{i} \cap P_{j}$ is normal in $G$, which cannot happen. Then, $|P_{i} \cap P_{j}| = 8$, which means $|P_{i}P_{j}| = 32$. We have that $|N| \geq 32$. As $P_{i}$ is a subgroup of $N$, $16 \mid |N|$. As $N$ is a subgroup of $G$, we have that $|N| \mid 112$. It follows that $|N| = 112$, hence, $P_{i} \cap P_{j}$ is normal in $G$, which is another contradiction. We conclude that there is no noncommutative simple group of order $112$. 
$$120,126,132,135,140,144,150,160$$
Let $G$ be a noncommutative group of order $135 = 3^{3} \cdot 7$. By Sylow III, $n_{7} \equiv 1 \mod 7$ and $n_{7} \mid 27$. Then, $n_{7} = 1$. Therefore, $G$ is simple. 
$$120,126,132,140,144,150,160$$
Let $G$ be a noncommutative group of order $126 = 2 \cdot 3^{2} \cdot 7$. We have that $n_{7} \equiv 1 \mod 7$ and $n_{7} \mid 18$. Then, $n_{7} = 1$. Hence, $G$ cannot be simple.  
$$120,132,140,144,150,160$$
Let $G$ be a simple noncommutative group of order $132 = 2^{2} \cdot 3 \cdot 11$. We have that $n_{2} \equiv 1 \mod 2$ and $n_{2} \mid 33$. Hence, $n_{2} \in \{1,3,11,33\}$. Additionally, $n_{3} \equiv 1 \mod 3$ and $n_{3} \mid 44$, hence, $n_{3} \in \{1,4,22\}$. Lastly, $n_{11} \equiv 12$ and $n_{11} \equiv 1 \mod 11$. Hence, $n_{11} \in \{1,12\}$. As $G$ is simple and $132 \geq 5!$, $n_{2} \in \{11,33\}, n_{3} = 22, n_{11} = 12$. We have that the number of elements in $G$ of order $3$ is $22(3-1) = 44$. The number of elements in $G$ of order $11$ is $12(11-1) = 120$. Hence, $G$ contains atleast $164$ elements. This is a contradiction. $G$ cannot be simple. 
$$120,140,144,150,160$$
Let $G$ be a noncommutative group of order $140 = 2^{2} \cdot 5 \cdot 7$. Then, $n_{5} \equiv 1 \mod 5$ and $n_{5} \mid 28$. Then, $n_{5} = 1$. $G$ cannot possible simple. 
$$120,144,150,160$$
Let $G$ be a noncommutative group of order $150 = 2 \cdot 3 \cdot 5^{2}$. Then, $n_{5} \equiv 1 \mod 5$ and $n_{5} \mid 6$. Then, $n_{5} \in \{1,6\}$. We have that $150$ does not divide $6!$, hence, $G$ cannot be simple. 
$$120,144,160$$
Let $G$ be a noncommutative group of order $160 = 2^{5} \cdot 5$. We have that $n_{2} \equiv 1 \mod 2$ and $n_{2} \mid 5$. Hence, $n_{2} \in \{1,5\}$. As $160$ does not divide $1!$ or $5!$, $G$ cannot be simple. 
$$120,144$$

\paragraph{25.} 

\subsection*{4.3 - Composition Series and Solvability}
\paragraph{1.} Let $n \in \mathbb{Z}$. We have that 
$$\mathbb{Z} \supset 2^{n-1}\mathbb{Z} \supset 2^{n-2}\mathbb{Z} \supset ... \supset 2\mathbb{Z} \supset \{1\}$$
is a normal series of length $n$ as $\mathbb{Z}$ is abelian. Therefore, $\ell(G)$ is not finite. 

\paragraph{2.}

\paragraph{3.} Note that $\mathbb{Z}/2\mathbb{Z}$ is simple, hence, it has  composition series $\mathbb{Z}/2\mathbb{Z} \supset \{[0]_{2}\}$. Thus, groups of order $2$ have composition series. Suppose that for all groups of order less than $n$, it has a composition series. Let $G$ be a group of order $n$. If $G$ is simple, then we are done and $G$ has a composition series. If $G$ has normal subgroups, let $N$ be a maximal normal subgroup of $G$. We note that $G/N$ is simple by the Correspondance Theorem, hence, has a composition series. $N$ is a group of order less than $n$ by Lagranges Theorem, hence, by assumption $N$ has a composition series. By Proposition 3.4, $G$ has a composition series. By the Principle of Strong Induction, all finite groups have a composition series. Note that $\mathbb{Z}$ does not have a composition series. Suppose that there exists a composition series of $\mathbb{Z}$, 
$$\mathbb{Z} = d_{0}\mathbb{Z} \supset d_{1}\mathbb{Z} \supset ... \supset d_{n}\mathbb{Z} = \{[0]_{1}\}$$
Then, $d_{i}\mathbb{Z}/d_{i+1}\mathbb{Z} \cong \mathbb{Z}/(d_{i+1}/d_{i})\mathbb{Z}$ is simple for all $i$. Hence, $d_{i+1}/d_{i} = p_{i}$ for some prime $p_{i}$ for all $i$. We have that
$$\prod_{i=0}^{n-1}p_{i} = \prod_{i=0}^{n-1}\frac{d_{i+1}}{d_{i}} = \frac{d_{n}}{d_{0}} = \frac{0}{1} = 0$$ 
Therefore, $p_{i} = 0$ for some $i$, which is a contradiction as $p_{i}$ is prime. 

\paragraph{4.} Let $x \in Q_{8}$ be the element of order $4$, and let $y \in D_{8}$ of order $4$. We have the following composition series
$$Q_{8} \supset \langle x \rangle \supset \langle x^{2} \rangle \supset \{1\}$$
$$D_{8} \supset \langle y \rangle \supset \langle y^{2} \rangle \supset \{e\}$$
Furthermore, $Q_{8}/\langle x \rangle \cong \langle x \rangle/\langle x^{2} \rangle \cong D_{8}/\langle y \rangle \cong \langle y \rangle/\langle y^{2} \rangle \cong \mathbb{Z}/2\mathbb{Z}$. Hence, $Q_{8}$ and $D_{8}$ are nonisomorphic groups with the same composition factors.

\paragraph{5.} Let $H, K$ be normal subgroups of a group $G$. Let $g \in G$. We have that $gHKg^{-1} = gHg^{-1}gKg^{-1} = HK$. Therefore, $HK$ is normal in $G$.

\paragraph{6.} Let $G_{1}, G_{2}$ be groups. We have that $(G_{1}\times G_{2})/G_{1} \cong G_{2}$ and $(G_{1} \times G_{2})/G_{2} \cong G_{2}$. By Proposition 3.4, $G_{1} \times G_{2}$ has composition series if and only if $G_{1}, G_{2}$ have composition series. Let
$$G_{1} = H_{0} \supset H_{1} \supset ... \supset H_{n} = \{1_{{G}_{1}}\}$$
$$G_{2} = K_{0} \supset K_{1} \supset ... \supset K_{m} = \{1_{G_{2}}\}$$
be composition series. Suppose that $H$ is normal in $H'$ and $K$ is normal in $K'$. Let $(g,g') \in H' \times K'$. We have that 
$$(g,g')(H \times K)(g,g')^{-1} = (g,g')(H \times K)(g^{-1},g'^{-1}) = (gHg^{-1}) \times (g'Kg'^{-1}) = H \times K$$
Therefore, $H \times K$ is normal in $H' \times K'$. Define the homomorphism $\varphi: H' \times K' \to (H'/H) \times (K'/K)$ by $(g,g') \mapsto (g + H, g' + K)$. We have that this is a surjective homomorphism with kernel $H \times K$. Hence, $(H' \times K')/(H \times K) \cong (H'/H) \times (K'/K)$. Suppose that $H, K$ are simple groups. Let $N \times M$ be a normal subgroup of $H \times K$. We have that for all $(g,1_{K}) \in H \times K$, $N \times M \supset (g,1_{K})(N \times M)(g, 1_{K})^{-1} = (gNg^{-1}) \times M$. Therefore, $N$ is normal in $H$. Similarly, $M$ is normal in $K$. We have the result that $N \times M$ is normal in $H \times K$ if and only if $N$ is normal in $H$ and $M$ is normal in $K$, and the result $(H \times K)/(N \times M) \cong (H/N) \times (K/M)$. Therefore, $H \times K$ is simple if and only if $H, K$ are simple and 
$$G_{1} \times G_{2} = H_{0}\times K_{0} \supset H_{1} \times K_{1} \supset ... \supset H_{n} \times K_{n} = \{1_{G_{1}}\} \times \{1_{G_{2}}\}$$
is a composition series of $G_{1} \times G_{2}$. By the Jordan-Holder theorem, any composition series of $G_{1} \times G_{2}$ is equivalent to the above composition series. 

\paragraph{7.}

\paragraph{8.} Let $\varphi:G_{1} \to G_{2}$ be a group homomorphism. Let $g,h \in G_{1}$. Then,
$$\varphi([g,h]) = \varphi(ghg^{-1}h^{-1}) = \varphi(g)\varphi(h)\varphi(g)^{-1}\varphi(h)^{-1} = [\varphi(g),\varphi(h)]$$
Let $x \in G_{1}'$. Then, $x = [g_{1},h_{1}]^{n_{1}}...[g_{k},h_{k}]^{n_{k}}$ for some $g_{1},...,g_{k},h_{1},...,h_{k} \in G_{1}$ and $n_{1},...,n_{k} \in \mathbb{Z}$. Then,
$$\varphi(x) = \varphi([g_{1},h_{1}]^{n_{1}}...[g_{k},h_{k}]^{n_{k}}) = \varphi([g_{1},h_{1}])^{n_{1}}...\varphi([g_{k},h_{k}])^{n_{k}} = [\varphi(g_{1}),\varphi(h_{1})]^{n_{1}}...[\varphi(g_{k}),\varphi(h_{K})]^{n_{k}} \in G_{2}'$$
Therefore, $\varphi(G_{1}') \subseteq G_{2}'$. 

\paragraph{9.} 

\paragraph{10.} Let $G$ be a group. Define inductively an increasing sequence $\{e\} = Z_{0} \subseteq Z_{1} \subseteq Z_{2} \subseteq \cdots$ of subgroups of $G$ as follows: for $i \geq 1$, $Z_{i}$ is the subgroup of $G$ corresponding to $Z(G/Z_{i-1})$.

\subparagraph{(i)}  We first note that $Z_{0} = \{e\}$ is normal in $Z_{1} = Z(G)$ and $Z_{1}$ is normal in $G$. Suppose that $Z_{i}$ is normal in $G$ for all $i < n$. We have that $Z_{n}$ corresponds to the subgroup $Z(G/Z_{n-1})$. We have that $Z_{n}/Z_{n-1}$ is normal in $G/Z_{n-1}$ as $Z_{n}$ corresponds to the centre of $G/Z_{n-1}$. Hence, $Z_{n}$ is normal in $G$ by the Third Isomorphism Theorem. 

\subparagraph{(ii)} Let $G$ be a group. Suppose that $G$ is nilpotent. We have that $G$ has the normal series
$$\{e\} = Z_{0} \subseteq Z_{1} \subseteq Z_{2} \subseteq \cdots \subseteq Z_{n} = G$$
for some $n \in \mathbb{N}$. Let $H = G/Z(G)$. Define inductively the increasing sequence $\{e\} = H_{1} \subseteq H_{2} \subseteq ...$ of subgroups of $H$ as follows: for $i \geq 2$, $H_{i}$ is the subgroup of $H$ corresponding to $Z(H/H_{i-1})$. Suppose that for all $i < n$, $H_{i} = Z_{i}/Z(G)$. We have that $H_{n}$ corresponds to the subgroup $Z(H/H_{n-1})$. Then, 
$$\frac{H_{n}}{H_{n-1}} \cong Z\qty(\frac{H}{H_{n-1}}) \cong Z\qty(\frac{G/Z(G)}{Z_{n-1}/Z(G)}) \cong Z\qty(\frac{G}{Z_{n-1}}) \cong \frac{Z_{n}}{Z_{n-1}} \cong \frac{Z_{n}/Z(G)}{H_{n-1}}$$
Then, $H_{n} = Z_{n}/Z(G)$. Therefore, we must have that $H$ has the series
$$\{e\} = H_{1} \subseteq H_{2} \subseteq \cdots \subseteq H_{n} = G/Z(G)$$
Therefore, $G/Z(G)$ is nilpotent. From the above chain of isomorphisms and using the same argument, we also have that if $G/Z(G)$ was nilpotent, then $G$ is nilpotent

\subparagraph{(iii)} Let $G$ be a $p$-group for some prime $p$. We have that $Z(G)$ is nontrivial and $G/Z(G)$ is also a $p$-group where $|G/Z(G)| < |G|$. Repeating, we have that $(G/Z(G))/(Z(G/Z(G)))$ is a $p$-group and is of order strictly less than $G/Z(G)$. Eventually, we reach the trivial group. $\{e\}$ is clearly nilpotent, hence, $G$ is nilpotent by the previous part.  

\subparagraph{(iv)} Let $G$ be a nilpotent group. $G$ has the central series
$$\{e\} = Z_{0} \subseteq Z_{1} \subseteq Z_{2} \subseteq \cdots \subseteq Z_{n} = G$$
for some $n \in \mathbb{N}$. We have that $Z_{i}/Z_{i-1}$ corresponds to the centre of a group, hence, it is abelian. Therefore, $G$ is solvable. 

\subparagraph{(v)} We have that $S_{3}$ has trivial centre. Hence, $S_{3}$ cannot be equal to $Z_{n}$ for some $n$. Therefore, $S_{3}$ is not nilpotent. 

\paragraph{11.} Let $H$ be a normal subgroup of a nilpotent group $G$. Let $r \geq 1$ be the smallest index such that there exists a nontrivial $h \in H \cap Z_{r}$. Let $g \in G$. We have that $[g,h] = ghg^{-1}h^{-1} \in H$ as $ghg^{-1}, h^{-1} \in H$. We note that $Z_{r}/Z_{r-1} = Z(G/Z_{r-1})$. Then, $ghZ_{r-1} = hgZ_{r-1}$ as $g \in G$ and $h \in Z_{r}$. Hence, $[g,h] = ghg^{-1}h^{-1} \in Z_{r-1}$. Therefore, $[g,h] \in H \cap Z_{r-1}$. By assumption, $[g,h] = 1$, so $h \in Z(G)$. Thus, $H$ has a nontrivial intersection with $Z(G)$. 

\paragraph{12.} Let $G$ be a finite nilpotent group and let $H$ be a proper subgroup of $G$. As $G$ is nilpotent, $Z(G)$ is trivial, otherwise, $G$ is the trivial group. Suppose that $H$ does not contain $Z(G)$. Then, there exists a $g \in Z(G)$ such that $g \not\in H$. Let $ghg^{-1} \in gHg^{-1}$. As $g \in Z(G)$, we have that $ghg^{-1} \in H$. Furthermore, let $h \in H$. Then, $h = hgg^{-1} = ghg^{-1} \in gHg^{-1}$. By definition, $h \in N_{G}(H)$. Hence, $H$ is properly contained in $N_{G}(H)$. Now, suppose that $H$ does contain $Z(G)$. There exists an index $r$ such that $H$ contains $Z_{r}$ but $H$ does not contain $Z_{r+1}$. There then exists an $X \in Z_{r+1}$ such that $x \not\in H$. From the previous exercise, not $[x,g] \in Z_{r}$ for any $g \in G$. Hence, $[x,g] \in H$. Thus, $xgx^{-1}g^{-1} \in H$. Suppose $h \in H$. Then, $xhx^{-1}h^{-1} \in H$. Thus, $xhx^{-1} \in H$, which means $x \in N_{G}(H)$. Therefore, $H$ is properly contained in $N_{G}(H)$. Let $P$ be a Sylow subgroup of $G$. We have that $N_{G}(N_{G}(P)) = N_{G}(P)$. However, $N_{G}(P)$ is properly contained in $N_{G}(N_{G}(P))$. Thus, $N_{G}(P)$ cannot be a proper subgroup of $G$. As $N_{G}(P)$ is not trivial, $N_{G}(P) = G$, that is, $P$ is normal in $G$. 

\paragraph{13.} Let $G$ be a group and let $H,K$ be subgroups of $G$ such that $K$ is characteristic in $H$ and $H$ is characteristic in $G$. Let $\varphi \in \text{Aut}_{\mathsf{Grp}}(G)$. We have that $\varphi(H) \subseteq H$, and so $\varphi_{H}$, the restriction of $\varphi$ to $H$, is an automorphism of $H$, that is, $\varphi_{H} \in \text{Aut}_{\mathsf{Grp}}(H)$. As $K$ is characteristic in $H$, $\varphi_{H}(K) \subseteq K$. We have that $\varphi(K) = \varphi_{H}(K)$ as $K$ is a subgroup of $H$. Therefore, $K$ is characteristic in $G$. We have that
$$G^{(n)} \text{ char } G^{(n-1)} \text{ char } ... \text{ char } G^{(1)} \text{ char } G$$
Via induction, $G^{(n)} \text{ char } G$ for all $n \in \mathbb{N}$. 

\paragraph{14.} Let $H$ be a nontrivial normal subgroup of a solvable group $G$. We have that $G$ has the derived series
$$\{e\} = G_{0} \subseteq G_{1} \subseteq \cdots \subseteq G_{n} = G$$
Let $r$ be the largest index such that $K = H \cap G_{r}$ is nontrivial. Let $x,y \in K$. We have that $[x,y] \in H$ and $[x,y] \in G_{r+1}$. Therefore, $[x,y] \in H \cap G_{r+1}$. By assumption, $[x,y] \in \{e\}$. Thus, $[x,y] = e$. It follows that $K$ is commutative. By a previous exercise, $G_{r}$ is normal. Hence, $K$ is normal. 

\paragraph{15.} Let $G$ be a group of order $p^{2}q$. Suppose that $p > q$. By Sylow III, $n_{p} \equiv 1 \mod p$ and $n_{p} \mid q$. Then, $n_{p} = 1$. Denote the unique normal $p$-Sylow subgroup by $N$. We have that $G/N$ is of order $q$, hence, abelian. Furthermore, $N$ is of order $p^{2}$, hence, abelian. $G/N$ and $N$ are then both solvable, thus, $G$ is solvable. Now, suppose that $p < q$. By Sylow III, $n_{q} \equiv 1 \mod q$ and $n_{q} \mid p^{2}$. We must have that $n_{q} \in \{1, p^{2}\}$. Suppose that $n_{q} = p^{2}$. Then, these Sylow subgroups contibute $p^{2}(q-1)$ elements of order $q$, hence, there can only be one Sylow subgroup of order $p^{2}$. Again, let $N$ be the subgroup of order $p^{2}$, and we see that $G$ is solvable. If $n_{q} = 1$, then let $M$ be the unique normal subgroup of order $q$ in $G$. We have that $M$ is abelian, and $G/M$ is also abelian as a group of order $p^{2}$. Hence, $G$ is solvable. Finally, let $p=q$. If $G$ is abelian, then $G$ is solvable. Suppose $G$ is noncommutative. As $G$ is a $p$-group, $Z(G)$ is nontrivial. $Z(G)$ is either of order $p$ or $p^{2}$. In both cases $Z(G)$ is abelian, and $G/Z(G)$ is abelian. Hence, $Z(G)$ and $G/Z(G)$ are solvable, which implies $G$ is solvable. Therefore, all groups of order $p^{2}q$ are solvable. 

\paragraph{16.} Note that $p$-groups and groups of order $p^{2}q$ are solvable. We look at groups of order
$$6,10,14,15,21,22,24,26,30,33,34,35,36,38,39,40,42,46,48,51,54,55,56,$$
$$57,58,62,65,66,69,70,72,74,77,78,80,82,84,85,86,87,88,90,91,93,94,$$
$$95,96,99,100,102,104,105,106,108,110,111,112,114,115,117,118,119$$
Note that groups of order $pq$ where $q \not\equiv 1 \mod p$ with $p < q$ are cyclic, hence abelian and then solvable. 
$$6,10,14,15,21,22,24,26,30,34,36,38,39,40,42,46,48,54,55,56,$$
$$57,58,62,66,70,72,74,78,80,82,84,86,88,90,93,94,$$
$$96,99,100,102,104,105,106,108,110,111,112,114,117,118$$
By Feit-Thompson, every group of odd order is solvable. 
$$6,10,14,22,24,26,30,34,36,38,40,42,46,48,54,56,58,62,66,70,72,74,78,80,82,$$
$$84,86,88,90,94,96,100,102,104,106,108,110,112,114,118$$
Let $G$ be a group of order $2p$. Then, $G$ contains a group of order $p$, $N$ say. We have that $[G:N] = 2$, hence, normal. We have that $G/N$ is abelian and $N$ is abelian. Hence, $G/N$ and $N$ are both solvable. Thus, $G$ is solvable. 
$$24,30,36,40,42,48,54,56,66,70,72,78,80,84,88,90,96,100,102,104,108,110,112,114$$
Let $G$ be a group of order $pq$ with $p < q$. By Sylow III, $n_{q} \equiv 1 \mod q$ and $n_{q} \mid p$. Then, $n_{q} = 1$. By $N$ be the normal subgroup of $G$ of order $q$. We have that $N$ is abelian, and $G/N$ is abelian. Therefore, $G$ is solvable. 
$$24,30,36,40,42,48,54,56,66,70,72,78,80,84,88,90,96,100,102,104,108,110,112,114$$
Let $G$ be a group of order $pqr$ with $p < q < r$ prime. $G$ is not simple or abelian, hence, it contains a normal subgroup $N$ of possible orders $pq,pr,qr,p,q,r$. In all cases, $N$ is solvable. We also have that $G/N$ is solvable as it has an order of $r,q,p,qr,pr,qp$. Then, $G$ is solvable. 
$$24,36,40,48,54,56,72,80,84,88,90,96,100,104,108,112$$
Let $G$ be a group of order $p^{2}q^{2}$, $p < q$ prime. We have that $G$ is not simple or abelian, hence, $G$ contains a normal subgroup of possible orders $p,p^{2},q,q^{2}, pq, pq^{2}, p^{2}q$, $N$ say. $G/N$ has possible orders $pq^{2}, q^{2}, p^{2}q, pq, p, q$ respectively. Hence, $N, G/N$ are both solvable. Therefore, $G$ is solvable. 
$$24,40,48,54,56,72,80,84,88,90,96,104,108,112$$
Let $G$ be a group of order $2^{n}\cdot 3$ with $n \geq 3$. We must have that $G$ is simple or abelian and contains a normal subgroup of order $2^{n}$, $N$ say. $N$ is a $p$-group, hence solvable. We also have that $G/N$ is cyclic, hence solvable. Therefore, $G$ is solvable.
$$40,54,56,72,80,84,88,90,104,108,112$$
Let $G$ be a group of order $40 = 2^{3} \cdot 5$. We have that $G$ is not simple or abelian, hence, $G$ contains a normal subgroup $N$ with $|N| \in \{2,2^{2},2^{3}, 2\cdot 5, 2^{2} \cdot 5\}$. Then, $N$ is solvable. It also follows that $G/N$ is solvable. With a similar argument, we can prove that $54 = 3^{3} \cdot 2, 56 = 2^{3} \cdot 7, 88 = 2^{3} \cdot 11, 104 = 2^{3} \cdot 13$ are solvable. 
$$72,80,84,90,108,112$$
Let $G$ be a group of order $72 = 2^{3} \cdot 3^{2}$. As $G$ is not simple or abelian, $G$ contains a normal subgroup $N$ with $|N| \in \{2,4,8,6,12,24,18,36\}$. $N$ is then solvable, and $G/N$ is solvable. Hence, $G$ is solvable. 
$$80,84,90,108,112$$
A group of order $80$ is either abelian or not simple. Let $G$ be a group of order $80$. Then, $G$ has a normal subgroup $N$ with $|N| \in \{1,2,4,8,16,5,10,20,40\}$. Hence, $N$ is solvable. We also have that $G/N$ is solvable. Thus, $G$ is solvable. 
$$84,90,108,112$$
Let $G$ be a group of order $84$. $G$ is either abelian or not simple. We have that $G$ has a normal subgroup $N$ with $|N| \in \{2,4,6,12,14,28,42\}$. Hence, $N$ is solvable. We also have that $G/N$ is solvable. Thus, $G$ is solvable. 
$$90,108,112$$
We prove the final ones in a similar way by noting we have proven all of the lesser orders are solvable. 

\paragraph{17.} Suppose the statement "Every finite group of odd order is solvable" holds. Let $G$ be a noncommutative finite simple group. Suppose that $G$ has odd order. We have that the commutator of $G$, $G'$, must be $G$ itself as $G$ is noncommutative and $G$ is simple. Hence, $G$ cannot be solvable, which is a contradiction. Therefore, $G$ has even order. For the converse, suppose the statement "Every noncommutative finite simple group has even order" holds. Let $G$ be a group of odd order. If $G$ is abelian, then $G$ is automatically solvable. Suppose $G$ is noncommutative. As $G$ has odd order, it is not simple, hence, $G$ has a normal subgroup $N$. $N$ is either abelian or noncommutative. If $N$ is abelian, then $N$ is solvable. If $N$ is noncommutative, then as $|N|$ divides $|G|$, we have that $N$ is of odd order, hence, not simple. $N$ contains a normal subgroup of order $M$. Via induction, we get a chain of subgroups that either terminate with an abelian group, or eventually we end up with a group of prime order, which is also abelian. Thus, $N$ is solvable. With the same argument, $G/N$ is also solvable. Therefore, $G$ is solvable. 

\subsection*{4.4 - The Symmetric Group}
\paragraph{1.} Let 
$$\sigma = \begin{pmatrix} 1 & 2 & 3 & 4 & 5 & 6 & 7 & 8 \\ 8 & 1 & 2 & 7 & 5 & 3 & 4 & 6 \end{pmatrix} \in S_{8}$$
We have that $\sigma = \begin{pmatrix} 1 & 8 & 6 & 3 & 2 & 1\end{pmatrix}\begin{pmatrix} 4 & 7 \end{pmatrix}\begin{pmatrix}
5\end{pmatrix}$. Then, $\sigma$ is of type $[6,2,1]$. 

\paragraph{2.}
\paragraph{3.}
\paragraph{4.}

\paragraph{5.} We have that $S_{1}$ is the trivial group, and the class formula is then $1 = 1$. $S_{2}$ is isomorphic to $\mathbb{Z}/2\mathbb{Z}$, which is abelian, hence, its class formula is $2 = 2$. $S_{3}$ is noncommutative and has trivial centre. The only possible class formula is $6 = 1 + 2 + 3$. $S_{4}$ has class formula $24 = 1 + 8 + 6 + 6 + 3$. $S_{5}$ has class formula $120 = 1 + 10 + 15 + 20 + 20 + 30 + 24$. $S_{6}$ has class formula $720 = 1 + 15 + 45 + 15 + 40 + 120 + 40 + 90 + 90 + 144 + 120$. 

\paragraph{6.} Let $N$ be a normal subgroup of $S_{4}$. We note that $S_{4}$ has the class formula $24 = 1 + 8 + 6 + 6 + 3$. By Lagranges Theorem, $|N| \in \{1,2,3,4,6,8,12,24\}$. As $N$ is normal, it is the union of conjugacy classes. $N$ must also contain the identity. We have that $4 = 1 + 3$ and $12 = 1 + 3 + 8$, and other divisors cannot be represented as a sum of numbers from the class formula. Hence, $|N| \in \{1,4,12,24\}$. 

\paragraph{7.} We first prove that $S_{n}$ is generated by $2$-cycles of the form $(k,k+1)$. Let $(a \ b)$ be a $2$-cycle in $S_{n}$. Without loss of generality, suppose $a > b$. We have that
$$(a \ b) = (b \ b+1)(b+1 \ b+2)...(a-2 \ a-1)(a-1 \ a)(a-2 \ a-1)...(b+1 \ b+2)(b \ b+1)$$
By Lemma 4.11, the set of transpotitions generate $S_{n}$, hence, $2$-cycles of the form $(k \ k+1)$ must generate $S_{n}$. Let $(1 \ 2), (1 \ 2 \ ... \ n) \in S_{n}$. We have that 
$$(k+1 \ k+2) = (1 \ 2 \ ... \ n)^{k}(1 \ 2)(1 \ 2 \ ... \ n)^{-k}$$
It follows that $S_{n} = \langle (1 \ 2), (1 \ 2 \ ... \ n)\rangle$. 

\paragraph{8.} Let $n \geq 2$. Let $H$ be the subgroup of $S_{n}$ fixing $1$. We have that for all $\sigma \in H$, $\sigma$ is a permutation of $A = \{2, 3, ..., n\}$. We can see that $H = S_{|A|}$. Therefore, $H \cong S_{n-1}$. Let $K$ be a subgroup of $S_{n}$ properly containing $H$. 

\paragraph{9.}

\paragraph{10.} Let $\sigma \in S_{n}$ be a permutation of type $[\lambda_{1}, ..., \lambda_{r}]$. Let $a_{1}, a_{2}, ..., a_{n}$ denote the multiplicities of $1,2,3,...,n$ respectively that appear in $[\lambda_{1},...,\lambda_{r}]$. We have that the length of the conjugacy class that contains $\sigma$ is
$$\frac{n!}{\prod_{i=1}^{n} (b_{i}!)(i^{b_{i}})}$$
where $b_{i} = 1$ if $a_{i} = 0$ and $b_{i} = a_{i}$ otherwise.

\paragraph{11.} Let $p$ be a prime integer. By the previous exercise, there are $(p-1)!$ $p$-cycles in $S_{p}$, which are all of order $p$. We have that each $p$-Sylow subgroup of $S_{p}$ contains $p-1$ elements of order $p$. Each $p$-Sylow subgroup has nontrivial intersection with eachother, hence, the number of $p$-Sylow subgroups are $(p-1)!/(p-1) = (p-2)!$. By the Third Sylow Theorem, $(p-2)! \equiv 1 \mod p$, hence, $(p-1)! \equiv 1 \mod p$.

\paragraph{12.}
\paragraph{13.}

\paragraph{14.} Let $n \geq 4$. We have that $Z(A_{n})$ is trivial for all $n \geq 5$ as $A_{n}$ is simple. Let $\sigma \in A_{4}$ be a nontrivial element. Suppose that $\sigma(a) = b$. Choose $c,d \neq a,b$. We have that $\sigma (b \ c \ d)(a) = b$ and $(b \ c \ d)\sigma(a) = c$. Hence, the center of $A_{4}$ is trivial.

\paragraph{15.}
\paragraph{16.}

\paragraph{17.} Possible types for elements of $A_{4}$ are $[1,1,1,1], [2,2], [3,1]$. There is $1$ element of type $[1,1,1,1]$ in $A_{4}$. There are $4!/(2\cdot 2 \cdot 2) = 3$ elements of type $[2,2]$ in $A_{4}$. There are $4!/3 = 8$ elements of type $[3,1]$ in $A_{4}$. Hence, the class formula of $A_{4}$ is 
$$|A_{4}| = 12 = 1 + 3 + 8$$
We can deduce that there is no normal subgroup of order $6$ in $A_{4}$ as you cannot form $6$ from $1,3,8$. 

\paragraph{18.} Let $n \geq 5$, and let $H$ be a subgroup of $A_{n}$ such that $[A_{n}:H] < n$. The action of $A_{n}$ on $A_{n}/H$ induces a homomorphism $\varphi:A_{n} \to S_{[A_{n}:H]}$. As $|A_{n}| = n!/2 > [A_{n}:H]! = |S_{[A_{n}:H]}|$, $\varphi$ cannot be injective. As $A_{n}$ is simple, $\ker\varphi$ must be $A_{n}$. This implies that $xH = H$ for all $x \in A_{n}$, which means that $H = A_{n}$. Therefore, if $H$ is a subgroup of $A_{n}$, $[A_{n}:H] \geq 5$. For $n \geq 3$, $A_{n}$ contains $A_{n-1}$ and $A_{n-1}$ is a nontrivial proper subgroup of $A_{n}$ of index $n$. 

\paragraph{19.} Let $n \geq 5$, and let $C$ be a set with $|C| = k < n$. Let $\rho$ be an action of $A_{n}$ on $C$. $\rho$ induces a homomorphism $\varphi: A_{n} \to S_{k}$. As $|A_{n}| = n!/2 > k! = |S_{k}|$, $\varphi$ cannot be injective. Hence, $\ker\varphi$ is nontrivial. As $A_{n}$ is simple, $\ker\varphi = A_{n}$. Therefore, $\rho$ is the trivial action. Let $N$ be the normal subgroup of $A_{4}$. We have that $A_{n}/N$ is of order $3$, and $A_{4}$ acts nontrivially on $A_{n}/N$. Let $\rho$ be an action of $A_{4}$ on a set $S$ where $|S| = 2$. We have that $\rho$ induces a homomorphism $\varphi:A_{4} \to S_{2}$. We have that $\ker\varphi = 1, N, A_{n}$. As $12 > 2$, $\varphi$ cannot be injective, hence, $\ker\varphi = N$ or $\ker\varphi = A_{n}$. As $3 > 2$, we cannot have $\ker\varphi = N$ as $A_{n}/N$ would be isomorphic to a subgroup of $S_{2}$. Hence, $\ker\varphi = A_{n}$. $\rho$ is then the trivial action. 

\paragraph{20.}

\paragraph{21.} We have that $A_{6}$ has the class formula
$$320 = 1 + 40 + 40 + 45 + 90 + 144$$
The only divisors of $320$ that you can form from numbers in the class formula that contain $1$ are $1$ and $320$. Hence, $A_{6}$ cannot contain a nontrivial normal subgroup. Hence, $A_{6}$ is simple.

\paragraph{22.}

\subsection*{4.5 - Products of Groups}

\paragraph{1.}
\paragraph{2.}
\paragraph{3.}

\paragraph{4.} Consider the sequence
$$0 \longrightarrow \mathbb{Z} \overset{2}{\longrightarrow} \mathbb{Z} \longrightarrow \mathbb{Z}/\mathbb{Z} \longrightarrow 0$$
We have that $\varphi:\mathbb{Z} \to \mathbb{Z}$ given by $\varphi(x) = 2x$ is injective with image $2\mathbb{Z}$. Note this is the kernel of the canonical projection from $\mathbb{Z}$ to $\mathbb{Z}/2\mathbb{Z}$. We also have that the canonical projection is surjective. Therefore, the sequence is exact. We have that every subgroup $H$ of $\mathbb{Z}$ such that $H \cap \mathbb{Z} = \{0\}$ must be the trivial subgroup, hence, the sequence does not split. 

\paragraph{5.}

\paragraph{6.} Let $N,H$ be groups and let $\theta:H \to \text{Aut}_{\mathsf{Grp}}(N), h \mapsto \theta_{h}$ be a homomorphism. Define an operation $\bullet_{\theta}$ on the set $N \times H$ as follows: for $n_{1},n_{2} \in N$ and $h_{1}, h_{2} \in H$, let 
$$(n_{1}, h_{1}) \bullet_{\theta} (n_{2}, h_{2}) := (n_{1}\theta_{h_{1}}(n_{2}), h_{1}h_{2})$$
Let $n_{1}, n_{2}, n_{3} \in N$ and $h_{1}, h_{2}, h_{3} \in H$, we have that
\begin{align*}[(n_{1},h_{1}) \bullet_{\theta} (n_{2},h_{2})]\bullet_{\theta} (n_{3},h_{3}) &= (n_{1}\theta_{h_{1}}(n_{2}),h_{1}h_{2}) \bullet_{\theta} (n_{3},h_{3}) \\
&= (n_{1}\theta_{h_{1}}(n_{2})\theta_{h_{1}h_{2}}(n_{3}),h_{1}h_{2}h_{3}) \\
&= (n_{1}\theta_{h_{1}}(n_{2})\theta_{h_{1}}(\theta_{h_{2}}(n_{3})),h_{1}h_{2}h_{3}) \\
&= (n_{1}\theta_{h_{1}}(n_{2}\theta_{h_{2}}(n_{3})),h_{1}h_{2}h_{3}) \\
&= (n_{1},h_{1})\bullet_{\theta}(n_{2}\theta_{h_{2}}(n_{3}),h_{2}h_{3}) \\
&= (n_{1},h_{1})\bullet_{\theta}[(n_{2},h_{2})\bullet_{\theta}(n_{3},h_{3})] \\
\end{align*}
Let $n \in N$ and $h \in H$. Note
$$(n,h) \bullet_{\theta} (1_{N},1_{H}) = (n\theta_{h}(1_{N}), h1_{H}) = (n1_{N},h) = (n,h)$$
$$(1_{N},1_{H})\bullet_{\theta} (n,h) = (1_{N}\theta_{1_{H}}(n),1_{H}h) = (\text{Id}(n),h) = (n,h)$$
Then, $(1_{H}, 1_{N})$ is the identity element in $N \rtimes_{\theta} H$. Finally,
$$(n,h) \bullet_{\theta} (\theta_{h^{-1}}(n^{-1}),h^{-1}) = (n\theta_{h}(\theta_{h^{-1}}(n^{-1})), hh^{-1}) = (nn^{-1},1_{H}) = (1_{N}, 1_{H})$$
$$(\theta_{h^{-1}}(n^{-1}),h^{-1}) \bullet_{\theta} (n,h) = (\theta_{h^{-1}}(n^{-1})\theta_{h^{-1}}(n),h^{-1}h) = \theta_{h^{-1}}(n^{-1}n),h^{-1}h) = (\theta_{h^{-1}}(1_{N}), 1_{H}) = (1_{N}, 1_{H})$$
Therefore, $N \rtimes_{\theta} H$ has inverses. It follows that $N \rtimes_{\theta} H$ is a group. 

\paragraph{7.} 

\paragraph{8.} Let $N, H$ be solvable groups, and let $G = N \rtimes_{\theta} H$. We have that $N \times 1_{H}$ is normal in $G$ by Proposition 5.10, and we also have that $G/(N \times 1_{H}) \cong 1_{N} \times H \cong H$. Note $N \cong N \times 1_{H}$, and so $N \times 1_{H}$ is solvable. We also have that $G/(N \times 1_{H})$ is solvable as $1_{N} \times H \cong H$ is solvable. Therefore, $G$ is solvable. 

\paragraph{9.} Let $N,H$ be groups, and let $G = N \rtimes_{\theta} H$ be an abelian group. Let $n_{1},n_{2} \in N$ and $h_{1},h_{2} \in H$. As $G$ is abelian, 
\begin{align*}
(1_{N},1_{H}) &= [(n_{1},h_{1}), (n_{2},h_{2})] \\ 
&= ((n_{1},h_{1}) \bullet_{\theta} (n_{2},h_{2}))\bullet_{\theta}((n_{2},h_{2})\bullet_{\theta}(n_{1},h_{1}))^{-1} \\
&= (n_{1}\theta_{h_{1}}(n_{2}),h_{1}h_{2}) \bullet_{\theta}(n_{2}\theta_{h_{2}}(n_{1}), h_{2}h_{1})^{-1} \\
&=(n_{1}\theta_{h_{1}}(n_{2}),h_{1}h_{2}) \bullet_{\theta} (\theta_{(h_{2}h_{1})^{-1}}((n_{2}\theta_{h_{2}}(n_{1}))^{-1}), (h_{2}h_{1})^{-1}) \\
&=(n_{1}\theta_{h_{1}}(n_{2})\theta_{h_{1}h_{2}}(\theta_{(h_{2}h_{1})^{-1}}((n_{2}\theta_{h_{2}}(n_{1}))^{-1})),h_{1}h_{2}(h_{2}h_{1})^{-1}) \\
&=(n_{1}\theta_{h_{1}}(n_{2})\theta_{h_{1}h_{2}(h_{2}h_{1})^{-1}}((n_{2}\theta_{h_{2}}(n_{1}))^{-1}),[h_{1},h_{2}]) \\
&=(n_{1}\theta_{h_{1}}(n_{2})\theta_{[h_{1},h_{2}]}(\theta_{h_{2}^{-1}}(n_{1})n_{2}^{-1}),[h_{1},h_{2}]) \\
&=(n_{1}\theta_{h_{1}}(n_{2})\theta_{[h_{1},h_{2}]h_{2}^{-1}}(n_{1})\theta_{[h_{1},h_{2}]}(n_{2}^{-1}),[h_{1},h_{2}]) \\
\end{align*}
Thus, $[h_{1},h_{2}] = 1_{H}$. We have that then 
$$1_{N} = n_{1}\theta_{h_{1}}(n_{2})\theta_{[h_{1},h_{2}]h_{2}^{-1}}(n_{1})\theta_{[h_{1},h_{2}]}(n_{2}^{-1}) = n_{1}\theta_{h_{1}}(n_{2})\theta_{h_{2}^{-1}}(n_{1})n_{2}^{-1}$$
Therefore, $n_{1}\theta_{h_{1}}(n_{2}) = n_{2}\theta_{h_{2}}(n_{1})$. Then, for any $n \in N, h \in H$, $\theta_{h}(n) = 1_{N}\theta_{h}(n) = n\theta_{h}(1_{N}) = n1_{N} = n$. $\theta$ is then the trivial morphism. Hence, $G \cong N \times H$. 

\paragraph{10.} Let $N$ be a normal subgroup of a finite group $G$, and assume $|N|$ and $|G/N|$ are relatively prime. Suppose in $G$ there exists a subgroup $H$ such that $|H| = |G/N|$. We have that $N \cap H$ is a subgroup of both $N$ and $H$, hence, $|N \cap H|$ must divide both $|N|$ and $|H|$. As $|H|$ and $|N|$ are relatively prime, $|N \cap H| = 1$, thus $N \cap H$ is trivial. As $N$ is normal in $G$, $NH$ is a subgroup of $G$ with order 
$$|NH| = \frac{|N||H|}{|N \cap H|} = \frac{|N||H|}{1} = |N||H| = |N||G/N| = |G|$$
Thus, $G = NH$. By Proposition 5.11, $G \cong N \rtimes H$.

\paragraph{11.} Let $n > 0$. We have that $D_{2n} = \langle x,y \mid x^{n} = y^{2} = (xy)^{2} = 1\rangle$. We have that $\langle x \rangle$ is normal in $D_{2n}$ as $\langle x \rangle$ has index $2$. If $n$ is odd, then $\langle x \rangle \cap \langle y \rangle = 1$ by order considerations. If $n$ is even, then the only possible way for $\langle x \rangle \cap \langle y \rangle$ is nontrivial is that if $x^{n/2} = y$. This implies $x^{n/2 + 1} = xy$ and so $x^{n+2} = 1$, which means $x^{2} = 1$. For $D_{4}$, $\langle x \rangle \cap \langle y \rangle$ is still trivial. In all cases, $\langle x \rangle \cap \langle y \rangle = 1$. As $|\langle x \rangle| = n$ and $|\langle y \rangle| = 2$, $D_{2n} = \langle x\rangle\langle y \rangle$. By Proposition 5.11, $D_{2n} \cong \langle x \rangle \rtimes \langle y \rangle \cong C_{n} \rtimes C_{2}$. 

\paragraph{12.} 
\paragraph{13.}
\paragraph{14.}
\paragraph{15.}
\paragraph{16.}
\paragraph{17.}

\subsection*{4.6 - Finite Abelian Groups}

\paragraph{1.}
\paragraph{2.}

\paragraph{3.} Let $G$ be a noncommutative group of order $p^{3}$ where $p$ is prime. As $G$ is a noncommutative $p$-group, $1 < |Z(G)| < p^{3}$. Hence, $|Z(G)| \in \{p, p^{2}\}$ by Lagranges Theorem. Suppose that $|Z(G)| = p^{2}$. We then have that $G/Z(G)$ is of order $p$ and is then cyclic. Thus, $G$ is abelian, which is a contradiction. We must have that $|Z(G)| = p$, and so $Z(G) \cong \mathbb{Z}/p\mathbb{Z}$. As $Z(G)$ is of order $p$, we must have that $G/Z(G)$ is of order $p^{2}$. As $G$ is noncommutative, $G/Z(G)$ cannot be abelian, thus, $G/Z(G) \cong \mathbb{Z}/p\mathbb{Z} \times \mathbb{Z}/p\mathbb{Z}$. 

\paragraph{4.} Abelian groups of order $400 = 2^{4} \cdot 5^{2}$ are 
$$\mathbb{Z}/400\mathbb{Z}, \ \mathbb{Z}/2\mathbb{Z} \times \mathbb{Z}/200\mathbb{Z}, \ \mathbb{Z}/4\mathbb{Z} \times \mathbb{Z}/100\mathbb{Z}, \ \mathbb{Z}/2\mathbb{Z} \times \mathbb{Z}/2\mathbb{Z} \times \mathbb{Z}/100\mathbb{Z}, \ \mathbb{Z}/2\mathbb{Z} \times \mathbb{Z}/2\mathbb{Z} \times \mathbb{Z}/2\mathbb{Z} \times \mathbb{Z}/50\mathbb{Z},$$
$$\mathbb{Z}/2\mathbb{Z} \times \mathbb{Z}/2\mathbb{Z} \times \mathbb{Z}/10\mathbb{Z} \times \mathbb{Z}/10\mathbb{Z} , \ \mathbb{Z}/2\mathbb{Z} \times \mathbb{Z}/10\mathbb{Z} \times \mathbb{Z}/20\mathbb{Z}, \ \mathbb{Z}/5\mathbb{Z} \times \mathbb{Z}/80\mathbb{Z}, \ \mathbb{Z}/10\mathbb{Z} \times \mathbb{Z}/40\mathbb{Z}, \ \mathbb{Z}/20\mathbb{Z} \times \mathbb{Z}/20\mathbb{Z}$$

\paragraph{5.}
\paragraph{6.}

\paragraph{7.} Let $p > 0$ be a prime integer, $G$ a finite abelian group, and denote $\rho:G \to G$ the homomorphism defined by $\rho(g) = pg$. Let $A$ be a finite abelian group such that $pA = 0$. We have that $A \cong \oplus_{i=1}^{n}\mathbb{Z}/d_{i}\mathbb{Z}$ for some $n$ and $d_{1},...,d_{n} \in \mathbb{Z}$. Then, 
$$0 = pA = p\bigoplus_{i=1}^{n}\mathbb{Z}/d_{i}\mathbb{Z} = \bigoplus_{i=1}^{n}p\mathbb{Z}/d_{i}\mathbb{Z}$$
Hence, $0 = p\mathbb{Z}/d_{i}\mathbb{Z}$ for all $d_{i}$. We must have that every nontrivial $x \in \mathbb{Z}/d_{i}\mathbb{Z}$ has order which divides $p$. Thus, $|x| = p$. It follows that $d_{i} = p$ for all $i$. Therefore, 
$$A \cong \bigoplus_{i=1}^{n}\mathbb{Z}/p\mathbb{Z}$$
for some $n$. Let $x \in \ker \rho$. We have that $px = 0$ by definition. Hence, $p\ker\rho = 0$. Let $x \in \coker\rho$. We have that $x$ corresponds to some $x' + \im\rho$ as $\coker\rho \cong G/\im\rho$. We have that $p(x' + \im\rho) = px' + \im\rho = \rho(x') + \im\rho = \im\rho$. Hence, $p\coker\rho = 0$. Note that 
$$\ker\rho \cong \bigoplus_{i=1}^{n}\mathbb{Z}/p\mathbb{Z}, \ \coker\rho \cong \bigoplus_{i=1}^{m}\mathbb{Z}/p\mathbb{Z}$$
for some $m,n$. By the First Isomorphism Theorem, $G/\ker\rho \cong \im\rho$, hence, $|G/\ker\rho| = |\im\rho|$. Then,
$$|G/\ker\rho| = |\im\rho| \implies |G|/|\ker\rho| = |\im\rho| = |G|/|\im\rho| = |\ker\rho| \implies |G/\im\rho| = |\ker\rho| \implies |\coker\rho| = |\ker\rho|$$
It follows that $m = n$. Therefore, 
$$\ker\rho \cong \bigoplus_{i=1}^{n}\mathbb{Z}/p\mathbb{Z} \cong \coker\rho$$
Let $H$ be a subgroup of order $p$ in $G$. Let $x \in H$. We have that $\rho(x) = px = 0$ as $H$ is of order $p$. Therefore, $H \subseteq \ker\rho$. Now, suppose that $H$ is a subgroup of index $p$ in $G$. Let $x \in \im\rho$. We have that $x = py$ for some $y$. We have that $x + \im\rho = py + \im\rho = p(y + \im\rho) = \im\rho$ as $G/H$ has order $p$. Therefore, $x \in \im\rho$ and $\im\rho \subseteq H$. Now, let $G_{p}$ be the set of subgroups of $G$ of order $p$. For every $H \in G_{p}$, we have that $H$ is contained in the kernel of $\rho$. Note that $\ker\rho \cong \coker\rho$, hence, every element of $G_{p}$, $H$, corresponds to a unique subgroup of $\coker\rho$, $H'$. We have that $H'$ then corresponds to some $K/\im\rho$ in $G/\im\rho$. As $H$ is of order $p$, we must have that $K$ is of index $p$. There is then a correspondance between a subgroup of order $p$ and a subgroup of index $p$. Therefore, the number of subgroups of order $p$ in $G$ is equal to the number of subgroups of index $p$ in $G$.  

\paragraph{8.}
\paragraph{9.}

\paragraph{10.} Let $G$ be a finite group of order $n$ and let $G^{\lor}:= \text{Hom}_{\mathsf{Grp}}(G, \mathbb{C}^{*})$. Let $\sigma \in \text{Hom}_{\mathsf{Grp}}(G, \mathbb{C}^{*})$ and let $g \in G$. We have that $\sigma(g)^{n} = \sigma(g^{n}) = \sigma(1_{G}) = 1$. Hence, $\sigma(g)$ is a root of $1$ in $\mathbb{C}$. The image of every $\sigma \in G^{\lor}$ is then a root of $1$ in $\mathbb{C}$. Let $C_{n}$ be the cyclic group of order $n$. Let $\sigma \in C_{n}^{\lor}$ and let $x \in C_{n}$ be a generator. We have that $\sigma(x^{k}) = \sigma(x)^{k}$ for all $0 \leq k < n$, hence, $\sigma$ is completely determined by $\sigma(x)$. We have that $\sigma(x)$ is a root of $1$ in $\mathbb{C}$ and is the solution to $z^{n} = 1$. There are $n$ possible $z$'s that satisfy this equation, thus, there are $n$ elements of $C_{n}^{\lor}$. Let $z_{0} = \exp(2\pi i/n)$ and let $\sigma_{0}:C_{n} \to \mathbb{C}^{*}$ be the homomorphism sending $x$ to $z_{0}$. We have that $\sigma_{0}^{k}(x) = \exp(2\pi i k/n)$ for $0 \leq k < n$ and we note that $\sigma_{0}^{i}(x) \neq \sigma_{0}^{j}(x)$ for all $0 \leq i < j < n$. Therefore, $C_{n}^{\lor}$ is generated by $\sigma_{0}$. It follows that $C_{n} \cong C_{n}^{\lor}$. Let $G_{1}, G_{2}$ be groups and let $K$ be an abelian group. We set to prove that
$$\text{Hom}_{\mathsf{Grp}}(G_{1} \times G_{2}, K) \cong \text{Hom}_{\mathsf{Grp}}(G_{1}, K) \times \text{Hom}_{\mathsf{Grp}}(G_{2}, K)$$
Let $Z$ be a group and $f_{1}:Z \to \text{Hom}(G_{1}, K), f_{2}:Z \to \text{Hom}(G_{2},K)$ be group homomorphism. Suppose that there is a homomorphism $\psi$ such that the following diagram commutes 
\[\begin{tikzcd}[ampersand replacement=\&]
	\&\&\& {\text{Hom}_{\mathsf{Grp}}(G_{2},K)} \\
	Z \&\& {\text{Hom}_{\mathsf{Grp}}(G_{1} \times G_{2},K)} \\
	\&\&\& {\text{Hom}_{\mathsf{Grp}}(G_{1},K)}
	\arrow["{f_{2}}"{description}, curve={height=-18pt}, from=2-1, to=1-4]
	\arrow["{f_{1}}"{description}, curve={height=18pt}, from=2-1, to=3-4]
	\arrow["\psi"{description}, from=2-1, to=2-3]
	\arrow["{\pi_{2}}"{description}, from=2-3, to=1-4]
	\arrow["{\pi_{1}}"{description}, from=2-3, to=3-4]
\end{tikzcd}\]
where $\pi_{1}$ is defined by $\sigma(g_{1},g_{2}) \mapsto \sigma(g_{1},0_{G_{2}})$ and $\pi_{2}$ is defined by $\sigma(g_{1},g_{2}) \mapsto \sigma(0_{G_{1}}, g_{2})$ for all $(g_{1},g_{2}) \in G_{1} \times G_{2}$. Let $z \in Z$ and $(g_{1},g_{2}) \in G_{1} \times G_{2}$. By commutativity, 
$$[f_{1}(z)](g_{1}) = [(\pi_{1} \circ \psi)(z)](g_{1}) = [\psi(z)](g_{1}, 0_{G_{2}})$$
$$[f_{2}(z)](g_{2}) = [(\pi_{2} \circ \psi)(z)](g_{2}) = [\psi(z)](0_{G_{1}}, g_{2})$$
Hence, for all $(g_{1},g_{2}) \in G_{1} \times G_{2}$ and $z\in Z$, we have that
$$[\psi(z)](g_{1},g_{2}) = [\psi(z)](g_{1},0_{G_{2}}) + [\psi(z)](0_{G_{1}}, g_{2}) = [f_{1}(z)](g_{1}) + [f_{2}(z)](g_{2})$$
This shows that $\psi$ is unique. We also note that $\psi$ is a homomorphism. Therefore, $\text{Hom}_{\mathsf{Grp}}(G_{1} \times G_{2}, K)$ satisfies the universal property for the product of $\text{Hom}_{\mathsf{Grp}}(G_{1}, K)$ and $\text{Hom}_{\mathsf{Grp}}(G_{2}, K)$. Hence, 
$$\text{Hom}_{\mathsf{Grp}}(G_{1} \times G_{2}, K) \cong \text{Hom}_{\mathsf{Grp}}(G_{1}, K) \times \text{Hom}_{\mathsf{Grp}}(G_{2}, K)$$
We note that
$$(G \times K)^{\lor} = \text{Hom}_{\mathsf{Grp}}(G \times K, \mathbb{C}^{*}) \cong \text{Hom}_{\mathsf{Grp}}(G, \mathbb{C}^{*}) \times \text{Hom}_{\mathsf{Grp}}(K,\mathbb{C}^{*}) = G^{\lor} \times K^{\lor}$$
Let $G$ be a finite abelian group. We have that $G \cong \oplus_{i=1}^{n}C_{d_{i}}$ for some integers $d_{i}$. Via induction,
$$G^{\lor} \cong \qty(\bigoplus_{i=1}^{n}C_{d_{i}})^{\lor} \cong \bigoplus_{i=1}^{n}C_{d_{i}}^{\lor} \cong \bigoplus_{i=1}^{n}C_{d_{i}} \cong G$$

\paragraph{11.}
\paragraph{12.}
\paragraph{13.}
\paragraph{14.}

\paragraph{15.} Let $G$ be a finite abelian group and let $a \in G$ be an element of maximal order in $G$. We have that
$$G \cong \bigoplus_{i=1}^{n}\mathbb{Z}/d_{i}\mathbb{Z}$$
where $d_{1} \mid ... \mid d_{n}$. $a$ must be of order $d_{n}$ as the representation is unique, and any $b \in G$ must be of order $d_{i}$, hence, $|b|$ divides $|a|$. 

\paragraph{16.}

\section*{V - Irreducibility and Factorisation in Integral Domains}
\subsection*{5.1 - Chain Conditions and Existence of Factorisations}
\paragraph{1.} Let $R$ be a Noetherian ring and let $I$ be an ideal of $R$. Let $J/I$ be an ideal of $R/I$. We have that $J$ is an ideal of $R$, and since $R$ is Noetherian, $J = (j_{1}, ..., j_{n})$ for some $j_{1},...,j_{n}$. Let $x + I \in J/I$. We have that 
$$x + I = (x_{1}j_{1} + ... + x_{n}j_{n}) + I = (x_{1}j_{1} + I) ... + (x_{n}j_{n} + I) = x_{1}(j_{1} + I) + ... + x_{n}(j_{n} + I)$$
for some $x_{1},...,x_{n} \in R$. Hence, $J/I = (j_{1} + I,..., j_{n} + I)$. It follows that $R/I$ is Noetherian. 

\paragraph{2.} Let $R$ be a commutative ring such that $R[x]$ is Noetherian. We have that $(x)$ is an ideal of $R[x]$, and, by the previous exercise, $R \cong R[x]/(x)$ is Noetherian. Therefore, $R$ is a Noetherian ring. 

\paragraph{3.}

\paragraph{4.} Let $R$ be the ring of real-valued continuous functions on the interval $[0,1]$. Let $I_{n} = \{f \in R \mid \forall x \in [0,1/n], f(x) = 0\}$. Let $f,g \in I_{n}$, and let $x \in [0,1/n]$. Then, $(f - g)(x) = f(x) - g(x) = 0 - 0 = 0$. Hence, $f-g \in I_{n}$. Furthermore, let $r \in R$ and $f \in I_{n}$. Let $x \in [0,1/n]$. Then, $(rf)(x) = r(x)f(x) = r(x)0 = 0$. Hence, $rf \in I_{n}$. It follows that $I_{n}$ is an ideal for all $n$. We note that each $I_{n}$ is properly contained in $I_{n+1}$ as for example $f$ defined by $f(x) = 0$ for $x \in [0,1/(n+1)]$ and $f(x) = x - (1/(n+1))$ otherwise is in $I_{n+1}$, but not in $I_{n}$. Hence, the sequence $I_{1} \subseteq I_{2} \subseteq I_{3} \subseteq ...$ cannot stabilise. $R$ cannot then be Noetherian. 

\paragraph{5.}

\paragraph{6.} Let $I$ be an ideal of $R[x]$, and let $A = \{0\} \cup \{a \in R \mid a \text{ is the leading coefficient of some element } f \in I\}$. Let $a,b \in A$ such that $a \neq b$. We have that there exists a polynomial $f$ of degree $n$ with leading coefficient $a$ and a polynomial $g$ of degree $m$ with leading coefficient $m$. Without loss of generality, suppose that $m \leq n$. We have that $x^{n-m}g$ is a polynomial of degree $n$ with leading coefficient $a$. Furthermore, as $a \neq b$, $f - g$ is a polynomial with leading coefficient $a-b$ that is in $I$ as $I$ is an ideal. Hence, $a-b \in A$. Note that $0 \in A$. We have that $(A,+)$ is a subgroup of $(R,+)$. Let $r \in R$ and $a \in A$ such that $ra \neq 0$. We have that there is a $f \in I$ such that $f$ has leading coefficient $a$. As $I$ is an ideal and $r \in R[x]$, $rf \in I$. $rf$ has leading coefficient $ra$, and so $ra \in A$. We then have that $A$ is an ideal of $R$. 

\paragraph{7.}
\paragraph{8.}
\paragraph{9.}

\paragraph{10.} Let $R$ be an Artinian ring, and let $I$ be an ideal of $R$. Let $J_{1} \supseteq J_{2} \supseteq J_{3} \supseteq ...$ be a decending chain of ideal in $R/I$. By the correspondance theorem, there exists ideals $A_{1}, A_{2}, ...$ such that $A_{1}/I, A_{2}/I, ...$ are isomorphic to $J_{1}, J_{2}, ...$, respectively. We have that the chain $A_{1} \supseteq A_{2} \supseteq ...$ must stabilise as $R$ is Artinian. The chain must stabilise to some $A_{n}$. Hence, $A_{n}/I = A_{n+1}/I = ...$, therefore, the chain $J_{1} \supseteq J_{2} \supseteq ...$ must stabilise too. It follows that $R/I$ is Artinian. Now, further suppose that $R$ is an Artinian integral domain. Let $r \in R$ be a non trivial element in $R$. The sequence $(r) \supseteq (r^{2}) \supseteq (r^{3}) \supseteq ...$ must stabilise, therefore, $(r^{n}) = (r^{n+1})$ for some $n \in \mathbb{Z}$. We have that $r^{n} = xr^{n+1}$ for some $x \in R$, which means that $r^{n}(1 - rx) = 0$. As $R$ is an integral domain, if $r^{n} = 0$, then $r = 0$, which contradicts our assumption, and if $1 - rx$, then $r$ is a unit. As $r$ was arbitrary, $R$ must be a field. Finally, let $R$ be an Artinian ring, and let $P$ be a prime ideal of $R$. We have that $R/P$ is Artinian, and $R/P$ must be an integral domain. As $R/P$ is an Artinian integral domain, it must be a field. Hence, $P$ is a maximal ideal. We conclude that Artinian rings have Krull dimension $0$. 

\paragraph{11.} 

\paragraph{12.} Let $R$ be an integral domain. Suppose that $a \in R$ is irreducible. Suppose that there is a proper principle ideal $(r)$ such that $(r)$ properly contains $(a)$. We have that $a \in (r)$ and so $a = rx$ for some $x \in R$. As $a$ is irreducible, $r$ is a unit or $x$ is a unit. If $r$ is a unit, $(r) = R$. If $x$ is a unit, then $r = ax^{-1} \in (a)$ and so $(a) = (r)$. It follows that $(a)$ is maximal among proper princple ideals of $R$. For the converse, suppose that $(a)$ is maximal among proper principle ideals of $R$. Suppose that $a = xy$ for some $x,y \in R$. We have that $a \in (y)$ as $a = xy$, and so $(a)$ is contained in $(y)$. By maximality, $(y) = R$ or $(y) = (a)$. If $(y) = R$, then $y$ is a unit. If $(y) = (a)$, then $y = ka$ for some $k \in R$. Then, $a = xy = xka$, and so $(1 - xk)a = 0$. As $R$ is an integral domain, $1 - xk = 0$ as $a \neq 0$. As $1 - xk = 0$, $x$ is then a unit. Therefore, $a$ is irreducible in $R$. 

\paragraph{13.} We have that $\mathbb{Z}$ is an integral domain, that is also a PID. By the previous exercise, if $p \in \mathbb{Z}$ is irreducible, then $(p)$ is maximal among proper principle ideals, and as $\mathbb{Z}$ is a PID, $(p)$ is maximal, hence, prime. $p$ is then prime. For the converse, if $p \in \mathbb{Z}$ is prime. Suppose that $p = ab$ for some $a,b \in \mathbb{Z}$. As $p = ab$, we must have that $p \mid ab$. BY primality, $p \mid a$ or $p \mid b$. Without loss of generality, assume $p \mid a$. Then, $a = px$ for some $x \in \mathbb{Z}$. We have that $p = ab = pxb$, and so $p - pxb = 0$. Then, $p(1 - xb) = 0$. As $\mathbb{Z}$ is an integral domain and $p \neq 0$, we have that $1 - xb = 0$, hence, $b$ is a unit. Therefore, $p$ is irreducible in $\mathbb{Z}$. 

\paragraph{14.} Let $R$ be a commutative ring and let $a,b \in R$. Suppose that there exists $a+(b) \in R/(b)$ is prime. Let $x+(a),y+(a) \in R/(a)$ such that $b + (a) \mid (x + (a))(y + (a))$. Then, $b + (a) \mid xy + (a)$. There then exists a $k+(a) \in R/(a)$ such that $xy + (a) = (k+(a))(b + (a)) = kb + (a)$. So, $xy - kb \in (a)$. Thus, there is a $q \in R$ such that $xy - kb = qa$, and we have that $xy - qa = kb$. We must have that $xy - qa \in (b)$ and $xy + (b) = qa + (b)$. Then, $a + (b) \mid (x + (b))(y + (b))$. As $a + (b)$ is prime, without loss of generality, assume $a + (b) \mid x + (b)$. There is then an $s + (b) \in R/(b)$ such that $x + (b) = (a + (b))(s + (b)) = as + (b)$. We then have that $x - as \in (b)$. There is then a $z \in R$ such that $x - as = zb$, and so $x - zb = as$, in which then $x - zb \in (a)$. We have that $x + (a) = zb + (a)$. Finally, $b+(a) \mid x + (a)$. Therefore, $b+(a)$ is prime. 

\paragraph{15.} Identify $S = \mathbb{Z}[x_{1},...,x_{n}]$ in a natural way with a subring of $R = \mathbb{Z}[x_{1},x_{2},x_{3},...]$. Suppose that $f \in S$ and nontrivial, and let $g \in R$ such that $(f) \subseteq (g)$ in $R$. Suppose that $g$ can be written as $g = p + r$ where $p$ contains no terms with the variables $x_{1}, ..., x_{n}$ and $r \in S$. As $f \in (f) \subseteq (g)$, we must have that $f = k\cdot g$ for some $k \in R$. Then, $f = kp + kr$. As $f \in S$ and $p \not\in S$, we must have that $kp = 0$. We must have that $k = 0$ or $p = 0$. Then, $f = kr$. As $r \in S$, we must also have that $k \in S$. If $k = 0$, then $f = 0$, which is a contradiction. Hence, $p = 0$. It follows that $g = r \in S$. Let 
$$(f_{1}) \subseteq (f_{2}) \subseteq (f_{3}) \subseteq ...$$
be a chain of ascending principle ideals in $R$. We have that $f_{1}$ is contained in some subring of $R$ corresponding to $S =\mathbb{Z}[x_{1},...,x_{m}]$ for some $m$. We then have that $f_{2}, f_{3}, ... \in S$. By Hilberts Basis Theorem, $S$ is Noetherian as $\mathbb{Z}$ is Noetherian. Then, the chain must stabilise. Therefore, $R$ is Noetherian, and so $R$ is a domain with factorisations.

\paragraph{16.}

\paragraph{17.} Let $\mathbb{Z}[\sqrt{-5}]$ be the subring of $\mathbb{C}$ defined by $\mathbb{Z}[\sqrt{-5}] = \{a + ib\sqrt{5} \mid a,b \in \mathbb{Z}\}$. 

\subparagraph{(i)} Define $\varphi:\mathbb{Z}[t] \to \mathbb{Z}[\sqrt{-5}]$ by the rule $f \mapsto f(\sqrt{-5})$. Let $f \in \ker\varphi$. If $\deg{f} \geq 2$, then $f = g(t^{2}+5) + r$ for some $g,r \in \mathbb{Z}[t]$ where $\deg{r} < 2$. As $f \in \ker\varphi$, we have that $0 = f(\sqrt{-5}) = g(\sqrt{-5})0 + r(\sqrt{-5}) = r(\sqrt{-5})$. Hence, $r(\sqrt{-5}) = 0$. As $\deg{r} < 2$, we have that $r(x) = at + b$ for some $a,b \in \mathbb{Z}$. Then, $a\sqrt{-5} + b = 0$. It follows that $a = b = 0$ as $a,b \in \mathbb{Z}$. Therefore, $f = g(t^{2} + 5)$, and so $f \in (t^{2} + 5)$. If $\deg f < 2$, then $f(t) = at + b$ for some $a,b$. We have that $f(\sqrt{-5}) = a\sqrt{-5} + b$ and so $f(t) = 0$ as $a,b \in \mathbb{Z}$. For the reverse inclusion, let $f \in (t^{2} + 5)$. Then, $f = g(t^{2} + 5)$ for some $g \in \mathbb{Z}[t]$. We have that $\varphi(f) = f(\sqrt{-5}) = g(\sqrt{-5})0 = 0$. Hence, $f \in \ker\varphi$. Thus, $\ker\varphi = (t^{2} + 5)$. By the first isomorphism theorem, 
$$\frac{\mathbb{Z}[t]}{(t^{2} + 5)} \cong \mathbb{Z}[\sqrt{-5}]$$ 

\subparagraph{(ii)} Note that $\mathbb{Z}$ is Noetherian as every ideal is principle, hence, finitely generated. By Hilbers basis theorem, $\mathbb{Z}[t]$ is Noetherian. By a previous exercise, we finally have that $\mathbb{Z}[t]/(t^{2}+t)$ is Noetherian. 

\subparagraph{(iii)} Define $N:\mathbb{Z}[\sqrt{-5}] \to \mathbb{Z}$ by $N(a + ib\sqrt{5}) = a^{2} + 5b^{2}$. Let $z_{1} = a_{1} + b_{1}i\sqrt{5}, z_{2} = a_{2} + b_{2}i\sqrt{5} \in \mathbb{Z}[\sqrt{-5}]$. We have that
\begin{align*}
N(z_{1}z_{2}) &= N((a_{1} + b_{1}i\sqrt{5})(a_{2} + b_{2}i\sqrt{5})) \\
&= N((a_{1}a_{2} - 5b_{1}b_{2}) + (a_{1}b_{2} + a_{2}b_{1})i\sqrt{5}) \\ 
&= a_{1}^{2}a_{2}^{2} - 10a_{1}a_{2}b_{1}b_{2} + 25b_{1}^{2}b_{2}^{2} + 5a_{1}^{2}b_{2}^{2} + 10a_{1}a_{2}b_{1}b_{2} + 5a_{2}^{2}b_{1}^{2} \\
&= a_{1}^{2}a_{2}^{2} + 25b_{1}^{2}b_{2}^{2} + 5a_{1}^{2}b_{2}^{2} + 5a_{2}^{2}b_{1}^{2} \\
&= (a_{1}^{2} + 5b_{1}^{2})(a_{2}^{2} + 5b_{2}^{2}) \\
&= N(a_{1} + b_{1}i\sqrt{5})N(a_{2} + b_{2}i\sqrt{5}) \\
&= N(z_{1})N(z_{2}) \\
\end{align*}

\subparagraph{(iv)} Suppose there are $u,v \in \mathbb{Z}[\sqrt{-5}]$ such that $uv = 1$. Note that $N(1) = 1$. Then, $1 = N(1) = N(uv) = N(u)N(v) = (a_{1}^{2} + 5b_{1}^{2})(a_{2}^{2} + 5b_{2}^{2})$ where $u = a_{1} + ib_{1}\sqrt{5}, v = a_{2} + ib_{2}\sqrt{5}$. As $N(z) \geq 0$ for all $z$, we have that $a_{1}^{2} + 5b_{1}^{2} = 1$ and $a_{2}^{2} + 5b_{2}^{2} = 1$. As $a_{1},a_{2},b_{1},b_{2} \in \mathbb{Z}$, we must have that $b_{1},b_{2} = 0$, hence, $a_{1},a_{2} = \pm 1$. We have that $1$ and $-1$ are units in $\mathbb{Z}[\sqrt{-5}]$. 

\subparagraph{(v)} Suppose that there exists $u,v \in \mathbb{Z}[\sqrt{-5}]$ such that $uv = 2$. We have that $2 = N(2) = N(uv) = N(u)N(v) = (a_{1}^{2} + 5b_{1}^{2})(a_{2}^{2} + 5b_{2}^{2})$ where $u = a_{1} + ib_{1}\sqrt{5}, v = a_{2} + ib_{2}\sqrt{5}$. As $a_{1},a_{2},b_{1},b_{2} \in \mathbb{Z}$ and $N(z) \geq 0$ for all $z$, we have that $a_{1}^{2} + 5b_{1}^{2} = 2$ and $a_{2}^{2} + 5b_{2}^{2} = 1$ without loss of generality. Note there are no solutions in $\mathbb{Z}$ to $a_{1}^{2} + 5b_{1}^{2} = 2$. Hence, no such $u,v$ exist. For a similar reason, we find that no such $u,v$ exist in $\mathbb{Z}[\sqrt{-5}]$ such that $3 = uv$. Suppose there are $u,v$ such that $1 + i\sqrt{5} = uv$. Then, $6 = N(1 + i\sqrt{5}) = N(u)N(v) = (a_{1}^{2} + 5b_{1}^{2})(a_{2}^{2} + 5b_{2}^{2})$. We have that $a_{1}^{2} + 5b_{1}^{2} = 6$ and $a_{2}^{2} + 5b_{2}^{2} = 1$ has the solution $a_{1} = b_{1} = 1$ and $a_{2} = 1, b_{2} = 0$, which correspond to $1 + i\sqrt{5} = (1 + i\sqrt{5})\cdot 1$. We also have that $a_{1}^{2} + 5b_{1}^{2} = 3$ and $a_{2}^{2} + 5b_{2}^{2} = 2$ has no solutions. Therefore, $1 + i\sqrt{5}$ is irreducible. For a similar reason, $1  - i\sqrt{5}$ is irreducible. We conclude that $2,3,1+i\sqrt{5}, 1-i\sqrt{5}$ are irreducible. 

\subparagraph{(vi)} Note that $6 \in (2), (3)$ but $6 = (1 - i\sqrt{5})(1 + i\sqrt{5})$, and $1-i\sqrt{5}, 1+i\sqrt{5} \not\in (2), (3)$. Hence, $(2),(3)$ are not prime. We also note that $6 \in (1 + i\sqrt{5}),(1 - i\sqrt{5})$ but $6 = 2\cdot 3$ and $2,3 \not\in (1 + i\sqrt{5}),(1 - i\sqrt{5})$. Hence, $2,3,1+i\sqrt{5},1-i\sqrt{5}$ are not prime in $\mathbb{Z}[\sqrt{-5}]$. This also shows that $\mathbb{Z}[\sqrt{-5}]$ is not a UFD. 

\subsection*{5.2 - UFDs, PIDs and Euclidean Domains}
\paragraph{1.} Let $R$ be a UFD, and let $a,b,c \in R$. 

\subparagraph{(i)} Suppose that $(a) \subseteq (b)$. We then have that $a \in (a) \subseteq (b)$, and so $a = xb$ for some $x \in R$. As $R$ is a UFD, $x = p_{1}p_{2}...p_{n}$ and $b = q_{1}q_{2}...q_{m}$ for irreducibles $p_{1},...,p_{n}$ and $q_{1},...,q_{m}$. We then have that $a = xb = p_{1}...p_{n}q_{1}...q_{m}$. By uniqueness, the multiset of irreducible factors of $b$ is contained in the multiset of irreducible factors of $a$. For the converse, suppose that the the multiset of irreducible factors of $b$ is contained in the multiset of irreducible factors of $a$. Then, $a = a_{1}...a_{n}b_{1}b_{2}...b_{n}$ where $b = b_{1}...b_{n}$ by assumption. It follows that $a \in (b)$. Therefore, $(a) \subseteq (b)$. 

\subparagraph{(ii)} Suppose that $a$ and $b$ are associates. Then, $(a) = (b)$. By the previous parts, we must have that the multiset of irreducible factors of $b$ is contained in the multiset of irreducible factors of $a$ and the multiset of irreducible factors of $a$ is contained in the multiset of irreducible factors of $b$. Therefore, the multiset of irreducible factors of $b$ is the same as the multiset of irreducible factors of $a$. Similarly, if the multiset of irreducible factors of $b$ is the same as the multiset of irreducible factors of $a$, then $(a) \subseteq (b)$ and $(b) \subseteq (a)$, hence, $(a) = (b)$. 

\subparagraph{(iii)} As $R$ is a UFD, $bc = p_{1}...p_{n}, b = q_{1}...q_{m}$ and $c = r_{1}...r_{k}$ for some irreducible $p_{1},...,p_{n}, q_{1}, ..., q_{m}$ and $r_{1}, ..., r_{k}$. We have that $p_{1}...p_{n}$ and $q_{1}...q_{m}r_{1}...r_{k}$ are factorisations into irreducibles for $bc$, and so $n = m+k$ and $p_{i}$ and $q_{j}$ are associates or $p_{i}$ and $r_{j}$ are associates. Therefore, the irreducible factors of $bc$ is the collection of all irreducible factors of $b$ and $c$. 

\paragraph{2.} Let $R$ be a UFD and let $a,b,c \in R$ such that $a \mid bc$ and $\gcd(a,b) = 1$. As $a \mid bc$, we have that $(bc) \subseteq (a)$. By Lemma 2.1, the multiset of irreducibles of $a$ is contained in the multiset of irreducibles of $bc$, which is the collection of all irreducibles of $b$ and $c$. As $\gcd(a,b) = 1$, the entirety of the irreducible factors of $a$ is contained in the irreducible factors of $c$. Therefore, $(c) \subseteq (a)$, and so $a \mid c$.  

\paragraph{3.}

\paragraph{4.} Let $x,y \in \mathbb{Z}[x,y]$. Suppose that there exists an $f$ such that $(x,y) \subseteq (f)$. We have that $x \in (f)$ and $y \in (f)$, so $x = fg$ and $y = fh$ for some $g,h\in \mathbb{Z}[x,y]$. As $\mathbb{Z}$ is an integral domain, $\mathbb{Z}[x,y]$ is an integral domain and so $\deg(f) + \deg(g) = 1$ and $\deg(f) + \deg(h) = 1$. Suppose that $\deg(f) = 1$, then $f = ax + by + c$ for some $a,b,c \in \mathbb{Z}$ and we have that $g = d, h = e$ for some $e,d \in \mathbb{Z}$. We have that $x = fg = (ax + by + c)d = adx + bdy + cd$. Then, $ad = 1, bd = 0$ and $cd = 0$. Note that $g \neq 0$, so $c = b = 0$. Furthermore, $y = fh = (ax + by + c)e = aex$. Hence, $ae = 0$. This implies that $a = 0$ as $h \neq 0$. It follows that $f = 0$, which cannot happen. Now suppose that $\deg(f) = 0$. Then, $f = a$ for some $a$. We have that $x = ag$ and $y = ah$, where $\deg(g) = \deg(h) = 1$. Let $g = bx + cy + d$. Then, $x = abx + acy + ad$, and so $ab = 1, ac = 0$ and $ad = 0$. As $f \neq 0$, we must have that $c = 0, d = 0$. Let $h = ex + iy + j$. Then, $y = ah = aex + aiy + aj$. It follows that $j = e = 0$ and $ai = 1$. So far we have that $g = bx, h = iy$ with $ab = ai = 1$. We must have that $f = \pm 1$, so $f$ is a unit. Therefore, $(f) = R = (1)$. It follows that $\gcd(x,y) = 1$. Now, suppose that there exist $f,g \in \mathbb{Z}[x,y]$ such that $fx + gy = 1$. We have that $f,g$ cannot exist as $fx+gy$ is always of degree greater than $1$ if they are not both nonzero. 

\paragraph{5.} Let $R$ be the subset of $\mathbb{Z}[t]$ consisting of polynomials of the form $f = a_{0} + a_{2}t^{2} + ... + a_{n}t^{n}$. Let $f = a_{0} + a_{2}t^{2} + ... + a_{m}t^{m}, g = b_{0} + b_{2}t^{2} + ... + b_{n}t^{n}$ be elements of $R$. Without loss of generality, suppose that $\deg f \geq \deg g$. We can write $g$ as $b_{0} + b_{2}t^{2} + ... + b_{n} + b_{n+1}t^{n+1} + ... + b_{m}t^{m}$ where $b_{i} = 0$ for $i > n$. Then, $f - g = (a_{0} - b_{0}) + (a_{2} - b_{2})t^{2} + ... + (a_{m} - b_{m})t^{m} \in R$. Furthermore, $fg = (a_{0} + a_{2}t^{2} + ... + a_{m}t^{m})(b_{0} + b_{2}t^{2} + ... + b_{n}t^{n}) = a_{0}b_{0} + (a_{0}b_{2} + a_{2}b_{0})t^{2} + ... + a_{m}b_{n}t^{n+m} \in R$. Hence, $R$ is a subring. It follows that $R$ is an integral domain as $\mathbb{Z}[t]$ is an integral domain. Note that the divisors of $t^{5}$ in $R$ are $\{\pm 1, \pm t^{2}, \pm t^{3}, \pm t^{5}\}$, and the divisors of $t^{6}$ in $R$ are $\{\pm 1, \pm t^{2}, \pm t^{3}, \pm t^{4}, \pm t^{6}\}$. The common divisors of $t^{5}$ and $t^{6}$ in $R$ are then $\{\pm 1, \pm t^{2}, \pm t^{3}\}$. We have that $t^{5} \mid t^{5}$ and $t^{5} \mid t^{6}$, however, $t^{5} \nmid x$ for any common divisor $x$ of $t^{5}$ and $t^{6}$ in $R$. 

\paragraph{6.}
\subparagraph{(i)} Let $R$ be an integral domain with the property that the intersection of any family of principal ideals in $R$ is necessarily a principal ideal. Let $x,y \in R$. Let $F$ be the set of common divisors of $x$ and $y$, and let
$$I = \bigcap_{a \in F}(a)$$
By assumption, $I = (d)$ for some $d \in R$. We have that for any $a \in F$, $a \mid x$ and $a \mid y$, hence, $(x), (y) \subseteq (a)$. Thus, $(x), (y) \subseteq \bigcap_{a \in F}(a) = (d)$, and so $d \mid x$ and $d \mid y$. Let $c \in R$ such that $c \mid x$ and $c\mid y$. By definition, $c \in F$. Hence, $(d) = \bigcap_{a \in F}(a) \subseteq (c)$. Hence, $c \mid d$. It follows that greatest common divisors in $R$ and $d = \gcd(x,y)$.

\subparagraph{(ii)}

\paragraph{7.} Let $R$ be a Noetherian domain, and assume for all $a,b \in R$, the greatest common divisor of $a$ and $b$ are a linear combination of $a$ and $b$. Let $I$ be an ideal of $R$. As $R$ is Noetherian, $I = (a_{1}, ..., a_{n})$ for $a_{1},...,a_{n} \in R$. Let $d = \gcd(a_{1}, a_{2})$. By assumption, we have that $d = xa_{1} + ya_{2}$ for some $x,y$. Let $r \in I$. Then, $r = a_{1}x_{1} + ... + a_{n}x_{n}$. We have that $d \mid a_{1}$ and $d \mid a_{2}$. Hence, $a_{1} = b_{1}d$ and $a_{2} = b_{2}d$. We have that
$$r = a_{1}x_{1} + a_{2}x_{2} + ... + a_{n}x_{n} = b_{1}dx_{1} + b_{2}dx_{2} + ... + a_{n}x_{n} = d(b_{1}x_{1} + b_{2}x_{2}) + ... + a_{n}x_{n} \in (d,a_{3},...,a_{n})$$
For the converse, let $r \in (d,a_{3}, ..., a_{n})$. Then, 
$$r = x_{0}d + x_{3}a_{3} + ... + x_{n}a_{n} = x_{0}(xa_{1} + ya_{2}) + x_{3}a_{3} + ... + x_{n}a_{n} = x_{0}xa_{1} + x_{0}ya_{2} + x_{3}a_{3} + ... + x_{n}a_{n} \in I$$
Therefore, $(a_{1}, a_{2}, ..., a_{n}) = (d, a_{3}, ..., a_{n})$.
Doing this process recursively, we have that $I = (d')$ for some $d'$. We conclude that $R$ is a PID. 

\paragraph{8.}

\paragraph{9.} Let $R$ be a UFD and let $P$ be a prime ideal of height $1$. Note that $P$ is not the $0$ ideal and we have that $(0) \subset P$. Let $a \in P$ be a nonzero element of $P$. If $(a) = P$, then we are done. Hence, assume the opposite i.e $(a) \neq P$. As $R$ is a UFD, $a = q_{1}...q_{n}$ for irreducibles $q_{1}, ..., q_{n} \in R$. As $a \in P$, we have that $q_{1}...q_{n} \in P$, hence, $q = q_{i} \in P$ for some $q$. Note that $q$ is prime as $R$ is a UFD. We have that $(q)$ is a prime ideal such that $(0) \subset (q) \subseteq P$. Thus, as $P$ is of height $1$, we must have that $(q) = P$ as $(q)$ is nonzero. 

\paragraph{10.} Let $R$ be a Noetherian domain and suppose that every prime ideal of height $1$ is principle. Let $a \in R$ be an irreducible element. By Krull's Hauptidealsatz, $a$ is contained in some prime ideal of height $1$, $P$ say. By assumption, $P = (x)$ for some $x \in R$. Hence, $a \in (x)$ and so $a = xy$ for some $y$. As $a$ is irreducible, $x$ is a unit or $y$ is a unit. As $P = (x)$ is prime, $x$ cannot be a unit, thus, $y$ is a unit. Therefore, $a$ and $x$ are associates, and so $(a) = (x)$. We have that $a$ is prime. As $R$ is Noetherian, we have that the a.c.c for principal ideals hold in $R$. By Theorem 2.5, its necessary that $R$ is a UFD.  

\paragraph{11.}

\paragraph{12.} Let $R$ be a commutative ring and suppose that $R[x]$ is a PID. Suppose that there is an $f \in R[x]$ such that $(x) \subset (f) \subset R$, where the inclusions are proper. We must have that $x \in (f)$ and so that $x = fg$ for some $g \in R[x]$. As $R[x]$ is a PID, $1 = \deg(x) = \deg(fg) = \deg(f) + \deg(g)$. Suppose that $\deg f = 1$. Then, $f = ax + b$ for some $a,b \in R$ and $g = c \in R$. Thus, $x = fg = (ax + b)c = acx + bc$. Hence, $ac = 1$ and $bc = 0$. We have that $c$ is a unit, and so $bc = 0 \implies b = 0$ as $R[x]$ is a PID. We have that $f = ax$ where $ab = 0$ for some $b \in R$. Then, $bf = bax = abx = x \in (x)$. Therefore, $(f) = (x)$, which is a contradiction. Now, suppose that $\deg f = 0$. We have that $f = a \in R$ and $g = bx + c$ where $b,c \in R$. Then, $x = fg = a(bx+c) = abx + ac$, hence, $ab = 1$ and $ac = 0$. Note $a$ is a unit, and $c =0$. As $a$ is a unit, $(f) = (a) = R$, which is another contradiction. We can conclude $f$ does not exist, and $(x)$ is a maximal ideal. $R \cong R[x]/(x)$ is then a field. 

\paragraph{13.}

\paragraph{14.}

\paragraph{15.} Let $R$ be a Euclidean domain. There exists a Euclidean valuation on $R$, $v$ say. For $a \neq 0$, let $\overline{v}(a) = \min\{v(ab) \mid b \in R\}$. Let $a,b \in R$ such that $b \nmid a$. Then, $a = qb + r$ for some $q,r$ with $r \neq 0$ and $v(b) > v(r)$. Let $r^{*}, q^{*} \in R$ such that $a = q^{*}b + r^{*}$ with $v(r^{*})$ minimal and $v(b) > v(r^{*})$. Suppose, for contradiction, $\overline{v}(r^{*}) \geq \overline{v}(b)$. Then, $\min\{v(r^{*}x) \mid x \in R\} \geq \min\{v(bx) \mid x \in R\}$. Let $x \in R$ such that $\overline{v}(b) = v(bx)$. Again, as $v$ is a Euclidean valuation and since $a \nmid b$, $a = bxp + k$ for some $p, k$ with $v(bx) > v(k)$. Thus, $v(k) < v(b) \leq \min\{v(by) \mid y \in R\} = \overline{v}(b) = \overline{v}(b) = v(bx)$. We note that $a = b(xp) + k$ with $v(b) > v(k)$. As $v(r^{*})$ is minimal, we have that $v(r^{*}) \leq v(k) < v(bx) = \overline{v}(b)$. Then, $\overline{v}(r^{*}) \leq v(r^{*}) < \overline{v}(b)$, which is a contradiction. We must have that $\overline{v}(r^{*}) < \overline{v}(b)$. It follows that $\overline{v}$ is a Euclidean valuation on $R$. Furthermore, we note that for nonzero $a,b \in R$, 
$$\overline{v}(ab) = \min\{v(abx) \mid x \in R\} \geq \min\{v(ay) \mid y \in R\} = \overline{v}(a)$$


\paragraph{16.} Let $R$ be a Euclidean domain with Euclidean valuation $v$, and assume that $v(ab) \geq v(b)$ for all nonzero $a,b \in R$. Let $x,y \in R$ be nonzero elements of $R$ that are associates. There exists some unit $u$ such that $x = uy$. Then, $v(x) = v(uy) \geq v(y)$ and $v(y) = v(u^{-1}x) \geq v(x)$. Hence, $v(x) = v(y)$. Let $w$ be a unit in $R$. We have that $w$ is an associate to $1$. Hence, $v(w) = v(1)$. Let $x \in R$ be a nonzero element. Then, $v(x) = v(x1) \geq v(1) = v(w)$. Therefore, $v(w)$ is minimal. 

\paragraph{17.} Let $R$ be a Euclidean domain that is not a field. Let $v$ be its associated Euclidean valuation. As $R$ is not a field, the set $A = \{v(x) \mid x \text{ is not a unit}\}$ is nonempty. Let $c \in R$ such that $v(c) \in A$ is minimal. Let $a \in R$. As $R$ is a Euclidean domain, there exists $r, q \in R$ such that $a = qc + r$ with either $r = 0$ or $v(r) < v(c)$. If $r = 0$, then we are done. Suppose that $r \neq 0$. Then, $v(r) < v(c)$ and so $v(r) \not\in A$ by minimality of $v(c)$. Hence, $r$ is a unit. Therefore, for all $a \in R$, there exists $q,r \in R$ such that $a = qc + r$ where either $r = 0$ or $r$ is a unit. 

\paragraph{18.}

\paragraph{19.} Let $v$ be a discrete valuation on a field $k$. 
\subparagraph{(i)} Let $R = \{a \in k^{*} \mid v(a) \geq 0\} \cup \{0\}$. Let $x, y \in R$ such that $x \neq y$. We then have that $x - y \in R$. Note that $v(1_{k}) = 0$ as $v$ is a homomorphism so that $1_{k} \in R$. Further note that $0 = v(1_{k}) = v(-1_{k} \cdot -1_{k}) = 2v(-1_{k})$. Hence, $v(-1_{k}) = 0$. We have that 
$$v(x - y) \geq \min\{v(x), v(-y)\} = \min\{v(x), v(-1_{k}) + v(y)\} = \min\{v(x), v(y)\} \geq 0$$
as $v(x), v(y) \geq 0$. Thus, $x - y \in R$. Furthermore, $v(xy) = v(x) + v(y) \geq 0$ as $v(x), y(y) \geq 0$. Hence, $R$ is a subring of $k$. 

\subparagraph{(ii)} Let $a,b \in R$ with $b \neq 0$. Suppose that $v(a) \geq v(b)$. Then, $v(a) - v(b) \geq 0$, and so $v(ab^{-1}) = v(a) - v(b) \geq 0$. Hence, $ab^{-1} \in R$. Then, $a = ab^{-1}b + 0$. Suppose now $v(a) < v(b)$. Note $a = b\cdot 0 + a$. It follows that $R$ is a Euclidean domain. 

\subparagraph{(iii)} Let $p$ be a fixed prime integer. Let $v_{p}:\mathbb{Q}^{*} \to \mathbb{Z}$ be a map of abelian groups defined by sending $a/b$, where $\gcd(a,b) = 1$, to $\max\{k \in \mathbb{Z} : p^{k} \mid a\}$. Let $k \in \mathbb{Z}$. Then, $v_{p}(p^{k}) = k$, hence, $v_{p}$ is surjective. Let $a_{1}/b_{1}, a_{2}/b_{2} \in \mathbb{Q}^{*}$. We have that $a_{1} = p^{m}z_{1}$ and $a_{2} = p^{n}z_{2}$ where $\gcd(z_{1},p) = \gcd(z_{2},p) = 1$. Then, 
$$v_{p}\qty(\frac{a_{1}}{b_{1}} \cdot \frac{a_{2}}{b_{2}}) = v_{p}\qty(\frac{p^{n}z_{1}p^{m}z_{2}}{b_{1}b_{2}}) = p^{n}p^{m} = v_{p}(p^{n})v_{p}(p^{m}) = v_{p}\qty(\frac{p^{n}z_{1}}{b_{1}})v_{p}\qty(\frac{p^{m}z_{2}}{b_{2}}) = v_{p}\qty(\frac{a_{1}}{b_{1}})v_{p}\qty(\frac{a_{2}}{b_{2}})$$
Furthermore, 
$$v_{p}\qty(\frac{a_{1}}{b_{1}} + \frac{a_{2}}{b_{2}}) = v_{p}\qty(\frac{a_{1}b_{2} + a_{2}b_{1}}{b_{1}b_{2}}) = v_{p}\qty(\frac{p^{n}z_{1}b_{2} + p^{m}z_{2}b_{1}}{b_{1}b_{2}}) = \min\{p^{n}, p^{m}\} = \min\left\{v_{p}\qty(\frac{p^{n}z_{1}}{b_{1}}), v_{p}\qty(\frac{p^{m}z_{2}}{b_{2}})\right\}$$
$$= \min\left\{v_{p}\qty(\frac{a_{1}}{b_{1}}), v_{p}\qty(\frac{a_{2}}{b_{2}})\right\}$$
Therefore, $v_{p}$ is a discrete valuation. Let $R = \{a/b \in \mathbb{Q}^{*} \mid v_{p}(a/b) \geq 0\} \cup \{0\}$ and let $R'$ be the set of rational numbers $a/b$ with $b$ not divisible by $p$. Let $a/b \in R$. Then, $v(a/b) \geq 0$ and so $p^{k} \mid a$ for some $k \geq 0$. As $\gcd(a,b) = 1$, $b$ cannot be divisible by $p$. Thus, $a/b \in R'$. Now let $a/b \in R'$. We have that $a = p^{k}z$ for some $k$ and where $\gcd(p,z) = 1$ and $k \geq 0$. We then have that $v_{p}(a/b) = v_{p}(p^{k}z/b) = k$ as $k$ is not reduced by $b$. Hence, $a/b \in R$. It follows that $R = R'$. Therefore, the set of rational numbers $a/b$ with $b$ not divisible by $p$ is a DVR. 

\paragraph{20.} Let $R$ be a DVR with discrete valuation $v$. Let $t \in R$ such that $v(t) = 1$, and let $I$ be an ideal of $R$. Let $n = \min\{v(x) \mid x \in I\}$. Note that $v(t^{n}) = nv(t) = n$. Let $x \in I$. We have that $v(x) \geq n = v(t^{n})$. Then, $v(xt^{-n}) \geq 0$. Hence, $xt^{-n} \in R$. Let $y = xt^{-n}$, then $yt^{n} = x$. Thus, $x \in (t^{n})$, and so $I \subseteq (t^{n})$. Now, let $y \in (t^{n})$. We have that $y = at^{n}$ for some $a \in R$. Let $x \in I$ such that $v(x)$ is minimal, that is, $v(x) = n$. Then, $v(y) = v(a) + v(t^{n}) = v(a) + n \geq n = v(x)$. Hence, $v(yx^{-1}) \geq 0$ and so $yx^{-1} \in R$. Let $yx^{-1} = z \in R$. Then, $y = zx \in I$. Therefore, $I = (t^{n})$.  

\paragraph{21.} Let $R$ be an integral domain. Suppose that $R$ admits a Dedekind-Hasse valuation, and let $v$ be such a valuation. Let $I$ be an ideal of $R$, and let $b$ be an element of $I$ such that $v(b)$ is minimal. Note that $b \in I$, so $(b) \subseteq I$. Let $a \in I$ such that $(a,b) \neq (b)$. Then, $as = bq + r$ with $v(r) < v(b)$ for some $s,q,r \in R$ such that $as + bq = r$. We have that $r = as + bq \in (a,b) \subseteq I$ as $a,b \in I$. This is a contradiction as $v(r) < v(b)$, but $v(b)$ is assumed to be minimal. Thus, $(a,b) = (b)$, and so $a \in I$. It follows that $I = (b)$, and $R$ is a PID. For the converse, suppose that $R$ is a PID. For nonzero $a \in R$, let $v(a)$ be the size of the multiset of irreducible factor of $a$. Note this is well-defined as $R$ is a UFD since it is a PID. Let $x, y \in R$. If $y \mid x$, then $(x,y) = (y)$. Suppose that $y \nmid x$. As $R$ is a PID, $(x,y) = (d)$ for some $d \in R$. As $R$ is a UFD, $\gcd(x,y)$ exists, and we have that $\gcd(x,y) = d$. We also have that $d = ax + by$ for some $a,b \in R$. We have that $d \mid y$, which means that $v(d) \leq v(y)$. Suppose that $v(d) = v(y)$, then $d = y$, and so $y = d = \gcd(x,y)$. Hence, $y \mid x$, which cannot happen. Thus, $v(d) < v(y)$. It follows that $v$ is a Dedekind-Hasse valuation. 

\paragraph{22.} Let $R \subseteq S$ be an inclusion of integral domains, and assume that $R$ is a PID. Let $a,b \in R$ and let $d \in R$ be a gcd for $a$ and $b$. Consider $(a,b)$ and $(d)$ as ideals of $R$. As $R$ is a PID, $(a,b) = (c)$ for some $c \in R$. As $d$ is a gcd of $a$ and $b$, we have that $(a,b) \subseteq (d)$ and $(d)$ is minimal with this property. Note that $(c) = (a,b) \subseteq (d)$, and $(d) \subseteq (c)$ by minimality as $(a,b) \subseteq (c)$. Hence, $(c) = (d)$, and so $(a,b) = (d)$. We have that $d \in (a,b)$, and so $ax + by = d$ for some $x,y \in R$. Now consider $(a,b)$ and $(d)$ as ideals of $S$. Let $t \in (a,b)$. Then, $t = ap + bq$ for some $p,q \in S$. As $d \mid a$ and $d \mid b$ in $R$, $a = r_{1}d$ and $b = r_{2}d$ for some $r_{1},r_{2} \in R$. Then, $t = ap + bq = r_{1}dp + r_{2}dq = d(r_{1}p + r_{2}q) \in (d)$. For the reverse inclusion, let $t \in (d)$. Then, $t = sd$ for some $s \in S$. And so $t = sd = s(ax + by) = sax + sby \in (a,b)$. Hence, $(a,b) = (d)$ as ideals of $S$. It follows that $d$ is the gcd of $a$ and $b$ in $S$. 

\paragraph{23.}

\paragraph{24.} Suppose, for contradiction, that the list of prime elements in $\mathbb{Z}$ were finite. Let $P = \{p_{1}, ..., p_{n}\}$ be such a list. Consider $x = p_{1}p_{2}...p_{n} + 1$. If $x$ was prime, then $x > p_{i}$ for all $1 \leq i \leq n$, and so $x \not\in P$. However, this is a contradiction as $x$ must be in $P$ as it is prime. Now, suppose $x$ was not prime. We note that $p_{i} \nmid x$ for all $i$ as it leaves a remainder of $1$. We must have that $x$ must contain a prime not in the list. Therefore, $P$ is not complete. We conclude that such a list $P$ cannot exist. 

\paragraph{25.}

\subsection*{5.3 - Intermezzo: Zorn's Lemma}
\paragraph{1.} Let $\preceq$ be a well-ordering on a non-empty set $Z$. Let $a,b \in Z$. Then, $\{a,b\}$ is a subset of $Z$, and, by assumption, must have a least element. If $a$ is the least element of $\{a,b\}$, then $a \preceq b$, and if $b$ is the least element of $\{a,b\}$, then $b \preceq a$. Hence, $(Z, \preceq)$ is a total order. 

\paragraph{2.} Let $\preceq$ be a total ordering on a non-empty set $Z$. Suppose that $\preceq$ is a well-ordering on $Z$. Let $z_{1} \succeq z_{2} \succeq ...$ be a decending chain in $Z$. Consider the set $A = \{z_{i} | i \in \mathbb{N}\} \subseteq Z$. By assumption, $A$ must have a least element, and such an element must be of the form $z_{k}$ for some $k \in \mathbb{N}$. We must have that $z_{m} \succeq z_{k}$ for all $m \in \mathbb{N}$. Thus, $z_{n} \succeq z_{k}$ for all $n > k$. Therefore, $z_{k} = z_{k+1} = ...$ and so the decending chain must stabilise. For the converse, suppose that $\preceq$ is not a well-ordering on $Z$. We have that there must exist a subset $A$ of $Z$ such that $A$ does not have a least element. Choose $z_{1} \in A$. As $A$ does not have a least element, there exists a $z_{2} \in A$ such that $z_{1} \succeq z_{2}$ and $z_{1} \neq z_{2}$. We keep finding $z_{i}$ in such a way to construct a decreasing chain of elements of $A$ that does not stabilise. Therefore, the converse holds. We have that $\preceq$ is a well-ordering on $Z$ if and only if every decending chain in $Z$ stabilises. 

\paragraph{3.} Suppose that the Axiom of Choice holds. Let $f:A \to B$ be a surjective function. Consider the set of preimages $A = \{f^{-1}(\{b\}) | b \in B\}$. As $f$ is surjective, we have that $f^{-1}(\{b\})$ is non-empty for all $b \in B$. We note that $A$ is a collection of non-empty disjoint sets. By the Axiom of Choice, for each $f^{-1}(\{b\})$, we can choose some $a' \in f^{-1}(\{b\})$ and we have that $f(a') = b$. Define $f^{-1}:B \to A$ by sending $b$ to such an $a'$. We have that for each $b \in B$, $(f \circ f^{-1})(b) = f(f^{-1}(b)) = f(a') = b$. Thus, $f^{-1}$ is a right inverse of $f$. Now, let $f:A \to B$ be a set-function with a right inverse, $f^{-1}:B \to A$. For every $b \in B$, we have that $f^{-1}(b) = a \in A$ and $f(a) = b$. Thus, $f$ is surjective. For the converse, suppose that every surjective function has a right inverse, and if a set-function has a right inverse, then it is surjective. Let $\mathfrak{A}$ be a collection of disjoint non-empty sets. Define $f:\bigcup_{A \in \mathfrak{A}}A \to \mathfrak{A}$ by sending $a \in \bigcup_{A \in \mathfrak{A}}A$ to the set $A$ in which $a \in A \in \mathfrak{A}$. $f$ is trivially surjective, thus, $f$ has a right inverse, $f^{-1}$ say. For each $A \in \mathfrak{A}$, we can choose $f^{-1}(A) = a$. Therefore, the Axiom of Choice holds.  

\paragraph{4.} Let $A$ be a set such that there exists a bijection $a_{n}:\mathbb{Z}^{> 0} \to A$ where $a \mapsto a_{i}$ with $i \in \mathbb{Z}^{> 0}$. Note that if $a \in A$, then $a = a_{i}$ for some $i \in \mathbb{Z}^{> 0}$ as $a_{n}$ is bijective. Define a relation $\preceq$ on $A$ by $a_{i} \preceq a_{j}$ if and only if $i \leq j$. We have that $a_{i} \preceq a_{i}$ as $i \leq i$, hence, $\preceq$ is transitive. Suppose that $a_{i} \preceq a_{j}$ and $a_{j} \preceq a_{k}$. Then, $i \leq j$ and $j \leq k$. As $\leq$ on $\mathbb{Z}^{> 0}$ is transtive, $i \leq k$. Hence, $a_{i} \preceq a_{k}$ and so $\preceq$ is transitive. Suppose now that $a_{i} \preceq a_{j}$ and $a_{j} \preceq a_{i}$. Then, $i \leq j$ and $j \leq i$. Hence, $i = j$ and so $a_{i} = a_{j}$, which means that $\preceq$ is antisymmetric. Therefore, $\preceq$ is an order relation. Let $A'$ be a subset of $A$. Consider the set $N = \{i \mid a_{i} \in A'\} \subseteq \mathbb{Z}^{> 0}$. By the well-ordering principle, $N$ has a least element, $i'$ say. For all $a_{i} \in A'$, we have that $a_{i'} \preceq a_{i}$ as $i' \leq i$ for all $i$. Hence, $a_{i'}$ is the least element of $A'$. Therefore, $\preceq$ is a well-order on $A$. It follows that $\mathbb{Z}$ and $\mathbb{Q}$ by taking your favourite bijection between $\mathbb{Z}, \mathbb{Q}$ and $\mathbb{Z}^{> 0}$. 

\paragraph{5.}
\paragraph{6.}
\paragraph{7.}
\paragraph{8.} Let $G$ be a nontrivial finitely generated group, and let $\mathfrak{F}$ be the family of proper subgroups of $G$. Note $\mathfrak{F}$ is not empty as it contains the trivial subgroup. Order $\mathfrak{F}$ via inclusion, and let $A$ be a chain of $\mathfrak{F}$. Let $H = \bigcup_{S \in A}S$. Let $x,y \in H$, then there exists proper subgroups $S, S'$ of $G$ such that $x \in S$ and $y \in S'$. As $A$ is a chain, $S \subseteq S'$ or $S' \subseteq S$. Without loss of generality, assume that $S \subseteq S'$. Then, $x,y \in S'$. Hence, $x - y \in S' \subseteq H$, and so $H$ is a subgroup of $G$. Assume that $H = G$. As $G$ is finitely generated, $G = \langle a_{1}, ..., a_{n}\rangle$ for some $a_{1}, ..., a_{n} \in G$. For each $a_{i} \in \{a_{1}, ..., a_{n}\}$, there is a proper subgroup $S_{i} \in A$ such that $a_{i} \in S_{i}$. As $A$ is a chain, for each $i,j$, $S_{i} \subseteq S_{j}$ or $S_{j} \subseteq S_{i}$. Let $S'$ be the maximal element among these $S_{i}$'s. Then, $a_{i} \in S'$ for all $i$, and so $S' = G$, which contradicts the assumption that $S'$ is a proper subgroup of $G$. Therefore, $H$ is a proper subgroup of $G$. Note $H$ is an upper bound for $A$ and by Zorn's Lemma, there exists a maximal element in $\mathfrak{F}$. Let $H$ be a proper subgroup of $(\mathbb{Q},+)$. Suppose that $H$ is maximal. Let $x \in \mathbb{Q} - H$. We must have that $\langle H, x\rangle = \mathbb{Q}$ by maximality of $H$. Let $y \in H$. There exists integers $a,b$ with $a,b \neq 0$ such that $y/x = a/b$. As $\langle H, x\rangle = \mathbb{Q}$, we have that $x/a = h + nx$ for some $h \in H$ and $n \in  \mathbb{Z}$. Then, $x = ah + anx = ah + nby \in H$ as $h \in H, y \in H$ and $a,n,b \in \mathbb{Z}$, which is a contradiction. Therefore, such a $H$ cannot exist. It follows that $(\mathbb{Q}, +)$ does not have maximal subgroups.  

\paragraph{9.} Let $R$ be the rng consisting of the abelian group $(\mathbb{Q}, +)$ and multiplication defined by $qr = 0$ for all $q,r \in \mathbb{Q}$. Trivially, if $I$ is an ideal of $R$, then $(I, +)$ is a subgroup of $(\mathbb{Q},+)$. Let $A$ be a subgroup of $\mathbb{Q}$. Let $x \in A$ and let $r \in R$. Then, $rx = 0 \in A$. Therefore, $A$ is an ideal of $R$. Hence, the ideals of $R$ are precicely the subgroups of $(\mathbb{Q},+)$. As $(\mathbb{Q},+)$ does not have any maximal subgroups, $R$ does not have any maximal ideals. 

\paragraph{10.}
\paragraph{11.}

\paragraph{12.} We first prove the following: "Let $(Z,\preceq)$ be a nonempty poset. Assume every chain in $Z$ has a lower bound; then there exists minimal elements, that is, there exists elements $u \in Z$ such that $a \preceq u \implies u = a$". Let $Z$ be a nonempty poset ordered by the relation $\preceq$, and assume that every chain in $Z$ has a lower bound. Define a relation on $Z$, $\preceq'$, by setting $a \preceq' b$ if and only if $b \preceq a$. For all $a \in Z$, we have that $a \preceq a$ as $\preceq$ is an order relation. Thus, $a \preceq' a$ for all $a \in Z$. Suppose that $a \preceq' b$ and $b \preceq' c$. Then, $b \preceq a$ and $c \preceq a$. As $\preceq$ is transitive, $c \preceq a$, and so $a \preceq' c$. Finally, suppose that $a \preceq' b$ and $b \preceq' a$. Then, $b \preceq a$ and $a \preceq b$. Hence, $a = b$. Let $A$ be a chain in $(Z, \preceq')$. For each $a,b \in A$, we have that $a \preceq' b$ or $b \preceq' a$, which means $b \preceq a$ or $a \preceq b$. Thus, $A$ is a chain in $(Z,\preceq)$. By assumption, $A$ has a lower bound with respect to $\preceq$, $u$ say. For all $a \in A$, $u \preceq a$, and so $a \preceq' u$ for all $a \in A$. Hence, $u$ is an upper bound of $A$ with respect to $\preceq'$. By Zorn's Lemma, there exists maximal elements in $(Z,\preceq')$. Let $m$ be such a maximal element in $(Z, \preceq')$. We have that if $a \in Z$ and $m \preceq' a$, then $m = a$. Hence, if $a \in Z$ and $a \preceq m$, we have that $m \preceq a$, and so $m = a$. Therefore, $m$ is a minimal element of $(Z,\preceq)$. Let $R$ be a commutative ring and let $K \subseteq R$ be a proper ideal. Let $\mathfrak{F}$ be the family of prime ideals containing $K$. Order $\mathfrak{F}$ via inclusion and let $A$ be a chain in $\mathfrak{F}$. Let $J = \bigcap_{I \in A}I$. We note that $J$ is an ideal as an intersection of a family of ideals. Let $x,y \in R$ such that $x, y \not\in J$. Then, there exists prime ideals $I, I'$ such that $x \not\in I$ and $y \not\in I'$. As $A$ is a chain, assume that $I \subseteq I'$ without loss of generality. Then, $x, y \not\in I$, and so $xy \not\in I$ by the primality of $I$. Therefore, $xy \not\in J$. It follows that $J$ is prime and is a lower bound of $A$. Therefore, $\mathfrak{F}$ must have minimal elements. 

\paragraph{13.} Let $R$ be a commutative ring, and let $N$ be its nilradical. Let $r \not\in N$. Let $\mathfrak{F}$ be the family of ideals of $R$ do not contain any power of $r^{k}$ of $r$ for $k > 0$. We order $\mathfrak{F}$ via inclusion. Let $A$ be a chain of ideals of $\mathfrak{F}$ and let $J = \bigcup_{I \in A}I$. Let $x,y \in J$, then there exists $I, I' \in A$ such that $x \in I$ and $y \in I'$. As $A$ is a chain, $I \subseteq I'$ or $I' \subseteq I$. Without loss of generality, assume $I \subseteq I'$. Then, $x,y \in I$. As $I'$ is an ideal, $x - y \in I \subseteq J$. Now, let $x \in J$ and $r \in R$. As $x \in J$, $x \in I$ for some ideal $I \in A$. Thus, $rx \in I \subseteq J$. It follows that $J$ is an ideal. Furthermore, $J$ does not contain any power $r^{k}$ of $r$ for $k > 0$ as no ideal $I \in A$ does. Note further that $J$ is an upper bound of $A$. By Zorn's Lemma, there exists maximal elements in $\mathfrak{F}$. Let $M$ be a maximal element of $\mathfrak{F}$. Suppose that there exists $x,y \not\in M$ however $xy \in M$. Let $\langle M, x\rangle$ be the ideal generated by $M$ and $x$. We have that $\langle M, x\rangle$ properly contains $M$, and by maximality of $M$, $\langle M, x\rangle$ must not be in $\mathfrak{F}$ and so contains a power of $r$. Hence, $r^{n} \in (x)$ for some $n$. Similarly, $r^{m} \in (y)$ for some $m$. Thus, $r^{n} = kx$ and $r^{m} = k'y$ for some $k,k' \in R$. We have that $xy \in M$ implies $kk'xy \in M$, thus, $r^{n+m} \in M$, which is a clear contradiction. Thus, $M$ must be prime. We conclude that if $r \not\in N$, then there exists a prime ideal of $R$, $\mathfrak{p}$, such that $\mathfrak{p}$ does not contain $r$. Therefore, $r$ is not contained in the intersection of all prime ideals. 

\paragraph{14.} Let $R$ be a commutative ring and let $J(R)$ be its Jacobson radical i.e the intersection of all maximal ideals of $R$. Suppose that $r \in J(R)$ and suppose that there is an $s \in R$ such that $1 + rs$ is not a unit. By Proposition 3.5, there is a maximal ideal $\mathfrak{m}$ of $R$ such that $\mathfrak{m} \supseteq (1 + rs) \ni 1 + rs$. However, as $r \in J(R)$, $r$ must be contained in every maximal ideal of $R$, thus, $r \in \mathfrak{m}$. And so we must have that $rs \in \mathfrak{m}$. As $1 + rs, rs \in \mathfrak{m}$, $1 \in \mathfrak{m}$, which contradicts the maximality of $\mathfrak{m}$. Therefore, $1 + rs$ must be a unit for all $s$. For the converse, let $r \in R$ and suppose that $1 + rs$ is a unit for all $s \in R$. Assume that $r \not\in \mathfrak{m}$ for some maximal ideal $\mathfrak{m}$. As $\mathfrak{m}$ is maximal, $\langle \mathfrak{m}, r\rangle = R$, and so there exists an element $m \in \mathfrak{m}$ and $x \in R$ such that $m + xr = 1$. Hence, $1 - xr \in \mathfrak{m}$. As $1 + rs$ is a unit for all $s$, $\mathfrak{m} = R$, which is contradiction. We must have that $r \in \mathfrak{m}$ for all maximal ideals $\mathfrak{m}$ of $R$. 

\paragraph{15.}

\subsection*{5.4 - Unique Factorisation In Polynomial Rings}
\paragraph{1.} Let $R$ be a ring, and let $I$ be an ideal of $R$. Define a map $f:R[x] \to (R/I)[x]$ by sending $a_{0} + a_{1}x + ... + a_{n}x^{n}$ to $(a_{0} + I) + (a_{1} + I)x + ... + (a_{n} + I)x^{n}$. This map is clearly surjective. Furthermore, $\ker f = RI[x]$. Hence, $R[x]/IR[x] \cong (R/I)[x]$. 

\paragraph{2.} Let $R$ be the ring of integers, and let $I = (2)$. We have that $R/I \cong \mathbb{F}_{2}$ is a field, hence, $I$ is maximal. However, $R[x]/IR[x] \cong \mathbb{F}_{2}[x]$ is not a field as $x$ is not invertible, hence, $IR[x]$ is not maximal. 

\paragraph{3.} Let $R$ be a PID and let $f \in R[x]$. If $f$ is very primitive, then it is clearly primite. Suppose that $f$ is primitive. Let $\mathfrak{p}$ be a prime ideal of $R$. As $R$ is a PID, then $\mathfrak{p}$ is a principle prime idel of $R$. As $f$ is primitive, $f \not\in \mathfrak{p}R[x]$. Therefore, $f$ is very primitive. Consider the UFD $R = \mathbb{Z}[x]$. Let $f \in \mathbb{Z}[x,y]$ be the polynomial $f = 2 + xy$. Note $f$ is not very primitive as $(2,x) \neq (1)$, however, $f$ is primitive as $\gcd(2,x) = 1$. 

\paragraph{4.} Let $R$ be a commutative ring and let $f,g \in R[x]$. Suppose that $fg$ is very primitive. Then, for all prime ideals $\mathfrak{p}$ of $R$, $fg \not\in \mathfrak{p}R[x]$. As $\mathfrak{p}R[x]$ is an ideal, $f,g \not\in \mathfrak{p}R[x]$. Hence, $f,g$ are very primitive. For the converse, suppose that both $f$ and $g$ are very primitive. For all prime ideals $\mathfrak{p}$ of $R$, we have that $f \not\in \mathfrak{p}R[x]$ and $g \not\in \mathfrak{p}R[x]$. By Corollary 4.2, $\mathfrak{p}R[x]$ is prime, hence, $fg \not\in \mathfrak{p}R[x]$. Thus, $fg$ is very primitive. 

\paragraph{5.}

\paragraph{6.}
\subparagraph{(i)} 


\subparagraph{(ii)}

\paragraph{7.} Let $S$ be a multiplicatively closed subset of a commutative ring $R$. Define a relation $\sim$ on the set of pairs $(a,s) \in R \times S$ by setting $(a,s) \sim (a',s')$ if and only if there exists a $t \in S$ such that $t(s'a - sa') = 0$. 

\subparagraph{(i)} We verify $\sim$ is an equivalence relation. We have that $(a,s)\sim (a,s)$ for all $(a,s) \in R \times S$ as $1(as - as) = 0$ and $1 \in S$. Suppose that $(a,s) \sim (a',s')$ for some $(a,s), (a',s') \in R \times S$. Then, there exists a $t \in S$ such that $t(as' - a's) = 0$. By multiplication of $-1 \in R$ on both sides, we obtain $t(sa' - s'a) = 0$. Hence, $(a',s') \sim (a,s)$. Finally, suppose that $(a,s) \sim (a',s')$ and $(a',s') \sim (a'',s'')$. There exists $t,t' \in S$ such that $t(s'a - sa') = 0$ and $t'(s''a' - s'a'') = 0$. Thus, $ts'a = tsa'$ and $t's''a' = t's'a''$. So $tt's's''a = tt's''a' = tst's'a''$, and $tt's'(s''a - sa'') = 0$. Note $tt's' \in S$ as $t,t',s' \in S$. Hence, $(a,s) \sim (a'',s'')$. Therefore, $\sim$ is an equivalence relation. 

\subparagraph{(ii)} Denote the equivalence class of $(a,s)$ by $a/s$, and let $S^{-1}R$ denote the set of equivalence classes. Define the operation $+, \cdot$ on the set of equivalence classes under $\sim$ as below
$$\frac{a}{s} + \frac{a'}{s'} = \frac{as' + a's}{ss'}, \ \ \frac{a}{s} \cdot \frac{a'}{s'} = \frac{aa'}{ss'}$$
Let $a/s, a'/s', b/t, b'/t' \in S^{-1}R$ such that $a/s = b/t$ and $a'/s' = b'/t'$. We note that there exists $u,u' \in S$ such that $u(at - bs) = 0$ and $u'(a't' - b's') = 0$. Then, 
\begin{align*} 
uu'(tt'(as' + a's) - ss'(bt' + b't)) &= uu'(tt'as' + tt'a's - ss'bt' - ss'b't) \\ 
&= uu'(s't'(at - bs) + st(a't' - b's')) \\
&= u's't'u(at - bs) + ustu'(a't' - b's') \\
& = 0
\end{align*}
Therefore, 
$$\frac{a}{s} + \frac{a'}{s'} = \frac{as' + a's}{ss'} = \frac{bt' + b't}{tt'} = \frac{b}{t} + \frac{b'}{t'}$$
Furthermore, 
\begin{align*}
uu'(aa'tt' - bb'ss') &= uu'(aa'tt' - a't'bs + a't'bs - bb'ss') \\
&= uu'(a't'(at - bs) + bs(a't' - b's')) \\
&= u'a't'u(at - bs) + ubsu'(a't' - b's') \\
&= 0
\end{align*}
Therefore, 
$$\frac{a}{s} \cdot \frac{a'}{s'} = \frac{aa'}{ss'} = \frac{bb'}{tt'} = \frac{b}{t} \cdot \frac{b'}{t'}$$
We have that $+, \cdot$ are well-defined operations. 

\subparagraph{(iii)}
We now prove $S^{-1}R$ forms a ring under the above operations. We have that $0/1 \in S^{-1}R$ and that for all $a/s \in S^{-1}R$, $a/s + 0/1 = (a1 + 0s)/s1 = a/s$. Hence, $0/1$ is an identity element under $+$. Furthermore, for each $a/s \in S^{-1}R$, $(-a)/s \in S^{-1}R$ and $a/s + (-a)/s = (as + (-a)s)/s^{2} = 0/s^{2}$. We have that $0/s^{2} = 0/1$ as for any $t \in S$, $t(0\cdot 1 - 0 \cdot s^{2}) = t0 = t$. Hence, $(-a)/s$ is the inverse element for any $a/s \in S^{-1}R$. We also have that for any triplet $a/s, a'/s', a''/s'' \in S^{-1}R$,
\begin{align*}
\qty(\frac{a}{s} + \frac{a'}{s'}) + \frac{a''}{s''} &= \frac{as' + a's}{ss'} + \frac{a''}{s''} \\
&= \frac{s''(as' + a's) + a''ss'}{ss's''} \\
&= \frac{as's'' + a'ss'' + a''ss'}{ss's''} \\
&= \frac{as's'' + s(a's'' + a''s')}{ss's''} \\ 
&= \frac{a}{s} + \frac{a's'' + a''s'}{s's''} \\ 
&= \frac{a}{s} + \qty(\frac{a'}{s'}+\frac{a''}{s''}) \\
\end{align*}
Therefore, $+$ is associative. Finally, we note that $a/s + a'/s' = (as' + a's)/ss' = (a's + as')/s's = a'/s' + a/s$. Therefore, $S^{-1}R$ forms an abelian group under $+$. Now, we note that $1/1 \in S^{-1}R$, and we have that for any $a/s \in S^{-1}R$, $a/s \cdot 1/1 = (a1)/(s1) = a/s$. Therefore, $S^{-1}R$ has an identity element. Next, we note that for any triplet $a/s, a'/s', a''/s'' \in S^{-1}R$
\begin{align*}
\qty(\frac{a}{s} \cdot \frac{a'}{s'}) \cdot \frac{a''}{s''} &= \frac{aa'}{ss'}\cdot\frac{a''}{s''} \\
&= \frac{aa'a''}{ss's''} \\
&= \frac{a}{s}\cdot\frac{a'a''}{s's''} \\
&= \frac{a}{s}\cdot\qty(\frac{a'}{s'}\cdot \frac{a''}{s''}) \\
\end{align*}
Therefore, $\cdot$ is associative. Note also $a/s \cdot a'/s' = aa'/ss' = a'a/s's = a'/s' \cdot a/s$. Finally, 
\begin{align*}
\frac{a}{s}\cdot \frac{a'}{s'} + \frac{a}{s}\cdot \frac{a''}{s''} &= \frac{aa'}{ss'} + \frac{aa''}{ss''} \\
&= \frac{aa'ss'' + aa''ss'}{s^{2}s's''} \\
&= \frac{s}{s}\cdot\frac{aa's'' + aa''s'}{ss's''} \\
&= \frac{1}{1}\cdot\frac{aa's'' + aa''s'}{ss's''} \\
&= \frac{aa's'' + aa''s'}{ss's''} \\
&= \frac{a}{s}\cdot\frac{a's'' + a''s'}{s's''} \\
&= \frac{a}{s}\cdot\qty(\frac{a'}{s'} + \frac{a''}{s''}) \\
\end{align*}
And so $S^{-1}R$ is a commutative ring. Define the map $\ell:R \to S^{-1}R$ by sending $a$ to $a/1 \in S^{-1}R$. We have that $\ell(1) = 1/1$ which is the multiplicative identity in $S^{-1}R$. Furthermore, 
$$\ell(a + b) = \frac{a+b}{1} = \frac{a1 + b1}{1\cdot 1} = \frac{a}{1} + \frac{b}{1} = \ell(a) + \ell(b)$$
$$\ell(ab) = \frac{ab}{1} = \frac{a\cdot b}{1 \cdot 1} = \frac{a}{1}\cdot \frac{b}{1} = \ell(a)\ell(b)$$
for any $a,b \in R$. Hence, $\ell$ is a ring homomorphism. 

\subparagraph{(iv)} Let $s \in S$. We have that 
$$\ell(s) \cdot \frac{1}{s} = \frac{s}{1}\cdot \frac{1}{s} = \frac{s1}{1s} = \frac{s}{s} = \frac{1}{1}$$
Hence, $\ell(s)$ is invertible. 

\subparagraph{(v)} Let $R'$ be a commutative ring and $f:R \to R'$ a ring homomorphism such that $f(s)$ is invertible for every $s \in S$. Suppose there exists a ring homomorphism $\alpha$ such that the following diagram commutes
\[\begin{tikzcd}[ampersand replacement=\&]
	{S^{-1}R} \&\& {R'} \\
	\& R
	\arrow["\alpha", from=1-1, to=1-3]
	\arrow["f"', from=2-2, to=1-3]
	\arrow["\ell", from=2-2, to=1-1]
\end{tikzcd}\]
We have that $f = \alpha \circ \ell$. For each $r \in R$, we have that $f(r) = (\alpha \circ \ell)(r) = \alpha(r/1)$. Hence, for any $a/s \in S^{-1}R$, 
$$\alpha\qty(\frac{r}{s}) = \alpha\qty(\frac{r}{1}\frac{1}{s}) = \alpha\qty(\frac{r}{1})\alpha\qty(\frac{1}{s}) = f(r)\alpha\qty(\frac{s}{1})^{-1} = f(r)f(s)^{-1}$$
which gives uniqueness of $\alpha$. We verify $\alpha$ is a ring homomorphism. Let $r/s, r'/s' \in S^{-1}R$. 
\begin{align*}\alpha\qty(\frac{r}{s} + \frac{r'}{s'}) &= \alpha\qty(\frac{rs' + r's}{ss'}) = f(rs' + r's)f(ss')^{-1} \\
&= (f(r)f(s') + f(r')f(s))f(s)^{-1}f(s')^{-1} \\
&= f(r)f(s)^{-1} + f(r')f(s')^{-1} \\
&= \alpha\qty(\frac{r}{s}) + \alpha\qty(\frac{r'}{s'}) \\
\end{align*}
\begin{align*}\alpha\qty(\frac{r}{s} \cdot \frac{r'}{s'}) &= \alpha\qty(\frac{rr'}{ss'}) = f(rr')f(ss')^{-1} \\
&= f(r)f(r')f(s)^{-1}f(s')^{-1} \\
&= f(r)f(s)^{-1}f(r')f(s')^{-1} \\
&= \alpha\qty(\frac{r}{s})  \cdot\alpha\qty(\frac{r'}{s'}) \\
\end{align*}
Finally, $\alpha(1/1) = f(1)f(1)^{-1} = 1$. Therefore, $\alpha$ is a ring homomorphism. It follows that $\ell$ is initial among ring homomorphisms $f:R \to R'$ such that $f(s)$ is invertible for every $s \in S$.

\subparagraph{(vi)} Suppose that $R$ is an integral domain and suppose $S$ is a multiplicative subset of $R$ such that $0 \not\in S$. Suppose there exists $a/s, a'/s' \in S^{-1}R$ such that $(a/s)\cdot (a'/s') = 0/1$. Then, $aa'/ss' = 0/1$ and so there exists a $t \in S$ such that $t(aa'1 - 0ss') = 0$. Hence, $taa' = 0$. As $R$ is an integral domain either $t = 0$ or $aa' = 0$. As $t \in S$ and $0 \not\in S$, $t \neq 0$, so that $aa' = 0$. As $R$ is an integral domain either $a = 0$ or $a' = 0$. Then, either $a/s = 0/1$ or $a'/s' = 0/1$. Therefore, $S^{-1}R$ is an integral domain. 

\subparagraph{(vii)} Let $R$ be a commutative ring and let $S$ be a multiplicative subset of $R$. Suppose that $0 \in S$. Then, $a/s = a'/s'$ for all $a,a' \in R$ and $s,s' \in S$ as $0(as' - a's) = 0$. Hence, $S^{-1}R$ is the zero ring. For the converse, suppose that $S^{-1}R$ is the zero ring. Then, $1/1 = 0/1$ and so there exists a $t \in S$ such that $t(1\cdot 1 - 0 \cdot 1) = 0$. Then, $t = 0$. And so $0 \in S$. 

\paragraph{8.}

\paragraph{9.} Let $R$ be a commutative ring, and $S$ a multiplicative subset of $R$

\subparagraph{(i)} Suppose that $I$ is an ideal of $R$ such that $I \cap S = \emptyset$. Let $I^{e} = S^{-1}I$. Let $x/s, y/s' \in I^{e}$. We have tht $xs', ys \in I$ as $x,y \in I$ and $I$ is an ideal. Thus, $xs' - ys \in I$. Hence, $(x/s) - (y/s') = (xs' - ys)/ss' \in I^{e}$. Furthermore, let $r/s \in S^{-1}R$ and $i/s' \in I^{e}$. Then, $(r/s)(i/s') = ri/ss' \in I^{e}$ as $ri \in I$. Suppose that $I^{e} = S^{-1}R$. Then, $1/1 \in I^{e}$, and so $1 \in I$. However, $1 \in S$, but we had assumed $I \cap S = \empty$. Therefore, $I^{e}$ must be a proper ideal of $S^{-1}R$. 

\subparagraph{(ii)} Let $\ell:R \to S^{-1}R$ be the natural homomorphism, and let $J$ be a proper ideal of $S^{-1}R$. Further, let $J^{c} = \ell^{-1}(J)$. As the preimage of an ideal is an ideal, $J^{c}$ is an ideal. Suppose that there exists an element $x \in J^{c} \cap S$. Then, $x \in J^{c}$ and $x \in S$. As $x \in J^{c}$, we have that $\ell(x) \in J$. As $x \in S$, $\ell(x)$ is a unit in $S^{-1}R$. Therefore, $J = R$. However, this is a contradiction as $J$ is a proper ideal. Hence, $J^{c} \cap S = \emptyset$. 

\subparagraph{(iii)} Let $J$ be a proper ideal of $S^{-1}R$. Let $x/s \in (J^{c})^{e}$. Then, $x \in \ell^{-1}(J)$ and so $\ell(x) = x/1 \in J$. As $J$ is an ideal, $x/s = (1/s)(x/1) \in J$. For the reverse inclusion, let $x/s \in J$. Then, as $J$ is an ideal, $x/1 = (s/1)(x/s) \in J$, and so $x \in \ell^{-1}(J) = J^{c}$. Hence, $x/s \in (J^{c})^{e}$. Now, let $x \in (I^{e})^{c}$. Then, $x/1 = \ell(x) = S^{-1}I$. Hence, $x \in I$ and so $x \in \{a \in R \mid (\exists s \in S) \ sa \in I\}$ as $1x = x$. For the reverse inclusion, let $x \in \{a \in R \mid (\exists s \in S) \ sa \in I\}$. There is a $s \in S$ such that $sx \in I$. Hence, $\ell(x) = x/1 = sx/s \in S^{-1}I$. Thus, $x \in (I^{e})^{c}$. 

\subparagraph{(iv)} Let $S = \{1, x, x^{2}, ...\}$ in $R = \mathbb{C}[x,y]$ and let $I = (xy)$. We have that $y \in (I^{e})^{c}$ as $xy \in (xy)$, however, $y \not\in (xy)$. Therefore, it does not necessarily hold that $(I^{e})^{c} = I$. 

\paragraph{10.} Let $R$ be a commutative ring and $S$ a multiplicative subset of $R$. We set to prove that the assignment $\mathfrak{p} \mapsto S^{-1}$ is an inclusion-preserving bijection between the set of prime ideals of $R$ disjoint from $S$ and the set of prime ideals of $S^{-1}R$. Let $\mathfrak{p}, \mathfrak{p}'$ be prime ideals of $R$ such that $\mathfrak{p} \subseteq \mathfrak{p}'$. Let $x/s \in S^{-1}\mathfrak{p}$. Then, $x \in \mathfrak{p} \subseteq \mathfrak{p}'$. Hence, $x/s \in S^{-1}\mathfrak{p}'$. Let $\mathfrak{p}$ be a prime ideal of $R$ disjoint from $S$. We have that $(\mathfrak{p}^{e})^{c} = \{a \in R \mid (\exists s \in S) \ sa \in \mathfrak{p}\} \supseteq \mathfrak{p}$. Let $x \in (\mathfrak{p}^{e})^{c}$. Then, there exists some $s \in S$ such that $sx \in \mathfrak{p}$. Either $s \in \mathfrak{p}$ or $x \in \mathfrak{p}$. If $s \in \mathfrak{p}$, then $x \in \mathfrak{p} \cap S$, which cannot occur. Hence, $x \in \mathfrak{p}$. It follows that $\mathfrak{p} = (\mathfrak{p}^{e})^{c}$. Therefore, the assignment has a left inverse, so that the assignment is injective. Let $J$ be a prime ideal of $S^{-1}R$. Let $x,y \in R$ such that $xy \in J^{c}$. Then, $\ell(xy) = xy/1 = (x/1)(y/1)$ is contained in $J$. Hence, either $x/1 \in J$ or $y/1 \in J$. Therefore, either $x \in J^{c}$ or $y \in J^{c}$. It follows that $J^{c}$ is a prime ideal in $R$. Also note that $(J^{c})^{e} = J$. Hence, the assignment has a right inverse. It follows that the assignment is surjective. We conclude that the assignment is a inclusion-preserving bijection. 

\paragraph{11.} Let $R$ be a commutative ring and let $\mathfrak{p}$ be a prime ideal of $R$. Let $S = R - \mathfrak{p}$. As $\mathfrak{p}$ is a prime ideal, $1 \not\in \mathfrak{p}$, so that $1 \in S$. Let $x,y \in S$. Then, $x, y \not\in \mathfrak{p}$, and so $xy \not\in \mathfrak{p}$, otherwise $x \in \mathfrak{p}$ or $y \in \mathfrak{p}$ by the primality of $\mathfrak{p}$. Therefore, $xy \in S$. It follows that $S$ is a multiplicative subset of $R$. By the previous exercise, there is a inclusion-preserving bijection between the set of prime ideals disjoint from $S$ and the set of prime ideals of $R_{\mathfrak{p}}$. A prime ideal disjoint from $S$ is the same as a prime ideal contained in $\mathfrak{p}$ as $S = R - \mathfrak{p}$. Therefore, there exists an inclusion-preserving bijection from the set of prime ideals contained in $\mathfrak{p}$ and the set of prime ideals of $R_{\mathfrak{p}}$. We claim the prime ideal of $R_{\mathfrak{p}}$ associated with $\mathfrak{p}$ in $R$, $\mathfrak{m}$, is maximal. Suppose there exists an ideal $I$ such that $\mathfrak{m} \subseteq I$. Then, there exists an ideal $J$ in $R$ that is associated with $I$ and $\mathfrak{p} \subseteq J$, as the association is inclusion preserving. However, $J$ is contained in $\mathfrak{p}$, hence, $J = \mathfrak{p}$ and so $\mathfrak{m} = I$. It follows that $\mathfrak{m}$ is a maximal ideal. Let $\mathfrak{m}'$ in $R_{\mathfrak{p}}$ be a maximal ideal. $\mathfrak{m}'$ is associated with some prime ideal $\mathfrak{p}'$ of $R$ that is contained in $\mathfrak{p}$. We have that $\mathfrak{p}' \subseteq \mathfrak{p}$, hence, $\mathfrak{m}' \subseteq \mathfrak{m}$. Thus, $\mathfrak{m} = \mathfrak{m}'$, which makes $\mathfrak{m}$ unique. We conclude that $R_{\mathfrak{p}}$ is a local ring. 

\paragraph{12.} Let $R$ be a commutative ring, and let $M$ be an $R$-module. Suppose that $M = 0$. Then, $M$ has no prime ideals, and it holds that $M_{\mathfrak{p}} = 0$ vacuously for all prime ideals $\mathfrak{p}$ of $R$. Now, suppose that for all prime ideals $\mathfrak{p}$ of $R$, $M_{\mathfrak{p}} = 0$. Let $\mathfrak{m}$ be a maximal ideal of $R$. Then, $\mathfrak{m}$ is a prime ideal, and so by assumption, $M_{\mathfrak{m}} = 0$. Now, suppose that for all maximal ideals $\mathfrak{m}$ of $R$, $M_{\mathfrak{m}} = 0$. Suppose, for contradiction, $M \neq 0$. There exists $x \in M$ such that $x \neq 0$. We have that the set $\{r \in R \mid rx = 0\}$ is a proper ideal of $R$, and by Proposition 3.5, it is contained in some maximal ideal $\mathfrak{m}$. By assumption, $M_{\mathfrak{m}} = 0$. Hence, $x/1 = 0/1$ in $M_{\mathfrak{m}}$. There then exists a $t \in R - \mathfrak{m}$ such that $tx = 0$. However, as $\mathfrak{m}$ contains $I$, $t$ cannot be an element of $R - \mathfrak{m}$ as $t \in I$. Therefore, $M = 0$. 

\paragraph{13.}

\paragraph{14.} Let $M$ be an $R$-module, $S$ a multiplicative subset of $M$, and $\hat{N}$ a submodule of $S^{-1}M$. Let $x/s \in (\hat{N}^{c})^{e}$. Then, $x \in \ell^{-1}(\hat{N})$ and so $x/1 \in \hat{N}$. Therefore, $x/s = (x/1)(1/s) \in \hat{N}$. For the converse, let $x/s \in \hat{N}$. We have that $x/1 = xs/s = s(x/s) \in \hat{N}$. Hence, $\ell(x) \in \hat{N}$. Thus, $x \in \hat{N}^{c}$. Therefore, $x/s \in (\hat{N}^{c})^{e}$. Suppose that $M$ is a Noetherian $R$-module, and let $\hat{N}$ be a submodule of $S^{-1}M$. We have that $\hat{N}^{c}$ is a submodule of $M$, hence, it is finitely generated, that is, $\hat{N}^{c} = \langle x_{1}, ..., x_{n}\rangle$ for some $x_{1},..., x_{n} \in M$. We have that $x/s \in (\hat{N}^{c})^{e}$. Then, $x \in \hat{N}^{c}$ and so $x = r_{1}x_{1} + ... + r_{n}x_{n}$ for some $r_{1}, ..., r_{n} \in R$. Thus, 
$$\frac{x}{s} = \frac{r_{1}x_{1} + ... + r_{n}x_{n}}{s} = \frac{r_{1}x_{1}}{s} + ... + \frac{r_{n}x_{n}}{s} = r_{1}\frac{x_{1}}{s} + ... + r_{n}\frac{x_{n}}{s} \in \langle x_{1}/s, ..., x_{n}/s \rangle$$
Therefore, $\hat{N} = (\hat{N}^{c})^{e} = \langle x_{1}/s, ..., x_{n}/s \rangle$ and so $S^{-1}M$ is Noetherian. 

\paragraph{15.}

\paragraph{16.} \textbf{HALF DONE} Let $R$ be a Noetherian domain, and let $s \in R$ be a prime element. Let $S = \{s^{n} | n \geq 0\}$. Suppose that $R$ is a UFD. We have that $S^{-1}R$ is a Noetherian domain by a previous exercise. Let $\hat{\mathfrak{p}}$ be a prime ideal of $S^{-1}R$ that is of height $1$. $\hat{\mathfrak{p}}^{c}$ is a prime ideal of $R$ of height $1$ as contraction is a inclusion preserving bijection between the set of prime ideals of $R$ disjoint from $S$, and the set of prime ideals of $S^{-1}R$. As $R$ is a Noetherian domain that is a UFD, $\hat{\mathfrak{p}}^{c}$ is principal, that is, $\hat{\mathfrak{p}}^{c} = (p)$ for some $p \in R$. Let $x \in S^{-1}(p)$. Then, $x = (rp)/s^{n}$ for some $r \in R, n \geq 0$. We have that $x = (rp)/s^{n} = (r/s^{n})(p/s) \in (p/s)$. Let $x \in (p/s)$. Then, $x = (r/s^{n})(p/s) = rp/s^{n+1} \in S^{-1}(p)$ where $r/s^{n} \in S^{-1}R$. Therefore, $\hat{\mathfrak{p}} = (\hat{\mathfrak{p}}^c)^{e}$ is principal. Therefore, $S^{-1}R$ is a UFD.   

\paragraph{17.} Let $F$ be a field and suppose that $F$ has characteristic $0$. The the map $\varphi:\mathbb{Z} \to F$ is injective and has kernel $\ker{\varphi} = \{0\}$. By the First Isomorphism Theorem $F$ contains an isomorphic copy of $\mathbb{Z}$, $Z$ say. We must have that $\mathbb{Q} = K(\mathbb{Z}) \cong K(Z)$, and by definition $K(\mathbb{Z}) = \mathbb{Q}$ is the smallest field containing $\mathbb{Z}$. As $F$ is a field, it must contain an isomorphic copy to $\mathbb{Q}$. For the converse, suppose that $F$ contains an isomorphic copy of $\mathbb{Q}$, $Q$ say. We have that the inclusion map $i:\hookrightarrow F$ is a ring homomorphism. Let $f:\mathbb{Z} \to Q$ be the unique ring homomorphism from $\mathbb{Z}$ to $Q$. As $\mathbb{Q}$ has characteristic $0$, $f$ has the trivial kernel. Furthermore, $i$ has the trivial kernel as a homomorphism of fields. Therefore, $f \circ i:\mathbb{Z} \to F$ has trivial kernel. It follows that $F$ has characteristic $0$ as $\mathbb{Z}$ is initial in $\mathsf{Ring}$. Now, suppose that $F$ has characteristic $p$ prime. The unique map $\varphi:\mathbb{Z} \to F$ has kernel $p\mathbb{Z}$, and so $\mathbb{Z}/p\mathbb{Z}$ embeds into $F$. For the converse, suppose that $F$ contains an isomorphic copy of $\mathbb{Z}/p\mathbb{Z}$. We have that the inclusion map $i:\mathbb{Z}/p\mathbb{Z} \to F$ is a ring homomorphism. Let $f:\mathbb{Z} \to \mathbb{Z}/p\mathbb{Z}$ be the canonical homomorphism. We have that $f$ has kernel $p\mathbb{Z}$ and $i$ has the trivial kernel as a homomorphism of fields. It follows that $i \circ f:\mathbb{Z} \to F$ has kernel $p\mathbb{Z}$, and so $F$ has characteristic $p$. 

\paragraph{18.} Let $R$ be an integral domain. Let $u \in R$ be a unit. There is a $v \in R$ such that $uv = 1 \in R$. View $u,v$ as constant polynomials in $R[x]$, then $uv = 1 \in R[x]$. Hence, the units of $R$ are units of $R[x]$ when viewed as constant polynomials. Let $f \in R[x]$ be a unit and let $g \in R[x]$ such that $fg = 1$. As $R$ is an integral domain, $R[x]$ is an integral domain. Therefore, $0 = \deg(1) = \deg(fg) = \deg(f) + \deg(g)$. It follows that $\deg(f) = \deg(g) = 0$, and so $f,g$ are constant polynomials and can be viewed as elements of $R$. As $fg = 1$, $f,g \in R$ are units in $R$.  

\paragraph{19.} Let $R$ be a commutative ring, and let $a \in R$ be a nilpotent element such that $x^{n} = 0$ for some $n > 0$. Let $u = 1 - x + x^{2} + ... + (-1)^{n-1}x^{n-1}$. We have that
\begin{align*}
(1+x)u &= (1 + x)(1 - x + x^{2} + ... + (-1)^{n-1}x^{n-1}) \\
&=1 - x + x^{2} + ... + (-1)^{n-1}x^{n-1} + x(1 - x + x^{2} + ... + (-1)^{n-1}x^{n-1}) \\
&=1 - x + x^{2} + ... + (-1)^{n-1}x^{n-1} + x - x^{2} + x^{3} + ... + (-1)^{n-1}x^{n} \\
&=1 + (-1)^{n-1}x^{n} \\
&=1
\end{align*}
Therefore, $1 + x$ is a unit. 

\paragraph{20.} Let $R$ be a commutative ring, and let $f = a_{0} + a_{1}x + ... + a_{d}x^{d} \in R[x]$. Suppose that $a_{0}$ is a unit, and $a_{1},...,a_{d}$ are nilpotent in $R$. Define $g = 1 + a_{1}a_{0}^{-1}x + ... + a_{d}a_{0}^{-1}x^{d}$. For any $0 < i \leq d$, we have that $a_{i}$ is nilpotent, and so there exists a $n$ such that $a_{i}^{n} = 0$. Furthermore, $(a_{i}a_{0}^{-1}x^{i})^{n} = a_{i}^{n}a_{0}^{-n}x^{ni} = 0a_{0}^{-n}x^{ni} = 0$, which means $a_{i}a_{0}^{-1}x^{i}$ is nilpotent. By the previous exercise, it follows $g$ is a unit. As $g$ is a unit, and $a_{0}$ is a unit, we have that $f = a_{0}g$ is a unit. For the converse, suppose that $f = a_{0} + a_{1}x + ... + a_{d}x^{d} \in R[x]$ is a unit. There exists a $g = b_{0} + b_{1}x + ... + b_{e}x^{e} \in R[x]$ such that $fg = 1$. Hence, 
$$fg = a_{0}b_{0} + (a_{1}b_{0} + a_{0}b_{1})x + ... + (a_{d-1}b_{e} + a_{d}b_{e-1})x^{d+e-1} + a_{d}b_{e}x^{d+e} = 1$$
It follows that $a_{0}b_{0} = 1$, so $b_{0}$ is a unit, and the coefficients of $x^{k}$ of $fg$ is $0$ for all $k > 0$. Note that $a_{d}b_{e} = 0$. Suppose for all $k < i$, $a_{d}^{k+1}b_{e-k} = 0$. We have that the coefficient of $x^{i}$ is $\sum_{m+n=i}a_{m}b_{n}$, and is equal to $0$. Thus, $a_{d}^{i}\sum_{m+n=i}a_{m}b_{n} = 0$, and so $a_{d}^{i+1}b_{e-i} = 0$ using the induction hypothesis. It follows by the principle of mathematical induction that $a_{d}^{i+1}b_{e-i} = 0$ for all $i$. Therefore, $a_{d}^{e+1}b_{0} = 0$. As $b_{0}$ is a unit, $a_{d}$ is nilpotent. Then, $-a_{d}x^{d}$ is nilpotent, and so $f - a_{d}x^{d}$ is a unit by a previous exercise. Applying this reasoning to $f - a_{d}x^{d}$, we find that $a_{d-1}$ is nilpotent. We find that $a_{1}, ..., a_{d}$ are nilpotent. Therefore, $f = a_{0} + a_{1}x + ... + a_{d}x^{d}$ is a unit in $R[x]$ if and only if $a_{0}$ is a unit, and $a_{1}, ..., a_{d}$ are nilpotent.  

\paragraph{21.}
\paragraph{22.}
\paragraph{23.}
\paragraph{24.}
\paragraph{25.} Let $f,g,h \in \mathbb{C}[t]$ be non-constant polynomials such that $f^{n} + g^{n} = h^{n}$ for some $n > 2$. Suppose that $f,g,h$ are not relatively prime. Let $d = \gcd(f,g,h)$. Then, $f = df', g = dg'$ and $h = dh'$ for some polynomials $f',g',h' \in \mathbb{C}[t]$. As $f^{n} + g^{n} = h^{n}$, we have that $(df')^{n} + (dg')^{n} = (dh')^{n}$. Hence, $d^{n}(f'^{n} + g'^{n} - h'^{n}) = 0$. As $\mathbb{C}$ is a field, $\mathbb{C}[t]$ is an integral domain, hence, $d^{n} = 0$ or $f'^{n} + g'^{n} = h'^{n}$. As $d \neq 0$, $d^{n} \neq 0$, thus, $f'^{n} + g'^{n} = h'^{n}$. In particular, we have found relatively prime polynomials such that they solve Fermats Last Theorem for complex polynomials. Without loss of generality, let $f,g,h$ be relatively prime non-constant polynomials such that $f^{n} + g^{n} = h^{n}$ with $n > 2$. Suppose futher $f,g,h$ have minimal degree. We have that 
$$f^{n} = h^{n} - g^{n} = h^{n}(1 - (g/h)^{n}) = h^{n}\prod_{i=1}^{n}(1 - \zeta^{i}(g/h)) = \prod_{i=1}^{n}(h - \zeta^{i}g)$$
As $\mathbb{C}$ is a field, $\mathbb{C}[t]$ is a UFD, hence, there exists irreducible polynomials $p_{1}, ..., p_{k}$ such that $f = p_{1}...p_{k}$. Furthermore, $f^{n} = p_{1}^{n}...p_{k}^{n}$, and using unique factorisation, $h - \zeta^{i}g = \alpha_{i}^{n}$ for all $i$ for some polynomial $\alpha_{i} \in \mathbb{C}[t]$. Let $h - g = a^{n}, h - \zeta g = b^{n}$ and $h - \zeta^{2} g = c^{n}$ for some polynomials $a,b,c \in \mathbb{C}[t]$. We have that 
$$(1 + \zeta)b^{n} = (1 + \zeta)(h - \zeta g) = h - \zeta g + \zeta h - \zeta^{2} g = (h - \zeta^{2} g) + \zeta (h - g) = c^{n} + \zeta a^{n}$$
And so $(1 + \zeta)b^{n} = c^{n} + \zeta a^{n}$. We can then find complex numbers $\lambda, \mu, \nu \in \mathbb{C}$ such that $(\lambda a)^{n} + (\mu b)^{n} = (\nu c)^{n}$. As $a^{n} = h - g$, the degree of $a$ is strictly less than the degree of $h,g$. Similarly, the degree of $b,c$ is strictly less that the degree of $h,g$. This contradicts the initial assumption. Therefore, there cannot exist non-constant polynomials $f,g,h \in \mathbb{C}[t]$ such that $f^{n} + g^{n} = h^{n}$. 

\subsection*{5.5 - Irreducibility of Polynomials}
\paragraph{1.} Let $f(x) \in \mathbb{C}[x]$. Suppose that $a \in \mathbb{C}$ is a complex number such that $f(a) = f'(a) = ... = f^{(n-1)}(a) = 0$ and $f^{(n)}(a) \neq 0$. By Taylor's Theorem, 
$$f(x) = f(a) + f'(a)(x-a) + ... + \frac{f^{(n-1)}(a)}{(n-1)!}(x-a)^{n+1} + \frac{f^{(n)}(a)}{n!}(x-a)^{n} + h(x)(x-a)^{n+1}$$
for some function $h(x)$. By assumption, 
$$f(x) = \frac{f^{(n)}(a)}{n!}(x-a)^{n} + h(x)(x-a)^{n+1} = (x-a)^{n}\qty(\frac{f^{(n)}(a)}{n!} + h(x)(x-a))$$
We have that $g(x) = \frac{f^{(n)}(a)}{n!} + h(x)(x-a)$ is non-zero at $x = a$ as $f^{(n)}(a) \neq 0$. Hence, $a \in \mathbb{C}$ is a root of $f$ with multiplicity $n$. For the converse, suppose that $a$ is a root of $f$ with multiplicity of $n$. Then, $f = (x-a)^{n}g$ for some $g \in \mathbb{C}[x]$ where $g(a) \neq 0$. For $0 \leq i \leq n-1$, we have that 
$$f^{(i)}(x) = \sum_{k=0}^{i}\binom{i}{k}\frac{d^{k}}{dx^{k}}[(x-a)^{n}]\frac{d^{i-k}}{dx^{i-k}}g = \sum_{k=0}^{i}\binom{i}{k}\frac{n!}{k!}(x-a)^{n-k}g^{(i-k)}$$
And it follows that $f^{(i)}(a) = 0$. For $i = n$, we have that 
$$f^{(n)}(x) = \sum_{k=0}^{n}\binom{n}{k}\frac{n!}{k!}(x-a)^{n-k}g^{n-k} = g(x) + \sum_{k=0}^{n-1}\binom{n}{k}\frac{n!}{k!}(x-a)^{n-k}g^{n-k}$$
Thus, $f^{(n)}(a) = g(a) \neq 0$. Therefore, $a \in \mathbb{C}$ is a root of $f$ if and only if $f(a) = f'(a) = ... = f^{(n-1)}(a) = 0$ and $f^{(n)}(a) \neq 0$. Suppose that $f(x) \in \mathbb{C}[t]$ has multiple roots i.e there exists an $a \in \mathbb{C}$ that is a root of $f$ with multiplicity $r > 1$. Then, $f(a) = f'(a) = 0$ from above. Thus, $(x-a) \mid f$ and $(x-a) \mid f'(a)$, and so $\gcd(f,f') \neq 1$. For the converse, suppose that $\gcd(f,f') \neq 1$. Then, there exists a $g \in \mathbb{C}[t]$ such that $g \mid f$ and $g \mid f'$, hence, $f = gh_{1}$ and $f' = gh_{2}$ for some $h_{1}, h_{2} \in \mathbb{C}[t]$. Taking the derivative of $f = gh_{1}$, we have that $g'h_{1} + gh_{1}' = f' = gh_{2}$. As $g \in \mathbb{C}[t]$ and $\mathbb{C}$ is algebraically closed, $g$ must have a root $a \in \mathbb{C}$. Then, $g'(a)h_{1}(a) + g(a)h_{1}'(a) = g(a)h_{2}(a)$ and so $g'(a)h_{1}(a) = 0$. If $g'(a) = 0$, then $g(a) = g'(a) = 0$, and so $g$ has a factor $(x-a)^{r}$ with $r > 1$. If $h_{1}(a) = 0$, then $x-a \mid h_{1}$ and $h_{1} = (x-a)h_{3}$. Since $g(a) = 0$, we have that $g = (x-a)g_{1}$. Hence, $f = (x-a)^{2}g_{1}h_{1}$, and so $f$ has multiple roots. Therefore, $f$ has multiple roots if and only if $\gcd(f,f') \neq 1$. 

\paragraph{2.} Let $F$ be a subfield of $\mathbb{C}$, and let $f(x) \in F[x]$ be an irreducible polynomial. As $f$ is irreducible, $\gcd(f,f') = 1$ in $F[x]$, otherwise, there would exist a polynomial $g \in F[x]$ that divides $f$, which contradicts the assumption of irreducibility. We have that $F[x] \subseteq \mathbb{C}[x]$ is an inculsion of integral domain and $F[x]$ is a PID. By exercise $2.22$, $\gcd(f,f') = 1$ in $\mathbb{C}[x]$. Hence, using the previous exercise, $f \in \mathbb{C}[x]$ has no multiple roots. 

\paragraph{3.}

\paragraph{4.} Notice that $f(x) = x^{4} + x^{2} + 1 = (x^{2} + x + 1)(x^{2} - x + 1)$. Hence, $f$ is reducible in $\mathbb{Z}[x]$. Suppose that $f$ has rational roots. Let $c = p/q \in \mathbb{Q}$ with $\gcd(p,q) = 1$ be such a root. By Proposition 5.5, $p \mid 1$ and $q \mid 1$ in $\mathbb{Z}$. Hence, $c \in \{-1,1\}$. We see that $f(1) = 3$ and $f(-1) = 3$ so $c$ cannot be rational. Therefore, $f$ does not have rational roots. 

\paragraph{5.}
\paragraph{6.}

\paragraph{7.} Let $R$ be an integral domain, and let $f \in R[x]$ be a polynomial of degree $d$. Let $r_{1}, ..., r_{d+1}$ be distinct elements of $R$. Suppose that $g \in R[x]$ is a polynomial of degree $d$ that agrees with $f$ at $r_{i}$ for each $i$. Then, $f - g$ is a polynomial of degree atmost $d$ with atleast $d+1$ roots. By Lemma $5.1$, this cannot occur unless $f - g$ is the zero polynomial. Therefore, $f = g$. Hence, $f$ is uniquely determined by its value at $d+1$ distinct elements of $R$. 

\paragraph{8.}
\paragraph{9.}
\paragraph{10.}
\paragraph{11.}
\paragraph{12.}
\paragraph{13.}
\paragraph{14.}
\paragraph{15.}
\paragraph{16.}
\paragraph{17.}

\paragraph{18.} Let $f \in \mathbb{Z}[x]$ be a cubic polynomial with odd leading coefficient, and both $f(0), f(1)$ are odd. Suppose that $f(x) = a_{3}x^{3} + a_{2}x^{2} + a_{1}x + a_{0}$. Consider the polynomial $g(x) = a_{3}^{-1}f(x) = x^{3} + b_{2}x^{2} + b_{1}x + b_{0}$. Note that for any $r \in \mathbb{Z}$, 
\begin{align*} g(r) - g(1) &= r^{3} + b_{2}r^{2} + b_{1}r + b_{0} - 1 - b_{2} - b_{1} - b_{0} \\
&= (r^{3} - 1) + b_{2}(r^{2} - 1) + b_{1}(r-1) \\
&= (r-1)[(r^{2} + r + 1) + b_{2}(r+1) + b_{1}]
\end{align*}
\begin{align*} g(r) - g(0) &= r^{3} + b_{2}r^{2} + b_{1}r + b_{0} - b_{0} \\
&= r(r^{2} + b_{2}r + b_{1}) \\
\end{align*}
Hence, $r \mid g(r) - g(0)$ and $(r-1) \mid g(r) - g(1)$. Suppose that $g$ is reducible. By Proposition 5.3, $g$ has a root in $\mathbb{Q}$. Let $c = p/q$ where $\gcd(p,q) = 1$ and $p,q \in \mathbb{Z}$ be such a root. By Proposition 5.5, $p \mid b_{0}$ and $q \mid 1$. Thus, $c = p \in \mathbb{Z}$. We have that $c \mid -g(0)$ and $c-1 \mid -g(1)$. By assumption, $g(0)$ is odd, so $c$ has to be odd. Furthermore, $g(1)$ is odd, so $c-1$ is odd. This is a contradiction, and so $c$ cannot exist. Therefore, $g$ is irreducible in $\mathbb{Q}$. As $f = a_{3}^{-1}g$, and $a_{3} \in \mathbb{Q}$ is a unit, since $g$ is irreducible in $\mathbb{Q}$, $f$ is irreducible in $\mathbb{Q}$. 

\paragraph{19.} Note that $\sqrt{2}$ is the root of the polynomial $f(x) = x^{2} - 2 \in \mathbb{Z}[x]$. By Eisensteins criterion, $f(x)$ is irreducible in $\mathbb{Z}[x]$. By Corollary 4.17, $f$ is irreducible in $\mathbb{Q}[x]$, and so $f = x^{2} - 2$ does not reduce to $(x-\sqrt{2})(x+\sqrt{2})$ in $\mathbb{Q}[x]$. Therefore, $\sqrt{2} \not\in \mathbb{Q}$. 

\paragraph{20.} Let $f(x) = x^{6} + 4x^{3} + 1 \in \mathbb{Z}[x]$. Consider 
\begin{align*}
    f(x+1) &= (x+1)^{6} + 4(x+1)^{3} + 1 \\
    &= x^{6} + 6x^{5} + 15x^{4} + 20x^{3} + 15x^{2} + 6x + 1 + 4(x^{3} + 3x^{2} + 3x + 1) + 1 \\
    &= x^{6} + 6x^{5} + 15x^{4} + 24x^{3} + 27x^{2} +18x + 6 \\
\end{align*}
Note that $3 \mid 6,15,24,27,18$, $3\nmid 1$ and $3^{2} \nmid 6$. Hence, $f(x+1)$ is irreducible by Eisensteins Criterion. Therefore, $f(x)$ is irreducible in $\mathbb{Z}[x]$. 

\paragraph{21.} Suppose that $n$ is not prime. There exists $p,q \in \mathbb{Z}$ such that $n = pq$. Then, 
\begin{align*} 1 + x + x^{2} + ... + x^{n-1} &= \frac{x^{n} - 1}{x - 1} \\
&= \frac{x^{pq} - 1}{x - 1} \\
&= \frac{(x^{p})^{q} - 1}{x - 1} \\
&= \frac{(x^{p} - 1)(1 + x^{p} + x^{2p} + ... + x^{p(q-1)})}{x-1} \\
&= (1 + x + x^{2} + ... + x^{p-1})(1 + x^{p} + x^{2p} + ... + x^{p(q-1)}) \\
\end{align*}
Therefore, $1 + x + x^{2} + ... + x^{n-1}$ is reducible in $\mathbb{Z}$.

\paragraph{22.} Let $R$ be a UFD, and let $a \in R$ be a non-unit element that is not divisible by the square of some irreducible element in its factorisation. Let $p$ be such an element. As $R$ is a UFD, since $p$ is irreducible, $p$ is prime. Let $f(x) = x^{n} - a$ for some $n \geq 1$. We have that $1 \not\in (p)$ and $-a \in (p)$. Furthermore, $-a \not\in (p)^{2}$ by assumption. Therefore, $f$ is irreducible by Eisensteins Criterion.

\paragraph{23.} Let $f(x,y) = y^{5} + x^{2}y^{3} + x^{3}y^{2} + x \in \mathbb{C}[x,y]$. We can view $\mathbb{C}[x,y]$ as $(\mathbb{C}[x])[y]$. We have that $(x) \subseteq \mathbb{C}[x]$ is a prime ideal as $\mathbb{C}[x]/(x) \cong \mathbb{C}$ is a field. Note that $x,x^{2},x^{3} \in \mathbb{C}[x]$, $1 \not\in (x)$, and $x \not\in (x)^{2}$. Therefore, by Eisensteins Criterion, $f(x,y)$ is irreducible in $\mathbb{C}[x,y]$. 

\paragraph{24.}

\subsection*{5.6 - Further Remarks and Examples}
\paragraph{1.} Let $I,J$ be ideals of a commutative ring $R$. Define the map $f_{1}:\textbf{0} \to I \cap J$ by $f_{1}(0) = 0$ where $\textbf{0}$ is the zero ring. We have that $\im{f_{1}} = \{0\}$. Let $f_{2}:I \cap J \to R$ be the inclusion map. We have that $\im{f_{1}} = \ker{f_{2}} = \{0\}$ and note that $\im{f_{2}} = I \cap J$. Define the map $\varphi:R \to R/I \times R/J$ by $r \mapsto (r + I, r + J)$. We have that 
\begin{align*}
    \ker \varphi &= \{x \in R \mid \varphi(x) = (I,J) \} \\
    &= \{x \in R \mid (x+I, x+J) = (I,J) \} \\
    &= \{x \in R \mid x \in I, x \in J\} \\
    &= \{x \in R \mid x \in I \cap J \} \\
    &= I \cap J \\
    &= \im{f_{2}}
\end{align*}
Now, define the map $f_{3}:R/I \times R/J \to R/I+J$ by $(x+I,y+J) \mapsto (x-y) + (I+J)$. We have that
\begin{align*}
    \ker f_{3} &= \{(x+I,y+J) \in R/I \times R/J \mid f_{3}(x+I,y+J) = I+J \} \\
    &= \{(x+I,y+J) \in R/I \times R/J \mid (x-y) + I+J = I+J \} \\
    &= \{(x+I,y+J) \in R/I \times R/J \mid x + I+J = y + I + J\} \\
    &= \{(x+I,y+J) \in R/I \times R/J \mid \exists(i,i' \in I, j,j' \in J), x + i + j = y + i' + j'\} \\
    &= \{(x+I,y+J) \in R/I \times R/J \mid \exists(i,i' \in I, j,j' \in J), x = y + (i'-i) + (j'-j)\} 
\end{align*}
Hence, if $(x+I, y+J) \in \ker f_{3}$, there exists $i,i' \in I$ and $j,j'$ so that $x = y + (i' - i) + (j' - j)$, hence, 
$(x+I, y+J) = (y + j'' + I, y + J)$ for some $j'' \in J$. We have that $\varphi(y + j'') = (y + j'' + I, y + J)$, hence, $(x+I,y+J) \in \im \varphi$. Furthermore, if $(x+I,y+J) \in \im\varphi$, then, there exists some $z \in R$ such that $(z+I,z+J) = (x+I, y+J)$ and so $x-z \in I$ and $y-z \in J$. Hence, $x-y \in I + J$, and so $(x+I,y+I) \in \ker f_{3}$. Therefore, $\im \varphi = \ker f_{3}$. Finally, we have that $f_{3}$ is surjective as if $x + I + J \in R/I+J$, then $f_{3}(x + I, J) = x + I + J$. It follows that the following is an exact sequence of $R$-modules
\[\begin{tikzcd}[ampersand replacement=\&]
	{\textbf{0}} \&\& {I \cap J} \&\& R \&\& {\frac{R}{I}\times \frac{R}{J}} \&\& {\frac{R}{I+J}} \&\& {\textbf{0}}
	\arrow[from=1-1, to=1-3]
	\arrow[from=1-3, to=1-5]
	\arrow["\varphi", from=1-5, to=1-7]
	\arrow[from=1-7, to=1-9]
	\arrow[from=1-9, to=1-11]
\end{tikzcd}\]
We now prove the Chinese Remainder Theorem for $k = 2$. Let $I,J$ be ideals of a commutative ring $R$ such that $I+J = (1)$. We obtain the exact sequence, 
\[\begin{tikzcd}[ampersand replacement=\&]
	{\textbf{0}} \&\& {I \cap J} \&\& R \&\& {\frac{R}{I}\times \frac{R}{J}} \&\& {\textbf{0}} \&\& {\textbf{0}}
	\arrow[from=1-1, to=1-3]
	\arrow[from=1-3, to=1-5]
	\arrow["\varphi", from=1-5, to=1-7]
	\arrow[from=1-7, to=1-9]
	\arrow[from=1-9, to=1-11]
\end{tikzcd}\]
By exactness, $\varphi$ must be surjective, and we obtain the isomorphism $R/I\cap J \cong R/I \times R/J$.

\paragraph{2.} Let $R$ be a commutative ring, and let $a \in R$ be an element in $R$ such that $a^{2} = a$. We have that $1 \in (a) + (1-a)$ as $a + 1 - a = 1$, thus, by the Chinese Remainder Theorem, $R/(a)(1-a) \cong R/(a) \times R/(1-a)$. Let $r \in (a)(1-a)$. Then $r = xay(1-a)$ for some $x,y \in R$. Note that $r = xay(1-a) = xya(1-a) = xy(a - a^{2}) = xy(a - a) = xy0 = 0$. Therefore, $r = 0$. It follows that $(a)(1-a)$ is trivial, hence, $R \cong R/(a)(1-a) \cong R/(a) \times R/(1-a)$. We have that $(a)$ is a ring with identity $a$ as for every $ax \in (a)$, we have that $aax = a^{2}x = ax$ and $(a)$ is an ideal of $R$. Define the map $\varphi:R \to (a)$ by $\varphi(x) = ax$. Note that $\varphi(r(1-a)) = ar(1-a) = r(a-a^{2}) = r(a - a) = 0$, hence, $(1-a) \subseteq \ker\varphi$. Suppose that $x \in \ker\varphi$. Then, $ax = 0$. Note that $x = ax + (1-a)x = (1-a)x \in (1-a)$. Therefore, we have that $\ker \varphi = (1-a)$. By the first isomorphism theorem, $R/(1-a) \cong (a)$. By a similar argument, $(1-a)$ can be viewed as a ring with identity $1-a$, and $R/(a) \cong (1-a)$. As $R \cong R/(a) \times R/(1-a)$, we have that 
$$R \cong (a) \times (1-a)$$

\paragraph{3.}

\paragraph{4.} Let $R$ be a finite commutative ring, and let $p$ be the smallest prime dividing $|R|$. Let $I_{1}, ..., I_{k}$ be proper ideals of $R$ such that $I_{i} + I_{j} = (1)$ for $i \neq j$. By the CRT, 
$$\frac{R}{I_{1}...I_{k}} \cong \frac{R}{I_{1}} \times ... \times \frac{R}{I_{k}}$$
Then, 
$$\frac{|R|}{|I_{1}...I_{k}|} = \qty|\frac{R}{I_{1}...I_{k}}| = \qty|\frac{R}{I_{1}} \times ... \times \frac{R}{I_{k}}| = \qty|\frac{R}{I_{1}}|...\qty|\frac{R}{I_{k}}| = \frac{|R|^{k}}{|I_{1}|...|I_{k}|}$$
Thus, $|R|^{k-1} = |I_{1}|...|I_{k}|/|I_{1}...I_{k}| \leq |I_{1}|...|I_{k}|$. As $p$ is the smallest prime divisor of $|R|$, by Lagranges Theorem, $|I_{i}| \leq |R|/p$ for each $i$. Hence, $|I_{1}|...|I_{k}| \leq (|R|/p)^{k}$. Therefore, $|R|^{k-1} \leq (|R|/p)^{k}$. By taking the logarithm base $p$ of both sides, we have that $(k-1)\log_{p}|R| \leq k(\log_{p}|R| - 1)$, and so $k \leq \log_{p}|R|$. 

\paragraph{5.} Let $\varphi:\mathbb{Z}[x] \to \mathbb{Z}[x]/(x) \times \mathbb{Z}[x]/(2)$ be the canonical map. Suppose there exists an $f \in \mathbb{Z}[x]$ such that $\varphi(f) = (1 + (x), (2))$. Then, $f - 1 \in (x)$ and $f \in (2)$. There exists $g,h \in \mathbb{Z}[x]$ such that $f-1 = xg$ and $f = 2h$. Hence, $xg + 1 = 2h$. Write $g = g_{0} + g_{1}x + ... + g_{n}x^{n}$ and $h = h_{0} + h_{1}x + ... + h_{m}x^{m}$. Thus, $1 + g_{0}x + g_{1}x^{2} + ... + g_{n}x^{n+1} = 2h_{0} + 2h_{1}x + ... + 2h_{m}x^{m}$. It follows that $2h_{0} = 1$, which means that $h_{0} = 1/2$. This is a contradiction as $h \in \mathbb{Z}$. Therefore, such an $f \in \mathbb{Z}$ cannot exist, and so $\varphi$ is not surjective.  

\paragraph{6.} Let $R$ be a UFD
\subparagraph{(i)} Let $a,b \in R$ such that $\gcd(a,b) = 1$. Let $x \in (ab)$. Then, $x = kab$ for some $k \in R$ and so $x = kab \in (a)$ and $x = kab \in (b)$. Thus, $x \in (a) \cap (b)$. For the converse, suppose that $x \in (a) \cap (b)$. Then, $x = ra = r'b$ for some $r,r' \in R$. As $\gcd(a,b) = 1$, we have that there exists $p,q \in R$ such that $ap + bq = 1$. Then, 
$$x = ra = ra1 = ra(ap + bq) = (ra)ap + rq(ab) = (r'b)ap + rq(ab) + r'p(ab) = ab(rq + r'p) \in (ab)$$
Therefore, $(a) \cap (b) = (ab)$. 

\subparagraph{(ii)}

\paragraph{7.} Note that
$$x^{100} + (x^{2}+1)\sum_{n=0}^{49}(-1)^{n}x^{2n} = 1$$
Consider $f = x^{100} + x(x^{2}+1)\sum_{n=0}^{49}(-1)^{n}x^{2n}$. We have that 
$$f \equiv x(x^{2}+1)\sum_{n=0}^{49}(-1)^{n}x^{2n} \mod x^{100} \equiv x(1 - x^{100}) \mod x^{100} \equiv x \mod x^{100}$$
$$f \equiv x^{100} \mod (x^{2}+1) \equiv 1 - (x^{2}+1)\sum_{n=0}^{49}(-1)^{n}x^{2n} \mod (x^{2}+1) \equiv 1 \mod (x^{2}+1)$$
We simplify $f$ as $f = -x^{101} + x^{100} + x$, and we have that $f$ satisfies our required properties. 

\paragraph{8.} Let $n \in \mathbb{Z}$ be a positive integer and $n = p_{1}^{a_{1}}...p_{r}^{a_{r}}$ its prime factorisation. 

\subparagraph{(i)} For $i \neq j$, we have that $\gcd(p_{i}^{a_{i}}, p_{j}^{a_{j}}) = 1$ as $p_{i}, p_{j}$ are prime. By Bezouts lemma, there exists $x,y$ such that $p_{i}^{a_{i}}x + p_{j}^{a_{j}}y = 1$. Hence, $(p_{i}^{a_{i}}) + (p_{j}^{a_{j}}) = (1)$. By the CRT, $\mathbb{Z}/(n) \cong \mathbb{Z}/(p_{1}^{a_{1}})...(p_{r}^{a_{r}}) \cong \mathbb{Z}/(p_{1}^{a_{1}}) \times ... \times \mathbb{Z}/(p_{r}^{a_{r}})$. 

\subparagraph{(ii)} Let $A,B$ be rings. Let $(x,y) \in A^{*} \times B^{*}$. Then, $x \in A^{*}$ and $y \in B^{*}$. We have that there exists $x^{-1} \in A$ and $y^{-1} \in B$ such that $xx^{-1} = 1 \in A$ and $yy^{-1} = 1 \in B$. We have that $(x,y)(x^{-1},y^{-1}) = (xx^{-1},yy^{-1}) = (1,1) \in A \times B$. Hence, $(x,y) \in (A \times B)^{*}$. For the reverse, suppos that $(x,y) \in (A \times B)^{*}$. There exists a $(x',y') \in A \times B$ such that $(x,y)(x',y') = (1,1)$. Hence, $xx' = 1$ and $yy' = 1$. Therefore, $x \in A^{*}$ and $y \in B^{*}$, so that $(x,y) \in A^{*} \times B^{*}$. It follows that $(A \times B)^{*} \cong A^{*} \times B^{*}$. Via induction, we have that $(A_{1} \times ... \times A_{n})^{*} \cong A_{1}^{*} \times ... \times A_{n}^{*}$ for rings $A_{1}, ..., A_{n}$. Therefore, 
$$(\mathbb{Z}/(n))^{*} \cong (\mathbb{Z}/(p_{1}^{a_{1}}))^{*} \times ... \times (\mathbb{Z}/(p_{r}^{a_{r}}))^{*}$$

\subparagraph{(iii)} Note that for any prime $p$ and integer $a$, there are $p^{a-1}$ integers that are not relatively prime to $p^{a}$ that are less than or equal to $p^{a}$, namely, $p, 2p, ..., p^{a-1}p$. Hence, there are $p^{a} - p^{a-1} = p^{a-1}(p - 1)$ integers that are relatively prime to $p^{a}$ that are less than or equal to $p^{a}$. It follows that $|(\mathbb{Z}/(p^{a}))^{*}| = \phi(p^{a}) = p^{a-1}(p-1)$. Therefore, 
$$\phi(n) = |(\mathbb{Z}/(n))^{*}| = |(\mathbb{Z}/(p_{1}^{a_{1}}))^{*}|...|(\mathbb{Z}/(p_{r}^{a_{r}}))^{*}| = \phi(p_{1}^{a_{1}})...\phi(p_{r}^{a_{r}}) = p_{1}^{a_{1}-1}(p_{1}-1)...p_{r}^{a_{r}-1}(p_{r}-1)$$

\paragraph{9.} Let $I$ be a nonzero ideal of $\mathbb{Z}[i]$. Let $z + I \in \mathbb{Z}[i]/I$. As $\mathbb{Z}[i]$ is a Euclidean domain, it is a PID, and so $I = (a)$ for some $a \in \mathbb{Z}[i]$. We have that $z = qa + r$ for some $q,r \in \mathbb{Z}[i]$ where either $r = 0$ or $N(r) < N(a)$. Hence, $z + I = (qa + r) + I = r + I$. There are a finite number of $r \in \mathbb{Z}[i]$ with $N(r) < N(a)$. Hence, $\mathbb{Z}[i]/I$ is finite. 

\paragraph{10.} Let $z,w \in \mathbb{Z}[i]$ be associate elements in $\mathbb{Z}[i]$. There exists a unit $u \in \mathbb{Z}[i]$ such that $z = uw$. Hence, $N(z) = N(uw) = N(u)N(w) = N(w)$. For a partial converse, suppose that $(z) = (w)$ and $N(z) = N(w)$. We have that there exists an $a \in \mathbb{Z}[i]$ such that $z = aw$. Hence, $N(w) = N(z) = N(aw) = N(a)N(w)$, so that $N(a) = 1$. It follows that  $a\overline{a} = 1$, thus, $a$ is a unit in $\mathbb{Z}[i]$. Therefore, $z$ and $w$ are associates. 

\paragraph{11.}

\paragraph{12.} Let $z,w \in \mathbb{Z}[i]$ with $w \neq 0$. Write $z = a + bi$ and $w = c + di$. Suppose that $w \mid z$ in $\mathbb{Z}[i]$. Then, $z = aw$ for some $a \in \mathbb{Z}[i]$. Suppose that $w \nmid z$. We have that $zw^{-1} = x + yi$ where $x = \frac{ac + bd}{c^{2} + d^{2}}$ and $y = \frac{bc-ad}{c^{2}+d^{2}}$. Let
$$e = \begin{cases} \lfloor x \rfloor  & \text{if } x \leq \floor{x} + 1/2 \\ \lceil x \rceil  & \text{if } x > \floor{x} + 1/2 \end{cases} \ f = \begin{cases} \floor{y} & \text{if } y \leq \floor{y} + 1/2 \\ \ceil{y} & \text{if } y > \floor{y} + 1/2 \end{cases}$$
Then, $|e - x| \leq 1/2$ and $|e - y| \leq 1/2$. Set $q = e + fi$. We have that
$$N(zw^{-1} - q) = N(x + yi - e - fi) = (x - e)^{2} + (y - f)^{2} \leq 1/2 < 1$$
Hence, $N(z - qw) < N(w)$. Set $r = z - qw$, then $z = qw + r$ with $N(r) < N(w)$. It follows that $\mathbb{Z}[i]$ is a Euclidean domain. 

\paragraph{13.} Denote the set $\{a + b\sqrt{2} \mid a,b \in \mathbb{Z}\} \subseteq \mathbb{C}$ by $\mathbb{Z}[\sqrt{2}]$.

\subparagraph{(i)} Let $a + b\sqrt{2}, c + d\sqrt{2} \in \mathbb{Z}[\sqrt{2}]$. We have that $a + b\sqrt{2} - (c + d\sqrt{2}) = (a - c) + (b - d)\sqrt{2} \in \mathbb{Z}[\sqrt{2}]$ and $(a + b\sqrt{2})(c + d\sqrt{2}) = (ac + 2bd) + (bc + ad)\sqrt{2} \in \mathbb{Z}[\sqrt{2}]$. Also note $0, 1 \in \mathbb{Z}[\sqrt{2}]$. Hence, $\sqrt{2}$ is a subring of $\mathbb{C}$. Define the map $\varphi:\mathbb{Z}[t] \to \mathbb{Z}[\sqrt{2}]$ by sending $t \mapsto \sqrt{2}$. Suppose that $f \in (t^{2} - 2)$. Then, $f = g(t)(t^{2} - 2)$ for some $g \in \mathbb{Z}[t]$, and so $\varphi(f) = g(t)0 = 0$. Hence, $f \in \ker\varphi$. For the reverse inclusion, suppose that $f \in \ker\varphi$. Write $f = \sum_{i=0}^{n}a_{i}t^{i}$. As $f \in \ker\varphi$, we have that 
\begin{align*} 
0 &= \sum_{i=0}^{n}a_{i}(\sqrt{2})^{i} \\
&= \sum_{i=0, i \text{ even}}^{n}a_{i}2^{\frac{i}{2}} + \sqrt{2}\sum_{i=0, i \text{ odd}}^{n}a_{i}2^{\frac{i-1}{2}}
\end{align*}
Hence, 
$$\sum_{i=0, i \text{ odd}}^{n}a_{i}2^{\frac{i-1}{2}} = \sum_{i=0, i \text{ even}}^{n}a_{i}2^{\frac{i}{2}} = 0$$
Now, 
\begin{align*} f(-\sqrt{2}) &= \sum_{i=0}^{n}a_{i}(-\sqrt{2})^{i} \\
&= \sum_{i=0, i \text{ even}}^{n}a_{i}2^{\frac{i}{2}} - \sqrt{2}\sum_{i=0, i \text{ odd}}^{n}a_{i}2^{\frac{i-1}{2}} \\
&= 0
\end{align*}
Therefore, $(t-\sqrt{2})(t + \sqrt{2}) = t^{2} - 2$ is a factor of $f$ and so $f = g(t)(t^{2} - 2)$ for some $g \in \mathbb{Z}[t]$. Thus, $f \in (t^{2} - 2)$. It follows that $\ker\varphi = (t^{2} - 2)$, and by the first isomorphism theorem, 
$$\mathbb{Z}[\sqrt{2}] \cong \frac{\mathbb{Z}[t]}{(t^{2}-2)}$$

\subparagraph{(ii)} Define the function $N:\mathbb{Z}[\sqrt{2}] \to \mathbb{Z}$ by $N(a + b\sqrt{2}) = a^{2} - 2b^{2}$. Let $z = a+b\sqrt{2}, w = c + d\sqrt{2} \in \mathbb{Z}[\sqrt{2}]$. Then, 
\begin{align*} 
N(zw) &= N((a+b\sqrt{2})(c+d\sqrt{2})) \\
&= N((ac + 2bd) + (bc + ad)\sqrt{2}) \\ 
&= (ac + 2bd)^{2} - 2(bc + ad)^{2} \\
&= (ac)^{2} + 4abcd + 4(bd)^{2} - 2(bc)^{2} - 4abcd - 2(ad)^{2} \\
&= (ac)^{2} + 4(bd)^{2} - 2(bc)^{2} - 2(ad)^{2} \\
&= a^{2}(c^{2} - 2d^{2}) - 2b^{2}(c^{2} - 2d^{2}) \\
&= (a^{2} - 2b^{2})(c^{2} - 2d^{2}) \\
&= N(a + b\sqrt{2})N(c + d\sqrt{2}) \\
&= N(z)N(w)
\end{align*}

\subparagraph{(iii)} We first prove that a $z \in \mathbb{Z}[\sqrt{2}]$ is a unit if and only if $N(z) = \pm 1$. Suppose that $z \in \mathbb{Z}[\sqrt{2}]$ is a unit. Then, there exists a $w \in \mathbb{Z}$ such that $zw = 1$. Hence, $1 = N(1) = N(zw) = N(z)N(w)$. Thus, as $N(z)$ is an integer, $N(z) \in \{-1,1\}$. For the converse, suppose that $N(z) = \pm 1$. Write $z = a + b\sqrt{2}$. Then, $N(z) = a^{2} - 2b^{2} = \pm 1$ so $(a + b\sqrt{2})(a - b\sqrt{2}) = \pm 1$. It follows that $z = a + b\sqrt{2}$ is a unit. Now, note that $N(1 + \sqrt{2}) = -1$, hence, $1 + \sqrt{2}$ is a unit. Consider $u_{n} = (1 + \sqrt{2})^{n}$ where $n \in \mathbb{N}$. Then, $N(u_{n}) = N((1 + \sqrt{2})^{n}) = N(1+\sqrt{2})^{n} = (-1)^{n}$. Thus, $u_{n}$ is a unit. It follows that $\mathbb{Z}[\sqrt{2}]$ has infinite many units. 

\subparagraph{(iv)} Let $z = a + b\sqrt{2} \in \mathbb{R}[\sqrt{2}] = \{a + b\sqrt{2} \mid a,b \in \mathbb{R}\}$. There exists a $x,y \in \mathbb{Z}$ such that $|a - x| \leq 1/2$ and $|b - y| \leq 1/2$. Let $w = x + y\sqrt{2}$. Then, 
$$|N(z - w)| = |N((a - x) + (b - y\sqrt{2}))| = |(a-x)^{2} - 2(b - y)^{2}| \leq |a-x|^{2} + 2|b-y|^{2} \leq \frac{3}{4} < 1$$
Therefore, for any $z \in \mathbb{R}[\sqrt{2}]$, there exists a $w \in \mathbb{Z}[\sqrt{2}]$ such that $|N(z - w)| < 1$. Now, let $z,w \in \mathbb{Z}[\sqrt{2}]$ with $w \neq 0$. We have that there exists a $q \in \mathbb{Z}[\sqrt{2}]$ such that $|N(zw^{-1} - q)| < 1$. Therefore, $|N(z - qw)| < |N(w)|$. Let $r = z - qw$. We have that $z = qw + r$ with $|N(r)| < |N(w)|$. Therefore, $\mathbb{Z}[\sqrt{2}]$ is a Euclidean domain. 

\paragraph{14.} 
\subparagraph{(i)} Define the norm $N:\mathbb{Z}[\sqrt{-2}] \to \mathbb{Z}$ by $N(a + b\sqrt{-2}) = a^{2} + 2b^{2}$. Let $z = a + b\sqrt{2} \in \mathbb{R}[\sqrt{-2}] = \{a + b\sqrt{-2} \mid a,b \in \mathbb{R}\}$. There exists a $x,y \in \mathbb{Z}$ such that $|a - x| \leq 1/2$ and $|b - y| \leq 1/2$. Let $w = x + y\sqrt{-2}$. Then, 
$$N(z - w) = N((a - x) + (b - y\sqrt{-2})) = (a-x)^{2} + 2(b - y)^{2} \leq |a-x|^{2} + 2|b-y|^{2} \leq \frac{3}{4} < 1$$
Therefore, for any $z \in \mathbb{R}[\sqrt{-2}]$, there exists a $w \in \mathbb{Z}[\sqrt{-2}]$ such that $N(z - w) < 1$. Now, let $z,w \in \mathbb{Z}[\sqrt{-2}]$ with $w \neq 0$. We have that there exists a $q \in \mathbb{Z}[\sqrt{-2}]$ such that $N(zw^{-1} - q) < 1$. Therefore, $N(z - qw) < N(w)$. Let $r = z - qw$. We have that $z = qw + r$ with $N(r) < N(w)$. Therefore, $\mathbb{Z}[\sqrt{-2}]$ is a Euclidean domain.

\subparagraph{(ii)} \textbf{FOR WHEN YOU'RE FEELING PARTICULARLY ADVENTUROUS}
\subparagraph{(iii)} 
\subparagraph{(iv)} 

\paragraph{15.} Let $k \in \mathbb{Z}$ and suppose that $n = a^{2} + b^{2}$ for some $a,b \in \mathbb{Z}$. We have that $\{[n^{2}]_{4} \mid n \in \mathbb{Z}\} = \{[0]_{4}, [1]_{4}\}$, and so $\{[n^{2}]_{4} + [m^{2}]_{4} \mid n,m \in \mathbb{Z}\} = \{[0]_{4}, [1]_{4}, [2]_{4}\}$. Therefore, $k \neq 3 \mod 4$. By taking the contrapositive, if $k \equiv 3 \mod 4$, then $k$ is not the sum of two squares. 

\paragraph{16.} Let $m,n \in \mathbb{Z}$ and suppose that there exists $a,b,c,d \in \mathbb{Z}$ such that $m = a^{2} + b^{2}$ and $n = c^{2} + d^{2}$. We have that
\begin{align*}
    mn &= (a^{2} + b^{2})(c^{2} + d^{2}) \\
    &= (a - bi)(a + bi)(c + di)(c - di) \\
    &= [(a-bi)(c+di)][(a+bi)(c-di)] \\
    &= [(ac + bd) + (ad - bc)i][(ac + bd) + (bc - ad)i] \\
    &= (ac + bd)^{2} + (ad - bc)^{2} + [(ad - bc)(ac + bd) + (ac + bd)(bc - ad)]i \\
    &= (ac + bd)^{2} + (ad - bc)^{2}
\end{align*}
Therefore, if $m,n$ can be represented as a sum of two squares, then their product, $mn$, can also be represented as a sum of two squares. 

\paragraph{17.} Let $n$ be a positive integer. 
\subparagraph{(i)} Suppose that $n$ is the sum of two squares. Then, $n = a^{2} + b^{2} = N(a + bi)$. Hence, $n$ is the norm of some complex number. Now, suppose that $n$ is the norm of some Gaussian integer. Then, $n = N(a+bi)$ for some $a,b \in \mathbb{Z}$. We have that $N(a+bi) = a^{2} + b^{2}$, thus, $n$ is the sum of two squares. 

\subparagraph{(ii)} Suppose that each integer prime factor $p$ of $n$ such that $p \equiv 3 \mod 4$ appears with an even power in $n$. Let $n = p_{1}^{a_{1}}...p_{m}^{a_{m}}$ be a prime factorisation of $n$. Suppose that $p_{i} \equiv 3 \mod 4$, then $p_{i}^{2} \equiv 1 \mod 4$. It follows that $p_{i}^{a_{i}} \equiv 1 \mod 4$ as $a_{i}$ is even by assumption. By Theorem 6.11, $p_{i}^{a_{i}}$ is then the sum of two squares. For any prime $p_{i}$ in the factorisation of $n$, if $p_{i} = 2$, then $2 = 1^{2} + 1^{2}$, and if $p_{i} \equiv 1 \mod 4$, then $p^{a_{i}} \equiv 1 \mod 4$, and it is the sum of two squares by Theorem 6.11. We have that every $p_{i}^{a_{i}}$ in the factorisation is the sum of two squares. By a previous exercise, the product of integers that can be written as a sum of two squares can also be written as a sum of two squares, hence, $n$ is the sum of two squares. For the converse, suppose that $n$ is the sum of two squares, in particular, suppose that there exists $a,b \in \mathbb{Z}$ such that $n = a^{2} + b^{2}$. By the previous part, $n$ is then the norm of a Gaussian integer, $n = N(z)$ say. We can factor $a^{2} + b^{2}$ as a product of primes $p_{1}^{a_{1}}...p_{m}^{a_{m}}$ where $a_{1}, ..., a_{m} \in \mathbb{N}$. Furthermore, as $\mathbb{Z}[i]$ is a UFD, $z$ has a prime factorisation, $w_{1}^{b_{1}}...w_{k}^{b_{k}}$ say. We then have that 
$$p_{1}^{a_{1}}...p_{m}^{a_{m}} = a^{2} + b^{2} = n = N(z) = N(w_{1}^{b_{1}}...w_{k}^{b_{k}}) = N(w_{1})^{b_{1}}...N(w_{k})^{b_{k}}$$
Suppose that $p_{i} \equiv 3 \mod 4$. By Lemma 6.7, for each $i$, $N(w_{i})$ is prime or the square of a prime. It follows that $p_{i}^{a_{i}} = N(w_{j_{1}})^{b_{j_{1}}}...N(w_{j_{v}})^{b_{j_{v}}}$ for some $j_{1},...,j_{v}$. We have that $p_{i}^{a_{i}} = N(w_{j_{1}}^{b_{j_{1}}}...w_{j_{v}}^{b_{j_{v}}})$ so $p_{i}^{a_{i}} \equiv 1 \mod 4$ as $p_{i}^{a_{i}}$ is the norm of a Gaussian integer and so the sum of two squares. It follows that $a_{i}$ must be even. 

\paragraph{18.} Suppose that $a^{2} = b^{2} \mod p$ where $a \neq b$ and $0 \leq a,b \leq (p-1)/2$. We have that $a^{2} - b^{2} = 0\mod p$. Since $p$ is prime, we have that $a = b \mod p$ or $a = -b \mod p$. As $a \neq b$ and $0 \leq a,b \leq (p-1)/2$, we cannot have that $a = b \mod p$. Assume $a = -b \mod p$. Then, $a = tp - b$ for some $t \in \mathbb{Z}$. As $0 \leq b \leq (p-1)/2$, we have that $(p(2t-1) + 1)/2 \leq tp - b \leq tp$. For any $t \in \mathbb{Z}$, we have that $(p+1)/2 \leq b$ or $b \leq 0$, which cannot occur. Therefore, it cannot occur that $a^{2} = b^{2} \mod p$ where $a \neq b$ and $0 \leq a,b \leq (p-1)/2$. Furthermore, suppose that $a \neq b$ and $0 \leq a,b \leq (p-1)/2$, and $-1-a^{2} = -1-b^{2} \mod p$. Then, $a^{2} = b^{2} \mod p$, which we have shown to be impossible. It follows that the numbers $a^{2}$ with $0 \leq a \leq (p-1)/2$ represent $(p+1)/2$ distinct classes modulo $p$, aswell as the numbers of the form $-1-b^{2}$ with $0 \leq b \leq (p-1)/2$. By the pigeonhole principle, there exists $a,b$ such that $a^{2} = -1 - b^{2} \mod p$ as there are in total $p+1$ congruence classes represented by $a^{2}$ or $-1-b^{2}$ and there are $p$ congruence classes in $\mathbb{Z}/p\mathbb{Z}$. Therefore, there exists an $n \in \mathbb{Z}$ such that $a^{2} = -1 - b^{2} + np$. Hence, there is an $n$ such that $np = 1 + a^{2} + b^{2}$. 

\paragraph{19.} Let $\mathbb{I} \subseteq \mathbb{H}$ be the set of quarternions of the form $\frac{a}{2}(1 + i + j + k) + bi + cj + dk$ with $a,b,c,d \in \mathbb{Z}$

\subparagraph{(i)} Let $\frac{a}{2}(1 + i + j + k) + bi + cj + dk, \frac{a'}{2}(1 + i + j + k) + b'i + c'j + d'k$ be elements of $\mathbb{I}$. Then, 
$$\frac{a}{2}(1 + i + j + k) + bi + cj + dk - \frac{a'}{2}(1 + i + j + k) + b'i + c'j + d'k$$
$$= \frac{a-a'}{2}(1+i+j+k) + (b-b') + (c - c')j + (d-d')k \in \mathbb{I}$$
as $a-a',b-b',c-c',d-d' \in \mathbb{Z}$. Now, let $a+bi+cj+dk, a'+b'i+c'j+d'k \in \mathbb{I}$. Furthermore, we have that
\begin{align*}
    (a+bi+cj+dk)(a'+b'i+c'j+d'k) &= aa' - bb' - cc' - dd' \\
    &+ (ab' + ba' + cd' - dc')i \\
    &+ (ac' - bd' + ca' + db')j \\
    &+ (ad' + bc' - cb' + da')k \in \mathbb{I}
\end{align*}
by looking at cases of $a,a',b,b',c,c',d,d'$. It follows that $\mathbb{I}$ is a subring of $\mathbb{H}$. 

\subparagraph{(ii)} Note that $N:\mathbb{H} \to \mathbb{R}^{+}$ is a homomorphism of the quarternions to the positive reals. Hence, it is multiplicative. It follows that for all $w_{1},w_{2} \in \mathbb{I}$, we have that $N(w_{1}w_{2}) = N(w_{1})N(w_{2})$ as $\mathbb{I}$ is a subring of $\mathbb{H}$. Let $w = \frac{a}{2}(1+i+j+k) + bi + cj + dk \in \mathbb{I}$. We have that
\begin{align*} N(w) &= N\qty(\frac{a}{2} + \qty(\frac{a}{2}+b)i + \qty(\frac{a}{2}+c)j + \qty(\frac{a}{2} + d)k) \\
&= \qty(\frac{a}{2})^{2} + \qty(\frac{a}{2}+b)^{2} + \qty(\frac{a}{2}+c)^{2} + \qty(\frac{a}{2}+d)^{2} \\
&= \frac{a^{2}}{4} + \frac{a^{2}}{4} + ab + b^{2} + \frac{a^{2}}{4} + ac + c^{2} + \frac{a^{2}}{4} + ad + d^{2} \\
&= a^{2} + b^{2} + c^{2} + d^{2} + a(b + c + d) \in \mathbb{Z}
\end{align*}

\subparagraph{(iii)} Let $u \in \mathbb{I}$ be a unit. There exists a $v \in \mathbb{I}$ such that $uv = 1 \in \mathbb{I}$. We have that $1 = N(1) = N(uv) = N(u)N(v)$. As $N(u), N(v)$ are positive integers, we have that $N(u) = 1$. We may write $u$ as $\frac{a}{2}(1+i+j+k) + bi + cj + dk$. Then, $N(u) = a^{2}+b^{2}+c^{2}+d^{2}+a(b+c+d) = 1$. As $a,b,c,d \in \mathbb{Z}$, we must have that $a,b,c,d \in \{-2,-1,0,1,2\}$. We note that if $a = 0$, then $N(u) = b^{2} + c^{2} + d^{2}$ and only one of $b,c,d$ can be non zero otherwise $N(u) > 1$. When $a = 0$, the only solutions are $(b,c,d) = (\pm 1,0,0), (0,\pm 1,0), (0,0,\pm 1)$. Take note of the solution $(a,b,c,d) = (\pm 1,0,0,0)$ aswell. Suppose that $a = \pm 1$, then $N(u) = 1 + b^{2} + c^{2} + d^{2} \pm (b + c + d)$. The only solutions are then $(a,b,c,d) = (\pm 1,\mp 1,0,0), (\pm 1,0,\mp 1,0), (\pm 1,0,0,\mp 1), (\pm 1,\mp 1,\mp 1,0), (\pm 1,\mp 1,0,\mp 1), (\pm 1, 0, \mp 1, \mp 1), (\pm 1, \mp 1, \mp 1, \mp 1)$. Finally, suppose that $a = \pm 2$. Then, $N(u) = 4 + b^{2} + c^{2} + d^{2} + \pm 2(b+c+d)$. The only solutions are $(a,b,c,d) = (\pm 2, \mp 1, \mp 1, \mp 1)$. Therefore, there are $24$ units of $\mathbb{I}$, namely, $\pm 1, \pm i, \pm j, \pm k, \frac{1}{2}(\pm 1 \pm i \pm j \pm k)$. 

\subparagraph{(iv)} Let $w \in \mathbb{I}$. Choose $z$ being of the form $\frac{a}{2}(\pm 1 \pm i \pm j \pm k)$ such that $\overline{w} + z$ is of the form $a + bi + cj + dk$ where $a,b,c,d$ are integers divisible by $2$. We have that $N(w) + wz = w\overline{w} + wz = w(\overline{w} + z)$ is of the form $p + qi + rj + sk$ where $p,q,r,s$ are integers as $\overline{w} + z$ is divisible by $2$ and it cancels with the half on the $w$ if $w$ does not have integer coefficients. And so $w(\overline{w} + z)$ can be written as a product of integer quarternions. Thus, $wz = w(\overline{w} + z) - N(w)$ is of the form $a + bi + cj + dk$ where $a,b,c,d \in \mathbb{Z}$. 

\paragraph{20.}
\subparagraph{(i)} We prove a preliminary result. Let $q = q_{1} + q_{2}i + q_{3}j + q_{4}k \in \mathbb{H}$. Choose some $n_{1} \in \mathbb{Z} \cup (\frac{1}{2} + \mathbb{Z})$ such that $|q_{1} - n_{1}| \leq 1/4$. If $n_{1}$ is an integer, we can find integers $n_{2},n_{3},n_{4}$ such that $|q_{i} - n_{i}| \leq 1/2$ for $i = 1,2,3$. Similarly, if $n_{1}$ is a half integer, we can find half integers $n_{2}, n_{3}, n_{4}$ such that $|q_{i} - n_{i}| \leq 1/2$ for $i = 1,2,3$. Let $z = n_{1} + n_{2}i + n_{3}j + n_{4}k$. We have that $z \in\mathbb{I}$ and
\begin{align*}
    N(q - z) &= N((q_{1} - n_{1}) + (q_{2} - n_{2})i + (q_{3} - n_{3})j + (q_{4} - n_{4})k) \\
    &= (q_{1} - n_{1})^{2} + (q_{2} - n_{2})^{2} + (q_{3} - n_{3})^{2} + (q_{4} - n_{4})^{2} \\
    &\leq (1/4)^{2} + (1/2)^{2} + (1/2)^{2} + (1/2)^{2} \\
    &= 13/16 \\
    &< 1
\end{align*}
Therefore, for any $q \in \mathbb{H}$, we can find a $z \in \mathbb{I}$ such that $N(q - z) < 1$. Let $z,w \in \mathbb{I}$ with $w \neq 0$. We have that there exists a $q \in \mathbb{I}$ such that $N(zw^{-1} - q) < 1$. We have that $N(z - qw) = N(w)N(zw^{-1} - q) < N(w)$, and $z = qw + z - qw$. Hence, we can find $q,r \in \mathbb{I}$ such that $z = qw + r$ with $N(r) < N(w)$.

\subparagraph{(ii)} Let $I$ be a non-trivial left-ideal of $\mathbb{I}$. Let $A = \{N(x) \mid x \in I\} \subseteq \mathbb{Z}^{\geq 0}$. $A$ has a minimal nonzero element, and so there exists a $w \in I$ such that $N(w)$ is minimal. We conjecture $I = \mathbb{I}w$. Let $rw \in \mathbb{I}w$. As $w \in I$, we must have that $rw \in I$. Hence, $\mathbb{I}w \subseteq I$. Now, let $z \in I$. We have that $z = qw + r$ for some $q,r \in \mathbb{I}$ where $N(r) < N(w)$. As $z,qw \in \mathbb{I}$, we must have that $r = z - qw \in \mathbb{I}$. As $N(w)$ is minimal among nonzero norms of elements of $I$, we must have that $N(r) = 0$. Thus, $r = 0$ and $z = qw \in \mathbb{I}w$. Therefore, $I = \mathbb{I}w$. 

\subparagraph{(iii)} Let $z,w \in \mathbb{I}$ with $w \neq 0$. We have that there exists $q,r$ such that $z = qw + r$ where $N(r) < N(w)$. Suppose that $d$ is a right divisor of $z,w$. Then, $z = z'd$ and $w = w'd$ for some $z',w'$. Hence, $r = z - qw = z'd - qw'd = (z' - qw')d$, and so $d$ is a right divisor of $r$. Now, suppose that $d$ is a right divisor of $w,r$. Then, $w = w'd$ and $r = r'd$ for some $w',r'$, and so $z = qw + r = qw'd + r'd = (qw' + r')d$. Thus, $d$ is a right divisor of $z$. It follows that the set of right divisors of $z$ and $w$ is the same as the set of right divisors of $w$ and $r$. Given $z,w \in \mathbb{I}$, we can apply division with remainder repeatedly:
$$z = q_{1}w + r_{1}$$
$$w = q_{2}r_{1} + r_{2}$$
$$r_{1} = q_{3}r_{2} + r_{3}$$
$$...$$
This process clearly terminates as $N(r_{i})$ is a positive integer and $N(r_{i+1}) < N(r_{i})$ for all $i$. Thus, the table of divisions with remainders must be as follows:
$$z = q_{1}w + r_{1}$$
$$w = q_{2}r_{1} + r_{2}$$
$$r_{1} = q_{3}r_{2} + r_{3}$$
$$...$$
$$r_{N-3} = r_{N-2}q_{N-2} + r_{N-1}$$
$$r_{N-2} = r_{N-1}q_{N-1}$$
with $r_{N-1} \neq 0$. From above, we must have that the set of right divisors of $z,w$ is the set as the set of right divisors of $r_{N-1}$. Therefore, the greatest common right divisor of $z,w$ must be $r_{N-1}$. Furthermore, from substition and working backwards, there exists $\alpha, \beta \in \mathbb{I}$ such that $\alpha z + \beta w = r_{N-1}$. 

\paragraph{21.} 
\subparagraph{(i)} Let $z \in \mathbb{I}$ and $n \in \mathbb{Z}$. Suppose that $(N(z), n) = 1$. Let $d$ be a common right divisor of $z$ and $n$. Then, $z = pd$ and $n = p'd$ for some $p,p' \in \mathbb{I}$. As $(N(z), n) = 1$, there exists $a,b \in \mathbb{Z}$ such that $aN(z) + bn = 1$. Thus, $a\overline{z} z + bn = 1$, and so $a\overline{z} pd + bp'd = 1$. Therefore, $d$ is a right divisor of $1$. We must have that the greatest common right divisor of $z$ and $n$ is $1$ (up to associates). For the converse, suppose that the greatest common right divisor of $z$ and $n$ is $1$. From the previous exercise, there exists $\alpha, \beta$ such that $\alpha z + \beta n = 1$. Then, $N(\alpha)N(z) = N(\alpha z) = N(1 - \beta n) = (1 - \beta n)(1 - \overline{\beta} n) = 1 - n(\beta + \overline{\beta}) + N(\beta)n = qn + 1$ for some $q \in \mathbb{Z}$. Let $d$ be a divisor of $N(z)$ and $n$. Then, $N(z) = pd$ and $n = p'd$ so that $N(\alpha)pd = qp'd + 1$. Thus, $d \mid 1$. It follows all divisors divide $1$, and so $(N(z), n) = 1$. 

\subparagraph{(ii)} Let $p$ be an odd prime. By a previous exercise, there exists a $n$ with $0 < n < p$ such that $np = 1 + a^{2} + b^{2}$ for some $a,b \in \mathbb{Z}$. Let $z = 1 + ai + bj$. We have that $N(z) = 1 + a^{2} + b^{2} = np$. Note that $p \mid N(z)$ and $p \mid p$, thus by the previous exercise, the greatest common right divisor of $z$ and $p$ is not a unit. Suppose that $up$ is a right divisor of $z$ where $u$ is a unit. Then, $z = (z_{0} + z_{1}i + z_{2}j + z_{3}k)up$ for some $w = z_{0} + z_{1}i + z_{2}j + z_{3}k \in \mathbb{I}$. We have that $N(z) = p^{2}N(w) = np$, and so $n = pN(w)$. As $0 < n < p$, this cannot occur as the norm of a integral quarternion is an integer. Therefore, an associate of $p$ cannot be a right divisor of $z$. It follows that the greatest common right divisor of $p$ and $z$ is not a unit and not an associate of $p$. 

\subparagraph{(iii)} Let $p$ be an odd prime. By the previous exercise, there exists a right divisor of $p$ that is not a unit or an associate of $p$, $q \in \mathbb{I}$ say. Then, $p = xq$ for some $x \in \mathbb{I}$. As $q$ is not a unit or an associate of $p$, $x$ is not a unit. Therefore, $p$ is not irreducible in $\mathbb{I}$. Suppose $p = 2$. Then, $p = (1 + i)(1 - i)$. We know that $1 + i, 1- i$ are not units in $\mathbb{I}$, and so $p$ is not irreducible. Let $q$ be a prime integer. $q$ is not irreducible so there exists $z,w \in \mathbb{I}$ that are not units and $q = zw$. We have that $q^{2} = N(zw) = N(z)N(w)$. As $q$ is prime and $N(z) \neq 1, N(w) \neq 1$, it must be that $N(z) = N(w) = q$. Therefore, every positive prime integer is the norm of some integral quarternions. 

\subparagraph{(iv)} Let $n \in \mathbb{N}$. Then, there exists primes $p_{1}, ..., p_{m}$ and $a_{1}, ..., a_{m} \in \mathbb{N}$ such that $n = p_{1}^{a_{1}}...p_{m}^{a_{m}}$. We have that for each $p_{i}$ there exists a $z_{i} \in \mathbb{I}$ such that $p_{i} = N(z_{i})$. Thus, 
$$n = N(z_{1})^{a_{1}}...N(z_{m})^{a_{m}} = N(z_{1}^{a_{1}}...z_{m}^{a_{m}})$$
Hence, $n$ is the norm of some integral quarternion.

\subparagraph{(v)} Let $n \in \mathbb{N}$. From above, $n = N(w)$ for some $w \in \mathbb{I}$. From a previous exercise, there exists a unit $u \in \mathbb{I}$ such that $uw = a + bi + cj + dk$ where $a,b,c,d \in \mathbb{Z}$. Thus, 
$$n = N(w) = N(u)N(w) = N(uw) = N(a + bi + cj + dk) = a^{2} + b^{2} + c^{2} + d^{2}$$
And we are done. 

\section*{VI - Linear Algebra}
\subsection*{6.1 - Free Modules Revisited}

\paragraph{4.} 
\subparagraph{(i)}

\subparagraph{(ii)} Let $V$ be a Lie algebra with Lie bracket $[\cdot, \cdot]:V \times V \to V$. Let $u,v \in V$. As $V$ is a vector space, $u+v \in V$, and so $[u+v,u+v] = 0$. Also note $[u,u] = [v,v] = 0$. We have that 
\begin{align*}
    [u,v] + [v,u] &= [u,u] + [u,v] + [v,u] + [v,v] \\
    &= [u,u+v] + [v,u+v] \\
    &= [u+v, u+v] \\
    &= 0
\end{align*}
Hence, $[u,v] = -[v,u]$. 

\subparagraph{(iii)} Suppose $V$ is a $k$-algebra where $k$ is a field. Define $[\cdot, \cdot]:V \times V \to V$ by $[u,v] = uv - vu$. Let $a,b \in k$ and $u,v,w \in V$. We have that 
\begin{align*} 
[au + bv,w] &= (au + bv)w - w(au + bv) \\
&= auw + bvw - awu - bwv \\
&= a(uw - wu) + b(vw - wv) \\
&= a[u,w] + b[v,w]
\end{align*}
\begin{align*}
[w,au+bv] &= w(au+bv) - (au+bv)w \\
&= awu + bwv - auw - bvw \\
&= a[w,u] + b[w,v]
\end{align*}
Furthermore, for all $v \in V$, we have that $[v,v] = vv - vv = 0$. Finally, for all $u,v,w \in V$, 
\begin{align*}
    [[u,v],w] + [[v,w],u] + [[w,u],v] &= [uv-vu,w] + [vw-wv,u] + [wu-uw,v] \\
    &= (uv-vu)w - w(uv-vu) + (vw-wv)u - u(vw - wv) + (wu - uw)v - v(wu - uw) \\
    &= uvw - vuw - wuv + wvu + vwu - wvu - uvw + uwv + wuv - uwv - vwu + vuw \\
    &= 0
\end{align*}
Therefore, $V$ is a Lie algebra with Lie bracket $[\cdot, \cdot]$. 

\subparagraph{(iv)}
\subparagraph{(v)}

\paragraph{12.} Let $V$ be a vector space over a field $k$, and let $R = \text{End}_{k-\mathsf{Vect}}(V)$ be its ring of endomorphisms. 
\subparagraph{(i)} Let $Z$ be an $R$-module and $f_{i}:Z \to R$ be $R$-module homomorphisms for $i=1,2,3,4$. If $\varphi(u,v) = (\psi_{1}(u,v), \psi_{2}(u,v)) \in \text{End}_{k-\mathsf{Vect}}(V \oplus V)$, define the maps $\pi_{i}:\text{End}_{k-\mathsf{Vect}}(V \oplus V) \to R$ for $i = 1,2,3,4$ by $\pi_{1}(\varphi) = \psi_{1}(u,0), \pi_{2}(\varphi) = \psi_{1}(0,v), \pi_{3}(\varphi) = \psi_{2}(u,0)$ and $\pi_{4}(\varphi) = \psi_{2}(0,v)$. Let $\varphi = (\psi_{1}, \psi_{2})$ and $\varphi' = (\psi_{1}', \psi_{2}')$ be elements of $\text{End}_{k-\mathsf{Vect}}(V \oplus V)$ and $r(u) \in R$. We have that
\begin{align*}
\pi_{1}(\varphi + \varphi') &= \pi_{1}((\psi_{1}, \psi_{2}) + (\psi_{1}',\psi_{2}')) \\
&= \pi_{1}((\psi_{1} + \psi_{1}', \psi_{2} + \psi_{2}')) \\
&= (\psi_{1} + \psi_{1}')(u,0) \\
&= \psi_{1}(u,0) + \psi_{1}'(u,0) \\
&= \pi_{1}((\psi_{1},\psi_{2})) + \pi_{1}((\psi_{1}',\psi_{2}')) \\
&= \pi_{1}(\varphi) + \pi_{1}(\varphi')
\end{align*}
\begin{align*}
\pi_{1}(r \cdot \varphi) &= \pi_{1}(r(u) \cdot (\psi_{1}, \psi_{2})) \\
&= \pi_{1}((r\circ \psi_{1}, r \circ \psi_{2})) \\
&= (r \circ \psi_{1})(u,0) \\
&= r(\psi_{1}(u,0)) \\
&= r \cdot \pi((\psi_{1}, \psi_{2})) \\
&= r \pi_{1}(\varphi)
\end{align*}
In a similar way, $\pi_{2}, \pi_{3}, \pi_{4}$ are also $R$-module homomorphisms. Suppose that the following diagram is commutative for $i = 1,2,3,4$:
\[\begin{tikzcd}[ampersand replacement=\&]
	\& {\text{End}_{k-\mathsf{Vect}}(V \oplus V)} \\
	Z \& {R=\text{End}_{k-\mathsf{Vect}}(V)}
	\arrow["{\pi_{i}}", from=1-2, to=2-2]
	\arrow["f", from=2-1, to=1-2]
	\arrow["{f_{i}}"', from=2-1, to=2-2]
\end{tikzcd}\]
for some $f:Z \to \text{End}_{k-\mathsf{Vect}}(V \oplus V)$. For $z \in Z$, we have that $f(z) \in \text{End}_{k-\mathsf{Vect}}(V \oplus V)$. For $u,v \in V$, we may write $f(z)[u,v] = (\psi_{1}(u,v), \psi_{2}(u,v))$. As the above diagram is commutative, we have that for each $z$ and $u,v$, $\pi_{1}(f(z)[u,v]) = f_{1}$ so $\psi_{1}(u,0) = f_{1}(z)[u]$. Similarly, $\psi_{1}(0,v) = f_{2}(z)[v], \psi_{2}(u,0) = f_{3}(z)[u]$ and $\psi_{2}(0,v) = f_{4}(z)[v]$. Hence, $f(z)[u,v] = (\psi_{1}(u,v), \psi_{2}(u,v)) = (\psi_{1}(u,0) + \psi_{1}(0,v), \psi_{2}(u,0) + \psi_{2}(0,v)) = (f_{1}(z)[u] + f_{2}(z)[v], f_{3}(z)[u] + f_{4}(z)[v])$. Hence, $f$ is unique. We prove that $f$ is a homomorphism. Let $z,w \in Z$ and $r \in R$. Then, 
\begin{align*} f(z+w)[u,v] &= (f_{1}(z+w)[u] + f_{2}(z+w)[v], f_{3}(z+w)[u] + f_{4}(z+w)[v]) \\
&= (f_{1}(z)[u] + f_{1}(w)[u] + f_{2}(z)[v] + f_{2}(w)[v], f_{3}(z)[u] + f_{3}(w)[u] + f_{4}(z)[v] + f_{4}(w)[v]) \\
&= (f_{1}(z)[u] + f_{2}(z)[v], f_{3}(z)[u] + f_{4}(z)[v]) + (f_{1}(w)[u] + f_{2}(w)[v], f_{3}(w)[u] + f_{4}(w)[v]) \\
&= f(z)[u,v] + f(w)[u,v]
\end{align*}
\begin{align*} f(rz) &= (f_{1}(rz)[u] + f_{2}(rz)[v], f_{3}(rz)[u] + f_{4}(rz)[v]) \\
&= (rf_{1}(z)[u] + rf_{2}(z)[v], rf_{3}(z)[u] + rf_{4}(z)[v]) \\
&= r(f_{1}(z)[u] + f_{2}(z)[v], f_{3}(z)[u] + f_{4}(z)[v]) \\
&= rf(z)
\end{align*}
using the properties of $R$-module homomorphisms as $f_{i}$ are $R$-module homomorphisms via assumption. It follows that $\text{End}_{k-\mathsf{Vect}}(V \oplus V)$ satisfies the universal property for $R^{4}$. Therefore, $\text{End}_{k-\mathsf{Vect}}(V \oplus V) \cong R^{4}$. 

\subparagraph{(ii)}


\end{document}

